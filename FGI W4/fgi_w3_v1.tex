\documentclass{article}
\usepackage[utf8]{inputenc}
\usepackage[ngerman]{babel}
\usepackage{lastpage}
\usepackage{color}   %May be necessary if you want to color links
\usepackage{hyperref}
% code snippets
\usepackage{listings}
% listing captions
\usepackage{caption}
% font: times new roman
\usepackage{times}
% tikz being tikz
\usepackage{tikz}
\usetikzlibrary{arrows,automata}
\usepackage{pgf}
% import math packages
\usepackage{amsmath}
\usepackage{amsfonts}
\usepackage{amssymb}
\usepackage{amsthm}
% contradiction lightning
\usepackage{stmaryrd}
% alignment options
\usepackage{ragged2e}
% page margins
\usepackage[margin=2.5cm]{geometry}

\definecolor{pblue}{rgb}{0.13,0.13,1}
\definecolor{pgreen}{rgb}{0,0.5,0}


\lstset{ %
language=Java,   							% choose the language of the code
basicstyle=\small\ttfamily,  				% the size of the fonts that are used for the code
numbers=left,                   			% where to put the line-numbers
numbersep=5pt,                  			% how far the line-numbers are from the code
backgroundcolor=\color{light-light-gray},   % choose the background color. You must add
frame=lrtb,           						% adds a frame around the code
tabsize=4,          						% sets default tabsize to 2 spaces
captionpos=b,           					% sets the caption-position to bottom
breaklines=true,        					% sets automatic line breaking
xleftmargin=1.5cm,							% space from the left paper edge
commentstyle=\color{pgreen},
keywordstyle=\color{pblue},
literate=%
    {Ö}{{\"O}}1
    {Ä}{{\"A}}1
    {Ü}{{\"U}}1
    {ß}{{\ss}}1
    {ü}{{\"u}}1
    {ä}{{\"a}}1
    {ö}{{\"o}}1
    {~}{{\textasciitilde}}1
}
\renewcommand{\lstlistingname}{Code}
\captionsetup[lstlisting]{font={footnotesize},margin=1.5cm,singlelinecheck=false } % removes "Listing 1: "
\definecolor{light-light-gray}{gray}{0.95}
\let\stdsection\section
\renewcommand\section{\stdsection}

% add line break for subtitle (size: large)
\title{Hausaufgabe der dritten Woche
    \\\large{
        Formale Grundlagen der Informatik I: Übungsgruppe 19
    }
}
\author{~\\
	\Large{Louis Kobras}\\
	\large{6658699}\\ %Matrikelnummer; wenn nicht für Uni, auskommentieren
	\large{4kobras@informatik.uni-hamburg.de} \\
	\\
	\Large{Utz Pöhlmann}\\
	\large{6663579} \\
	\large{4poehlma@informatik.uni-hamburg.de}
}

% leave empty for no date on title page
% comment for auto-generated date
\date{}


\begin{document}
	\maketitle
	\newpage
	\section*{Aufgabe 3.3}
	\label{sec:a3.3}
	$w \in \{0,1\}*$;~~~~$|w|_0=|w|_1$\\
	Angenommen, \texttt{L} $\in$ \texttt{REG}, $\Rightarrow$Pumping-Lemma gilt.\\
	Sei $k$ die Zahl des Pumping-Lemmas. Wir betrachten die Zahl $a^k b^k$.\\
	$uv=1^s~~~~|~s \le k~~~~~~~~~~~~~(|uv|=s)$\\
	$w=1^t~~~~|t \le s; t \ge 1~~~~~(|v|=t)$\\ \\
	(3) 
	\begin{align*}
		\Rightarrow~~&1^{s+t^i} \\
		\Rightarrow~~&1^{s+t=k}~~~~~~|i=1 \\
		\Rightarrow~~&1^{k+t}~~~~~~~~|i=2 \\
		\Rightarrow~~&1^{k+t\cdot (i-1)}~~~~~|i \in \mathbb{N}
	\end{align*}
	Für das Wort $1^{k+t\cdot (i-1)}a^k$ gilt $|1|=|w| \Leftrightarrow i=1$.\\
	Für zum Beispiel $i=0$ gilt es \textit{nicht}, da $k+t\cdot (0-1)=k+t\cdot(-1)=k-t$ ist und \texttt{L}$\ge1$ (laut (2)) sein muss.\\
	$\Rightarrow$mind. 1 Zerlegung gefunden$\Rightarrow$\texttt{L}$\notin$ \texttt{REG}.
	\section*{Aufgabe 3.4}
	\label{sec:a3.4}
	\begin{tikzpicture}[->,>=stealth',shorten >=1pt,auto,node distance=2.8cm,semithick]
	    \node[initial, state]   (A)                 {$z_0$};
	    \node[state]            (B) [right of=A]    {$z_1$};
	
	    \path (A) edge [loop above] 			node [align=left]{$\lambda, \perp|\lambda$\\
	    														  $0,A|\lambda$\\
	    														  $1,A|AA$\\
	    														  $1,\perp|A\perp$}				(A)
	    	  (A) edge [above, bend left=10]	node 			 {$0,\perp|A\perp$}							(B)
	    	  (B) edge [loop above]				node [align=left]{$\lambda, \perp|\lambda$\\
	    	  										  			  $1,A|\lambda$\\
	    	  										  			  $0,A|AA$} 					(B)
	    	  (B) edge [below, bend left=10]	node 			 {$1, \perp|A\perp$}		(A)
	    	  ;
	\end{tikzpicture}
	\\
	
	$L_\lambda(A) \subset M_1$: \\
	Sei $\omega \in L_\lambda(A)$, so wurde $\omega$ akzeptiert.
	Nach der Konstruktion von A und da nur mit $0,\perp$ nach $z_1$ gewechselt werden kann, bleibt der Automat in $z_0$, solange mehr 1en als 0en gelesen wurden.
	Dies wird genau dann erreicht, wenn mit einer 0 gestartet wurde oder durch lesen von mehr 0en als 1en alle \textit{A}s vom Stack genommen wurden.\\
	Werden dann nach dem Leeren des Stacks weiter 1en gelesen, wird der Stack in $z_1$ mit \textit{A}s gefüllt, welche durch das Lesen von 0en wieder gelöscht werden.
	Tritt dann wieder der Fall ein, dass mehr 1en als 0en gelesen werden, so geht der Automat zurück nach $z_0$, wo durch 0en wieder \textit{A}s hochgezählt werden. \\
	Dies wird solange wiederholt, bis nach dem Lesen von genau gleich vielen 0en und 1en der Keller leer und gleichzeitig das Wort zuende sind. \\
	$\Rightarrow L(a) \subset M_1$. \\
	\\
	$M_1 \subset L_\lambda(A)$:\\
	Sei $\omega \in M_1$, so ist $|\omega|_0=|\omega|_1$. $\omega$ ist eine Konkatenation von Worten $\omega_1,...,\omega_n; n \in \mathbb{N}^{+}$, für die gilt: $|\omega_i|_0=|\omega_i|_1;i > 1; \omega_i$ kann nicht geteilt werden, ohne dass diese Bedingungen verletzt werden. \\
	Nach Konstruktion gilt: Für jedes $\omega_i$ wird bei einer gelesenen 1 ein \textit{A} auf den Stack gepusht; für eine gelesene 0 wird ein \textit{A} gepopt.\\
	Wird $0,\perp$ gelesen, so findet ein Zustandswechsel von $z_0$ nach $z_1$ statt, wo dann für jede gelesene 0 ein \textit{A} gepusht und für jede gelesene 1 ein \textit{A} gepopt wird.\\
	Bei $1,\perp$ findet dann wiederum ein Zustandswechsel von $z_1$ nach $z_0$ statt, wo wiederum nach o.g. Verfahren mit \textit{A} auf dem Stack gearbeitet wird. \\
	Bei $\lambda, \perp$, wird $\perp$ gepopt und der Automat ist fertig, also wird das Wort akzeptiert.\\
	$\Rightarrow L_\lambda(A) \subset M_1~~~~\Rightarrow~~M_1=L_\lambda[A]$.
	\section*{Aufgabe 3.5}
	\label{sec:a3.5}
		\subsection*{3.5.1}
		\label{ssec:3.5.1}
			\[\hat{\delta}(\{z_s,z_0\},01)=\hat{\delta}(\{z_s,z_0,z_e\},1)=\hat{\delta}(\{z_s,z_1,z_e\},\lambda)=\{z_s,z_1,z_e\} \]
			\[\hat{\delta}(\{z_0,z_1\},01)=\hat{\delta}(\{z_e \},1)=\hat{\delta}(\{z_e \},\lambda)=\{z_e\}\]
		\subsection*{3.5.2}
		\label{ssec:3.5.2}
		NFA \textit{A}$=(\{z_s,z_0,z_1,z_e\},\{0,1\},Z \times \Sigma \rightarrow Z, z_s, Z_{End})$\\
			\begin{tikzpicture}[->,>=stealth',shorten >=1pt,auto,node distance=2.8cm,semithick]
			    \node[initial,state] (1)			  {$\{z_s\}$};
			    \node[state]		 (2) [right of=1] {$\{z_s,z_0\}$};
			    \node[state]		 (3) [below of=1] {$\{z_s,z_1\}$};
			    \node[state]		 (4) [right of=2] {$\{z_s,z_0,z_e\}$};
			    \node[state]		 (5) [right of=3] {$\{z_s,z_1,z_e\}$};
				\path
					(1)	edge [above]				node {0} (2)
					(1) edge [above]				node {1} (3)
					(2) edge [above]				node {0} (4)
					(2) edge [above, bend left=10]  node {1} (3)
					(3) edge [above]				node {1} (5)
					(3) edge [above, bend left=10]  node {0} (2)
					(4) edge [loop below]			node {0} (4)
					(4) edge [above, bend left=10]  node {1} (5)
					(5) edge [loop below]			node {1} (5)
					(5) edge [below, bend left=10]  node {0} (4)				
				;
			\end{tikzpicture}
		\subsection*{3.5.3}
		\label{ssec:3.5.3}
			\begin{tikzpicture}[->,>=stealth',shorten >=1pt,auto,node distance=2.8cm,semithick]
				\node[state] (1)				{$z_0$};
				\node[state] (2) [right of=1]	{$z_1$};
				\node[state] (3) [below of=1]	{$z_2$};
				\node[state] (4) [right of=3]	{$z_3$};
				\path
					(1) edge [loop above] 	node {a} (1)
					(1) edge [bend left=10] node {a} (3)
					(2) edge		        node {a} (3)
					(2) edge			    node {a} (4)
					(3) edge [bend left=10] node {b} (1)
					(3) edge 				node {b} (4)
					(4) edge [loop below]	node {b} (4)
				;
			\end{tikzpicture} \\
			\\
		Wir können weder Start- noch Endzustand rekonstruieren, da wir nicht wissen, ob das Wort vom Automaten akzeptiert wird, oder ob es sich überhaupt um ein vollständiges Wort handelt bzw. ob es ein Teilwort ist.\\
		Wir können zudem die \textit{a}-Kanten von $z_2$ und $z_3$ nicht bestimmen, ebenso wie die \textit{b}-Kante von $z_1$, da keine entsprechenden Rechnungen vorliegen.\\
		Ebenfalls können wir nicht wissen, ob es noch weitere Zustände gibt.\\
		Wir wissen lediglich, dass $z_0$ keine \textit{b}-Kante besitzt.
\end{document}