\documentclass[parskip,12pt,paper=a4,sffamily]{article}
%alternate documentclass:
%\documentclass[parskip,12pt,paper=a4,sffamily]{scrartl}
\usepackage[utf8]{inputenc}
\usepackage[ngerman]{babel}
\usepackage{lastpage}
\usepackage{color}   %May be necessary if you want to color links
\usepackage{hyperref}
% code snippets
\usepackage{listings}
% listing captions
\usepackage{caption}
% font: times new roman
%\usepackage{times}
% tikz being tikz
\usepackage{tikz}
\usetikzlibrary{arrows,automata}
\usepackage{pgf}
% import math packages
\usepackage{amsmath}
\usepackage{amsfonts}
\usepackage{amssymb}
\usepackage{amsthm}
% contradiction lightning
\usepackage{stmaryrd}
% alignment options
\usepackage{ragged2e}
% page margins
\usepackage[margin=2.5cm]{geometry}

\definecolor{pblue}{rgb}{0.13,0.13,1}
\definecolor{pgreen}{rgb}{0,0.5,0}


\lstset{ %
language=Tex,   							% choose the language of the code
basicstyle=\small\ttfamily,  				% the size of the fonts that are used for the code
numbers=left,                   			% where to put the line-numbers
numbersep=5pt,                  			% how far the line-numbers are from the code
backgroundcolor=\color{light-light-gray},   % choose the background color. You must add
frame=lrtb,           						% adds a frame around the code
tabsize=4,          						% sets default tabsize to 2 spaces
captionpos=b,           					% sets the caption-position to bottom
breaklines=true,        					% sets automatic line breaking
xleftmargin=1.5cm,							% space from the left paper edge
commentstyle=\color{pgreen},
keywordstyle=\color{pblue},
literate=%
    {Ö}{{\"O}}1
    {Ä}{{\"A}}1
    {Ü}{{\"U}}1
    {ß}{{\ss}}1
    {ü}{{\"u}}1
    {ä}{{\"a}}1
    {ö}{{\"o}}1
    {~}{{\textasciitilde}}1
}
\renewcommand{\lstlistingname}{Code}
\captionsetup[lstlisting]{font={footnotesize},margin=1.5cm,singlelinecheck=false } % removes "Listing 1: "
\definecolor{light-light-gray}{gray}{0.95}
\let\stdsection\section
\renewcommand\section{\stdsection}

% add line break for subtitle (size: large)
\title{Was ist \LaTeX ?
    \\\large{
        Warum Informatiker darunter etwas Anderes verstehen als Menschen
    }
}
\author{~\\
	\Large{Louis Kobras}\\
	\large{4kobras@informatik.uni-hamburg.de}\\
	\\
	\Large{Hauke Stieler}\\
	\large{4stieler@informatik.uni-hamburg.de}
}

% leave empty for no date on title page
% comment for auto-generated date
\date{\today}


\begin{document}
	\maketitle
	\newpage
	\tableofcontents
	\clearpage
	\section{Einführung}
	\label{sec:intro}
	\section{Dokumentklassen}
	\label{sec:class}
	\subsection{Beamer}
	\label{ssec:beamer}
	\subsection{Article}
	\label{ssec:article}
	\section{Pakete}
	\label{ssec:package}
	\section{Mathmode}
	\label{sec:math}
	Für eine ausgiebige Nutzung des sogenannten \textsc{Mathmode} sollten folgene Pakete genutzt werden:
	\begin{lstlisting}
% import math packages
\usepackage{amsmath}
\usepackage{amsfonts}
\usepackage{amssymb}
\usepackage{amsthm}	
	\end{lstlisting}
	Dieser Code-Abschnitt ist im Präambel einzufügen.
	Der \textsc{Mathmode} kann in Latex durch verschiedene Methoden initiiert werden.
	Für Rechnungen, die sich über eine einzelne Zeile erstrecken, eignet sich der Operator
	\lstinline{\[} zum Öffnen sowie der dazugehörende schließende Operator \lstinline{\]}.
	Sind nur kurze Abschnitte innerhalb eines Satzes einzufügen, so liegt der \lstinline{$}-Operator nahe.
	Hier sind der Operator zum Öffnen und der zum Schließen identisch.\\
	\textit{Anmerkung: In \TeX wird der Operator \lstinline{$$} verwendet. Dies ist in \LaTeX nicht der Fall.}\\
	Diese beiden Operator-Sets sind in \LaTeX von Grund auf enthalten.
	Das eben genannte Package \textsc{amsmath} liefert einige Erweiterungen für den \lstinline{\begin}-Befehl,
	durch welche mehrzeilige Gleichungsblöcke sowie erweiterte Layout- und Formatierungsoptionen zur Verfügung stehen.
	\section{Source-Code}
	\label{sec:source}
	\section{(TikZ)}
	\label{sec:tikz}
	Optional, verhandelbar
	\section{Ein Quellenverzeichnis erstellen}
	\label{sec:ref}
	\section{Tools}
	\label{sec:tools}
	Selbstverständlich steht man mit \LaTeX nicht alleine da.
	Das wichtigste Hilfsmittel ist natürlich \textbf{Google} oder eine andere Suchmaschine deines Vertrauens.
	Doch neben Google geben wir hier zwei sehr nützliche Hilfsmittel, die empfehlenswert sind, wenn man viel und häufig mit \LaTeX arbeitet.
	\subsection{Detexify}
	\label{ssec:detex}
	\textbf{Detexify} ist ein (noch) Web-Service, der die Suche nach den Befehlen für Sonderzeichen erheblich erleichtert.
	Die Funktionsweise ist einfach: Mit der Maus zeichnet man ein Symbol in ein Feld, der Service interpretiert die Zeichnung und gibt einem die Interpretationen sowie die dazugehörigen Befehle, das den Befehl enthaltende Paket und ob Text- oder Mathmode-Symbol zurück.\\
	\textit{[Grafiken]Leere Seite - nach Zeichnung}
	\subsection{\TeX maker}
	\label{ssec:texmaker}
	\textbf{\TeX maker} ist eine IDE für \LaTeX.
	Nicht mehr und nicht weniger.
	Dropdown-Menüs für Befehlsreferenzen, mehr Dropdown-Menüs für Zusatzfunktionen, Hotkey für Compile-Aktionen und ein eingebauter Viewer für das Ausgabe-PDF.
	\TeX maker ist für sämtliche Plattformen frei verfügbar und verfügt über eine Funktion, um importierte Pakete automatisch herunterzuladen, sofern sie auf dem System nicht gefunden wurden.
\end{document}