\documentclass[parskip,12pt,paper=a4,sffamily]{article}
%alternate documentclass:
%\documentclass[parskip,12pt,paper=a4,sffamily]{scrartl}
\usepackage[utf8]{inputenc}
\usepackage[ngerman]{babel}
\usepackage{lastpage}
\usepackage{color}   %May be necessary if you want to color links
\usepackage{hyperref}
% code snippets
\usepackage{listings}
% listing captions
\usepackage{caption}
% font: times new roman
\usepackage{times}
% tikz being tikz
\usepackage{tikz}
\usetikzlibrary{arrows,automata}
\usepackage{pgf}
% import math packages
\usepackage{amsmath}
\usepackage{amsfonts}
\usepackage{amssymb}
\usepackage{amsthm}
% contradiction lightning
\usepackage{stmaryrd}
% alignment options
\usepackage{ragged2e}
% page margins
\usepackage[margin=2.5cm]{geometry}

\definecolor{pblue}{rgb}{0.13,0.13,1}
\definecolor{pgreen}{rgb}{0,0.5,0}


\lstset{ %
language=Java,   							% choose the language of the code
basicstyle=\small\ttfamily,  				% the size of the fonts that are used for the code
numbers=left,                   			% where to put the line-numbers
numbersep=5pt,                  			% how far the line-numbers are from the code
backgroundcolor=\color{light-light-gray},   % choose the background color. You must add
frame=lrtb,           						% adds a frame around the code
tabsize=4,          						% sets default tabsize to 2 spaces
captionpos=b,           					% sets the caption-position to bottom
breaklines=true,        					% sets automatic line breaking
xleftmargin=1.5cm,							% space from the left paper edge
commentstyle=\color{pgreen},
keywordstyle=\color{pblue},
literate=%
    {Ö}{{\"O}}1
    {Ä}{{\"A}}1
    {Ü}{{\"U}}1
    {ß}{{\ss}}1
    {ü}{{\"u}}1
    {ä}{{\"a}}1
    {ö}{{\"o}}1
    {~}{{\textasciitilde}}1
}
\renewcommand{\lstlistingname}{Code}
\captionsetup[lstlisting]{font={footnotesize},margin=1.5cm,singlelinecheck=false } % removes "Listing 1: "
\definecolor{light-light-gray}{gray}{0.95}
\let\stdsection\section
\renewcommand\section{\stdsection}

% add line break for subtitle (size: large)
\title{Formale Grundlagen der Informatik I
    \\\large{
        Abgabe der Hausaufgaben
    }
    \\\large{
    	Übungsgruppe 24 am \today
    }
}
\author{~\\
	\Large{Louis Kobras}\\
	\large{6658699}\\ %Matrikelnummer; wenn nicht für Uni, auskommentieren
	\large{4kobras@informatik.uni-hamburg.de}\\
	\\
	\Large{Utz Pöhlmann}\\
	\large{6663579}\\
	\large{4poehlma@informatik.uni-hamburg.de}\\
	\\
	\Large{}Philipp Quach\\
	\large{6706421}\\
	\large{4quach@informatik.uni-hamburg.de}\\
}

% leave empty for no date on title page
% comment for auto-generated date
\date{\today}


\begin{document}
	\maketitle
	\newpage
	\section*{Aufgabe 7.3}
	\label{sec:a7.3}
	\subsection*{Aufgabe 7.3.1}
	\label{ssec:a7.3.1}
	%-----AUFGABE-----%
	Zeigen Sie, dass aus $L_1,L_2\in P$ auch $L_1\cap L_2 \in P$ und $\overline{L_1}\in P$ folgt, dass also $P$ gegenüber Durchschnitts- und Komplementbildung abgeschlossen ist.\\
	\\
	\\
	%-----LÖSUNG-----%
	Sei $\mu$ in $L_1 \cap L_2$.
	Dann ist $\mu$ in $L_1$ oder $L_2$, aber auch in $P$.\\
	Sei $\mu$ in $L_1$.
	Dann ist $\mu$ in $P$, aber auch in $L_1 \cap L_2$.\\
	Sei $\mu$ in $L_2$.
	Dann ist $\mu$ in $P$, aber auch in $L_1 \cap L_2$.\\
	Daraus folgt, dass $L_1,L_2 \in P\Rightarrow L_1 \cap L_2 \in P.$\\
	\\
	Sei $L_1 \underline{\subset} P$. 
	Dann gilt: $L_1=P \setminus \overline{L_1}$.
	Daraus ergibt sich $\overline{L_1} = P \setminus L_1~~~\Rightarrow~\overline{L_1}\in P$.
	\subsection*{Aufgabe 7.3.2}
	\label{ssec:a7.3.2}
	%-----AUFGABE-----%
	Zeigen Sie, dass aus $L_1,L_2\in NP$ auch $L_1 \cap L_2 \in NP$ folgt.\\
	\\
	\\
	%-----LÖSUNG-----%
	Sei $\mu$ in $L_1 \cap L_2$.
	Dann ist $\mu$ in $L_1$ oder $L_2$, aber auch $NP$.\\
	Sei $\mu$ in $L_1$.
	Dann ist $\mu$ in $NP$, aber auch in $L_1 \cap L_2$.\\
	Sei $\mu$ in $L_2$.
	Dann ist $\mu$ in $NP$, aber auch in $L_1 \cap L_2$.\\
	Daraus folgt, dass $L_1,L_2 \in NP\Rightarrow L_1 \cap L_2 \in NP.$\\
	%-------------------------------------------------%
	\section*{Aufgabe 7.4}
	\label{ssec:a7.4}
	%-----AUFGABE-----%
	Zeigen Sie, dass das Färbungsproblem (siehe unten) in \textit{NP} liegen,
	indem Sie einen Verifikationsalgorithmus mit polynomialer Laufzeit fur dieses Problem angeben.
	Beachten Sie dabei, dass Sie das Zertifikat spezifizieren müssen.
	\begin{itemize}
		\item Das Färbungsproblem \\\begin{itemize}
			\item[-] Eingabe: Ein ungerichteter Graph $G=(V,E)$ und ein $k\in\mathbb{N}$
			\item[-] Frage: Kann $G$ mit $k$ Farben gefärbt werden, d.h. gibt es eine Funktion $c:V\rightarrow \{1,...,k\}$ derart, dass $c(u)\ne c(v)$ für jede Kante $\{u,v\}\in E$ gilt?
		\end{itemize}
	\end{itemize}
	%-----LÖSUNG-----%
	%-------------------------------------------------%
	\section*{Aufgabe 7.5}
	\label{sec:a7.5}
	Betrachten Sie das folgende Problem:\\
	\textbf{Gegeben:} Zwei Graphen $G_1=(V_1,E_1) und G_2=(V_2,E_2)$.\\
	\textbf{Frage:} Gibt es Teilmengen $V\underline{\subset}V_1$ und $E\underline{\subset}E_1$ derart, dass $|V|=|V_2|$ und $|E|=|E_2|$ gilt und eine bijektive Abbildung
	$f:V_2\rightarrow V$ existiert mit $\{u,v\}\in E_2$ genau dann, wenn $\{f(u),f(v)\}\in E$?
	\subsection*{Aufgabe 7.5.1}
	\label{ssec:a7.5.1}
	%-----AUFGABE-----%
	Zeigen Sie, dass das Problem in \textit{NP} liegt, indem Sie einen \textit{NP}-Algorithmus angeben, der das Problem löst.\\
	\\
	\\
	%-----LÖSUNG-----%
	\subsection*{Aufgabe 7.5.2}
	\label{ssec:a7.5.2}
	%-----AUFGABE-----%
	Beweisen Sie, dass das Problem \textit{NP}-hart (und damit insgesamt \textit{NP}-vollständig) ist, indem Sie eine Reduktion von einem Ihnen bekannten \textit{NP}-vollständigen Problem angeben.
	\\
	\\
	\\
	%-----LÖSUNG-----%
	%-------------------------------------------------%
	\section*{Aufgabe 7.6}
	\label{sec:a7.6}
	%-----AUFGABE-----%
	Sei A ein Algorithmus, der eine konstante Anzahl von Aufrufen von Unterroutinen enthält.
	Zählt man jeden dieser Aufrufe als einen Schritt, so sei A ein Polynomialzeit-Algorithmus.
	Zeigen Sie, dass A insgesamt in polynomialer Zeit läuft, wenn die Unterroutinen in polynomialer Zeit laufen.
	Zeigen Sie ferner, dass ein Algorithmus mit exponentieller Laufzeit entstehen kann, wenn ein Algorithmus eine polynomiale Anzahl von Aufrufen von Unterroutinen mit polynomialer Laufzeit enthält.\\
	\\
	\\
	%-----LÖSUNG-----%
\end{document}