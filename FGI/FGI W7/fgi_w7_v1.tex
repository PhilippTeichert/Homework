\documentclass[parskip,12pt,paper=a4,sffamily]{article}
%alternate documentclass:
%\documentclass[parskip,12pt,paper=a4,sffamily]{scrartl}
\usepackage[utf8]{inputenc}
\usepackage[ngerman]{babel}
\usepackage{lastpage}
\usepackage{color}   %May be necessary if you want to color links
\usepackage{hyperref}
% code snippets
\usepackage{listings}
% listing captions
\usepackage{caption}
% font: times new roman
\usepackage{times}
% tikz being tikz
\usepackage{tikz}
\usetikzlibrary{arrows,automata}
\usepackage{pgf}
% import math packages
\usepackage{amsmath}
\usepackage{amsfonts}
\usepackage{amssymb}
\usepackage{amsthm}
% contradiction lightning
\usepackage{stmaryrd}
% alignment options
\usepackage{ragged2e}
% page margins
\usepackage[margin=2.5cm]{geometry}

\definecolor{pblue}{rgb}{0.13,0.13,1}
\definecolor{pgreen}{rgb}{0,0.5,0}


\lstset{ %
language=Java,   							% choose the language of the code
basicstyle=\small\ttfamily,  				% the size of the fonts that are used for the code
numbers=left,                   			% where to put the line-numbers
numbersep=5pt,                  			% how far the line-numbers are from the code
backgroundcolor=\color{light-light-gray},   % choose the background color. You must add
frame=lrtb,           						% adds a frame around the code
tabsize=4,          						% sets default tabsize to 2 spaces
captionpos=b,           					% sets the caption-position to bottom
breaklines=true,        					% sets automatic line breaking
xleftmargin=1.5cm,							% space from the left paper edge
commentstyle=\color{pgreen},
keywordstyle=\color{pblue},
literate=%
    {Ö}{{\"O}}1
    {Ä}{{\"A}}1
    {Ü}{{\"U}}1
    {ß}{{\ss}}1
    {ü}{{\"u}}1
    {ä}{{\"a}}1
    {ö}{{\"o}}1
    {~}{{\textasciitilde}}1
}
\renewcommand{\lstlistingname}{Code}
\captionsetup[lstlisting]{font={footnotesize},margin=1.5cm,singlelinecheck=false } % removes "Listing 1: "
\definecolor{light-light-gray}{gray}{0.95}
\let\stdsection\section
\renewcommand\section{\stdsection}

% add line break for subtitle (size: large)
\title{Formale Grundlagen der Informatik I
    \\\large{
        Abgabe der Hausaufgaben
    }
    \\\large{
    	Übungsgruppe 24 am \today
    }
}
\author{~\\
	\Large{Louis Kobras}\\
	\large{6658699}\\ %Matrikelnummer; wenn nicht für Uni, auskommentieren
	\large{4kobras@informatik.uni-hamburg.de}\\
	\\
	\Large{Utz Pöhlmann}\\
	\large{6663579}\\
	\large{4poehlma@informatik.uni-hamburg.de}\\
	\\
	\Large{}Philipp Quach\\
	\large{6706421}\\
	\large{4quach@informatik.uni-hamburg.de}\\
}

% leave empty for no date on title page
% comment for auto-generated date
\date{\today}


\begin{document}
	\maketitle
	\newpage
	\section*{Aufgabe 6.4}
	\label{sec:a6.4}
	\subsection*{Aufgabe 6.4.1}
	\label{ssec:a6.4.1}
	\begin{tikzpicture}[->,>=stealth',shorten >=1pt,auto,node distance=3.0cm,semithick]
		\node [initial, state]		(A)					{$z_1$};
		\node [state]				(B) [right of=A]	{$z_2$};
		\node [state]				(C) [right of=B]	{$z_3$};
		\node [accepting, state]	(D) [right of=C]	{$z_E$};
		\node [state]				(E) [right of=D]	{$z_{F}$};
		
		\path
			(A) edge [loop above]		node [align=left]	{$a,\perp|\perp$}						(A)
			(A) edge					node [align=left]	{$b,\perp|b\perp$}						(B)
			(A) edge [bend right=30]	node [align=left]	{$\lambda,\perp|\perp$}					(D)
			(B) edge [loop above]		node [align=left]	{$b,b|B$ \\ $b,B|bB$}					(B)
			(B) edge					node [align=left]	{$a,B|\lambda$}							(C)
			(B) edge [bend right=30]	node [align=left]	{$a,b|\lambda$}							(E)
			(C) edge [loop above]		node [align=left]	{$a,B|\lambda$}							(C)
			(C) edge					node [align=left]	{$\lambda,\perp|\perp$} 				(D)
			(D) edge					node [align=left]	{$b,\perp|\perp$ \\ $a,\perp|\perp$}	(E)
		;
	\end{tikzpicture}\\
	$z_E$ ist ein Endzustand und $z_F$ ist ein Fehlerzustand.\\
	\\
	$L\underline{\subset}L(A)$\\
	In $z_1$ wird $a^n$ gelesen: $n\in\mathbb{N}$\\
	In $z_2$ wird $b^{2m}$ gelesen und alle 2 $b$s ein $B$ auf den Stack gepusht.\\
	Nach $z_2$ kann ein $\lambda$ gelesen werden für den Fall $n=0$ und ein $b$ für den Fall $m\in\mathbb{N}$.
	Für $m=0$ gibt es eine $\lambda$-Kante nach $z_E$ von $z_1$.
	Sonst geht es mit $a,B|\lambda$ nach $z_3$, wo für jedes $a$ ein $B$ gelöscht wird.
	Wenn dann der Stack leer ist, geht es in $z_E$.
	Sollte dann das Wort nicht zu Ende gelesen sein, geht es in $z_F$.\\
	\\
	$L(A)\underline{\subset}L$\\
	Zuerst werden beliebig viele $a$s gelesen (auch $0$)($\Rightarrow a^n|n\in\mathbb{N}_0$),
	danach bielebig viele $b$s (auch $0$ möglich durch $\lambda,\perp|\perp$ nach $z_E$)
	und alle 2 $b$s ein $B$ auf das Band geschrieben. ($\Rightarrow b^{2m}|m\in\mathbb{N}_0$).\\
	Dann kann für jedes $B$ wieder ein $a$ gelesen werden, bei einer ungeraden Anzahl $b$s und einem $a$ dahinter wird abgebrochen.\\
	Dann wird für jedes $B$ ein $a$ gelesen ($\Rightarrow a^m$).
	Sobald danach noch ein Buchstabe kommt, brechen wir ab, somit sind wir fertig.
	\subsection*{Aufgabe 6.4.2}
	\label{ssec:a6.4.2}
	%-------------------------------------------------%
	\newpage
	\section*{Aufgabe 6.5}
	\label{sec:a6.5}
	\subsection*{Aufgabe 6.5.1}
	\label{ssec:a6.5.1}
	\subsection*{Aufgabe 6.5.2}
	\label{ssec:a6.5.2}
	\subsection*{Aufgabe 6.5.3}
	\label{ssec:a6.5.3}
	%-------------------------------------------------%
	\newpage
	\section*{Aufgabe 6.5}
	\label{sec:a6.5}
\end{document}