\documentclass[parskip,12pt,paper=a4,sffamily]{article}
%alternate documentclass:
%\documentclass[parskip,12pt,paper=a4,sffamily]{scrartl}
\usepackage[utf8]{inputenc}
\usepackage[ngerman]{babel}
\usepackage{lastpage}
\usepackage{color}   %May be necessary if you want to color links
\usepackage{hyperref}
% code snippets
\usepackage{listings}
% listing captions
\usepackage{caption}
<<<<<<< HEAD
=======
% font: times new roman
\usepackage{times}
>>>>>>> 21a8c6cd8f4ae65a4b60804810bf98a1251d878b
% tikz being tikz
\usepackage{tikz}
\usetikzlibrary{arrows,automata}
\usepackage{pgf}
% import math packages
\usepackage{amsmath}
\usepackage{amsfonts}
\usepackage{amssymb}
\usepackage{amsthm}
% contradiction lightning
\usepackage{stmaryrd}
% alignment options
\usepackage{ragged2e}
% page margins
\usepackage[margin=2.5cm]{geometry}

\definecolor{pblue}{rgb}{0.13,0.13,1}
\definecolor{pgreen}{rgb}{0,0.5,0}


\lstset{ %
language=Java,   							% choose the language of the code
basicstyle=\small\ttfamily,  				% the size of the fonts that are used for the code
numbers=left,                   			% where to put the line-numbers
numbersep=5pt,                  			% how far the line-numbers are from the code
backgroundcolor=\color{light-light-gray},   % choose the background color. You must add
frame=lrtb,           						% adds a frame around the code
tabsize=4,          						% sets default tabsize to 2 spaces
captionpos=b,           					% sets the caption-position to bottom
breaklines=true,        					% sets automatic line breaking
xleftmargin=1.5cm,							% space from the left paper edge
commentstyle=\color{pgreen},
keywordstyle=\color{pblue},
literate=%
    {Ö}{{\"O}}1
    {Ä}{{\"A}}1
    {Ü}{{\"U}}1
    {ß}{{\ss}}1
    {ü}{{\"u}}1
    {ä}{{\"a}}1
    {ö}{{\"o}}1
    {~}{{\textasciitilde}}1
}
\renewcommand{\lstlistingname}{Code}
\captionsetup[lstlisting]{font={footnotesize},margin=1.5cm,singlelinecheck=false } % removes "Listing 1: "
\definecolor{light-light-gray}{gray}{0.95}
\let\stdsection\section
\renewcommand\section{\stdsection}

% add line break for subtitle (size: large)
\title{Formale Grundlagen der Informatik I
    \\\large{
        Abgabe der Hausaufgaben\\
        Übungsgruppe 24 am Freitag, d. \today
    }
}
\author{~\\
	\Large{Louis Kobras}\\
	\large{6658699}\\ %Matrikelnummer; wenn nicht für Uni, auskommentieren
	\large{4kobras@informatik.uni-hamburg.de}\\
	\\
	\Large{Utz Pöhlmann}\\
	\large{6663579}\\ %Matrikelnummer; wenn nicht für Uni, auskommentieren
	\large{4poehlma@informatik.uni-hamburg.de}\\
	\\
	\Large{Philipp Quach}\\
	\large{6706421}\\ %Matrikelnummer; wenn nicht für Uni, auskommentieren
	\large{4quach@informatik.uni-hamburg.de}
}

% leave empty for no date on title page
% comment for auto-generated date
\date{\today}


\begin{document}
	\maketitle\thispagestyle{empty}
	\newpage
	\pagenumbering{arabic}
	\newgeometry{top=2.5cm, bottom =2.5cm, left=2cm, right=3.5cm}
<<<<<<< HEAD
	\section*{Aufgabe 10.4}
	\label{sec:a10.4}
	\begin{flushright}
	\large{[~~~~/2]}
	\end{flushright}
	%-----AUFGABE-----%
	BeweisennSie, dass eine Inferenzregel $R=\frac{F_1,\dots,F_n}{G}$ genau dann korrekt ist,
	wenn $\{F_1,\dots,F_n\} \vDash G$ gilt.
	(Nutzen Sie dazu die Definition der Korrektheit einer Inferenzregel auf Folie 31.)
	\-\\\\\\
	%-----RECHNUNG-----%
	%++++++++++++++++++++++++++++++++++++++++++++++++++++++++++%
	\section*{Aufgabe 10.5}
	\label{sec:a10.5}
	\begin{flushright}
	\large{[~~~~/3]}
	\end{flushright}
	%-----AUFGABE-----%
	\subsection*{Aufgabe 10.5.1}
	\label{ssec:a10.5.1}
	Seien $F=((A \Leftrightarrow B ) \land B \land \lnot C)$ und $G=((B \lor \lnot C) \Leftrightarrow \lnot C ) \land \lnot C \land \lnot (B \lor \lnot C)$.
	Geben Sie eine Substitution \textit{sub} an mit $sub(F)=G$ oder begründen Sie, warum dies nicht möglich ist.
	\-\\\\\\
	%-----RECHNUNG-----%
	%-----AUFGABE-----%
	\subsection*{Aufgabe 10.5.2}
	\label{ssec:a10.5.2}
	Zeigen Sie, dass für jede Formel \textit{F} und jede Substitution \textit{sub} gilt:
	Wenn \textit{F} eine Tautologie ist, dann ist auch $sub(F)$ eine Tautologie.
	Vervollständigen Sie dazu den Beweis aus der Vorlesung.
	Führen Sie insb. die dort nicht ausgeführte strukturelle Induktion.
	\-\\\\\\
	%-----RECHNUNG-----%
	%++++++++++++++++++++++++++++++++++++++++++++++++++++++++++%
	\section*{Aufgabe 10.6}
	\label{sec:a10.6}
	\begin{flushright}
	\large{[~~~~/7]}
	\end{flushright}
	%-----AUFGABE-----%
	\subsection*{Aufgabe 10.6.1}
	\label{ssec:10.6.1}
	Zeigen oder Widerlegen Sie, dass die folgenden Inferenzregeln korrekt sind:
	\begin{equation*}
		\begin{split}
			\frac{A \Rightarrow B, B \Rightarrow A}{\lnot B \lor A}
		\end{split}
		{}~~~~~~~~~~~~~~~~{}
		\begin{split}
			\frac{(A \lor B) \Rightarrow C, \lnot C \land \lnot B}{A \lor B}
		\end{split}
	\end{equation*}
	\-\\\\\\
	%-----RECHNUNG-----%
	%-----AUFGABE-----%
	\subsection*{Aufgabe 10.6.2}
	\label{ssec:10.6.2}
	Sei $\mathcal{C}=(\mathcal{L}_{AL},Ax,\mathcal{R})$ ein Kalkül der Aussagenlogik mit $Ax=\{A\Rightarrow B(\Rightarrow A)\}$ und $R=\{\frac{F,F\Rightarrow G}{G},\frac{F \land G}{G}\}$.
	Sei ferner $M=\{A\land B,(C\Rightarrow (A\land B))\Rightarrow (B \land A)\}$.\\
	Zeigen Sie $M\vdash_{\mathcal{c}}A$ durch Angabe einer Ableitung.
	\-\\\\\\
	%-----RECHNUNG-----%
=======
	\section*{Aufgabe *}
	\label{sec:a*}
	\begin{flushright}
	\large{[~~~~/*]}
	\end{flushright}
	%-----AUFGABE-----%
	\-\\\\\\
	%-----RECHNUNG-----%
	%++++++++++++++++++++++++++++++++++++++++++++++++++++++++++%
>>>>>>> 21a8c6cd8f4ae65a4b60804810bf98a1251d878b
\end{document}