\documentclass[parskip,12pt,paper=a4,sffamily]{article}
%alternate documentclass:
%\documentclass[parskip,12pt,paper=a4,sffamily]{scrartl}
\usepackage[utf8]{inputenc}
\usepackage[ngerman]{babel}
\usepackage{lastpage}
\usepackage{color}   %May be necessary if you want to color links
\usepackage{hyperref}
% code snippets
\usepackage{listings}
% listing captions
\usepackage{caption}
% font: times new roman
\usepackage{times}
% tikz being tikz
\usepackage{tikz}
\usetikzlibrary{arrows,automata}
\usepackage{pgf}
% import math packages
\usepackage{amsmath}
\usepackage{amsfonts}
\usepackage{amssymb}
\usepackage{amsthm}
% contradiction lightning
\usepackage{stmaryrd}
% alignment options
\usepackage{ragged2e}
% page margins
\usepackage[margin=2.5cm]{geometry}

\definecolor{pblue}{rgb}{0.13,0.13,1}
\definecolor{pgreen}{rgb}{0,0.5,0}


\lstset{ %
language=Java,   							% choose the language of the code
basicstyle=\small\ttfamily,  				% the size of the fonts that are used for the code
numbers=left,                   			% where to put the line-numbers
numbersep=5pt,                  			% how far the line-numbers are from the code
backgroundcolor=\color{light-light-gray},   % choose the background color. You must add
frame=lrtb,           						% adds a frame around the code
tabsize=4,          						% sets default tabsize to 2 spaces
captionpos=b,           					% sets the caption-position to bottom
breaklines=true,        					% sets automatic line breaking
xleftmargin=1.5cm,							% space from the left paper edge
commentstyle=\color{pgreen},
keywordstyle=\color{pblue},
literate=%
    {Ö}{{\"O}}1
    {Ä}{{\"A}}1
    {Ü}{{\"U}}1
    {ß}{{\ss}}1
    {ü}{{\"u}}1
    {ä}{{\"a}}1
    {ö}{{\"o}}1
    {~}{{\textasciitilde}}1
}
\renewcommand{\lstlistingname}{Code}
\captionsetup[lstlisting]{font={footnotesize},margin=1.5cm,singlelinecheck=false } % removes "Listing 1: "
\definecolor{light-light-gray}{gray}{0.95}
\let\stdsection\section
\renewcommand\section{\stdsection}

% add line break for subtitle (size: large)
\title{Formale Grundlagen der Informatik I
    \\\large{
        Abgabe der Hausaufgaben\\
        Übungsgruppe 24 am \today
    }
}
\author{~\\
	\Large{Louis Kobras}\\
	\large{6658699}\\ %Matrikelnummer; wenn nicht für Uni, auskommentieren
	\large{4kobras@informatik.uni-hamburg.de}\\
	\\
	\Large{Utz Pöhlmann}\\
	\large{6663579}\\ %Matrikelnummer; wenn nicht für Uni, auskommentieren
	\large{4poehlma@informatik.uni-hamburg.de}\\
	\\
	\Large{Philipp Quach}\\
	\large{6706421}\\ %Matrikelnummer; wenn nicht für Uni, auskommentieren
	\large{4quach@informatik.uni-hamburg.de}
}

% leave empty for no date on title page
% comment for auto-generated date
\date{\today}


\begin{document}
	\maketitle
	\newpage
	\newgeometry{top=2.5cm, bottom =2.5cm, left=2cm, right=3.5cm}
	\section*{Aufgabe 8.3}
	\label{sec:a8.3}
	\begin{flushright}
	\large{[~~~~/4]}
	\end{flushright}
	%-----AUFGABE-----%
	Geben Sie für jede der folgenden Formeln jeweils an, ob diese erfüllbar ist, falsifizierbar, kontingent, allgemeingültig oder unerfüllbar.
	\begin{enumerate}
		\item $((A \Rightarrow B) \Rightarrow A)$
		\item $((A \land B) \Leftrightarrow (\lnot A \lor \lnot B))$
		\item $(((((A \land B) \land C) \land D) \land E) \Rightarrow (\lnot A \lor E))$
		\item $(((C \Rightarrow B) \lor A) \land (A \lor \lnot B))$
	\end{enumerate}
	\-\\\\\\
	%-----RECHNUNG-----%
	\subsection*{1.}
	\label{ssec:8.3.1}
	\begin{equation*}
		\begin{array}{cc|cc}
			A & B & A \Rightarrow B & (A\Rightarrow B)\Rightarrow A\\ \hline
			0 & 0 & 1 & 0 \\
			0 & 1 & 1 & 0 \\
			1 & 0 & 0 & 1 \\
			0 & 1 & 1 & 1
		\end{array}
	\end{equation*}
	Die Formel ist erfüllbar und falsifizierbar, aber insbesondere kontingent.
	\subsection*{2.}
	\label{ssec:8.3.2}
	\begin{equation*}
		\begin{array}{cc|ccc}
			A & B & (A \land B) & (\lnot A \lor \lnot B) & (A \land B) \Leftrightarrow (\lnot A \lor \lnot B) \\ \hline
			0 & 0 & 0 & 1 & 0 \\
			0 & 1 & 0 & 1 & 0 \\
			1 & 0 & 0 & 1 & 0 \\
			1 & 1 & 1 & 0 & 0
		\end{array}
	\end{equation*}
	Die Formel ist falsifizierbar und insbesondere unerfüllbar (Kontradiktion).
	\subsection*{3.}
	\label{ssec:8.3.3}
	$(((A \land B ) \land C ) \land D ) \land E~~:=~~F$
	\begin{equation*}
		\begin{array}{ccccc|ccc}
			A & B & C & D & E & F & (\lnot A \lor E ) & F \Rightarrow (\lnot A \lor E ) \\ \hline
			1 & 1 & 1 & 1 & 1 & 1 & 1 & 1 \\
			0 & * & * & * & * & 0 & ? & 1 \\
			* & 0 & * & * & * & 0 & ? & 1 \\
			* & * & 0 & * & * & 0 & ? & 1 \\
			* & * & * & 0 & * & 0 & ? & 1 \\
			* & * & * & * & 0 & 0 & ? & 1
		\end{array}
	\end{equation*}
	Wie hier zu sehen ist, ist der als \textit{F} definierte Ausdruck genau dann wahr, wenn [A-E] wahr ist.
	Da in diesem Fall auch der Ausdruck $ (\lnot A \lor E ) $ wahr ist, ist die Implikation ebenfalls wahr.
	Wird nun eines der Literale [A-E] falsch, so wird auch \textit{F} falsch; und da $ 0 \Rightarrow *$ immer wahr ist, ist somit auch der untersuchte Ausdruck$ [F] \Rightarrow (\lnot A \lor E ) $ wahr für alle [A-E], die unwahr sind.
	Der Zustand von $ (\lnot A \lor E ) $ interessiert dann nicht weiter.
	Ebenso ist der Zustand der einzelnen Literale uninteressant: Sobald ein einzelnes Literal unwahr ist, ist der gesamte Ausdruck \textit{F} falsch, folglich die \textit{don't care}s. \\
	Somit ergibt sich: Die Formel ist erfüllbar und insbesondere allgemeingültig (Tautologie).
	\subsection*{4.}
	\label{ssec:8.3.4}
	$((C \Rightarrow B) \lor A)~~:=~~ F$
	\begin{equation*}
		\begin{array}{ccc|cccc}
			A & B & C & (C \Rightarrow B ) & F & (A \lor \lnot B) & F \land (A \lor \lnot B) \\ \hline
			0 & 0 & 0 & 1 & 1 & 1 & 1 \\
			0 & 0 & 1 & 0 & 0 & 1 & 0 \\
			0 & 1 & 0 & 1 & 1 & 0 & 0 \\
			0 & 1 & 1 & 1 & 1 & 0 & 0 \\
			1 & 0 & 0 & 1 & 1 & 1 & 1 \\
			1 & 0 & 1 & 0 & 1 & 1 & 1 \\
			1 & 1 & 0 & 1 & 1 & 1 & 1 \\
			1 & 1 & 1 & 1 & 1 & 1 & 1
		\end{array}
	\end{equation*}
	Die Formel ist erfüllbar und falsifizierbar und insbesondere kontingent.
	%++++++++++++++++++++++++++++++++++++++++++++++++++++++++++%
	\section*{Aufgabe 8.4}
	\label{sec:a8.4}
	\begin{flushright}
	\large{[~~~~/5]}
	\end{flushright}
	%-----AUFGABE-----%
	Seien \textit{T,K,F,G} und \textit{H} aussagenlogiche Formeln, die keine Aussagensymbole gemein haben.
	Sei ferner \textit{T} eine Tautologie, \textit{K} eine Kontradiktion und \textit{F,G} und \textit{H} kontingente Formeln.
	Zu welcher semantischen Kategorie (tautologisch, kontradiktorisch, kontinget) gehören dann die folgenden Formeln?
	Begründen Sie dabei stets Ihre Aussage!\\
	\begin{enumerate}
		\item $(F\lor \lnot G)$
		\item $(K\Rightarrow(F\lor \lnot G))$
		\item $((F\Rightarrow G)\Rightarrow(F\lor G))$
		\item $((T\Leftrightarrow \lnot K)\Rightarrow F)$
		\item $((T\Leftrightarrow K)\Rightarrow(((F\lor\lnot G)\land T)\Leftrightarrow((\lnot F \lor G)\land)G)))$
	\end{enumerate}
	\-\\\\\\
	%-----RECHNUNG-----%
	\subsection*{1.}
	\label{ssec:8.5.1}
	\textit{F} ist kontingent.\\
	\textit{G} ist kontingent $\Rightarrow \lnot G$ ist auch kontingent.\\
	$\Rightarrow F \lor \lnot G$ ist auch kontingent, weil beide Teilbedingungen kontingent sind.
	\subsection*{2.}
	\label{ssec:8.5.2}
	Die Formel ist eine Tautologie, da \textit{K} immer $0$ ist und $0 \Rightarrow *$ immer wahr ist.
	\subsection*{3.}
	\label{ssec:8.5.3}
	\textit{F} kontingent und \textit{G} kontingent $\Rightarrow (F \lor G)$ auch kontingent.\\
	\textit{F} kontingent und \textit{G} kontingent $\Rightarrow (F \Rightarrow G)$ auch kontingent. \\
	Der Ausdruck \textit{Kontingent impliziert Kontingent} ist ebenfalls kontingent.
	\subsection*{4.}
	\label{ssec:8.5.4}
	\textit{K} ist immer falsch, woraus folgt, dass $\lnot K$ immer wahr ist.\\
	\textit{T} ist immer wahr, woraus folgt, dass $ T \Leftrightarrow \lnot K$ immer wahr ist.\\
	\textit{wahr} bedeutet, dass ein Ausdruck kontingent ist, da $wahr \Rightarrow wahr$ immer wahr zurückgibt, $ wahr \Rightarrow falsch$ jedoch falsch.
	\subsection*{5.}
	\label{ssec:8.5.5}
	$(T \Leftrightarrow K)$ ist immer falsch, da \textit{K} immer falsch ist und $falsch \Rightarrow *$ immer wahr ist.\\
	Daraus folgt, dass diese Formel allgemeingültig ist (tautologisch).
	%++++++++++++++++++++++++++++++++++++++++++++++++++++++++++%
	\section*{Aufgabe 8.5}
	\label{sec:a8.5}
	\begin{flushright}
	\large{[~~~~/3]}
	\end{flushright}
	%-----AUFGABE-----%
	Gegeben sei eine kontextfreie Grammatik mit den folgenden Produktionen
	\begin{equation*}
		\begin{array}{lll}
			S~&~\rightarrow~~&~~\lnot S|(S \lor S)|(S \land S)|(S \Rightarrow S)|(S \Leftrightarrow S)|T	\\
			T~&~\rightarrow~~&~~A|B|C|D|E
		\end{array}
	\end{equation*}
	Dabei sollen \textit{A,B,C,D,D,(,),$\lnot$,$\lor$,$l\and$,$\Rightarrow$,$\Leftrightarrow$} Terminale und \textit{S} und \textit{T} Nonterminale sein.\\
	Diese Grammatik erzeugt alle aussagenlogischen Formeln, die nur die atomaren Formeln \textit{A,B,C,D} und \textit{E} enthalten.\\
	Beweisen Sie mittels struktureller Induktion eine Richtung dieser Behauptung, nämlich dass jede aussagenlogische Formel, die nur die atomaren Formeln \textit{A,B,C,D} und \textit{E} enthält, von der Grammatik generiert werden kann.
	\-\\\\\\
	%-----RECHNUNG-----%
	\underline{Induktionsanfang:} B:G gilt für jede atomare Formel:\\
	\-~~~~$S \rightarrow T$\\
	\-~~~~$T \rightarrow A|B|C|D|E$\\
	$\Rightarrow G$ gilt für jede atomare Formel aus $\mathbb{D}$ ($\mathbb{D} := $Definitionsbereich)\\
	\\
	\underline{Induktionsannahme:} $B(M) \land B(N)$ gilt für zwei Formeln $M,N$ ($B(X) := $ Behauptung für \textit{X}).\\
	\\
	\underline{Induktionsschritt:} $B(\lnot M$ ist : $S \rightarrow \lnot S$ und danach $S \rightarrow M$.
	Dies geht, da $B(M)$ gilt, also möglich ist.\\
	$B(M \circ N)$ ist: $S \rightarrow S \circ S $, wonach das erste \textit{S} zu $S \rightarrow M$ und das zweite \textit{S} zu $S\rightarrow N$ wird.\\
	$S \rightarrow M$ geht, da $B(M)$ gilt; und $S \rightarrow N$ geht, da $B(N)$ gilt. \\
	$ \circ \in \{ \lor, 1, \Rightarrow, \Leftrightarrow \}$
\end{document}