\documentclass[parskip,12pt,paper=a4,sffamily]{article}
%alternate documentclass:
%\documentclass[parskip,12pt,paper=a4,sffamily]{scrartl}
\usepackage[utf8]{inputenc}
\usepackage[ngerman]{babel}
\usepackage{lastpage}
\usepackage{color}   %May be necessary if you want to color links
\usepackage{hyperref}
% code snippets
\usepackage{listings}
% listing captions
\usepackage{caption}
% font: times new roman
\usepackage{times}
% tikz being tikz
\usepackage{tikz}
\usetikzlibrary{arrows,automata}
\usepackage{pgf}
% import math packages
\usepackage{amsmath}
\usepackage{amsfonts}
\usepackage{amssymb}
\usepackage{amsthm}
% contradiction lightning
\usepackage{stmaryrd}
% alignment options
\usepackage{ragged2e}
% page margins
\usepackage[margin=2.5cm]{geometry}

\definecolor{pblue}{rgb}{0.13,0.13,1}
\definecolor{pgreen}{rgb}{0,0.5,0}


\lstset{ %
language=Java,   							% choose the language of the code
basicstyle=\small\ttfamily,  				% the size of the fonts that are used for the code
numbers=left,                   			% where to put the line-numbers
numbersep=5pt,                  			% how far the line-numbers are from the code
backgroundcolor=\color{light-light-gray},   % choose the background color. You must add
frame=lrtb,           						% adds a frame around the code
tabsize=4,          						% sets default tabsize to 2 spaces
captionpos=b,           					% sets the caption-position to bottom
breaklines=true,        					% sets automatic line breaking
xleftmargin=1.5cm,							% space from the left paper edge
commentstyle=\color{pgreen},
keywordstyle=\color{pblue},
literate=%
    {Ö}{{\"O}}1
    {Ä}{{\"A}}1
    {Ü}{{\"U}}1
    {ß}{{\ss}}1
    {ü}{{\"u}}1
    {ä}{{\"a}}1
    {ö}{{\"o}}1
    {~}{{\textasciitilde}}1
}
\renewcommand{\lstlistingname}{Code}
\captionsetup[lstlisting]{font={footnotesize},margin=1.5cm,singlelinecheck=false } % removes "Listing 1: "
\definecolor{light-light-gray}{gray}{0.95}
\let\stdsection\section
\renewcommand\section{\stdsection}

% add line break for subtitle (size: large)
\title{Formale Grundlagen der Informatik I
    \\\large{
        Abgabe der Hausaufgaben\\
        Übungsgruppe 24 am \today
    }
}
\author{~\\
	\Large{Louis Kobras}\\
	\large{6658699}\\ %Matrikelnummer; wenn nicht für Uni, auskommentieren
	\large{4kobras@informatik.uni-hamburg.de}\\
	\\
	\Large{Utz Pöhlmann}\\
	\large{6663579}\\ %Matrikelnummer; wenn nicht für Uni, auskommentieren
	\large{4poehlma@informatik.uni-hamburg.de}\\
	\\
	\Large{Philipp Quach}\\
	\large{6706421}\\ %Matrikelnummer; wenn nicht für Uni, auskommentieren
	\large{4quach@informatik.uni-hamburg.de}
}

% leave empty for no date on title page
% comment for auto-generated date
\date{\today}


\begin{document}
	\maketitle
	\newpage
	\newgeometry{top=2.5cm, bottom =2.5cm, left=2cm, right=3.5cm}
	\section*{Aufgabe 8.3}
	\label{sec:a8.3}
	\begin{flushright}
	\large{[~~~~/4]}
	\end{flushright}
	%-----AUFGABE-----%
	Geben Sie für jede der folgenden Formeln jeweils an, ob diese erfüllbar ist, falsifizierbar, kontingent, allgemeingültig oder unerfüllbar.
	\begin{enumerate}
		\item $((A \Rightarrow B) \Rightarrow A)$
		\item $((A \land B) \Leftrightarrow (\lnot A \lor \lnot B))$
		\item $(((((A \land B) \land C) \land D) \land E) \Rightarrow (\lnot A \lor E))$
		\item $(((C \Rightarrow B) \lor A) \land (A \lor \lnot B))$
	\end{enumerate}
	\-\\\\\\
	%-----RECHNUNG-----%
	%++++++++++++++++++++++++++++++++++++++++++++++++++++++++++%
	\section*{Aufgabe 8.4}
	\label{sec:a8.4}
	\begin{flushright}
	\large{[~~~~/5]}
	\end{flushright}
	%-----AUFGABE-----%
	Seien \textit{T,K,F,G} und \textit{H} aussagenlogiche Formeln, die keine Aussagensymbole gemein haben.
	Sei ferner \textit{T} eine Tautologie, \textit{K} eine Kontradiktion und \textit{F,G} und \textit{H} kontingente Formeln.
	Zu welcher semantischen Kategorie (tautologisch, kontradiktorisch, kontinget) gehören dann die folgenden Formeln?
	Begründen Sie dabei stets Ihre Aussage!\\
	\begin{enumerate}
		\item $(F\lor \lnot G)$
		\item $(K\Rightarrow(F\lor G))$
		\item $((F\Rightarrow G)\Rightarrow(F\lor G))$
		\item $((T\Leftrightarrow \lnot K)\Rightarrow F)$
		\item $((T\Leftrightarrow K)\Rightarrow(((F\lor\lnot G)\land T)\Leftrightarrow((\lnot F \lor G)\land)G)))$
	\end{enumerate}
	\-\\\\\\
	%-----RECHNUNG-----%
	%++++++++++++++++++++++++++++++++++++++++++++++++++++++++++%
	\section*{Aufgabe 8.5}
	\label{sec:a8.5}
	\begin{flushright}
	\large{[~~~~/3]}
	\end{flushright}
	%-----AUFGABE-----%
	Gegeben sei eine kontextfreie Grammatik mit den folgenden Produktionen
	\begin{equation*}
		\begin{array}{lll}
			S~&~\rightarrow~~&~~\lnot S|(S \lor S)|(S \land S)|(S \Rightarrow S)|(S \Leftrightarrow S)|T	\\
			T~&~\rightarrow~~&~~A|B|C|D|E
		\end{array}
	\end{equation*}
	Dabei sollen \textit{A,B,C,D,D,(,),$\lnot$,$\lor$,$l\and$,$\Rightarrow$,$\Leftrightarrow$} Terminale und \textit{S} und \textit{T} Nonterminale sein.\\
	Diese Grammatik erzeugt alle aussagenlogischen Formeln, die nur die atomaren Formeln \textit{A,B,C,D} und \textit{E} enthalten.\\
	Beweisen Sie mittels struktureller Induktion eine Richtung dieser Behauptung, nämlich dass jede aussagenlogische Formel, die nur die atomaren Formeln \textit{A,B,C,D} und \textit{E} enthält, von der Grammatik generiert werden kann.
	\-\\\\\\
	%-----RECHNUNG-----%
\end{document}