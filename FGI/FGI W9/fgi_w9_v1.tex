\documentclass[parskip,12pt,paper=a4,sffamily]{article}
%alternate documentclass:
%\documentclass[parskip,12pt,paper=a4,sffamily]{scrartl}
\usepackage[utf8]{inputenc}
\usepackage[ngerman]{babel}
\usepackage{lastpage}
\usepackage{color}   %May be necessary if you want to color links
\usepackage{hyperref}
% code snippets
\usepackage{listings}
% listing captions
\usepackage{caption}
% font: times new roman
%\usepackage{times}
% tikz being tikz
\usepackage{tikz}
\usetikzlibrary{arrows,automata}
\usepackage{pgf}
% import math packages
\usepackage{amsmath}
\usepackage{amsfonts}
\usepackage{amssymb}
\usepackage{amsthm}
% contradiction lightning
\usepackage{stmaryrd}
% alignment options
\usepackage{ragged2e}
% page margins
\usepackage[margin=2.5cm]{geometry}

\definecolor{pblue}{rgb}{0.13,0.13,1}
\definecolor{pgreen}{rgb}{0,0.5,0}


\lstset{ %
language=Java,   							% choose the language of the code
basicstyle=\small\ttfamily,  				% the size of the fonts that are used for the code
numbers=left,                   			% where to put the line-numbers
numbersep=5pt,                  			% how far the line-numbers are from the code
backgroundcolor=\color{light-light-gray},   % choose the background color. You must add
frame=lrtb,           						% adds a frame around the code
tabsize=4,          						% sets default tabsize to 2 spaces
captionpos=b,           					% sets the caption-position to bottom
breaklines=true,        					% sets automatic line breaking
xleftmargin=1.5cm,							% space from the left paper edge
commentstyle=\color{pgreen},
keywordstyle=\color{pblue},
literate=%
    {Ö}{{\"O}}1
    {Ä}{{\"A}}1
    {Ü}{{\"U}}1
    {ß}{{\ss}}1
    {ü}{{\"u}}1
    {ä}{{\"a}}1
    {ö}{{\"o}}1
    {~}{{\textasciitilde}}1
}
\renewcommand{\lstlistingname}{Code}
\captionsetup[lstlisting]{font={footnotesize},margin=1.5cm,singlelinecheck=false } % removes "Listing 1: "
\definecolor{light-light-gray}{gray}{0.95}
\let\stdsection\section
\renewcommand\section{\stdsection}

% add line break for subtitle (size: large)
\title{Formale Grundlagen der Informatik I
    \\\large{
        Abgabe der Hausaufgaben\\
        Übungsgruppe 24 am Freitag, d. \today
    }
}
\author{~\\
	\Large{Louis Kobras}\\
	\large{6658699}\\ %Matrikelnummer; wenn nicht für Uni, auskommentieren
	\large{4kobras@informatik.uni-hamburg.de}\\
	\\
	\Large{Utz Pöhlmann}\\
	\large{6663579}\\ %Matrikelnummer; wenn nicht für Uni, auskommentieren
	\large{4poehlma@informatik.uni-hamburg.de}\\
	\\
	\Large{Philipp Quach}\\
	\large{6706421}\\ %Matrikelnummer; wenn nicht für Uni, auskommentieren
	\large{4quach@informatik.uni-hamburg.de}
}

% leave empty for no date on title page
% comment for auto-generated date
\date{\today}


\begin{document}
	\maketitle
	\newpage
	\newgeometry{top=2.5cm, bottom =2.5cm, left=2cm, right=3.5cm}
	\section*{Aufgabe 9.4}
	\label{sec:a9.4}
	\begin{flushright}
	\large{[~~~~/5]}
	\end{flushright}
	%-----AUFGABE-----%
	\subsection*{9.4.1}
	\label{ssec:9.4.1}
	Beweisen oder widerlegen Sie die folgende Folgerbarkeitsbeziehung mittels einer Wahrheitstafel:
	\[ \{ \lnot B, A \Rightarrow B \} \vDash \lnot A \]
	\-\\
	%-----RECHNUNG-----%
	$A \Rightarrow B~~:=~~F$
	\begin{equation*}
		\begin{array}{cc|ccc}
			A & B & \lnot B & F & \lnot A \\ \hline
			0 & 0 & 1 & 1 & 1 \\
			0 & 1 & 0 & 1 & 1 \\
			1 & 0 & 1 & 0 & 0 \\
			1 & 1 & 0 & 1 & 0
		\end{array}
	\end{equation*}
	Nur in der ersten Zeile wurden $\lnot B$ und $F$ wahr;
	dort ist auch $\lnot A$ wahr.
	Damit ist die Aussage bewiesen.
	\subsection*{9.4.2}
	\label{ssec:9.4.2}
	Beweisen oder widerlegen Sie die folgende Äquivalenz mittels einer Wahrheitstafel:
	\[ ( A \Rightarrow B ) \Rightarrow A \equiv A \Rightarrow ( B \Rightarrow A ) \]
	\-\\
	%-----RECHNUNG-----%
	$A \Rightarrow B~~:=~~F$\\
	$F \Rightarrow B~~:=~~G$\\
	$B \Rightarrow A~~:=~~H$\\
	$A \Rightarrow H~~:=~~I$
	\begin{equation*}
		\begin{array}{cc|cccc}
			A & B & F & G & H & I \\ \hline
			0 & 0 & 1 & 0 & 1 & 1 \\
			0 & 1 & 1 & 0 & 0 & 1 \\
			1 & 0 & 0 & 1 & 1 & 1 \\
			1 & 1 & 1 & 1 & 1 & 1
		\end{array}
	\end{equation*}
	$\Rightarrow~~~ (A \Rightarrow B)\Rightarrow A \equiv A$ \\
	$\Rightarrow~~~ A \Rightarrow (B\Rightarrow A) \equiv T$ \\
	$ A \not\equiv T~~\Rightarrow$ widerlegt (siehe Spalten 4 und 6).
	\subsection*{9.4.3}
	\label{ssec:9.4.3}
	Beweisen oder widerlegen Sie die folgenden Aussagen: Wenn $ F \equiv G $ und $ G \vDash H $ gilt, dann gilt auch $ F \vDash H $.
	\-\\\\
	%-----RECHNUNG-----%
	% C
	$F \equiv G$ bedeutet: Wenn \textit{F} wahr ist, ist \textit{G} auch wahr; wenn \textit{F} falsch ist, ist \textit{G} auch falsch $\Rightarrow H$ bleibt \textit{H}.\\
	\\
	% A
	$G \vDash H$ bedeutet: Wenn \textit{G} wahr ist und \textit{H} wahr ist, gilt dies; wenn \textit{G} wahr ist, \textit{H} jedoch falsch ist, gilt dies nicht; wenn \textit{H} wahr ist, gilt dies.\\
	\\
	% B
	$F \vDash H$ bedeutet: Wenn \textit{F} wahr ist und \textit{H} wahr ist, gilt dies; wenn \textit{F} wahr ist und \textit{H} falsch ist, gilt dies nicht; wenn \textit{F} falsch ist, gilt dies.
	\subsection*{9.4.4}
	\label{ssec:9.4.4}
	Beweisen oder widerlegen Sie die folgenden Aussagen: Wenn $ F \equiv G $ und $ G \vDash H $ gilt, dann gilt auch $ H \vDash F $.
	\-\\\\
	%-----RECHNUNG-----%
	$H \vDash F$ bedeutet: Wenn \textit{H} wahr ist und \textit{F} wahr ist, gilt dies; wenn \textit{H} wahr ist, \textit{F} jedoch falsch ist, gilt dies nicht; wenn \textit{H} falsch ist, gilt dies.\\
	\\
	Dies stimmt nicht überein mit \texttt{9.4.3} und zeigt somit auf, dass die Aussage nicht korrekt ist $\Rightarrow$ widerlegt.
	\subsection*{9.4.5}
	\label{ssec:9.4.5}
	Beweisen oder widerlegen Sie die folgenden Aussagen: Wenn $ F_1 \equiv F_2 $ und $ G_1 \equiv G_2 $ gilt, dann gilt $ F_1 \vDash G_1 $ genau dann, wenn $ F_2 \vDash G_2 $.
	\-\\\\
	%-----RECHNUNG-----%
	$F_1 \equiv F_2$ bedeutet: Wenn $F_1$ wahr ist, ist $F_2$ auch wahr; wenn $F_1$ falsch ist, ist $F_1$ auch falsch $\Rightarrow F_1$ bleibt $F_2$.\\
	\\
	$G_1 \equiv G_2$ bedeutet: Wenn $G_1$ wahr ist, ist $G_2$ auch wahr; wenn $G_1$ falsch ist, ist $G_2$ auch falsch $\Rightarrow G_1$ bleibt $G_2$.\\
	\\
	$F_1 \vDash G_1$ bedeutet: Wenn $F_1$ wahr ist und $G_1$ wahr ist, gilt dies; wenn $F_1$ wahr ist, $G_1$ jedoch falsch ist, gilt dies nicht; wenn $G_1$ wahr ist, gilt dies.\\
	\\
	$F_2 \vDash G_2$ bedeutet: Wenn $F_2$ wahr ist und $G_2$ wahr ist, gilt dies; wenn $F_2$ wahr ist, $G_2$ jedoch falsch ist, gilt dies nicht; wenn $G_2$ wahr ist, gilt dies.\\
	\\
	Da der dritte und vierte Absatz zueinander kongruent sind, ist die Aussage bewiesen.
	%++++++++++++++++++++++++++++++++++++++++++++++++++++++++++%
	\section*{Aufgabe 9.5}
	\label{sec:a9.5}
	\begin{flushright}
	\large{[~~~~/4]}
	\end{flushright}
	%-----AUFGABE-----%
	\subsection*{9.5.1}
	\label{ssec:9.5.1}
	Bilden Sie zu
	\[ F := ( \lnot C \Rightarrow \lnot ( \lnot A \lor B ) ) \land ( A \lor C ) \]
	durch Äquivalenzumformungen nach dem Verfahren aus der Vorlesung eine äquivalente Formel in konjunktiver Normalform.
	Geben Sie dabei bei jeder Umformung an, welche Umformungsregel Sie anwenden.
	\-\\\\
	%-----RECHNUNG-----%
	\begin{equation*}
		\begin{array}{ll}
			(\lnot C \Rightarrow \lnot ( \lnot A \lor B ) ) \land ( A \lor C ) 	& \text{Eliminieren von }\Rightarrow \\
			(\lnot \lnot C \lor \lnot ( \lnot A \lor B ) ) \land (A \lor C)		& \text{Aufheben von Doppelten Negationen} \\
			( C \lor \lnot ( \lnot A \lor B ) ) \land ( A \lor C )				& \text{Anwenden des de Morgan-Gesetzes} \\
			( C \lor ( \lnot \lnot A \land \lnot B ) ) \land ( A \lor C )		& \text{Aufheben von Doppelten Negationen} \\
			( (  C \lor ( A \land \lnot B ) ) \land ( A \lor C )				& \text{Anwenden des Distributiv-Gesetzes} \\
			( ( C \lor A ) \land ( C \lor \lnot B ) ) \land ( A \lor C )		& \text{Anwenden des Assoziativgesetzes} \\
			( C \lor A ) \land ( C \lor \lnot B ) \land ( C \lor A )			& \text{Streichen doppelt vorkommender Terme} \\
			( C \lor A ) \land 8 C \lor \lnot B )								& \text{\textbf{KNF}}
		\end{array}
	\end{equation*}
	\subsection*{9.5.2}
	\label{ssec:9.5.2}
	Bilden Sie zu
	\[ G := ( ( A \Leftrightarrow B ) \lor \lnot B ) \land \lnot C \]
	eine äquivalente Formel in disjunktiver Normalform mit der in der Vorlesung behandelten Wahrheitstafelmethode.
	\-\\\\
	%-----RECHNUNG-----%
	$((A \Leftrightarrow B) \lor \lnot B)~~:=~~F$
	\begin{equation*}
		\begin{array}{ccc|ccccc|ll}
			A & B & C & \lnot B & \lnot C & A \Leftrightarrow C & F & F \land \lnot C & {} \\ \hline
			0 & 0 & 0 & 1 & 1 & 1 & 1 & 1 & \leftarrow & (\lnot A \land \lnot B \land \lnot C) \\
			0 & 0 & 1 & 1 & 0 & 1 & 1 & 0 & {} & {} \\
			0 & 1 & 0 & 0 & 1 & 0 & 0 & 0 & {} & ~~~~~~~~\lor \\
			0 & 1 & 1 & 0 & 0 & 0 & 0 & 0 & {} & {} \\
			1 & 0 & 0 & 1 & 1 & 0 & 1 & 1 & \leftarrow & ( A \land \lnot B \land \lnot C ) \\
			1 & 0 & 1 & 1 & 0 & 0 & 1 & 0 & {} & ~~~~~~~~\lor \\
			1 & 1 & 0 & 0 & 1 & 1 & 1 & 1 & \leftarrow & ( A \land B \land \lnot C )\\
			1 & 1 & 1 & 0 & 0 & 1 & 1 & 0 & {} & {} \\
		\end{array}
	\end{equation*}
	\\
	\begin{equation*}
		\begin{array}{ll}
			( \lnot A \land \lnot B \land \lnot C) \lor ( A \land \lnot B \land \lnot C ) \lor ( A \land B \land \lnot C ) 	& \text{Anwenden des Distributivgesetzes} \\
			( ( \lnot B \land \lnot C ) \land ( \lnot A \lor A ) ) \lor ( A \land B \land \lnot C )							& (\lnot A \lor A)=1\Rightarrow\text{ kann gekürzt werden} \\
			( ( \lnot B \land \lnot C ) ) \lor ( A \land B \land \lnot C )													& \text{Omittieren von Klammern} \\
			( \lnot B \land \lnot C ) \lor ( A \land B \land \lnot C )														& \text{\textbf{DNF}}
		\end{array}
	\end{equation*}
	%++++++++++++++++++++++++++++++++++++++++++++++++++++++++++%
	\section*{Aufgabe 9.6}
	\label{sec:a9.6}
	\begin{flushright}
	\large{[~~~~/3]}
	\end{flushright}
	%-----AUFGABE-----%
	Wenden Sie auf die folgenden Hornformeln den Markierungsalgorithmus an, um die Erfüllbarkeit zu prüfen.
	Geben Sie dabei an in welchem Schritt Aussagesymbole markiert werden und geben sie ferner sofern möglich eine erfüllende Belegung an.
	\begin{enumerate}
		\item $ ( \lnot A \lor G \lor \lnot D ) \land ( \lnot B \lor A ) \land ( \lnot E \lor \lnot F ) \land ( \lnot D \lor C \lor \lnot B ) \land ( \lnot D \lor B ) \land D \land ( A \lor \lnot F ) \land ( \lnot C \lor \lnot D \lor E \lor \lnot B ) $
		\item $ ( \lnot G \lor E \lor \lnot D ) \land ( \lnot F \lor C ) \land ( \lnot B \lor \lnot D \lor \lnot E \lor A ) \land B \land ( F \lor \lnot  G \lor \lnot D ) \land G \land ( D \lor \lnot B ) \land (\lnot E \lor \lnot F ) $
	\end{enumerate}
	\-\\\\\\
	%-----RECHNUNG-----%
	\subsection*{9.6.1}
	\label{ssec:9.6.1}
	\begin{enumerate}
		\item D
		\item B (D$\Rightarrow$B)
		\item C ((D$\land$B)$\Rightarrow$C),A(B$\Rightarrow$A)
		\item G ((D$\land$A)$\Rightarrow$G),E((C$\land$B$\land$D)$\Rightarrow$E)
	\end{enumerate}
	Erfüllende Belegung \textit{A} mit $(A)=A(B)=A(C)=A(D)=A(E)=A(G)=1~\land~A(F)=0$
	\subsection*{9.6.2}
	\label{ssec:9.6.2}
	\begin{enumerate}
		 \item G,B
		 \item D(B$\Rightarrow$D)
		 \item E((G$\land$D)$\Rightarrow$E),F((G$\land$D)$\Rightarrow$F)
		 \item C(F$\Rightarrow$C),A((B$\land$D$\land$E)$\Rightarrow$A) \dots OH!
	\end{enumerate}
	unerfüllbar: $(\lnot E \lor \lnot F)$
	%++++++++++++++++++++++++++++++++++++++++++++++++++++++++++%
\end{document}