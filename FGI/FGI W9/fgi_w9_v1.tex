\documentclass[parskip,12pt,paper=a4,sffamily]{article}
%alternate documentclass:
%\documentclass[parskip,12pt,paper=a4,sffamily]{scrartl}
\usepackage[utf8]{inputenc}
\usepackage[ngerman]{babel}
\usepackage{lastpage}
\usepackage{color}   %May be necessary if you want to color links
\usepackage{hyperref}
% code snippets
\usepackage{listings}
% listing captions
\usepackage{caption}
% font: times new roman
%\usepackage{times}
% tikz being tikz
\usepackage{tikz}
\usetikzlibrary{arrows,automata}
\usepackage{pgf}
% import math packages
\usepackage{amsmath}
\usepackage{amsfonts}
\usepackage{amssymb}
\usepackage{amsthm}
% contradiction lightning
\usepackage{stmaryrd}
% alignment options
\usepackage{ragged2e}
% page margins
\usepackage[margin=2.5cm]{geometry}

\definecolor{pblue}{rgb}{0.13,0.13,1}
\definecolor{pgreen}{rgb}{0,0.5,0}


\lstset{ %
language=Java,   							% choose the language of the code
basicstyle=\small\ttfamily,  				% the size of the fonts that are used for the code
numbers=left,                   			% where to put the line-numbers
numbersep=5pt,                  			% how far the line-numbers are from the code
backgroundcolor=\color{light-light-gray},   % choose the background color. You must add
frame=lrtb,           						% adds a frame around the code
tabsize=4,          						% sets default tabsize to 2 spaces
captionpos=b,           					% sets the caption-position to bottom
breaklines=true,        					% sets automatic line breaking
xleftmargin=1.5cm,							% space from the left paper edge
commentstyle=\color{pgreen},
keywordstyle=\color{pblue},
literate=%
    {Ö}{{\"O}}1
    {Ä}{{\"A}}1
    {Ü}{{\"U}}1
    {ß}{{\ss}}1
    {ü}{{\"u}}1
    {ä}{{\"a}}1
    {ö}{{\"o}}1
    {~}{{\textasciitilde}}1
}
\renewcommand{\lstlistingname}{Code}
\captionsetup[lstlisting]{font={footnotesize},margin=1.5cm,singlelinecheck=false } % removes "Listing 1: "
\definecolor{light-light-gray}{gray}{0.95}
\let\stdsection\section
\renewcommand\section{\stdsection}

% add line break for subtitle (size: large)
\title{Formale Grundlagen der Informatik I
    \\\large{
        Abgabe der Hausaufgaben\\
        Übungsgruppe 24 am Freitag, d. \today
    }
}
\author{~\\
	\Large{Louis Kobras}\\
	\large{6658699}\\ %Matrikelnummer; wenn nicht für Uni, auskommentieren
	\large{4kobras@informatik.uni-hamburg.de}\\
	\\
	\Large{Utz Pöhlmann}\\
	\large{6663579}\\ %Matrikelnummer; wenn nicht für Uni, auskommentieren
	\large{4poehlma@informatik.uni-hamburg.de}\\
	\\
	\Large{Philipp Quach}\\
	\large{6706421}\\ %Matrikelnummer; wenn nicht für Uni, auskommentieren
	\large{4quach@informatik.uni-hamburg.de}
}

% leave empty for no date on title page
% comment for auto-generated date
\date{\today}


\begin{document}
	\maketitle
	\newpage
	\newgeometry{top=2.5cm, bottom =2.5cm, left=2cm, right=3.5cm}
	\section*{Aufgabe 9.4}
	\label{sec:a9.4}
	\begin{flushright}
	\large{[~~~~/5]}
	\end{flushright}
	%-----AUFGABE-----%
	\subsection*{9.4.1}
	\label{ssec:9.4.1}
	Beweisen oder widerlegen Sie die folgende Folgerbarkeitsbeziehung mittels einer Wahrheitstafel:
	\[ \{ \lnot B, A \Rightarrow B \} \vDash \lnot A \]
	\-\\
	%-----RECHNUNG-----%
	\subsection*{9.4.2}
	\label{ssec:9.4.2}
	Beweisen oder widerlegen Sie die folgende Äquivalenz mittels einer Wahrheitstafel:
	\[ ( A \Rightarrow B ) \Rightarrow A \equiv A \Rightarrow ( B \Rightarrow A ) \]
	\-\\
	%-----RECHNUNG-----%
	\subsection*{9.4.3}
	\label{ssec:9.4.3}
	Beweisen oder widerlegen Sie die folgenden Aussagen: Wenn $ F \equiv G $ und $ G \vDash H $ gilt, dann gilt auch $ F \vDash H $.
	\-\\\\
	%-----RECHNUNG-----%
	\subsection*{9.4.4}
	\label{ssec:9.4.4}
	Beweisen oder widerlegen Sie die folgenden Aussagen: Wenn $ F \equiv G $ und $ G \vDash H $ gilt, dann gilt auch $ H \vDash F $.
	\-\\\\
	%-----RECHNUNG-----%
	\subsection*{9.4.5}
	\label{ssec:9.4.5}
	Beweisen oder widerlegen Sie die folgenden Aussagen: Wenn $ F_1 \equiv F_2 $ und $ G_1 \equiv G_2 $ gilt, dann gilt $ F_1 \vDash G_1 $ genau dann, wenn $ F_2 \vDash G_2 $.
	\-\\\\
	%-----RECHNUNG-----%
	%++++++++++++++++++++++++++++++++++++++++++++++++++++++++++%
	\section*{Aufgabe 9.5}
	\label{sec:a9.5}
	\begin{flushright}
	\large{[~~~~/4]}
	\end{flushright}
	%-----AUFGABE-----%
	\subsection*{9.5.1}
	\label{ssec:9.5.1}
	Bilden Sie zu
	\[ F := ( \lnot C \Rightarrow \lnot ( \lnot A \lor B ) ) \land ( A \lor C ) \]
	durch Äquivalenzumformungen nach dem Verfahren aus der Vorlesung eine äquivalente Formel in konjunktiver Normalform.
	Geben Sie dabei bei jeder Umformung an, welche Umformungsregel Sie anwenden.
	\-\\\\
	%-----RECHNUNG-----%
	\subsection*{9.5.1}
	\label{ssec:9.5.1}
	Bilden Sie zu
	\[ G := ( ( A \Leftrightarrow B ) \lor \lnot B ) \land \lnot C \]
	eine äquivalente Formel in disjunktiver Normalform mit der in der Vorlesung behandelten Wahrheitstafelmethode.
	%++++++++++++++++++++++++++++++++++++++++++++++++++++++++++%
	\section*{Aufgabe 9.6}
	\label{sec:a9.6}
	\begin{flushright}
	\large{[~~~~/3]}
	\end{flushright}
	%-----AUFGABE-----%
	Wenden Sie auf die folgenden Hornformeln den Markierungsalgorithmus an, um die Erfüllbarkeit zu prüfen.
	Geben Sie dabei an in welchem Schritt Aussagesymbole markiert werden und geben sie ferner sofern möglich eine erfüllende Belegung an.
	\begin{enumerate}
		\item $ ( \lnot A \lor G \lor \lnot D ) \land ( \lnot B \lor A ) \land ( \lnot E \lor \lnot F ) \land ( \lnot D \lor C \lor \lnot B ) \land ( \lnot D \lor B ) \land D \land ( A \lor \lnot F ) \land ( \lnot C \lor \lnot D \lor E \lor \lnot B ) $
		\item $ ( \lnot G \lor E \lor \lnot D ) \land ( \lnot F \lor C ) \land ( \lnot B \lor \lnot D \lor \lnot E \lor A ) \land B \land ( F \lor \lnot  G \lor \lnot D ) \land G \land ( D \lor \lnot B ) \land (\lnot E \lor \lnot F ) $
	\end{enumerate}
	\-\\\\\\
	%-----RECHNUNG-----%
	%++++++++++++++++++++++++++++++++++++++++++++++++++++++++++%
\end{document}