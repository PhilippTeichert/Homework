\documentclass[parskip,12pt,paper=a4,sffamily]{article}
%alternate documentclass:
%\documentclass[parskip,12pt,paper=a4,sffamily]{scrartl}
\usepackage[utf8]{inputenc}
\usepackage[ngerman]{babel}
\usepackage{lastpage}
\usepackage{color}   %May be necessary if you want to color links
\usepackage{hyperref}
% code snippets
\usepackage{listings}
% listing captions
\usepackage{caption}
% tikz being tikz
\usepackage{tikz}
\usetikzlibrary{arrows,automata}
\usepackage{pgf}
% import math packages
\usepackage{amsmath}
\usepackage{amsfonts}
\usepackage{amssymb}
\usepackage{amsthm}
% contradiction lightning
\usepackage{stmaryrd}
% alignment options
\usepackage{ragged2e}
% page margins
\usepackage[margin=2.5cm]{geometry}

\definecolor{pblue}{rgb}{0.13,0.13,1}
\definecolor{pgreen}{rgb}{0,0.5,0}


\lstset{ %
language=Java,   							% choose the language of the code
basicstyle=\small\ttfamily,  				% the size of the fonts that are used for the code
numbers=left,                   			% where to put the line-numbers
numbersep=5pt,                  			% how far the line-numbers are from the code
backgroundcolor=\color{light-light-gray},   % choose the background color. You must add
frame=lrtb,           						% adds a frame around the code
tabsize=4,          						% sets default tabsize to 2 spaces
captionpos=b,           					% sets the caption-position to bottom
breaklines=true,        					% sets automatic line breaking
xleftmargin=1.5cm,							% space from the left paper edge
commentstyle=\color{pgreen},
keywordstyle=\color{pblue},
literate=%
    {Ö}{{\"O}}1
    {Ä}{{\"A}}1
    {Ü}{{\"U}}1
    {ß}{{\ss}}1
    {ü}{{\"u}}1
    {ä}{{\"a}}1
    {ö}{{\"o}}1
    {~}{{\textasciitilde}}1
}
\renewcommand{\lstlistingname}{Code}
\captionsetup[lstlisting]{font={footnotesize},margin=1.5cm,singlelinecheck=false } % removes "Listing 1: "
\definecolor{light-light-gray}{gray}{0.95}
\let\stdsection\section
\renewcommand\section{\stdsection}

% add line break for subtitle (size: large)
\title{Formale Grundlagen der Informatik I
    \\\large{
        Abgabe der Hausaufgaben\\
        Übungsgruppe 24 am Freitag, d. \today
    }
}
\author{~\\
	\Large{Louis Kobras}\\
	\large{6658699}\\ %Matrikelnummer; wenn nicht für Uni, auskommentieren
	\large{4kobras@informatik.uni-hamburg.de}\\
	\\
	\Large{Utz Pöhlmann}\\
	\large{6663579}\\ %Matrikelnummer; wenn nicht für Uni, auskommentieren
	\large{4poehlma@informatik.uni-hamburg.de}\\
	\\
	\Large{Philipp Quach}\\
	\large{6706421}\\ %Matrikelnummer; wenn nicht für Uni, auskommentieren
	\large{4quach@informatik.uni-hamburg.de}
}

% leave empty for no date on title page
% comment for auto-generated date
\date{\today}


\begin{document}
	\maketitle\thispagestyle{empty}
	\newpage
	\pagenumbering{arabic}
	\newgeometry{top=2.5cm, bottom =2.5cm, left=2cm, right=3.5cm}
	\section*{Aufgabe 11.4}
	\label{sec:a11.4}
	\begin{flushright}
	\large{[~~~~/*]}
	\end{flushright}
	%-----AUFGABE-----%
	\subsection{11.4.1}
	\label{ssec:a11.4.1}
	\-\\\\\\
	%-----RECHNUNG-----%
	%-----AUFGABE-----%
	\subsection{11.4.2}
	\label{ssec:a11.4.2}
	\-\\\\\\
	%-----RECHNUNG-----%
	%++++++++++++++++++++++++++++++++++++++++++++++++++++++++++%
	\section*{Aufgabe 11.5}
	\label{sec:a5}
	\begin{flushright}
	\large{[~~~~/*]}
	\end{flushright}
	%-----AUFGABE-----%
	Geben Sie kurz und kompakt die wichtigsten Argumente aus dem Beweis des Resolutionssatzes wieder.
	(Sie sollen hier nicht den Beweis wiedergeben, sondern aus dem Beweis die Vorgehensweise und die wichtigsten Argumente herausarbeiten und diese mit Ihren Worten wiedergeben.)
	\-\\\\\\
	%-----RECHNUNG-----%
	\textbf{Voraussetzungen:} $\square\in Res^*(F)$, die Definition von $Res^*(F)$ und $F \equiv Res^n(F)\text{ mit }n:\square\in Res^n(F)\Rightarrow K_1, K_2 \in Res^n(F); K_1=\{L\} \land K_2=\{\overline{L}\}$, was aus der Definition folgt, sowie das Vorliegen der Struktur als KNF.\\
	Diese Voraussetzungliste impliziert, dass die Korrektheit der Resolution erforderlich ist.\\
	\\
	Mit Induktion kann gezeigt werden, dass $\square\in Res^*(F)$ für jede Formelmenge mit $n$ atomaren Formeln gilt (da für $n=0$ die Formelmenge leer ist, ist sie grundsätzlich nicht erfüllbar).\\
	Es müssen zwei Umformungen gebildet werden, die zunächst $A_{n+1}$ durch $\epsilon$ ersetzen in dem Sinne, dass bei $F_0$ $A_{n+1}$ gleich 0 gesetzt wird und folglich, da eine KNF vorliegt, jedes Vorkommen von 0 in einer Klausel ignoriert werden kann und jedes Vorkommen von $\lnot0=1$ eine Klausel automatisch wahr macht, Vorkommen von $A_{n+1}$ wegfallen und Vorkommen von $\lnot A_{n+1}$ ihre ganze Klausel wegfallen lassen, da die immer wahr ist.
	Analog wird bei $F_1$ $A_{n+1}$ gleich 1 gesetzt.
	Folglich werden Klauseln, die $A_{n+1}$ enthalten, komplett omittiert, während Vorkommen von $\lnot A_{n+1}$ gestrichen werden.\\
	Somit erhält man zwei Formelmengen $F_0$ und $F_1$, die nur die Fomeln $A_1,\dots,A_n$ enthalten.
	Für diese Formelmenge wurde der Resolutionssatz bereits gezeigt.
	Somit ist ersichtlich, dass der Resolutionssatz auch für eine Formelmenge der Größe $n+1$ gilt, unabhängig der Belegung der Formel $A_{n+1}$.
	%++++++++++++++++++++++++++++++++++++++++++++++++++++++++++%
	\section*{Aufgabe 11.6}
	\label{sec:a11.6}
	\begin{flushright}
	\large{[~~~~/*]}
	\end{flushright}
	%-----AUFGABE-----%
	\subsection{11.6.1}
	\label{ssec:a11.6.1}
	\-\\\\\\
	%-----RECHNUNG-----%
	%-----AUFGABE-----%
	\subsection{11.6.2}
	\label{ssec:a11.6.2}
	\-\\\\\\
	%-----RECHNUNG-----%
	%-----AUFGABE-----%
	\subsection{11.6.3}
	\label{ssec:a11.6.3}
	\-\\\\\\
	%-----RECHNUNG-----%
\end{document}