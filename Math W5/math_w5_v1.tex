\documentclass[parskip,12pt,paper=a4,sffamily]{scrartcl}
%alternate documentclass:
%\documentclass[parskip,12pt,paper=a4,sffamily]{article}
\usepackage[utf8]{inputenc}
\usepackage[ngerman]{babel}
\usepackage{lastpage}
\usepackage{color}   %May be necessary if you want to color links
\usepackage{hyperref}
% code snippets
\usepackage{listings}
% listing captions
\usepackage{caption}
% font: times new roman
\usepackage{times}
% tikz being tikz
\usepackage{tikz}
% import math packages
\usepackage{amsmath}
\usepackage{amsfonts}
\usepackage{amssymb}
\usepackage{amsthm}
% contradiction lightning
\usepackage{stmaryrd}
% alignment options
\usepackage{ragged2e}
% page margins
\usepackage[margin=2.5cm]{geometry}

\definecolor{pblue}{rgb}{0.13,0.13,1}
\definecolor{pgreen}{rgb}{0,0.5,0}


\lstset{ %
language=Java,   							% choose the language of the code
basicstyle=\small\ttfamily,  				% the size of the fonts that are used for the code
numbers=left,                   			% where to put the line-numbers
numbersep=5pt,                  			% how far the line-numbers are from the code
backgroundcolor=\color{light-light-gray},   % choose the background color. You must add
frame=lrtb,           						% adds a frame around the code
tabsize=4,          						% sets default tabsize to 2 spaces
captionpos=b,           					% sets the caption-position to bottom
breaklines=true,        					% sets automatic line breaking
xleftmargin=1.5cm,							% space from the left paper edge
commentstyle=\color{pgreen},
keywordstyle=\color{pblue},
literate=%
    {Ö}{{\"O}}1
    {Ä}{{\"A}}1
    {Ü}{{\"U}}1
    {ß}{{\ss}}1
    {ü}{{\"u}}1
    {ä}{{\"a}}1
    {ö}{{\"o}}1
    {~}{{\textasciitilde}}1
}
\renewcommand{\lstlistingname}{Code}
\captionsetup[lstlisting]{font={footnotesize},margin=1.5cm,singlelinecheck=false } % removes "Listing 1: "
\definecolor{light-light-gray}{gray}{0.95}
\let\stdsection\section
\renewcommand\section{\stdsection}

% add line break for subtitle (size: large)
\title{Hausaufgaben zum 07. Mai 2015
    \\\large{
        Mathematik II für Studierende der Informatik\\
        (Analysis und Lineare Algebra)
    }
}
\author{~\\
	\Large{Louis Kobras}\\
	\large{6658699}\\ %Matrikelnummer; wenn nicht für Uni, auskommentieren
	\large{4kobras@informatik.uni-hamburg.de}\\
	\\
	\Large{Utz Pöhlmann}\\
	\large{4poehlma@informatik.uni-hamburg.de}\\
	\large{6663579} \\{}
}

% leave empty for no date on title page
% comment for auto-generated date
\date{06.05.2015}


\begin{document}
	\maketitle
	\newpage
	\section*{Aufgabe 1}
	\label{sec:a1}
	    %%%%AUFGABE%%%%
	    Bestätigen Sie für die Matrizen
	    \[
	        A=\begin{pmatrix}
	            5&7\\9&-1\\8&2
	        \end{pmatrix},~~~
	        B=\begin{pmatrix}
	            1&-1\\3&2
	        \end{pmatrix},~~~
	        C=\begin{pmatrix}
	            1&2\\3&6
	        \end{pmatrix}
	    \]
	    die Gültigkeit des Assoziativgesetzes $A(BC)=(AB)C$.
	    %%%%RECHNUNG%%%%
	    \begin{equation*}
	    	\begin{array}{c|c}
	    		\begin{split}
	    			\begin{array}{l}
	    				Berechnung~von~BC:\\
			    		\begin{pmatrix}
	        			    1&-1\\3&2
				        \end{pmatrix}
				        ~~\cdot~~
						\begin{pmatrix}
				            1&2\\3&6
				        \end{pmatrix}\\
				        =~~
				        \begin{pmatrix}
			        		-5&-10\\9&18
						\end{pmatrix}\\ \\
						Multiplikation~mit~A:\\
						\begin{pmatrix}
			    	        5&7\\9&-1\\8&2
				        \end{pmatrix}
						~~\cdot~~
						\begin{pmatrix}
				        	-5&-10\\9&18
						\end{pmatrix}\\
						=~~
						\begin{pmatrix}
							38&76\\-54&-108\\-22&-44
						\end{pmatrix}
	    			\end{array}
	    		\end{split}
	    		{}~~~~~&~~~~~{}
	    		\begin{split}
	    			\begin{array}{l}
	    				Berechnung~von~AB:\\
	    				\begin{pmatrix}
			    	        5&7\\9&-1\\8&2
				        \end{pmatrix}
				        ~~\cdot~~
				        \begin{pmatrix}
				            1&-1\\3&2
				        \end{pmatrix}\\
				        =~~
				        \begin{pmatrix}
				        	26&4\\6&-20\\14&-12
				        \end{pmatrix}\\ \\
	    				Multiplikation~mit~C:\\
	    				\begin{pmatrix}
				        	26&4\\6&-20\\14&-12
				        \end{pmatrix}
				        ~~\cdot~~
				        \begin{pmatrix}
				            1&2\\3&6
				        \end{pmatrix}\\
	        			=~~
	        			\begin{pmatrix}
	        				38&76\\-54&-108\\-22&-44
	        			\end{pmatrix}
	    			\end{array}
	    		\end{split}
	    	\end{array}
	    \end{equation*}
	    Wie man hier sehen kann, kommt bei beiden Rechnungen die gleiche Matrix heraus.
	    Ergo ist $A(BC)=(AB)C$, und somit gilt hier das Assoziativgesetz.
	\section*{Aufgabe 2}
	\label{sec:a2}
	    %%%%AUFGABE%%%%
	    Gegeben seien die Matrizen
	    \[
	        A=\begin{pmatrix}
	            1&2\\3&5
	        \end{pmatrix},~~~
	        B=\begin{pmatrix}
	            1&-1\\3&2
	        \end{pmatrix},~~~
	        C=\begin{pmatrix}
	            5&7\\7&-1\\8&2
	        \end{pmatrix}.
	    \]
	    Bestätigen Sie für diese Matrizen die Gültigkeit des Distributivgesetzes
	    \[
	        C(A+B)=CA+CB
	    \]
	    %%%%RECHNUNG%%%%
	    \begin{equation*}
	    	\begin{array}{c|c}
	    		\begin{split}
	    			\begin{array}{l}
	    				Berechnen~von~A+B:\\
	    					\begin{pmatrix}
					            1&2\\3&5
					        \end{pmatrix}
					        ~~+~~
					        \begin{pmatrix}
					        	1&-1\\3&2
					        \end{pmatrix}\\
					        =~~
					        \begin{pmatrix}
					        	2&1\\6&7
					        \end{pmatrix}\\ \\
	    				Multiplizieren~mit~C:\\
	    					\begin{pmatrix}
					        	2&1\\6&7
					        \end{pmatrix}
					        ~~\cdot~~
					        \begin{pmatrix}
					            5&7\\7&-1\\8&2
					        \end{pmatrix}\\
					        =~~
					        \begin{pmatrix}
					        	52&54\\8&0\\28&22
					        \end{pmatrix}
	    			\end{array}
	    		\end{split}
	    		{}~~~~~&~~~~~{}
	    		\begin{split}
	    			\begin{array}{l}
	    				Berechnen~von~CA:\\
	    					\begin{pmatrix}
					            5&7\\7&-1\\8&2
					        \end{pmatrix}
					        ~~\cdot~~
					        \begin{pmatrix}
					            1&2\\3&5
					        \end{pmatrix}\\
					        =~~
					        \begin{pmatrix}
					        	26&45\\4&9\\14&26
							\end{pmatrix}\\ \\
	    				Berechnen~von~CB:\\
	    					\begin{pmatrix}
					            5&7\\7&-1\\8&2
					        \end{pmatrix}
					        ~~\cdot~~
					        \begin{pmatrix}
					        	1&-1\\3&2
					        \end{pmatrix}\\
					        =~~
					        \begin{pmatrix}
					        	26&9\\4&-9\\14&-4
							\end{pmatrix}\\ \\
	    				Addieren~von~CA~und~CB:\\
	    					\begin{pmatrix}
	    						26&45\\4&9\\14&26
	    					\end{pmatrix}
	    					~~+~~
	    					\begin{pmatrix}
	    						26&9\\4&-9\\14&-4
							\end{pmatrix} \\
							=~~
					        \begin{pmatrix}
					        	52&54\\8&0\\28&22
							\end{pmatrix}\\ \\
	    			\end{array}
	    		\end{split}
	    	\end{array}
	    \end{equation*}
	    Wir sehen, dass in beiden Fällen die gleiche Matrix herauskommt.
	    Folglich gilt das Distributivgesetz.
	\newpage
	\section*{Aufgabe 3}
	\label{sec:a3}
	    %%%%AUFGABE%%%%
	    Bestimmen Sie den Zeilenrang und den Spaltenrang der folgenden Matrix, indem Sie eine maximale
        Menge von linear unabhängigen Zeilen und eine maximale Menge von linear unabhängigen Spalten
        auswählen.
        \[
            \begin{pmatrix}
                3 & 4 & 5 & 6 \\
                2 & 3 & 4 & 5 \\
                1 & 2 & 3 & 4 \\
                4 & 7 & 6 & 4 
            \end{pmatrix}                        
        \]
	    %%%%RECHNUNG%%%%
	    \[v_1(3,4,5,6)\]
	    \[v_2(2,3,4,5)\]
	    \[v_3(1,2,3,4)\]
	    \[v_4(4,7,6,4)\]
	    \[v_1~~\Rightarrow~~lin.~unabh.\]
	    \[v_1,~v_2:\]
	    \begin{equation*}
	        \begin{array}{c|c}
	            \begin{split}
	                \begin{array}{l}
	                    \begin{array}{lcccl}
	                        I~~~~&3&2&0&~~~~~~~~(I:3)\\
	                        II~~~&4&3&0&~\\
	                        III~~&5&4&0&~\\
	                        IV~~~&6&5&0&~
	                    \end{array}\\\\
	                    \begin{array}{lcccl}
	                        I~~~~&1&\frac{2}{3}&0&~\\
	                        II~~~&4&3&0&~~~~~~~~(II-4\cdot I)\\
	                        III~~&5&4&0&~~~~~~~~(III-5\cdot I)\\
	                        IV~~~&6&5&0&~~~~~~~~(IV-6\cdot I)
	                    \end{array}\\\\
	                    \begin{array}{lcccl}
	                        I~~~~&1&\frac{2}{3}&0&~\\
	                        II~~~&0&\frac{1}{2}&0&~~~~~~~~(II\cdot 3)\\
	                        III~~&0&\frac{2}{3}&0&~\\
	                        IV~~~&0&1&0&~
	                    \end{array}
	                \end{array}
	            \end{split}
	        {}~~~~~&~~~~~{}
	            \begin{split}
	                \begin{array}{l}
    	                \begin{array}{lcccl}
	                        I~~~~&1&\frac{2}{3}&0&~\\
	                        II~~~&0&1&0&~\\
	                        III~~&0&\frac{2}{3}&0&~~~~~~~~(III-\frac{2}{3}II)\\
	                        IV~~~&0&1&0&~~~~~~~~(IV-II)
	                    \end{array}\\\\
	                    \begin{array}{lcccl}
	                        I~~~~&1&\frac{2}{3}&0&~\\
	                        II~~~&0&1&0&~\\
	                        III~~&0&0&0&~\\
	                        IV~~~&0&0&0&~
	                    \end{array}
	                \end{array}
	            \end{split}
	        \end{array}
	    \end{equation*}
	    Einzige Lösung: $\lambda_1=\lambda_2=0~~\Rightarrow~~v_1,~v_2~lin.~unabh.$\\
	    \newpage
	    \[v_1,~v_2,~v_3:\]
	    \begin{equation*}
	        \begin{array}{c|c}
	            \begin{split}
	            \begin{array}{l}	            
	                \begin{array}{lccccl}
	                    I~~~~&3&2&1&0&~~~~~~~~(I:3) \\
	                    II~~~&4&3&2&0&~ \\
	                    III~~&5&4&3&0&~ \\
	                    IV~~~&6&5&4&0&~ \\
	                \end{array}\\\\
	                \begin{array}{lccccl}
	                    I~~~~&1&\frac{2}{3}&0&~ \\
	                    II~~~&4&3&2&0&~~~~~~~~(II-4\cdot I) \\
	                    III~~&5&4&3&0&~~~~~~~~(III-5\cdot I) \\
	                    IV~~~&6&5&4&0&~~~~~~~~(IV-6\cdot I) \\
	                \end{array}\\\\
	                \begin{array}{lccccl}
	                    I~~~~&1&\frac{2}{3}&\frac{1}{3}&0&~ \\
	                    II~~~&0&\frac{1}{3}&\frac{2}{3}&0&~~~~~~~~(II\cdot 3) \\
	                    III~~&0&\frac{2}{3}&\frac{4}{3}&0&~ \\
	                    IV~~~&0&1&2&0&~ \\
	                \end{array}
	                \end{array}
	            \end{split}
	        {}~~~~~&~~~~~{}
	            \begin{split}
	                \begin{array}{l}
	                \begin{array}{lccccl}
	                    I~~~~&1&\frac{2}{3}&\frac{1}{3}&0&~ \\
	                    II~~~&0&1&2&0&~ \\
	                    III~~&0&\frac{2}{3}&\frac{4}{3}&0&~~~~~~~~(III-\frac{2}{3}\cdot II) \\
	                    IV~~~&0&1&2&0&~~~~~~~~(IV-II) \\
	                \end{array}\\\\
	                \begin{array}{lccccl}
	                    I~~~~&1&\frac{2}{3}&\frac{1}{3}&0&~~~~~~~~(I-\frac{2}{3}\cdot II) \\
	                    II~~~&0&1&2&0&~ \\
	                    III~~&0&0&0&0&~ \\
	                    IV~~~&0&0&0&0&~ \\
	                \end{array}\\\\
	                \begin{array}{lccccl}
	                    I~~~~&1&0&-1&0&~ \\
	                    II~~~&0&1&2&0&~ \\
	                    III~~&0&0&0&0&~ \\
	                    IV~~~&0&0&0&0&~ \\
	                \end{array}
	                \end{array}
	            \end{split}
	        \end{array}
	    \end{equation*}
	    Es gibt nicht-triviale Lösungen ($x-z=0,~y+2z=0$): $\Rightarrow~~v_1,~v_2,~v_3~~lin.abh.$\\
	    \\
%	    \[v_1,~v_2,~v_4\]
%	    \begin{equation*}
%	        \begin{array}{c|c}
%	            \begin{split}
%	                \begin{array}{l}
%	                    \begin{array}{lccccl}
%	                        I~~~~&&0&~\\
%	                        II~~~&&0&~\\
%	                        III~~&&0&~\\
%	                        IV~~~&&0&~
%	                    \end{array}\\\\
%	                    \begin{array}{lccccl}
%	                        I~~~~&&0&~\\
%	                        II~~~&&0&~\\
%	                        III~~&&0&~\\
%	                        IV~~~&&0&~
%	                    \end{array}\\\\
%	                    \begin{array}{lccccl}
%	                        I~~~~&&0&~\\
%	                        II~~~&&0&~\\
%	                        III~~&&0&~\\
%	                        IV~~~&&0&~
%	                    \end{array}\\\\
%	                    \begin{array}{lccccl}
%	                        I~~~~&&0&~\\
%	                        II~~~&&0&~\\
%	                        III~~&&0&~\\
%	                        IV~~~&&0&~
%	                    \end{array}\\\\ 
%	                    \begin{array}{lccccl}
%	                        I~~~~&&0&~\\
%	                        II~~~&&0&~\\
%	                        III~~&&0&~\\
%	                        IV~~~&&0&~
%	                    \end{array}
%	                \end{array}
%	            \end{split}
%	            {}~~~~~&~~~~~{}
%	            \begin{split}
%	                \begin{array}{l}
%        	            \begin{array}{lccccl}
%	                        I~~~~&&0&~\\
%	                        II~~~&&0&~\\
%	                        III~~&&0&~\\
%	                        IV~~~&&0&~
%	                    \end{array}\\\\
%	                    \begin{array}{lccccl}
%	                        I~~~~&&0&~\\
%	                        II~~~&&0&~\\
%	                        III~~&&0&~\\
%	                        IV~~~&&0&~
%	                    \end{array}\\\\
%	                    \begin{array}{lccccl}
%	                        I~~~~&&0&~\\
%	                        II~~~&&0&~\\
%	                        III~~&&0&~\\
%	                        IV~~~&&0&~
%	                    \end{array}\\\\
%	                    \begin{array}{lccccl}
%	                        I~~~~&&0&~\\
%	                        II~~~&&0&~\\
%	                        III~~&&0&~\\
%	                        IV~~~&&0&~
%	                    \end{array}
%	                    \begin{array}{lccccl}
%	                    \end{array}
%	                \end{array}
%	            \end{split}
%	        \end{array}
%	    \end{equation*}
%	\section*{Aufgabe 4}
%	\label{sec:a4}
%	    %%%%AUFGABE%%%%
%	    Bestimmen Sie den Zeilenrang und den Spaltenrang der Matrix aus Hausaufgabe 3,
%	    indem Sie die Matrix mit Hilfe des Gauß-Jordan-Verfahrens auf Zeilenstufenform bringen.
%	    %%%%RECHNUNG%%%%
%	\section*{Aufgabe 5}
%	\label{sec:a5}
%	    %%%%AUFGABE%%%%
%	    %%%%RECHNUNG%%%%
%	    In der Vorlesung wurde gezeigt, dass eine lineare Abbildung $f: V \rightarrow W$ eindeutig bestimmt ist
%        durch die Bilder einer der Elemente einer Basis von $V$ . Wir betrachten die lineare Abbildung $f: \mathbb{R}^2 \rightarrow \mathbb{R}^2$
%        mit $f((1,1))=(3,4)$ und $f((-2,2))=(2,0)$. Geben Sie eine $(2 \times 2)$-Matrix $A$, so dass
%        für alle $x  \in \mathbb{R}^2$ die Gleichung $Ax = f(x)$ gilt. \\
%        Hinweis: Ermitteln Sie zunächst Bilder $f(e_1)$ und $f(e_2)$ der Einheitsvektoren $e_1 = (1,0)$ und $e_2 =
%        (0, 1)$ unter der Abbildung $f$.
\end{document}