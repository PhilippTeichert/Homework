\documentclass[twoside]{article}
\usepackage[utf8]{inputenc}
\usepackage[T1]{fontenc}
\usepackage[ngerman]{babel}
\usepackage[margin=2.5cm]{geometry}

% import math packages
\usepackage{amsmath}
\usepackage{amsfonts}
\usepackage{amssymb}
\usepackage{amsthm}
% contradiction lightning
\usepackage{stmaryrd}
% better tilde
\usepackage{textcomp}
% algorithms and pseudo code
\usepackage{algorithmic}
\usepackage{algorithm}
%\usepackage{clrscode3e}
% formatting and layout
\usepackage{color}
\usepackage{fancyhdr}
% references and links
\usepackage{hyperref}
\usepackage{lastpage}
% settings
\usepackage{perpage}
\MakePerPage{footnote}
% custom text colors
\definecolor{pblue}{rgb}{0.13,0.13,1}
\definecolor{pgreen}{rgb}{0,0.5,0}
% tikz
\usepackage{tikz}
\usetikzlibrary{%
	arrows,			% arrow types
	automata,		% states and transitions
	positioning,	% position shifting
	shapes,			% circles and stuff for Venn diagrams
	backgrounds		% coloring and stuff
}
\usepackage{pgf}







\begin{document}






%#+-------------------------------------------------+#%
%#+						MAKROS				     	+#%
%#+-------------------------------------------------+#%


%% requires color package and the custom colors defined here 
\newcommand{\todo}[1]{
	\addcontentsline{toc}{subsubsection}{TODO: #1}
	\textcolor{pgreen}{\texttt{\-\\ \-\\//TODO: #1\-\\ \-\\}}
}
\newcommand{\sect}[3]{	%	custom section (level 1)
	\newpage
	\addcontentsline{toc}{section}{Zettel #1 (#2)}
	\section*{Zettel Nr. #1 (Ausgabe: #2, Abgabe: #3)}
	\label{sec:#1}
	\fancyfoot[LO,RE]{Abgabe: #3}
}
\newcommand{\sus}[1]{
	\addcontentsline{toc}{subsection}{Hausübung #1}
	\subsection*{Hausübung #1}
	\label{ssec:\thesection.#1}
}
\newcommand{\susP}[2]{
	\sus{#1}
	\points{#2}
}
\newcommand{\sss}[1]{
	\addcontentsline{toc}{subsubsection}{Teilaufgabe #1}
	\subsubsection*{Teilaufgabe #1}
	\label{sssec:\thesection.\thesubsection.#1}
}
\newcommand{\points}[1]{
	\begin{flushright}
		\begin{Large}
			[~~~~\string| ~#1~]
		\end{Large}
	\end{flushright}
}
%% custom short command for blue text
\newcommand{\tcb}[1]{
	\textcolor{blue}{#1}
}
%% custom short command for green text
\newcommand{\tcg}[1]{
	\textcolor{green}{#1}
}
%% custom command for better inline fractions
\newcommand*\ilfrac[2]{{}^{#1}\!/_{#2}}
%% shorter sum
\newcommand{\mysum}[2]{
	\sum_{i= #1}^{#2}
}
%% shorter variant and expectation
\newcommand{\var}[1]{%
	\textbf{Var}[$#1$]
}
\newcommand{\e}[1]{%
	\textbf{E}[$#1$]
}
\newcommand{\cov}[2]{%
	\textbf{COV}$[#1,#2]$
}



%#+-------------------------------------------------+#%
%#+						VARIABLEN			     	+#%
%#+-------------------------------------------------+#%

% flags for compiling
%% set counts other than 'True' to 1 to print the corresponding section
\newcount\True
\True=1
%% set to false to omit title page
\newcount\Title
\Title=1
%% set to false to omit point tables
\newcount\Table
\Table=1
%% set to false to omit empty page behind title
\newcount\Empty
\Empty=1
%% set to false to omit the table of contents
\newcount\TOC
\TOC=0
%% set to false to omit the first task sheet
\newcount\ZettelEins
\ZettelEins=0
%% set to false to omit the second task sheet
\newcount\ZettelZwei
\ZettelZwei=0
%% set to false to omit the third task sheet
\newcount\ZettelDrei
\ZettelDrei=0
%% set to false to omit the fourth task sheet
\newcount\ZettelVier
\ZettelVier=0
%% set to false to omit the fifth task sheet
\newcount\ZettelFuenf
\ZettelFuenf=0
%% set to false to omit the sixth task sheet
\newcount\ZettelSechs
\ZettelSechs=0
%% set to false to omit the seventh task sheet
\newcount\ZettelSieben
\ZettelSieben=0
%% set to false to omit the eigth task sheet
\newcount\ZettelAcht
\ZettelAcht=0
%% set to false to omit the ninth task sheet
\newcount\ZettelNeun
\ZettelNeun=1
%% set to false to omit the tenth task sheet
\newcount\ZettelZehn
\ZettelZehn=0
%% set to false to omit the eleventh task sheet
\newcount\ZettelElf
\ZettelElf=0
%% set to false to omit the twelfth task sheet
\newcount\ZettelZwoelf
\ZettelZwoelf=0


% global variables
%% Fach-Daten
\newcommand{\fachname}{Stochastik 1 für Studierende der Informatik}
\newcommand{\fachnummer}{MATH3-Inf}
\newcommand{\veranstaltungsnummer}{65-832}
\newcommand{\stinegruppe}{Übungsgruppe 2}
\newcommand{\termin}{Dienstag, 14.15 - 15.00\\Geom 431}
%% Gruppenmitglied 1
\newcommand{\memOneName}{Utz Pöhlmann}
\newcommand{\memOneMail}{4poehlma@informatik.uni-hamburg.de}
\newcommand{\memOneNr}{6663579}
%% Gruppenmitglied 2
\newcommand{\memTwoName}{Louis Kobras}
\newcommand{\memTwoMail}{4kobras@informatik.uni-hamburg.de}
\newcommand{\memTwoNr}{6658699}
%% Gruppenmitglied 3
\newcommand{\memThreeName}{Felix Gebauer}
\newcommand{\memThreeMail}{4gebauer@informatik.uni-hamburg.de}
\newcommand{\memThreeNr}{6671660}
%% Gruppenmitglied 4
\newcommand{\memFourName}{}
\newcommand{\memFourMail}{}
\newcommand{\memFourNr}{}
%% Datum
\newcommand{\datum}{\today\\}
%% leere Seite
\newcommand{\leereSeite}{
	\newpage
	\thispagestyle{empty}
	\mbox{}
	\newpage
}






%#+-------------------------------------------------+#%
%#+						FORMATIERUNG		     	+#%
%#+-------------------------------------------------+#%

\newcommand{\fach}{
	\begin{Huge}
		\fachname\\
	\end{Huge}
	\begin{LARGE}
		Modul: \fachnummer\\
		Veranstaltung: \veranstaltungsnummer\\
	\end{LARGE}
}

\newcommand{\gruppe}{
	\begin{LARGE}
		\stinegruppe\\
	\end{LARGE}
	\begin{Large}
		\termin\\
	\end{Large}
}

\newcommand{\memberOfGroup}[3]{
	\begin{center}
		\begin{Large}
			#1
		\end{Large}\-\\
		#2\-\\
		#3\-\\
	\end{center}
	\vspace{.5cm}
}
\newcommand{\datumf}{
	\begin{Large}
		\datum\-\\
	\end{Large}
}

% setting up footers and headers
%% setting page style so that footers and headers can be used
\pagestyle{fancy}
%% overwrite default values
\fancyhead{}
\fancyfoot{}
%% first group member in upper left corner
\lhead{\memOneName\-\\\memOneNr}
%% second group member in the upper center
\chead{\memTwoName\-\\\memTwoNr}
%% third group member in upper right corner
\rhead{\memThreeName\-\\\memThreeNr}
%% lower left corner (empty)
\lfoot{}
%% subject name in lower center
\cfoot{\fachname}
%% lower right corner (empty)
\rfoot{}
%% sets the page number to appear in either the lower left or lower right corner,
%% depending on whether the page number is even or odd, in the format "page X of Y"
\fancyfoot[LE,RO]{Seite \thepage ~von \pageref{LastPage}}
%% thin separation line over the footer
\renewcommand{\footrulewidth}{0.4pt}

% preparing some circles for Venn diagrams and stuff
\def\firstcircle{(0,0) circle (1.5cm)}
\def\secondcircle{(1,2) circle (1.5cm)}
\def\thirdcircle{(2,0) circle (1.5cm)}


%#+-------------------------------------------------+#%
%#+						DECKBLATT  			     	+#%
%#+-------------------------------------------------+#%
\ifnum\Title=\True
\thispagestyle{empty}
\-\vspace{0.5cm}
\begin{center}
	\fach
	\vspace{1.5cm}
	\gruppe
	\vspace{1.5cm}
	% group members
	\memberOfGroup{\memOneName}{\memOneMail}{\memOneNr}
	\memberOfGroup{\memTwoName}{\memTwoMail}{\memTwoNr}
%	\memberOfGroup{\memThreeName}{\memThreeMail}{\memThreeNr}
%	\memberOfGroup{\memFourName}{\memFourMail}{\memFourNr}
	% 1 cm to next element
	\vspace{1cm}
	\datumf
	\vspace{1cm}
	
	\ifnum\Table=\True
	\textbf{Punkte für die Hausübungen:}\\
	\vspace{1cm}
	\ifnum\ZettelEins=\True
	\begin{tabular}{c|c|c|c}
	~1.1~&~1.2~&~1.3~&~$\Sigma$~	\\	\hline
		 &	   &	 &
	\end{tabular}
	\fi
	\ifnum\ZettelZwei=\True
	\begin{tabular}{c|c|c|c|c}
	~2.1~&~2.2~&~2.3~&~2.4~&~$\Sigma$~	\\	\hline
		 &	   &	 &		&
	\end{tabular}
	\fi
	\ifnum\ZettelDrei=\True
	\begin{tabular}{c|c|c}
	~3.1~&~3.2~&~$\Sigma$~	\\	\hline
		 &	   &
	\end{tabular}
	\fi
	\ifnum\ZettelVier=\True
	\begin{tabular}{c|c|c|c}
	~4.1~&~4.2~&~4.3~&~$\Sigma$~	\\	\hline
		 &	   &	 &
	\end{tabular}
	\fi
	\ifnum\ZettelFuenf=\True
	\begin{tabular}{c|c|c|c}
	~5.1~&~5.2~&~5.3~&~$\Sigma$~	\\	\hline
		 &	   &	 &
	\end{tabular}
	\fi
	\ifnum\ZettelSechs=\True
	\begin{tabular}{c|c|c|c}
	~6.1~&~6.2~&~6.3~&~$\Sigma$~	\\	\hline
		 &	   &	 &
	\end{tabular}
	\fi
	\ifnum\ZettelSieben=\True
	\begin{tabular}{c|c|c|c}
	~7.1~&~7.2~&~7.3~&~$\Sigma$~	\\	\hline
		 &	   &	 &
	\end{tabular}
	\fi
	\ifnum\ZettelAcht=\True
	\begin{tabular}{c|c|c|c|c}
	~8.1~&~8.2~&~8.3~&~8.4~&~$\Sigma$~	\\	\hline
		 &	   &	  &		&
	\end{tabular}
	\fi
	\ifnum\ZettelNeun=\True
	\begin{tabular}{c|c|c|c|c}
	~9.1~&~9.2~&~9.3~&~9.4~&~$\Sigma$~	\\	\hline
		 &	   &	 &		&
	\end{tabular}
	\fi
	\ifnum\ZettelZehn=\True
	\begin{tabular}{c|c|c|c}
	~10.1~&~10.2~&~10.3~&~$\Sigma$~	\\	\hline
		 &	   &	 &
	\end{tabular}
	\fi
	\ifnum\ZettelElf=\True
	\begin{tabular}{c|c|c|c}
	~11.1~&~11.2~&~11.3~&~$\Sigma$~	\\	\hline
		 &	   &	 &
	\end{tabular}
	\fi
	\ifnum\ZettelZwoelf=\True
	\begin{tabular}{c|c|c|c}
	~12.1~&~12.2~&~12.3~&~$\Sigma$~	\\	\hline
		 &	   &	 &
	\end{tabular}
	\fi
	\fi
	
\end{center}
\newpage
\fi



%#+-------------------------------------------------+#%
%#+						INHALT  			     	+#%
%#+-------------------------------------------------+#%
\ifnum\Empty=\True
\leereSeite
\fi
\ifnum\TOC=\True
\pagestyle{empty}
\tableofcontents
\newpage
\fi
\pagestyle{fancy}
\pagenumbering{arabic}

%#+-------------------------------------------------+#%
%#+						ZETTEL 1					+#%
%#+-------------------------------------------------+#%
\ifnum\ZettelEins=\True
\sect{1}{05. April 2016}{12. April 2016}
%%%%%%%%%%%%%%%%%%%%%%%%%%%%%
%
%%%% Aufgabe 1
%
%%%%%%%%%%%%%%%%%%%%%%%%%%%%%
\susP{1.1}{8}
(Zufallsexperimente, 2+2+2+2 Punkte).
Handelt es sich in den folgenden Situationen um Zufallsexperimente?
Begründen Sie ihre Antwort.
\begin{enumerate}
	%%%%% Aufgabe 1a ---------------
	\item[a)] Sonnenaktivität am 07.04.2016 um 10:00.\\
	%%%%% Bearbeitung 1a ----------
	\textbf{Kein Zufallsexperiment}, da die Situation nicht rekonstruierbar ist.
	
	%%%%% Aufgabe 1b ---------------
	\item[b)] Verkehrssituation am Schlump Donnerstags 10:00.\\
	%%%%% Bearbeitung 1b ----------
	\textbf{Zufallsexperiment}, da die Situation rekonstruiert und dementsprechend das Experiment beliebig oft wiederholt werden kann.
	
	%%%%% Aufgabe 1c ---------------
	\item[c)] Gleichzeitiger Wurf von drei fairen Würfeln, Beobachtung der Augensumme.\\
	%%%%% Bearbeitung 1c ----------
	\textbf{Zufallsexperiment}, da die Situation rekonstruiert und dementsprechend das Experiment beliebig oft wiederholt werden kann.
	
	%%%%% Aufgabe 1d ---------------
	\item[d)] Lebenszeit römischer Kaiser nach Inthronisierung.\\
	%%%%% Bearbeitung 1d ----------
	\textbf{Kein Zufallsexperiment}, da die Umstände für jeden Kaiser eindeutig sind und dementsprechend die Situation nicht rekonstruierbar ist.
	
	
\end{enumerate}


%%%%%%%%%%%%%%%%%%%%%%%%%%%%%
%
%%%% Aufgabe 2
%
%%%%%%%%%%%%%%%%%%%%%%%%%%%%%
\susP{1.2}{12}
(Zufall in der Praxis, 6+6 Punkte).
Stellen Sie in den beiden folgenden Situationen dar, an welchen Stellen sich Zufallseinflüsse auswirken.
Geben Sie darüber hinaus kurz an, welche Ziele (ggf. aus Sicht der unterschiedlichen Parteien) erreicht werden sollen.
\begin{enumerate}
	%%%%% Aufgabe 2a ---------------
	\item[a)] Ein Flughafen hat eine Landebahn, die Flugzeuge müssen beim Landen einen gewissen zeitlichen Abstand halten (vorangegangenes Flugzeug muss die Landebahn verlassen und sich weit genug entfernt haben, Wirbelschleppen müssen \dq verflogen\dq  sein, \dots).
	Nach Flugplan kommen die Flugzeuge gleichmäßig an, der Abstand zwischen zwei Ankünften ist im Mittel etwas größer als der notwendige zeitliche Abstand zwischen zwei Landungen.
	Ist eine Landung für ein Flugzeug noch nicht möglich, da die Landebahn noch nicht wieder freigegeben ist, muss das Flugzeug eine Warteschleife fliegen.\\
	\vspace{.5cm}\\
	%%%%% Bearbeitung 2a ----------
	Das primäre Zufallsereigniss ist die Verspätung von Fliegern\footnote{siehe analog dazu Beispiel 1.3.1 im Skript, Version 05.04.2016}.
	Zufallsereignisse können sich insofern auf den Flughafen auswirken, dass Leute verspätet in ihren Flieger steigen können, ankommende Passagiere ihr Gepäck nicht zeitnah erhalten, da sich das Gepäckstück in einem anderen, verzufallten Flugzeug befinden kann, oder Passagiere ihren Anschluss verpassen.
	Eine zufällige Bombendrohung kann zu einer kompletten Evakuierung des Flughafens verleiten, wodurch der Betrieb vollständig zum Stillstand kommt.
	Zufällig kann es sein, dass zwei anfliegende Flugzeuge nicht richtig koordiniert werden und es zu einer (beinahe-)Kollision kommt.
	Der Gegenfall ist, dass ein Flugzeug zufällig fälschlicherweise in die Warteschlange eingereiht wird.
	
	Die Ziele der Passagiere sind, dass sie ihr Gepäck erhalten und ihre Anschlussgelegenheiten rechtzeitig erreichen.
	
	Die Ziele der Flughafenbetreiber sind ein reibungsloser Betrieb und zufriedene Passagiere.
	%%%%% Aufgabe 2b ---------------
	\item[b)]
	Die Universität Hamburg betreibt das System STiNE zur Vorlesungsplanung, -information und -unter-stützung auf einem Server.
	Von Studierenden und Dozenten kommen Anfragen an und werden bearbeitet.
	Wird eine Anfrage nicht nach einer bestimmten Zeit erfolgreich bearbeitet, so wird sie unerfolgreich abgelehnt.\\
	\vspace{.5cm}\\
	%%%%% Bearbeitung 2b ----------
	Das primäre Zufallsereignis ist die Anfrage einer Person an das System\footnote{analog Skript, Bsp. 1.3.2, V. 05.04.2016}.
	Hierbei gibt es drei Zufallsgrößen: Art, Umfang und Anzahl der Anfragen.
	Die Art der Anfrage bindet unterschiedlich viele Ressourcen gleichzeitig, während der Umfang die Dauer der Ressourcenbindung beeinflusst.
	Die Anzahl der Anfragen gibt an, wie viele Anfragen das System gleichzeitig bearbeiten können muss.
	Ein weiteres Zufallsereignis ist der Zeitpunkt der Anfrage, welcher in einem Zeitraum der Unerreichbarkeit bzw. Nichtverfügbarkeit des Systems liegen kann.
	
	Ziel des Betreibers ist ein sinnvolles LoS\footnote{Skript, Bsp. 1.3.2, V. 05.04.2016}.
	
	Ziel des Service-Nutzers ist eine angemessene Bearbeitungszeit des Systems sowie dass er die Anfrage nicht mehrfach stellen muss, bis sie erfolgreich bearbeitet wird.
	
	
\end{enumerate}
%%%%%%%%%%%%%%%%%%%%%%%%%%%%%
%
%%%% Aufgabe 3
%
%%%%%%%%%%%%%%%%%%%%%%%%%%%%%
\susP{1.3}{5}
(Mengenoperationen und Venn-Diagramme, 1+1+1+2 Punkte).
In einem Venn-Diagramm werden die Grundmengen symbolisch durch geometrische Objekte, meistens Kreise oder Ellipsen dargestellt.
Die Resultate betrachteter Mengenverknüpfungen werden
dann farblich oder durch Markierung hervorgehoben.
Beispielsweise veranschaulicht das folgende Diagramm den Schnitt $A \cap B$ zweier Mengen $A$ und $B$.
\begin{center}
	\begin{tikzpicture}
		\draw \firstcircle node {$A$};
		\draw \thirdcircle node {$B$};
		\begin{scope}[fill opacity=0.5]
			\clip \firstcircle;
			\fill[gray] \thirdcircle;
		\end{scope}
	\end{tikzpicture}
\end{center}
Zeichnen Sie die entsprechenden Diagramme für die folgenden Operationen.
\begin{enumerate}
	%%%%% Aufgabe 3a ---------------
	\item[a)] $A \setminus B = \{x \in A: x \not \in B\}$ für zwei Mengen $A,B$.\\
	%%%%% Bearbeitung 3a ----------
	\begin{tikzpicture}
	\begin{scope}[shift={(0cm,0cm)}]
        \begin{scope}[even odd rule]% first circle without the second
            \clip \thirdcircle (-3,-3) rectangle (3,3);
        	\fill[yellow] \firstcircle;
        \end{scope}
        \draw \firstcircle node {$A$};
        \draw \thirdcircle node {$B$};
    \end{scope}
    \end{tikzpicture}
	
	
	%%%%% Aufgabe 3b ---------------
	\item[b)]$A \Delta B := (A \setminus B) \cup (B \setminus A)$ für zwei Mengen $A,B$.
	($A \Delta B$ heißt auch \textit{symmetrische Mengendifferenz}.)\\
	%%%%% Bearbeitung 3b ----------
	\begin{tikzpicture}
	\begin{scope}[shift={(0cm,0cm)}]
        \begin{scope}[even odd rule]% first circle without the second
            \clip \thirdcircle (-3,-3) rectangle (3,3);
        	\fill[red] \firstcircle;
        \end{scope}
        \begin{scope}[even odd rule]% first circle without the second
            \clip \firstcircle (-3,-3) rectangle (5,3);
        	\fill[red] \thirdcircle;
        \end{scope}
        \draw \firstcircle node {$A$};
        \draw \thirdcircle node {$B$};
    \end{scope}
    \end{tikzpicture}
	
	
	%%%%% Aufgabe 3c ---------------
	\item[c)]$A \cap B \cap C$ für drei Mengen $A,B,C$.\\
	%%%%% Bearbeitung 3c ----------
	\begin{tikzpicture}
		\begin{scope}
    		\clip \firstcircle;
    		\clip \secondcircle;
			\fill[green] \thirdcircle;
		\end{scope}
		\draw \firstcircle node {$A$};
		\draw \secondcircle node {$B$};
		\draw \thirdcircle node {$C$};
	\end{tikzpicture}
	
	
	%%%%% Aufgabe 3d ---------------
	\item[d)]$(A \cap B) \cup (A \cap C) \cup (B \cap C)$ für drei Mengen $A,B,C$.\\
	%%%%% Bearbeitung 3d ----------
	\begin{tikzpicture}
		\draw \firstcircle node {$A$};
		\draw \secondcircle node[above] {$B$};
		\draw \thirdcircle node[below] {$C$};
		\begin{scope}[fill opacity=0.5]
			\clip \firstcircle;
			\fill[blue] \thirdcircle;
		\end{scope}
		\begin{scope}[fill opacity=0.5]
			\clip \firstcircle;
			\fill[blue] \secondcircle;
		\end{scope}
		\begin{scope}[fill opacity=0.5]
			\clip \secondcircle;
			\fill[blue] \thirdcircle;
		\end{scope}
	\end{tikzpicture}
\end{enumerate}
\fi
%#+-------------------------------------------------+#%
%#+						ZETTEL 2					+#%
%#+-------------------------------------------------+#%
\ifnum\ZettelZwei=\True
\sect{2}{12. April 2016}{19. April 2016}
\susP{2.1}{6}
(Ereignisse und Mengen, 3+3 Punkte).
Es sei $\Omega \neq \emptyset$ eine Ergebnismenge, außerdem seen $A,B,C \subset \Omega$ Ereignisse.
\begin{enumerate}
	%%%%% Aufgabe 1a ----------
	\item[a)] Beschreiben Sie das Ereignis $A \cup (B \cap C)$ verbal.\\
	\vspace{.5cm}\\ 
	%%%%% Bearbeitung 1a ----------
	Es tritt das Ereignis $A$ ein oder die Ereignisse $B$ und $C$.
	
	%%%%% Aufgabe 1b ----------
	\item[b)]Beschreiben Sie das Ereignis \dq Höchstens zwei der Ereignisse $A,B,C$ treten ein\dq ~mengentheoretisch.
	%%%%% Bearbeitung 1b ----------
	\[
		(A \cup B \cup C) \setminus (A \cap B \cap C)
	\]
\end{enumerate}

\susP{2.2}{4}
(Warten auf Zahl, 2+2 Punkte).
In einem Zufallsexperiment wird eine Münze solange geworfen, bis zum ersten Mal \dq Zahl\dq ~erscheint, die möglichen Ausgänge sind die natürlichen Zahlen, d.h. $\mathbb{N}$ ist die Ergebnismenge.
\begin{enumerate}
	%%%%% Aufgabe 2a ------------
	\item[a)]
	Stellen Sie das Ereignis \dq Der erste Wurf, bei dem Zahl erscheint, hat eine ungerade Nummer\dq ~als Menge dar.\\\vspace{.5cm}\\
	%%%%% Bearbeitung 2a ------------
	\[
		\{k, k \in \mathbb{N}, 2 \not | k\}
	\]
	$k$ ist die Anzahl der Würfe, nach denen zum ersten Mal \textit{Zahl} erscheint.
	%%%%% Aufgabe 2b ------------	
	\item[b)] Stellen Sie das Ereignis \dq Spätestens nach 10 Würfen ist einmal Zahl erschienen\dq ~als Menge dar.\\\vspace{.5cm}\\
	%%%%% Bearbeitung 2b ------------
	\[
		\{k, k \in \mathbb{N}, k \leq 10\}
	\]
	$k$ ist die Anzahl der Würfe, nach denen zum ersten Mal \textit{Zahl} erscheint.
\end{enumerate}

\susP{2.3}{7}
(Wahrscheinlichkeitsmaße, 2+5 Punkte).
Über $\Omega=\{1,2,3\}$ soll ein Wahrscheinlichkeitsmaß $P$ definiert werden.
\begin{enumerate}
	%%%%% Aufgabe 3a ------------
	\item[a)] Vervollständigen Sie die folgende Tabelle so, dass $P$ ein Wahrscheinlichkeitsmaß wird.\\\vspace{.5cm}\\
	\begin{center}\begin{tabular}{|c|c|c|c|c|c|c|c|c|}\hline
		$A$		&	$\emptyset$	&	$\{1\}$			&	$\{2\}$				&	$\{3\}$				&	$\{1,2\}$			&	$\{1,3\}$		&	$\{2,3\}$			&	$\{1,2,3\}$		\\ \hline
		$P(A)$	&	\tcg{$0$}	&	$\frac{1}{3}$	&	\tcb{$\frac{1}{2}$}	&	\tcb{$\frac{1}{6}$}	&	\tcb{$\frac{5}{6}$}	&	$\frac{1}{2}$	&	\tcb{$\frac{2}{3}$}	&	\tcg{$1$}		\\ \hline
	\end{tabular}\end{center}
	$P(\emptyset)=0$ und $P(\Omega)=1$ sind nach Definition gegeben.
	
	Es gilt $ \frac{1}{2} = P(\{1,3\}) = P(\{1\} \cup \{3\}) = P(\{1\}) + P(\{3\}) - P(\{1\} \cap \{3\}) = \frac{1}{3} + x - 0 \Rightarrow P(\{3\}) = \frac{1}{6}$.
	
	Desweiteren kann über $P(\Omega)=P(\{1,2,3\})=P(\{1\} \cap \{2\} \cap \{3\})=1$ der Wert von $P(\{2\})$ folgendermaßen ermittelt werden: $1 = P(\Omega) = P(\{1\} \cap \{2\} \cap \{3\}) = P(\{1\} \cap \{3\} \cap \{2\}) = P((\{1\} \cap \{3\}) \cap \{2\}) = P(\{1,3\}) + P(\{2\}) - P((\{1\} \cap \{3\}) \cup \{2\}) = \frac{1}{2} + x \Rightarrow x = P(\{2\}) = \frac{1}{2}$.
	
	Über den Additionssatz gilt dann $\frac{1}{3} + \frac{1}{2} (- 0) = P(\{1\}) + P(\{2\}) - P(\{1\} \cap \{2\}) = P(\{1\} \cap \{2\}) = P(\{1,2\}) = \frac{5}{6}$.
	
	Analog dazu $\frac{1}{2} + \frac{1}{6} (- 0) = P(\{2\}) + P(\{3\}) - P(\{1\} \cap \{3\}) = P(\{2\} \cap \{3\}) = P(\{2,3\}) = \frac{2}{3}$.
	
	%%%%% Aufgabe 3b ------------
	\item[b)] In einer anderen Situation kennen Sie über $\Omega = \{1,2,3\}$ nur Angaben zu $P(\{1\})$ und $P(\{2,3\})$.
	Begründen Sie, warum diese Information nicht ausreicht, um $P:2^{\Omega} \rightarrow \mathbb{R}$ eindeutig festzulegen.\\\vspace{.5cm}\\
	%%%%% Bearbeitung 3b -----------
	$P$ ist nicht eindeutig festlegbar, da keine eindeutige Aussage über $P(\{2\})$ und $P(\{3\})$ getroffen werden kann; es ist lediglich deren Vereinigung bekannt.
	Ohne definitive Werte für $P(\{2\})$ und $P(\{3\})$ lassen sich deren Vereinigungen mit $P(\{1\})$ nicht berechnen, weswegen die Tabelle nicht vervollständigt werden kann.
	
	
	Diese Information reicht nicht aus, um P eindeutig festzulegen, da man nur 2 Informationen gegeben hat, die weder eine Schnittmenge haben, um auf eine dritte schließen zu können, noch jedes bis auf ein Elementarereignis gegeben ist, um auf das letze schließen zu können.
	Somit sind keine weiteren Informationen extrahierbar.
	
	
\end{enumerate}

\susP{2.4}{8}
(Rechnen mit Wahrscheinlichkeiten, 3+5 Punkte).
Es werden zwei faire Würfel geworfen, dabei werden die folgenden Ereignisse betrachtet.
\begin{itemize}
	\item $A$ sei das Ereignis \dq Pasch gewürfelt\dq ~, d.h. beide Würfel zeigen die gleiche Augenzahl.
	Es gilt $P(A)=\frac{6}{36}=\frac{1}{6}$.
	\item $B$ sei das Ereignis \dq Maximum der Augenzahlen ist $\leq 3$, es gilt $P(B) = \frac{9}{36} = \frac{1}{4}$.
	\item $C$ sei das Ereignis \dq Augensumme 7 gewürfelt\dq ~, es gilt $P(C) = \frac{6}{36} = \frac{1}{6}$.
	\item $D$ sei das Ereignis \dq Augensumme 11 gewürfelt\dq ~, es gilt $P(D) = \frac{2}{36} = \frac{1}{18}$.
\end{itemize}
\begin{enumerate}
	\item[a)] Es gilt außerdem $P(A \cap B) = \frac{3}{36} = \frac{1}{12}$.
	Nutzen Sie diese Information und den Additionssatz, um $P(A \cup B)$ zu berechnen.\\\vspace{.5cm}\\
	Additionssatz: $P(A \cup B) = P(A) + P(B) - P(A \cap B)$.
	\[
		P(A \cup B) = P(A) + P(B) - P(A \cap B) = \frac{6}{36} + \frac{9}{36} - \frac{3}{36} = \frac{12}{36} = \frac{1}{3}
	\]
	\item[b)] Begründen Sie, dass $A,C,D$ paarweise disjunkt sind.
	Berechnen Sie anschließend $P(A \cup C \cup D)$.\\\vspace{.5cm}\\
	Da bei $C$ und $D$ unterschiedliche Summen gefordert sind, kann es nicht ein Ergebnis geben, welches auf beide Szenarien zutrifft.
	Die Summen, die in $C$ und $D$ gefordert werden, sind beide ungerade.
	Bei $A$ wird ein Pasch gefordert, d.h. beide Würfel zeigen die Augenzahl $i$.
	Demnach gilt für die Augensumme $i+i=2i$, welches eine gerade Zahl ist, womit sie weder den Wert 7 noch den Wert 11 annehmen kann.
	\begin{align*}
		P(A \cup C \cup D)	=& P(A) + P(C) + P(D) - (P(A \cap C) + P(A \cap D) + P(C \cap D)) + P(A \cap C \cap D)\\
							=& \frac{6}{36} + \frac{6}{36} + \frac{2}{36} - (0 + 0 + 0) + 0 = \frac{14}{36} = \frac{7}{18}
	\end{align*}
\end{enumerate}
\fi

\ifnum\ZettelDrei=\True
\sect{3}{19. April 2016}{26. April 2016}
\susP{3.1}{15}
(Die minimale Augenzahl, 5+5+3+2 Punkte).
Es werden zwei faire Würfel geworfen, der Ergebnisraum ist $\Omega = \{1,2,3,4,5,6\}^2$, das Wahrscheinlichkeitsmaß durch $P(\{(i,j)\})=\frac{1}{36}$ für alle $(i,j) \in \Omega$ definiert.
Die Zufallsvariable $X$ beschreibe nun das Merkmal \dq minimale Augenzahl\dq , d.h
\[
	X : \Omega \rightarrow \Omega ', X((i,j)) = \operatorname{min}\{i,j\}
\]
mit $\Omega ' = \{1,2,3,4,5,6\}$.
\begin{enumerate}
	\item[a)] Bestimmen Sie $X^{-1}(\{x\})$ für $x=1,2,3,4,5,6$.
	\begin{itemize}
		\item	1:	$\{(1,1),(1,2),(1,3),(1,4),(1,5),(1,6),(6,1),(5,1),(4,1),(3,1),(2,1)\}$
		\item	2:	$\{(2,2),(2,3),(2,4),(2,5),(2,6),(6,2),(5,2),(4,2),(3,2)\}$
		\item	3:	$\{(3,3),(3,4),(3,5),(3,6),(6,3),(5,3),(4,3)\}$
		\item	4:	$\{(4,4),(4,5),(4,6),(6,4),(5,4)\}$
		\item	5:	$\{(5,5),(5,6),(6,5)\}$
		\item	6:	$\{(6,6)\}$
	\end{itemize}
	\item[b)] Charakterisieren Sie die Verteilung von $X$, d.h. ergänzen Sie die Tabelle
	\begin{center}
		\begin{tabular}{|c|c|c|c|c|c|c|}
		\hline
		x	&	1	&	2	&	3	&	4	&	5	&	6	\\	\hline
		$P(X=x)$	& $\ilfrac{11}{36}$ & $\ilfrac{9}{36}$ & $\ilfrac{7}{36}$ & $\ilfrac{5}{36}$ & $\ilfrac{3}{36}$ & $\ilfrac{1}{36}$	\\	\hline
		\end{tabular}
	\end{center}
	\item[c)] Ermitteln Sie $P(X \leq 3)$ und bestimmen Sie die Wahrscheinlichkeit dafür, dass die minimale Augenzahl ungerade ist.
		\begin{itemize}
			\item $P(X \leq 3) = P(X = 1) + P(X = 2) + P(X = 3) = \ilfrac{11+9+7}{36} = \ilfrac{27}{36}$
			\item $P(X = x, 2 \nmid x) = P(X = 1) + P(X = 3) + P(X = 5) = \ilfrac{11+7+3}{36} = \ilfrac{21}{36}$
		\end{itemize}
	\item[d)] Hätte $\Omega ' = \{1,2,3,4,5,6,7,8,9,10\}$ gewählt werden dürfen?
	\begin{itemize}
		\item Ja, da ein Abbild kein Urbild braucht ($X^{-1}(\{7\})=X^{-1}(\{8\})=X^{-1}(\{9\})=X^{-1}(\{10\})=\emptyset$).
	\end{itemize}
\end{enumerate}
\susP{3.2}{10}
(Archäologie, 10 Punkte).
Bei Ausgrabungen in China haben Archäologen vor einigen Jahren einen Behälter entdeckt, in dem sich eine wohl 2400 Jahre alte \dq Suppe \dq ~befand.
Die mit Knochen bestückte grünliche Flüssigkeit befand sich in einem Kessel, der in einem Grab in der Stadt Xian entdeckt wurde, wie die Zeitung \dq Global Times\dq berichtete.
Der Fund wurde bei Ausgrabungen im Rahmen des Ausbaus des Flughafens der Stadt Xian gemacht, die für ihre Terrakotta-Armee bekannt ist.
Eine Untersuchung sollte nun zeigen, welche Zutaten sich in der Flüssigkeit befanden, und ob es sich tatsächlich um eine Suppe handelt (soweit die Realität \dots)


Wissenschaftler und Studierende der Universität Hamburg mischten bei diesen Untersuchungen mit.
Sie überlegten vorher, ob in dem Behälter wenigstens eine der Zutaten ``Reis'', ``Peking-Ente'' oder ``Brokkoli'' zu identifizieren sei.
Dabei nehmen sie an, dass die Wahrscheinlichkeit, dass Reis verwendet wurde, bei $\frac{4}{5}$ liegt, die für Peking-Ente bei $\frac{1}{2}$ und die bei Brokkoli bei $\frac{1}{4}$.
Die Wahrscheinlichkeit, dass Reis und Peking-Ente verwendet wurden, liegt bei $\frac{9}{20}$, die Wahrscheinlichkeit für Reis und Brokkoli bei $\frac{3}{20}$, die Wahrscheinlichkeit für Peking-Ente und Brokkoli bei $\frac{1}{20}$.
Schließlich liegt die Wahrscheinlichkeit, dass alle drei Zutaten verwendet wurden, ebenfalls bei $\frac{1}{20}$.
Hätten Sie hier helfen können?
Ermitteln Sie die gesuchte Wahrscheinlichkeit.
\begin{center}

	\begin{tikzpicture}
		\draw \firstcircle node[left] {$P=\ilfrac{1}{2}$};
		\draw \secondcircle node {$R=\ilfrac{4}{5}$};
		\draw \thirdcircle node[right] {$B=\ilfrac{1}{4}$};
	\end{tikzpicture}
\end{center}
Stehe $P$ für \textit{Peking-Ente}, $R$ für \textit{Reis} und $B$ für \textit{Brokkoli}.
Neben der Beschriftung des Venn-Diagramms gilt außerdem:
\begin{itemize}
	\item $P(\{R\} \cap \{B\}) = \ilfrac{3}{20}$
	\item $P(\{R\} \cap \{P\}) = \ilfrac{9}{20}$
	\item $P(\{P\} \cap \{B\}) = \ilfrac{1}{20}$
	\item $P(\{R\} \cap \{B\} \cap \{P\}) = \ilfrac{1}{20}$
\end{itemize}
Die Wahrscheinlichkeit, dass mindestens eine dieser Zutaten verwendet wurde, liegt somit bei
$P(\{B\} \cup \{P\} \cup \{R\}) = P(\{B\}) + P(\{P\}) + P(\{R\}) - (P(\{R\} \cap \{B\}) + P(\{R\} \cap \{P\}) + P(\{P\} \cap \{B\})) + P(\{R\} \cap \{B\} \cap \{P\}) = \ilfrac{5}{20} + \ilfrac{10}{20} + \ilfrac{16}{20} - (\ilfrac{3}{20} + \ilfrac{9}{20} + \ilfrac{1}{20}) + \ilfrac{1}{20} = \ilfrac{19}{20}$.

Die Wahrscheinlichkeit, dass mindestens zwei dieser Zutaten verwendet wurden, liegt bei $P(\{R\} \cap \{B\}) + P(\{R\} \cap \{P\}) + P(\{P\} \cap \{B\}) - 2 \cdot P(\{R\} \cap \{B\} \cap \{P\}) = \ilfrac{3+9+1-1}{20} = \ilfrac{12}{20}$.

Wie nach Aufgabe ist die Wahrscheinlichkeit dafür, dass alle drei Zutaten verwendet wurden, gerade $P(\{R\} \cap \{B\} \cap \{P\}) = \ilfrac{1}{20}$.
\fi



\ifnum\ZettelVier=\True
\sect{4}{26. April 2016}{3. Mai 2016}
\susP{4.1}{10}
(Fußball, 4+3+3 Punkte).
Der Trainer eines Fußballklubs hat in seinem Aufgebot drei Torhüter, sieben Verteidiger, acht Mittelfeldspieler und vier Stürmer.
\paragraph{a.}
Auf wieviele Arten kann der Trainer ein Team zusammenstellen, wenn er im 4-4-2-Systm spielen will, also mit einem Torhüter, je vier Verteidigern und Mittelfeldspielern sowie zwei Stürmern?
\[
	\begin{array}{ll}
		 &	\binom{3}{1} \cdot \binom{7}{4} \cdot \binom{8}{4} \cdot \binom{4}{2}	\\
		=&	\ilfrac{3!}{1!2!} \cdot \ilfrac{7!}{4!3!} \cdot \ilfrac{8!}{4!4!} \cdot \ilfrac{4!}{2!2!}	\\
		=&	3 \cdot (7 \cdot 5) \cdot (2 \cdot 5 \cdot 7) \cdot 6\\
		=&	3 \cdot 35 \cdot 70 \cdot 6\\
		=&	210 \cdot 210\\
		=&	44100\text{ Möglichkeiten}
	\end{array}
\]

\paragraph{b.}
Mit welcher Wahrscheinlichkeit ist der bei den Fans beliebte Stürmer Abel Bebel im Team, wenn der als Feierbiest bekannte Trainer nach feuchtfröhlichem Beisammensein seine Trainingseindrücke vergisst und aus allen im 4-4-2-System möglichen Aufstellungen rein Zufällig eine auswählt?
\\\vspace{.3cm}\\
Hier wird einer der Stürmer gesetzt, der Trainer hat also noch 1 Slot, den er mit einem von 3 Stürmern besetzen kann, übrig.
Der Rest ändert sich nicht.
\[
	\begin{array}{ll}
		 &	\binom{3}{1} \cdot \binom{7}{4} \cdot \binom{8}{4} \cdot \binom{3}{1}	\\
		=&	3 \cdot 35 \cdot 70 \cdot 3\\
		=&	22050
	\end{array}
\]
Es gibt also 22050 Fälle, die Abel Bebel enthalten, und insgesamt 44100.
Also ist er in der Hälfte der Fälle enthalten, womit er zu 50\% auf dem Platz steht.

\paragraph{c.}
Auf wieviele Arten kann der Trainer die vier ausgewählten Verteidiger auf den vier Positionen seiner Abwehrkette verteilen, wenn er davon ausgeht, dass alle ausgewählten Verteidiger jede Position spielen können?
\[
	4! = 24
\]


\susP{4.2}{7}
(Ein neues Spiel 77, 3+4 Punkte).
Sie arbeiten bei einer Lotterie.
Diese möchte eine Variante der bekannten Fernsehlotterie ,,Spiel 77 '' auf den Markt bringen:
\begin{itemize}
	\item Es werden wie gewohnt 7 Ziffern von 0 bis 9 mit Zurücklegen gezogen, jede Ziffer kann also mehrfach vorkommen.
	\item Im Unterschied zum bekannten Spiel werden die Ziffern am Ende der Größe nach aufgsteigend sortiert, aus der Ziehung 5,0,7,9,1,7,5 wird also die Zahl 0155779 zusammengesetzt.
\end{itemize}
Die sortierte siebenstellige Zahl ist dann das Ergebnis der Ziehung, die Menge aller solchen sortierten Zahlen wird mit $\Omega$ bezeichnet.
\paragraph{a.}
Wie viele Ergebnisse gibt es, d.h. wie viele Elemente enthält $\Omega$?
\[
	|\Omega|=\binom{n+k-1}{k}=\binom{10+7-1}{7}=\binom{16}{7}=11440
\]
\paragraph{b.}
Ist auf $\Omega$ eine Laplace-Annahme gerechtfertigt?
\paragraph{Nein,} da $P(X=\dq 0000000\dq)$ geringer ist als $P(X=\dq 1234567\dq)$.
Dies ist bedingt durch den Umstand der Sortierung, womit z. B. die Ereignisse $X=\dq 1234567\dq$ und $X=\dq 7654321\dq$ auf das gleiche Ergebnis abbilden, jedoch Ergenisse wie $X=\dq iiiiiii\dq, i \in [0,1,\dots,9]$ jeweils nur ein Urbild haben.






\susP{4.3}{8}
(Zwei Wahrscheinlichkeitsmaße, 8 Punkte).
Es sei $\Omega \neq \emptyset$ eine diskrete Menge, $P,Q: 2^{\Omega} \rightarrow \mathbb{R}$ seien zwei Wahrscheinlichkeitsmaße, und es sei $\alpha \in [0,1]$.
Zeigen Sie, dass durch
\[
	R(A) := \alpha P(A) + (1-\alpha)Q(A),~~~~~A \subset \Omega
\]
ebenfalls ein Wahrscheinlichkeitsmaß $R:2^{\Omega} \rightarrow \mathbb{R}$ definiert wird.
Warum gilt dies für $\alpha \not \in [0,1]$ nicht?
\[
	\begin{array}{c}
		\begin{array}{ll}
			R(\Omega)	&=\alpha \cdot P(\Omega)+(1-\alpha)\cdot Q(\Omega)~~~~~~~~P(\Omega),Q(\Omega) = 1\text{, da es Wahrscheinlichkeitsmaße sind}\\
						&=\alpha+1-\alpha\\
						&=1
		\end{array}\\\\
		\begin{array}{c}
			\alpha \in [0,1]\text{ muss gelten, da } R \rightarrow [0,1] \text{ sonst nicht gilt}
		\end{array}\\\\
		\begin{array}{ll}
			R(\overset{\infty}{\underset{i=1}{\cup}}A_i)	&=	\alpha \cdot P(\overset{\infty}{\underset{i=1}{\cup}} A_i)+ (1-\alpha) \cdot Q(\overset{\infty}{\underset{i=1}{\cup}} A_i)\\
															&=	\alpha \cdot \sum_{i=1}^{\infty}P(A_i) + (1-\alpha) \cdot \sum_{i=1}{\infty}Q(A_1)\\
															&=	\sum_{i=1}^{\infty} \alpha \cdot P(A_i) + (1-\alpha) \cdot Q(A_i)\\
															&=	\sum_{i=1}^{\infty} R(A_i)
		\end{array}
	\end{array}
\]

\fi



\ifnum\ZettelFuenf=\True
\sect{5}{03. Mai 2016}{10. Mai 2016}
\susP{5.1}{11}
	(Stichprobenentnahme, 5+6 Punkte).
	Aufgrund von Ungenauigkeiten in der Produktion sind in jedem 1000er Pack einer bestimmten Sorte Schrauben immer 70 dabei, die nicht den Qualitätsanforderungen entsprechen.
	Sie entnehmen einem vollen 1000er Pack acht Schrauben.
	Da Sie diese nicht zurücklegen, sondern verwenden, wissen Sie, dass die Anzahl $X$ der defekten Schrauben unter den acht entnommenen einer hypergeometrischen Verteilung folgt.
	
	\textit{Hinweis:} Sie werden einen Computer oder Taschenrechner benötigen.

	\sss{a)}
	Geben Sie die Parameter der hypergeometrischen Verteilung an, und bestimmen Sie die Wahrscheinlichkeit, dass höchstens zwei Schrauben defekt sind.
	
	\[
		X \textasciitilde Hg_{8,70,1000} = \frac{\binom{70}{n} \cdot \binom{1000-70}{8-n}}{\binom{1000}{8}} = P(X=n)
	\]
	\[
		P(X \leq 2) = \mysum{0}{2} \frac{\binom{70}{i} \binom{1000-70}{8-i}}{\binom{1000}{8}}\approx 0.9857 \hat{=} 98.57\%
	\]
	
	\sss{b)}
	Verwenden Sie eine geeignete Binomialverteilung zur Approximation, und bestimmen Sie darauf basierend eine Näherung für die gesuchte Wahrscheinlichkeit.
	Wie bewerten Sie die Approximation?
	\[
		P(X\leq 2) = \mysum{0}{2}\binom{8}{i}\cdot\left(\frac{70}{1000}\right)^i\cdot\left(1-\frac{70}{1000}\right)^{8-i} \approx 0.9853 \hat{=} 98.53\%
	\]
	Abweichung minimal; Binomialverteilung eignet sich als Näherung, wenn die Varianz größer als 9 ist (hier gegeben durch $V=1000\cdot0.01\cdot0.99=9.9$) und der Bruch $\ilfrac{n}{N}$ kleiner als 0.05 (hier gegeben durch $\ilfrac{n}{N}=\ilfrac{8}{1000}=0.008 \leq 0.05$).
\susP{5.2}{10}
	(Glücksspiel, 5+5 Punkte).
	In einem Glücksspiel haben Sie eine Gewinnwahrscheinlichkeit von $\ilfrac{1}{100}$.
	Sie spielen ein Jahr lang wöchentlich, also 52 mal.
	$X$ bezeichne die Anzahl von Spielen, bei denen Sie gewinnen.
	
	\textit{Hinweis:} Auch hier werden Sie einen Computer oder Taschenrechner benötigen.
	
	\sss{a)}
	Welcher Verteilung folgt die Zufallsvariable $X$?
	Bestimmen Sie die Wahrscheinlichkeit dafür, dass Sie
	\begin{itemize}
		\item höchstens einmal gewinnen,
		\item mehr als einmal gewinnen.
	\end{itemize}
	\[
		X \textasciitilde{} Bin_{52,0.01} = P(X=n)=\binom{52}{n}\cdot 0.01^n \cdot 0.99^{52-n}
	\]
	\[
		\begin{array}{lllll}
			P(X\leq 1)&=&\mysum{0}{1}\binom{52}{i} \cdot 0.01^i \cdot 0.99^{52-i} &\approx& 0.9044 \hat{=} 90.44\%\\
			\\
			P(X > 1) &=& 1-P(X\leq 1) &\approx& 1-0.9044=0.0956 
		\end{array}
	\]
	
	\sss{b)}
	Setzen Sie eine geeignete Poisson-Verteilung zur Approximation ein, und berechnen Sie mit dieser eine Näherung dafür, dass Sie mehr als einmal gewinnen.
	\[
		Pois_{\lambda}(k)=\frac{\lambda^k}{k!}e^{-\lambda}=\frac{\lambda^k}{k!} \cdot \frac{1}{\mysum{0}{\infty}\frac{\lambda^k}{k!}}
	\]
	\[
		\lambda=n \cdot p=52\cdot0.01=0.52
	\]
	\[
		Pois_{0.52}(i>1)=\mysum{2}{52}\ilfrac{0.52^i}{i!}~e^{-0.52}\approx0.0963 \hat{=} 9.63\%
	\]
\susP{5.3}{4}
	(Eine Zähldichte, 4 Punkte).
	Die Funktion $f:\{1,\dots,n\} \rightarrow \mathbb{R}$ mit $k \rightarrow p_k$ soll eine Zähldichte werden.
	Äquivalent formuliert soll $(p_k)_{k=1}^{n}$ ein Wahrscheinlichkeitsvektor werden.
	Dabei ist $p_k=c \cdot k$ mit einer Konstanten $c$ gesetzt.
	
	Bestimmen Sie $c$ so, dass die Anforderungen an eine Zähldichte bzw. einen Wahrscheinlichkeitsvektor erfüllt sind.\\
	

	\textit{Hinweis:} Es gilt $\sum_{k=1}^{n}k=\ilfrac{1}{2}n(n+1)$.
	\\\vspace{.4cm}\\
	\textit{Anmerkung: $i := k$.}
	Es gilt: 
	\[
		p_i = c \cdot i \Leftrightarrow \mysum{1}{n} c \cdot i = 1
	\]
	Durch den Hinweis erhalten wir:
	\[
		1 = \mysum {1}{n}ci = c\cdot \mysum{1}{n} i \Leftrightarrow c = \frac{1}{\mysum{1}{n} i} = \frac{1}{\frac{n(n+1)}{2}} = \frac{2}{n(n+1)}
	\]
	\paragraph{Nichtnegativität.}
	$i$ ist stets positiv, ebenso wie $n$.
	Beide Werte werden an keiner Stelle subtrahiert.
	Damit nimmt der Term nie einen negativen Wert an.
	\paragraph{Normiertheit.}
	Für die Normiertheit muss gelten: $\mysum{1}{n}c\cdot i=1$.
	\begin{equation*}
		\begin{array}{rll}
			1 	&=&	\mysum{1}{n} \frac{2}{n(n+1)}\cdot i\\
				&=&	\frac{2}{n(n+1)}\mysum{1}{n}i\\
				\Rightarrow \mysum{1}{n}i	&=& \frac{n(n+1)}{2}
		\end{array}
	\end{equation*}
	Zusammen mit $c$ ergibt sich dann
	\[
		c \cdot \mysum{1}{n}i = \frac{n(n+1)}{2} \cdot \frac{2}{n(n+1)} = 1 \qed
	\]
\fi
\ifnum\ZettelSechs=\True
\sect{6}{10. Mai 2016}{24. Mai 2016}
\susP{6.1}{9}
(Roulette, 3+4+2 Punkte).
Beim Roulette sind die möglichen Ausgänge die Zahlen {0, . . . , 36}, bei guten Roulette-Tischen ist eine Laplace-Annahme gerechtfertigt.
Im Spiel können Sie auf verschiedene Gruppen von Zahlen (einzelne Zahlen, corner, split, rot/schwarz, . . . ) wetten, allen Wettmöglichkeiten ist gemeinsam, dass sie die (grüne) 0 nicht enthalten.
Als allgemeine Auszahlungsregel gilt: Haben Sie auf eine Gruppe aus k Zahlen gesetzt, erhalten Sie (Ihren Einsatz mitgerechnet) das $\ilfrac{36}{k}$-fache Ihres Einsatzes zurück.
Setzen Sie $n$ Geldeinheiten ein, so beträgt Ihr Gewinn im Erfolgsfall $n \cdot (\ilfrac{36}{k}-1)$, im Misserfolgsfall verlieren Sie ihren Einsatz, Ihr Gewinn beträgt also $-n$ Geldeinheiten.
\sss{a)}
In der einfachsten Variante setzen sie auf ,,rot'' oder ,,schwarz''.
Beide Zahlengruppen umfassen jeweils 18 der 37 möglichen Ereignisse.
Setzen Sie $n$ Geldeinheiten ein, so können Sie im Erfolgsfall $n$ Geldeinheiten Gewinn machen, im Misserfolgsfall verlieren Sie $n$ Geldeinheiten.
Bestimmen Sie den Erwartungswert des Gewinns.
\[
	\mu(x) = n \cdot \left(\frac{36}{18}-1 \right) \cdot \frac{18}{37} - n \cdot \frac{18}{37} - n \cdot \frac{1}{37} \approx -0.027n
\]
\sss{b)}
Betrachten Sie nun den allgemeinen Fall mit dem Setzen auf $k$ Zahlen und bestimmen Sie auch hier den Erwartungswert Ihres Gewinns.
\[
	\mu(x) = n \cdot \left(\frac{36}{k}-1 \right) \cdot \frac{k}{37} - n \cdot \frac{37-k}{37} = 35n-\frac{2kn}{37} = n \cdot \left(35-\frac{2k}{37} \right)
\]
\sss{c)}
Wie hängt der Erwartungswert des Gewinns von $n$ ab?
Wie hängt er von $k$ ab?\\
\vspace{.2cm}\\
Je größer $k$ wird, desto größer der zu erwartende Verlust.
Der Faktor, mit dem der Gewinn ausgezahlt wird, ist $f(k)$, die Auszahlung ist von $n$ abhängig in $n \cdot f(k)$.
$n$ wirkt deutlich stärker in die Auszahlung als $k$.


\susP{6.2}{7}
(Die Varianz der Poisson-Verteilung, 7 Punkte).
Es sei $Z$\textasciitilde$Pois_{\lambda}$.
Bestimmen Sie \textbf{Var}[Z].
\textit{Hinweis:} Den Erwartungswert haben Sie bereits in der Präsenzübung bestimmt.
\\\vspace{.3cm}\\
$Var[Z] = E[X^{2}]- E[X]^{2} \overset{da~E[X]=\lambda}{=} \mysum{z=0}{\infty} z^{2}\cdot \frac{\lambda^{z}\cdot e^{-\lambda}}{z!} - \lambda^{2}$

Da für $z=0$ die ganze Summe den Wert 2 annimmt, kann die Untergrenze inkrementiert werden.

$Var[Z] = \mysum{z=1}{\infty} z^{2}\cdot \frac{\lambda^{z}\cdot e^{-\lambda}}{z!} - \lambda^{2} = \mysum{z=1}{\infty} z^{2}\cdot \frac{\lambda\cdot\lambda^{z-1}\cdot e^{-\lambda}}{z\cdot(z-1)!} - \lambda^{2} = \lambda \cdot \mysum{z=1}{\infty} z\cdot \frac{\lambda^{z-1}\cdot e^{-\lambda}}{(z-1)!} - \lambda^{2}$

Redekrementierung der Untergrenze ergibt einen Term $(z+1)$, welcher mit dem Bruch ausmultipliziert wird, wodurch zwei Summen entstehen.

$Var[Z] = \lambda \cdot \mysum{z=1}{\infty} z\cdot \frac{\lambda^{z-1}\cdot e^{-\lambda}}{(z-1)!} - \lambda^{2} = \lambda \cdot \mysum{z=0}{\infty} (z+1) \cdot \frac{\lambda^{z}\cdot e^{-\lambda}}{z!} - \lambda^{2} = \lambda \cdot \mysum{z=0}{\infty} z \cdot \frac{\lambda^{z}\cdot e^{-\lambda}}{z!} +  \lambda \cdot \mysum{z=0}{\infty} \frac{\lambda^{z}\cdot e^{-\lambda}}{z!} - \lambda^{2}$

Die erste Summe des Terms ist eben $E[Z]=\lambda$ und die zweite Summe ist die Summe aller Wahrscheinlichkeiten, somit 1.
Demnach gilt also:

$Var[Z] = \lambda \cdot \mysum{z=0}{\infty} z \cdot \frac{\lambda^{z}\cdot e^{-\lambda}}{z!} +  \lambda \cdot \mysum{z=0}{\infty} \frac{\lambda^{z}\cdot e^{-\lambda}}{z!} - \lambda^{2} = \lambda \cdot \lambda + \lambda \cdot 1 - \lambda^{2}= \lambda \Rightarrow Var[Z]=\lambda \qed$

\susP{6.3}{9}
(Noch ein Glücksspiel, 3+2+4 Punkte).
Sie möchten ein neues Glücksspiel anbieten.
Dazu stehen Ihnen geometrische Objekte zur Verfügung, die mit Wahrscheinlichkeit $\frac{1}{2},\frac{1}{3},\frac{1}{4},\frac{1}{5},\frac{1}{6}\dots$ jeweils einen Smiley anzeigen.
Der Ablauf des Spiels sieht wie folgt aus: In der ersten Runde wird das Objekt mit Smiley-Wahrscheinlichkeit $\frac{1}{2}$ gewürfelt, in der zweiten Runde das mit Smiley-Wahrscheinlichkeit $\frac{1}{3}$, \dots, allgemein wird in der $n$-ten Runde das Objekt mit Smiley-Wahrscheinlichkeit $\frac{1}{n+1}$ gewürfelt.
Das Spiel endet, sobald zum ersten Mal ein Smiley erscheint.
Sie bezeichnen mit $X$ die Runde, in der zum ersten Mal ein Smiley gewürfelt wird.
Nach etwas Überlegung wissen Sie, dass dann
\[
	P(X=k)=\frac{1}{k(k+1)},~~~~~~k \in \mathbb{N}
\]
gilt (Sie müssen diese Formel nicht begründen).
Die Spieler setzen zu Beginn einen festen Betrag, und erhalten in Abhängigkeit von $X$ am Ende des Spiels eine Auszahlung.
Dabei ziehen Sie folgende Möglichkeiten in Betracht:
\begin{enumerate}
	\item[a)] Ist $X=k$, so zahlen Sie $k$ Geldeinheiten aus.
	\item[b)] Ist $X=k$, so zahlen Sie $k^2$ Geldeinheiten aus.
	\item[c)] Ist $X=k$, so zahlen Sie $\frac{8}{k+2}$ Geldeinheiten aus.
\end{enumerate} 
Geben Sie jeweils an, wie hoch Sie den Einsatz zu Beginn des Spiels wählen müssen, um ein faires Spiel zu erzeugen.\\
\vspace{.2cm}\\
\textit{Hinweis:} Es gilt
\[
	\sum_{k=1}^{n}\frac{1}{i(i+1)(i+2)}=\frac{n(n+3)}{4(n+1)(n+2)}.
\]
\-\\\vspace{.3cm}\\
Bei einem fairen Spiel macht keine Seite nennenswerten Gewinn, bei langen Versuchsreihen mitteln sich Gewinn und Verlust einer jeden Partei auf 0.
Demnach muss der Erwartungswert gleich dem Einsatz sein, damit ein Spiel fair ist, denn nur dann machen im Schnitt weder Spieler noch Bank Gewinn oder Verlust.
Es gilt: $\mu(X) = \mysum{1}{\infty}i\cdot p_i \overset{Distributivgesetz}{=}\Leftrightarrow \ilfrac{\mu(X)}{\mysum{1}{\infty}p_i} = \mysum{1}{\infty}i \overset{\mysum{1}{\infty}p_i=1}{\Leftrightarrow} \mu(X) = \mysum{1}{\infty}i$

Es wird also jeweils der Erwartungswert bestimmt; der Einsatz ist diesem dann gleichzusetzen.
\sss{a)}
\[
	\frac{k}{k(k+1)}=\frac{1}{k+1}
\]
\[
	\mu(x)=1
\]
Der Einsatz sollte also \textbf{1 Geldeinheiten} betragen.
\sss{b)}
\[
	\frac{k^2}{k(k+1)}=\frac{k}{k+1}=k+1
\]
\[
	\mu(x)=x
\]
Es lässt sich also \textbf{kein faires Spiel} erzeugen mit dieser Auszahlung.
\sss{c)}
\[
	\frac{\frac{8}{k+2}}{k(k+1)}=\frac{8}{k(k+1)(k+2)}=8\cdot\frac{1}{k(k+1)(k+2)}
\]
\[
	\mu(x)=2
\]
Der Einsatz sollte also \textbf{2 Geldeinheiten} betragen.
\fi



\ifnum\ZettelSieben=\True
\sect{7}{24. Mai 2016}{31. Mai 2016}
\susP{7.1}{4}
(Erwartungswerte müssen nicht existieren, 4 Punkte).
Die Zufallsvariable $Z$ möge Werte in $\mathbb{Z} \setminus \{0\}$ annehmen können, dabei möge
\[
	\mathbb{P}(Z=k) = \mathbb{P}(Z=-k) = \frac{1}{2k(k+1)}
\]
für alle $k \in \mathbb{N}$ gelten.
Zeigen Sie, dass \textbf{E}[$Z$] nicht existiert, konkreter:
Zeigen Sie, dass die Reihe zur Berechnung von \textbf{E}[$Z$] nicht absolut konvergiert, und auch nicht gegen $+\infty$ oder $-\infty$ konvergiert.\\
\vspace{-.2cm}\\
\textit{Hinweise:}
\begin{itemize}
	\item Es reicht aus, die Reihe zur Berechnung von \textbf{E}[$Z$] in zwei Summen zu zerlegen, von denen eine gegen $+\infty$ und eine gegen $-\infty$ strebt.
	\item Die harmonische Reihe divergiert, insbesondere gilt
	\[
		\sum_{k=1}^{\infty} \frac{1}{k+1}=\infty
	\]
\end{itemize}
\vspace{.4cm}
\begin{equation*}
	\begin{array}{rlll}
		\mathbb{E}[Z]&=    &\sum_{k \in \mathbb{Z}\setminus\{0\}}(k \cdot P(Z = k))&\\
		    &=    &\sum_{k = 1}^{\infty}(-k \cdot P(Z = -k)) + \sum_{k=1}^{\infty}(k \cdot P(Z = k))&\\
		    &=    &\sum_{k = 1}^{\infty}(-k \cdot \frac{1}{2k(k+1)})) + \sum_{k=1}^{\infty}(k \cdot \frac{1}{2k(k+1)})&\\
		    &=    &\sum_{k = 1}^{\infty}(\frac{-k}{2k(k+1)})) + \sum_{k=1}^{\infty}(\frac{k}{2k(k+1)})&\\
		    &=    &\sum_{k = 1}^{\infty}(-\frac{1}{2(k+1)})) + \sum_{k=1}^{\infty}(\frac{1}{2(k+1)})&\\
		    &=    &\sum_{k = 1}^{\infty}(-\frac{1}{2}\cdot\frac{1}{k+1})) + \sum_{k=1}^{\infty}(\frac{1}{2}\cdot\frac{1}{k+1})&\\
		    &=    &-\frac{1}{2}\cdot\sum_{k = 1}^{\infty}(\frac{1}{k+1})) + \frac{1}{2}\cdot\sum_{k=1}^{\infty}(\frac{1}{k+1})&\\
		    &=    &\frac{1}{2}\cdot\sum_{k=1}^{\infty}(\frac{1}{k+1}) - \frac{1}{2}\cdot\sum_{k = 1}^{\infty}(\frac{1}{k+1}))&\\
		    &=    &\frac{1}{2}\cdot\left(\sum_{k=1}^{\infty}(\frac{1}{k+1}) - \cdot\sum_{k = 1}^{\infty}(\frac{1}{k+1})\right)& /\sum_{k=1}^{\infty}(\frac{1}{k+1})=\infty\\
		    &=    &\frac{1}{2}\cdot\left(\infty - \infty\right)&\\
	\end{array}
\end{equation*}
Da man $\infty - \infty$ nicht genau bestimmen kann, hat diese Aufgabe keine Lösung.
Somit konvergiert die Reihe auch nicht.


\susP{7.2}{13}
(Erwartungswert und Varianz der Gleichverteilung, 3+5+1+4 Punkte).
Die Zufallsvariable $X$ mit Werten in der Menge $\{a,a+1,\dots,b\}$ sei gleichverteilt, d.h.
\[
	P(X=k))\frac{1}{b-a+1},~~~~k=a,\dots,b.
\]
\begin{enumerate}
	\item[a)] Zeigen Sie \textbf{E}[$X$]$=\frac{a+b}{2}$.
	\item[b)] Bestimmen Sie \textbf{Var}[$X$].
	\item[c)] Wenden Sie ihre Resultate auf das Würfeln eines fairen Würfels an.
		Stimmen die Ergebnisse mit denen aus der Vorlesung überein?
	\item[d)] Sie haben sechs Würfel, der erste ist mit den Zahlen von $1,\dots,6$ beschiftet, der zweite mit den Zahlen von $2,\dots,7$, usw., der sechste mit den Zahlen von 6 bis 11.
		Sie würfeln mit allen gleichzeitig und bezeichnen mit $X_k$ die Augensumme, die der $k$-te Würfel anzeigt sowie mit $Z$ die Augensumme aller sechs Würfel.
		Bestimmen Sie \textbf{E}[$X_k$] für $k=1,\dots,6$ sowie \textbf{E}[$Z$].
\end{enumerate}
\vspace{-.2cm}\-\\
\textit{Hinweise:}
\begin{itemize}
	\item Verwenden Sie $Y=X-a$, um $X$ auf eine auf $\{0,\dots,b-a\}$ gleichverteilte Zufallsvariable $Y$ zu transformieren.
		Berechnen Sie zunächst \textbf{E}[$Y$] und \textbf{Var}[$Y$] und schließen Sie dann auf \textbf{E}[$X$] und \textbf{Var}[$Y$] zurück.
	\item Verwenden Sie die Summenformeln
	\[
		\begin{array}{ccc}
			\sum_{k=1}^{n}k = \frac{1}{2} n(n+1)	&	\text{~~~~und~~~~}	&	\sum_{k=0}^{n}k^2=\frac{1}{6}n(n+1)(2n+1).
		\end{array}
	\]
\end{itemize}
\sss{a)}
Sei $Y = X - a$.
Dann gilt: $\Omega'' = \{0, 1, ..., b-a\}$.
\begin{equation*}
    \begin{array}{rlll}
        \mathbb{E}[Y]&=    &\sum_{k \in \Omega''}(k \cdot P(Y = k))&\\
            &=    &\sum_{k = 0}^{b-a}(k \cdot \frac{1}{(b-a)-(a-a)+1})&\\
            &=    &\sum_{k = 0}^{b-a}(k \cdot \frac{1}{b-a-0+1})&\\
            &=    &\sum_{k = 0}^{b-a}(k \cdot \frac{1}{b-a+1})&\\
            &=    &\sum_{k = 0}^{b-a}(k) \cdot \frac{1}{b-a+1}& |~Hinweis~2\\
            &=    &\frac{1}{2} \cdot (b - a) \cdot((b - a) + 1) \cdot \frac{1}{b-a+1}&\\
            &=    &\frac{b-a}{2} \cdot \frac{b - a + 1}{b-a+1}&\\
            &=    &\frac{b - a}{2}&\\
    \end{array}
\end{equation*}
Nun gilt nach anfänglicher Definition für $\mathbb{E}[X] =\mathbb{E}[Y] + a$.
\begin{equation*}
    \begin{array}{rlll}
        \mathbb{E}[X]&=    &\mathbb{E}[Y] + a&\\
            &=    &\frac{b - a}{2} + a&\\
            &=    &\frac{b - a}{2} + \frac{2}{2} \cdot a&\\
            &=    &\frac{b - a}{2} + \frac{2\cdot a}{2}&\\
            &=    &\frac{b - a}{2} + \frac{a + a}{2}&\\
            &=    &\frac{b - a + a + a}{2}&\\
            &=    &\frac{b + a}{2}&\\
    \end{array}
\end{equation*}

\sss{b)}
Sei $Y = X - a$.
Dann gilt: $\Omega'' = \{0, 1, ..., b-a\}$.
\begin{equation*}
    \begin{array}{rlll}
        \textbf{Var}[Y]&=    &\mathbb{E}[Y^2] - \mathbb{E}[Y]^2&\\
            &=    &\sum_{k \in \Omega''}(k^2 \cdot P(Y = k^2)) - (\frac{b - a}{2})^2&\\
            &=    &\sum_{k = 0}^{b-a}(k^2 \cdot (\frac{1}{(b-a)-(a-a)+1})^2) - (\frac{b - a}{2})^2&\\
            &=    &\sum_{k = 0}^{b-a}(k^2 \cdot (\frac{1}{b-a-0+1})^2) - (\frac{b - a}{2})^2&\\
            &=    &\sum_{k = 0}^{b-a}(k^2 \cdot (\frac{1}{b-a+1})^2) - (\frac{b - a}{2})^2&\\
            &=    &\sum_{k = 0}^{b-a}(k^2) \cdot (\frac{1}{b-a+1})^2 - (\frac{b - a}{2})^2& |~Hinweis~2\\
            &=    &\frac{1}{6}\cdot (b - a)\cdot((b - a) + 1)\cdot(2\cdot (b - a) + 1) \cdot (\frac{1}{b-a+1})^2 - (\frac{b - a}{2})^2&\\
            &=    &\frac{1}{6}\cdot (b - a)\cdot(b - a + 1)\cdot(2\cdot (b - a) + 1) \cdot \frac{1}{b-a+1} \cdot \frac{1}{b-a+1} - (\frac{b - a}{2})^2&\\
            &=    &\frac{b-a}{6} \cdot \frac{b-a+1}{b-a+1} \cdot \frac{b-a+b-a+1}{b-a+1} - (\frac{b - a}{2})^2&\\
            &=    &\frac{b-a}{6} \cdot \frac{b-a+b-a+1}{b-a+1} - (\frac{b - a}{2})^2&\\
            &=    &\frac{b-a}{6} \cdot \frac{2 \cdot (b-a)+1}{b-a+1} - (\frac{b - a}{2})^2&\\
            &=    &\frac{b-a}{6} \cdot \frac{2 \cdot (b-a)+1}{b-a+1} - \frac{b - a}{2} \cdot \frac{b - a}{2}&\\
            &=    &\frac{2 \cdot ((b-a)^2)+b-a}{6\cdot(b - a + 1)} - \frac{(b-a)^2}{4}&\\
    \end{array}
\end{equation*}
Nun gilt nach anfänglicher Definition für $\textbf{Var}[X] =\textbf{Var}[Y]$, da das $a$ wegfällt.
\begin{equation*}
    \begin{array}{rlll}
        \textbf{Var}[X]&=    &\textbf{Var}[Y]&\\
            &=    &\frac{2 \cdot ((b-a)^2)+b-a}{6\cdot(b - a + 1)} - \frac{(b-a)^2}{4}&\\
    \end{array}
\end{equation*}

\sss{c)}
\begin{equation*}
    \begin{array}{rlll}
        \mathbb{E}[W$\textit{ü}$rfel]&=    &\frac{b + a}{2}&/b=6;~a=1\\
            &=    &\frac{6 + 1}{2}&\\
            &=    &\frac{7}{2}&\\
            &=    &3,5&\\
    \end{array}
\end{equation*}
Dieses Ergebnis stimmt mit dem aus der Vorlesung überein.
\begin{equation*}
    \begin{array}{rlll}
        \textbf{Var}[W$\textit{ü}$rfel]&=    &\frac{2 \cdot ((b-a)^2)+b-a}{6\cdot(b - a + 1)} - \frac{(b-a)^2}{4}&/b=6;~a=1\\
            &=    &\frac{2 \cdot ((6-1)^2)+6-1}{6\cdot(6 - 1 + 1)} - \frac{(6-1)^2}{4}&\\
            &=    &-\frac{85}{18}&\\
            &=    &-4,7\overline{2}&\\
    \end{array}
\end{equation*}
Dieses Ergebnis stimmt mit dem aus der Vorlesung nicht überein.

\sss{d)}

\begin{equation*}
    \begin{array}{rlll}
        \mathbb{E}[X_k]&=    &\sum_{k \in \Omega'}(k\cdot P(X_k = k))&\\
            &=    &\sum_{k=1}^{6}(\sum_{i=k}^{k+6}(i) \cdot \frac{1}{6})&\\
    \end{array}
\end{equation*}
In der Formel wurde $k$ durch $\sum_{i=k}^{k+6}(i)$ ersetzt, da es nicht um die Nummer des Würfels, sondern um die Zahl, die er zeigt, geht.
\\
Die Ergebnisse sind: $\mathbb{E}[X_1]=3.5$, $\mathbb{E}[X_2]=4.5$, $\mathbb{E}[X_3]=5.5$, $\mathbb{E}[X_4]=6.5$, $\mathbb{E}[X_5]=7.5$ und $\mathbb{E}[X_6]=8.5$.
\begin{equation*}
    \begin{array}{rlll}
        \mathbb{E}[Z]&=    &\sum_{k = 1}^{6}(\mathbb{E}[X_k]))&\\
    \end{array}
\end{equation*}
Das Ergebnis lautet: $\mathbb{E}[Z] = 36$.

\susP{7.3}{8}
(Noch einmal der Zonk, 4+4 Punkte).
In einer Gameshow wählt ein Kandidat zwischen den drei Toren 1, 2 und 3 aus, hinter zweien davon ist der Zonk (Trostpreis), hinter einem der Hauptgewinn.
Das zugehörige Tor wird zu Beginn zufällig gleichverteilt ausgewählt.
Der Kandidat wählt ein Tor, anschließend öffnet der Showmaster ein Tor folgenden Regeln:
\begin{itemize}
	\item Das geöffnete Tor ist nicht das vom Kandidaten gewählte Tor.
	\item Hinter dem geöffneten Tor ist ein Zonk.
	\item Hat der Showmaster die Wahl zwischen mehreren Toren, so wählt er das Tor mit der größeren Nummer.
\end{itemize}
Anschließend erhält der Kandidat immer die Möglichkeit, das Tor zu wechseln.
\begin{enumerate}
	\item[a)] Ein Kandidat wählt zu Beginn Tor 1, der Showmaster öffnet Tor 3.
		Sollte der Kandidat nun zu Tor 2 wechseln?
	\item[b)] Ein anderer Kandidat wählt zu Beginn ebenfalls Tor 1, der Showmaster öffnet Tor 2.
		Sollte dieser Kandidat nun zu Tor 3 wechseln?
\end{enumerate}
Geben Sie für beide Kandidaten an, wie hoch die Gewinnwahrscheinlichkeit nach einer Umentscheidung in Folge des geöffneten Tors ist.

\sss{a)}
Nach Hinweis 3 ist die Chance des Kandidaten auf einen Gewinn vor dem Wechseln 50\%.
Nach dem Wechseln ändert sich diese Chance nicht.
\[
	\begin{array}{lll}
		P (A_1 | {A_3}^c)	&=	\frac{P(A_1)}{P(A_1\cup A_2)}	&\\
									&=	\frac{\ilfrac{1}{3}}{\ilfrac{2}{3}}	&= \frac{1}{2}
	\end{array}
\]
Die Gewinnwahrscheinlichkeit für die Tore 1 und 2 ist gleich, nämlich 50\%.
Somit ist es egal, ob er wechselt.
\sss{b)}
Nach Hinweis 3 ist dies genau das, was er tun sollte.
Denn demnach müsste der Hauptgewinn hinter Tor 3 liegen.
Nach den Hinweisen 2 und 3 würde der Moderator Tor 3 öffnen, sofern dahinter ein Zonk läge.
Da er das nicht tut, liegt folglich kein Zonk dahinter.
Somit ist die Gewinnwahrscheinlichkeit bei einem Wechsel bei 100\%.
\fi
\newpage
\chead{}
\rhead{\memTwoName\\\memTwoNr}
\ifnum\ZettelAcht=\True
\sect{8}{31. Mai 2016}{7. Juni 2016}
\susP{8.1}{7}
(Bedingte Wahrscheinlichkeiten, 4+3 Punkte).
Bei der digitalen Datenübertragung können aufgrund verrauschter Kanäle auf der physikalischen Ebene Fehler auftreten, die sich dann als Verfälschung übermittelter Datenpakete auswirken.
Das neu entwickelte Verfahren DEPP (Detection of Error-in-Packet Probabilities), ein Verfahren zur Erkennung von
Datenübertragungsfehlern, zeigt 99\% aller auftretenden Übertragungsfehler richtigerweise als Fehler an.
Leider weist das Verfahren fälschlicherweise 0.1\% aller korrekten Datenübertragungen als fehlerhaft aus.
Das Verfahren wird nun zur Prüfung eines Nachrichtenübertragungskanals eingesetzt, bei dem 1\% Übertragungsfehler auftreten. 
\begin{enumerate}
	\item[a)] Wie groß ist die Wahrscheinlichkeit, dass das Verfahren bei einer Datenübertragung einen Fehler anzeigt?
	\item[b)] Wie groß ist die Wahrscheinlichkeit, dass eine Datenübertragung tatsächlich fehlerhaft ist, falls das Verfahren einen Fehler anzeigt?
\end{enumerate}
\vspace{.3cm}\-\\
A := ``Es liegt ein Fehler vor.''\\
B := ``Das System erkennt einen Fehler.''
\paragraph{a)}
Mit einer Wahrscheinlichkeit von 1\% tritt ein Fehler auf, welcher mit einer Wahrscheinlichkeit von 99\% erkannt wird $ \Rightarrow0.01\cdot 0.99$\\
Mit einer Wahrscheinlichkeit von 99\% tritt kein Fehler auf, welcher mit einer Wahrscheinlichkeit von 0.1\% trotzdem als Fehler erkannt wird $\Rightarrow0.99\cdot 0.001$
\[
	P(B)=0.99\cdot0.01+0.99\cdot0.001\approx0.01089\hat{\approx}1.1\%
\]
\paragraph{b)}
Formel von Bayes:
\[
	P(A|B)=\frac{P(B|A)\cdot P(A)}{P(B)}
\]
$P(B|A)$ liegt bei 99\% (vlg. Aufgabenstellung).\\
$P(A)$ ist 1\% (vlg. Aufgabenstellung).\\
$P(B)$ wurde in a) ausgerechnet.
\[
	P(A|B)=\frac{0.99\cdot0.01}{0.01089}=\frac{10}{11}\hat{\approx}91\%
\]
\susP{8.2}{5}
(Stochastische Unabhängigkeit, 2+3 Punkte).
Im Folgenden sind jeweils zwei Ereignisse $A,B$ gegeben.
Überprüfen Sie auf stochastische Unabhängigkeit.
\begin{enumerate}
	\item[a)] Es wird ein fairer Würfel geworfen.
	$A$ sei das Ereignis, dass die Augenzahl gerade ist, $B$ sei das Ereignis, dass die Augenzahl durch drei teilbar ist.
	\item[b)] Es werden zwei faire Würfel geworfen.
	$A$ sei das Ereignis, dass die Augensumme 6 ist, $B$ sei das Ereignis, dass mindestens ein Würfel eine 3 zeigt.
\end{enumerate}
\vspace{.3cm}\-\\
\paragraph{a)} $P(A|B)$ ist die Wk. dafür, dass die Augenzahl sowohl gerade, als auch durch drei Teilbar ist.
Dafür gibt es bei einem Würfel nur einen Fall, nämlich gerade die Augenzahl $6=2\cdot3$.
Da der Würfel fair ist, gilt $P(A\cap B)=P(X=6)=\frac{1}{6}$.
\[
	P(A|B)=\frac{P(A\cap B)}{P(B)}=\frac{\frac{1}{6}}{\frac{1}{3}}=\frac{3}{6}=\frac{1}{2}=P(A)
\]
Definition von stoch. Unabhängigkeit:
\[
	P(A|B)=P(A) \Leftrightarrow P(A \cap B) = P(A) \cdot P(B)
\]
\[
	P(A) \cdot P(B) = \frac{1}{2} \cdot \frac{1}{3} = \frac{1}{6} \overset{nach~oben}{=}P(A\cap B)
\]
Damit ist die Definition erfüllt, $A$ und $B$ sind also stoch. unabhängig.
\paragraph{b)}
$P(B)=\frac{11}{36}, P(A)=\frac{5}{36}$.
\[
	P(B|A)=\frac{1}{5}\text{, da es 5 Fälle für A gibt, von denen jedoch nur einer für B günstig ist.}
\]
\[
	P(A|B)=\frac{1}{11}\text{, da es 11 Fälle für B gibt, von denen jedoch nur einer für A günstig ist.}
\]
Da $P(A|B) \neq P(A)$, sind sie nach Definition (vgl. oben) nicht stochastisch unabhängig.

\susP{8.3}{8}
(Smileys revisited, 3+3+2 Punkte).
In einer früheren Aufgabe wurde folgendes Glücksspiel beschrieben:\\
\vspace{.1cm}\\
Ihnen stehen geometrische Objekte zur Verfügung, die mit Wahrscheinlichkeit $\frac{1}{2},\frac{1}{3},\frac{1}{4},\frac{1}{5},\frac{1}{6}\dots$ jeweils einen Smiley anzeigen.
Der Ablauf des Spiels sieht wie folgt aus: In der ersten Runde wird das Objekt mit Smiley-Wahrscheinlichkeit $\frac{1}{2}$ gewürfelt, in der zweiten Runde das mit Smiley-Wahrscheinlichkeit $\frac{1}{3}$, \dots, allgemein wird in der $n$-ten Runde das Objekt mit Smiley-Wahrscheinlichkeit $\frac{1}{n+1}$ gewürfelt.
Das Spiel endet, sobald zum ersten Mal ein Smiley erscheint.
Sie bezeichnen mit $X$ die Runde, in der zum ersten Mal ein Smiley gewürfelt wird.\\
\vspace{.1cm}\\
Anschließend wurde ohne Beweis $P(X=k)=\frac{1}{k(k+1)}$ behauptet.
Dies soll hier nun begründet werden.
\begin{enumerate}
	\item[a)] Begründen Sie zunächst anhand der Beschreibung
	\[
		P(X=k | X > k-1)=\frac{1}{k+1}
	\]
	und folgern Sie daraus
	\[
		P(X > k | X > k-1) = \frac{k}{k+1}
	\]
	\item[b)] Zeigen Sie $P(X>k)=\frac{1}{k+1}$ für alle $k \in \mathbb{N}$ per Induktion.
	\item[c)] Folgern Sie schließlich $P(X=k)=\frac{1}{k(k+1)}$ für alle $k \in \mathbb{N}$.
\end{enumerate}
\vspace{.3cm}\-\\
\paragraph{a)}
$X>k-1$ bedeutet, dass wir mindestens $k$ Runden brauchen.
Die Wahrscheinlichkeit, in der $k$-ten Runde zu gewinnen, beträgt $\frac{1}{k+1}$.\\
Somit schließt 
\[
	\Rightarrow P(X=k | X > k-1)=(X=k)=\frac{1}{k+1}
\]
\\
$P(X>k)$ meint die Wahrscheinlichkeit, in einer späteren als der $k$-ten Runde zu gewinnen.
\[
	\sum_{i=k + 1}^{\infty} \left(\frac{1}{k+i}\right) = \sum_{i=1}^{\infty}\left(\frac{1}{2k+1}\right)
\]
Aus der Divergenz der harmonischen Reihe folgt:
\[
	\sum_{k=1}^{\infty} \left(\frac{1}{2k+1}\right)=\infty
\]

\[
	\Rightarrow P(X > k | X > k-1) = \frac{k}{k+1}
\]
\susP{8.4}{5}
(Minimum und Maximum, 5 Punkte).
Betrachten Sie einmal mehr den gleichzeitigen Wurf zweier fairer Würfel.
$X$ sei die minimale Augenzahl, $Y$ die maximale Augenzahl.
Charakterisieren Sie die gemeinsame Verteilung von $(X,Y)$ durch Ausfüllen der Tabelle. \\
\\
Annahme: Tupel (x,y) $\lor$ (y,x)
\[
	\begin{array}{|c|cccccc|}\hline
	Y \setminus X	&	1	&	2	&	3	&	4	&	5	&	6	\\ \hline
	1&\ilfrac{1}{36}&0&0&0&0&0\\
	2&\ilfrac{2}{36}&\ilfrac{1}{36}&0&0&0&0\\
	3&\ilfrac{2}{36}&\ilfrac{2}{36}&\ilfrac{1}{36}&0&0&0\\
	4&\ilfrac{2}{36}&\ilfrac{2}{36}&\ilfrac{2}{36}&\ilfrac{1}{36}&0&0\\
	5&\ilfrac{2}{36}&\ilfrac{2}{36}&\ilfrac{2}{36}&\ilfrac{2}{36}&\ilfrac{1}{36}&0\\
	6&\ilfrac{2}{36}&\ilfrac{2}{36}&\ilfrac{2}{36}&\ilfrac{2}{36}&\ilfrac{2}{36}&\ilfrac{1}{36}\\\hline
	\end{array}
\]
Annahme: Tupel (i,j); x <= i,j <= y
\[
	\begin{array}{|c|cccccc|}\hline
	Y \setminus X	&	1	&	2	&	3	&	4	&	5	&	6	\\ \hline
	1&\ilfrac{1}{36}&0&0&0&0&0\\
	2&\ilfrac{4}{36}&\ilfrac{1}{36}&0&0&0&0\\
	3&\ilfrac{9}{36}&\ilfrac{4}{36}&\ilfrac{1}{36}&0&0&0\\
	4&\ilfrac{16}{36}&\ilfrac{9}{36}&\ilfrac{4}{36}&\ilfrac{1}{36}&0&0\\
	5&\ilfrac{25}{36}&\ilfrac{16}{36}&\ilfrac{9}{36}&\ilfrac{4}{36}&\ilfrac{1}{36}&0\\
	6&\ilfrac{36}{36}&\ilfrac{25}{36}&\ilfrac{16}{36}&\ilfrac{9}{36}&\ilfrac{4}{36}&\ilfrac{1}{36}\\\hline
	\end{array}
\]
\vspace{.3cm}\\
\fi
\ifnum\ZettelNeun=\True
\sect{9}{7. Juni 2016}{14. Juni 2016}
\susP{9.1}{9}
(Berechnung Kovarianz, 3+3+3 Punkte).
Bestimmen Sie in folgenden Situationen \cov{X}{Y}.\\
\begin{minipage}[t]{0.48\textwidth}
\begin{enumerate}
	\item[a)]
		\begin{tabular}{|cc|ccc|c|}\hline
				&		&	&$X$			&				&\\
				&		&0	&1				&2				&\\\hline
				%%%%%%%%%%%%%%%%%%%%%%%%%%%%%%%%%%%%%%%%%%%%%%%%
				&0	{}	&0				&$\frac{1}{4}$	&0				&\\
			$Y$	&1	{}	&$\frac{1}{4}$	&0				&$\frac{1}{4}$	&\\
				&2	{}	&0				&$\frac{1}{4}$	&0				&\\\hline
				&	{}	&				&				&				&\\\hline
		\end{tabular}
	\item[b)]
		\begin{tabular}{|cc|ccc|c|}\hline
				&		&	&$X$			&				&\\
				&		&0	&1				&2				&\\\hline
				%%%%%%%%%%%%%%%%%%%%%%%%%%%%%%%%%%%%%%%%%%%%%%%%
				&0	{}	&$\frac{1}{3}$	&0				&0				&\\
			$Y$	&1	{}	&0				&$\frac{1}{3}$	&0				&\\
				&2	{}	&0				&0				&$\frac{1}{3}$	&\\\hline
				&	{}	&				&				&				&\\\hline
		\end{tabular}
	\item[c)]
		\begin{tabular}{|cc|ccc|c|}\hline
				&		&	&$X$			&				&\\
				&		&0	&1				&2				&\\\hline
				%%%%%%%%%%%%%%%%%%%%%%%%%%%%%%%%%%%%%%%%%%%%%%%%
				&0	{}	&0				&0				&$\frac{1}{3}$	&\\
			$Y$	&1	{}	&0				&$\frac{1}{3}$	&0				&\\
				&2	{}	&$\frac{1}{3}$	&0				&0				&\\\hline
				&	{}	&				&				&				&\\\hline
		\end{tabular}
\end{enumerate}
\end{minipage} ~~
\begin{minipage}[t]{0.48\textwidth}
\begin{enumerate}
	\item[a)]
		\begin{tabular}{|cc|ccc|c|}\hline
				&		&	&$X$			&				&\\
				&		&0	&1				&2				&\\\hline
				%%%%%%%%%%%%%%%%%%%%%%%%%%%%%%%%%%%%%%%%%%%%%%%%
				&0	{}	&0						&$\frac{1}{4}$		&0					&$\ilfrac{1}{4}$\\
			$Y$	&1	{}	&$\frac{1}{4}$			&0					&$\frac{1}{4}$		&$\ilfrac{2}{4}$\\
				&2	{}	&0						&$\frac{1}{4}$		&0					&$\ilfrac{1}{4}$\\\hline
				&	{}	&$\frac{1}{4}$			&$\ilfrac{2}{4}$	&$\ilfrac{1}{4}$	&1\\\hline
		\end{tabular}
	\item[b)]
		\begin{tabular}{|cc|ccc|c|}\hline
				&		&	&$X$			&				&\\
				&		&0	&1				&2				&\\\hline
				%%%%%%%%%%%%%%%%%%%%%%%%%%%%%%%%%%%%%%%%%%%%%%%%
				&0	{}	&$\frac{1}{3}$		&0					&0					&$\ilfrac{1}{3}$\\
			$Y$	&1	{}	&0					&$\frac{1}{3}$		&0					&$\ilfrac{1}{3}$\\
				&2	{}	&0					&0					&$\frac{1}{3}$		&$\ilfrac{1}{3}$\\\hline
				&	{}	&$\ilfrac{1}{3}$	&$\ilfrac{1}{3}$	&$\ilfrac{1}{3}$	&1\\\hline
		\end{tabular}
	\item[c)]
		\begin{tabular}{|cc|ccc|c|}\hline
				&		&	&$X$			&				&\\
				&		&0	&1				&2				&\\\hline
				%%%%%%%%%%%%%%%%%%%%%%%%%%%%%%%%%%%%%%%%%%%%%%%%
				&0	{}	&0					&0					&$\frac{1}{3}$		&$\ilfrac{1}{3}$\\
			$Y$	&1	{}	&0					&$\frac{1}{3}$		&0					&$\ilfrac{1}{3}$\\
				&2	{}	&$\frac{1}{3}$		&0					&0					&$\ilfrac{1}{3}$\\\hline
				&	{}	&$\ilfrac{1}{3}$	&$\ilfrac{1}{3}$	&$\ilfrac{1}{3}$	&1\\\hline
		\end{tabular}
\end{enumerate}
\end{minipage}\\
\vspace{.3cm}\\
\paragraph{a)}
\cov{X}{Y}=\e{XY}-\e{X}\e{Y}.
\begin{align*}
	\Rightarrow \textbf{COV}[X,Y]	&= 0 \cdot P(XY=0) + 1 \cdot P(XY=1) + 2 \cdot P(XY=2) + 4 \cdot P(XY=4) \\
									&-\left[ \left(0 \cdot P(X=0) + 1 \cdot P(X=1) + 2 \cdot P(X=2)\right) \cdot \left(0 \cdot P(Y=0) + 1 \cdot P(Y=1) + 2 \cdot P(Y=2)\right) \right]\\
									&= 0\cdot0 + 1 \cdot 0 + 2 \cdot \frac{2}{4} + 4 \cdot 0 - \left[ \left(0 \cdot \frac{1}{4} + 1 \cdot \frac{2}{4} + 2 \cdot \frac{1}{4} \right) \cdot \left( 0 \cdot \frac{1}{4} + 1 \cdot \frac{2}{4} + 2 \cdot \frac{1}{4} \right) \right]\\
									&= \frac{4}{4} - \left[\frac{4}{4} \cdot \frac{4}{4} \right]\\
									&= 1 - 1 \cdot 1\\
									&= 0
\end{align*}
\paragraph{b)}
\begin{align*}
	\textbf{COV}[X,Y]	&=	0+\frac{1}{3}+0+\frac{4}{3} - \left[ 3 \cdot \frac{1}{3} \cdot 3 \cdot \frac{1}{3} \right]\\
						&= \frac{5}{3} - 1\\
						&= \frac{2}{3}
\end{align*}
\paragraph{c)}
\begin{align*}
	\textbf{COV}[X,Y]	&=	0+\frac{1}{3}+0+0-\left[ 3 \cdot \frac{1}{3} \cdot 3 \cdot \frac{1}{3} \right]\\
						&=	\frac{1}{3} - 1\\
						&=	-\frac{2}{3}
\end{align*}
\susP{9.2}{4}
(Vorzeichen Kovarianz, 2+2 Punkte).
Der Vektor $(X,Y)$ sei auf der im jeweiligen Aufgabenteil dargestellten Punktemenge Laplace-Verteilt.
Geben Sie - ohne Rechnung - eine begründete Vermutung für das Vorzeichen von \cov{X}{Y} bzw. das Vorzeichen des Korrelationskoeffizienten ab.\\
%\begin{minipage}[t]{0.48\textwidth}
%	\begin{enumerate}
%		\item[a)]
%			\begin{tikzpicture}
%				\draw
%						(0,0) coordinate (origin)
%						(5.3,0) coordinate (x-limit)
%						(0,5.3) coordinate (y-limit)
%						
%						(0.55,0.5)	coordinate (a) node {•}
%						(0.55,1.05)	coordinate (b) node {•}
%						(1.05,1.05)	coordinate (c) node {•}
%					;
%				\draw (origin) -> (x-limit);
%				\draw (origin) -> (y-limit);
%			\end{tikzpicture}
%	\end{enumerate}
%\end{minipage}
%~~
%\begin{minipage}[t]{0.48\textwidth}
%	\begin{enumerate}
%		\item[b)] .
%	\end{enumerate}
%\end{minipage}\\
\vspace{.3cm}\\
\textit{Anmerkung:} Diesmal aus naheliegenden Gründen ohne Skizze. Kommt vielleicht noch, wenn ich vor Abgabedatum Lust darauf hab, das Ding zu malen...\\
\paragraph{a)}Da, wenn $X$ steigt, $Y$ auch steigt, ist \cov{X}{Y} positiv.
\paragraph{b)}Da, wenn $X$ steigt, $Y$ fällt, ist \cov{X}{Y} negativ.
\susP{9.3}{6}
(Die Summe Poisson-verteilter Zufallsvariablen, 6 Punkte).
Es seien $X \textasciitilde Pois_{\lambda}$ und $Y \textasciitilde Pois_{\mu}$ unabhängig.
Zeigen Sie, dass auch $X+Y$ Poisson-verteilt ist und bestimmen Sie den Parameter.\\
\\
\textit{Hinweis:} Der binomische Satz besagt
\[
	\sum_{k=0}^{n}\binom{n}{k}x^ky^{n-k}=(x+y)^n
\]
für alle $n \in \mathbb{N}_0$ sowie alle $x,y \in \mathbb{R}$.\\
\vspace{.3cm}\\
$Z = X + Y$.\\
\[\begin{array}{lll}
	P(Z=k)	&=	P(X+Y=k)																										\\
			&=	\sum_{i=0}^{k}P(Z=k|Y=i) \cdot P(Y=i)																			\\
			&=	\sum_{i=0}^{k}P(X=k-i) \cdot P(Y=i)																			\\
			&=	\sum_{i=0}^{k}e^{-\lambda} \cdot \frac{\lambda^{k-i}}{(k-i)!} \cdot e^{-\mu} \cdot \frac{\mu^i}{i!}			&|~\text{Summe erweitern mit }\frac{k!}{k!}\\
			&=	\sum_{i=0}^{k}e^{-(\lambda+\mu)} \cdot \frac{k! \cdot \lambda^{k-i} \cdot \mu^i}{k! \cdot (k-i)! \cdot i!}	&|~\text{Binomialkoeffizenten ausklammern}\\
			&=	\sum_{i=0}^{k}e^{-(\lambda+\mu)} \cdot \binom{k}{i} \cdot \frac{\lambda^{k-i}\cdot\mu^i}{k!}					&|~\frac{1}{k!}\cdot e^{-(\lambda+\mu)}\text{ aus Summe rausziehen}\\
			&=	\frac{1}{k!} \cdot e^{-(\lambda+\mu)} \cdot \sum_{i=0}^{k} \binom{k}{i} \cdot \lambda^{k-i}\cdot\mu^i		&|~\text{Hinweis einsetzen}\\
			&=	\frac{1}{k!} \cdot e^{-(\lambda+\mu)} \cdot (\lambda+\mu)^k													\\
			&=	\frac{(\lambda+\mu)^k}{k!} \cdot e^{-(\lambda+\mu)}															&\qed
\end{array}\]


\susP{9.4}{6}
(Das Simpson-Paradox, 4+2 Punkte).
Eine kleine Universität bietet nur zwei Studiengänge (A und B) an.
Aus Erfahrung ist bekannt, dass sich 80\% aller Frauen für Studiengang A interessieren, aber nur 30\% aller Männer.
Ebenfalls aus Erfahrung ist bekannt, dass die Erfolgsquote von Bewerbungen von Frauen bei Studiengang A bei 30\%, die von Männern bei 20\% liegt, bei Studiengang B werden sowohl Frauen als auch Männer mit einer Wahrscheinlichkeit von 70\% akzeptiert.
\begin{enumerate}
	\item[a)] Bestimmen Sie, welcher Anteil aller sich bewerbenden Frauen einen Studienplatz erhält, und bestimmen Sie, welcher Anteil aller sich bewerbenden Männer einen Studienplatz erhält.
	\item[b)] Wenn Sie sich nciht verrechnet haben, werden Sie festgestellt haben, dass Frauen eine niedrigere Erfolgsquote bei der Bewerbung haben als Männer, obwohl sie bei jedem einzelnen Studiengang mindestens die gleiche Erfolgsquote haben.
	Woran liegt das?
\end{enumerate}
\textit{Anmerkung:} Tatsächlich führte ähnliches Datenmaterial (mit mer als zwei Studiengängen) wegen ausschließlicher berachtung der totalen Erfolgsquoten bereits zu Diskriminierungsklagen (z.B. an der Universität Berkeley).
\paragraph{a)}
\subparagraph{Annahmen.}
Unsere überprüfte Gruppe wird als 100\% angesehen.\\
Jeder aus der Gruppe interessiert sich für genau 1 Studiengang.
\subparagraph{Lösung.}
Anteil der Frauen, die einen Studienplatz erhalten: $80\% \cdot 30\% + (100\%-80\%) \cdot 70\% = 38\%$.\\
Anteil der Männer, die einen Studienplatz erhalten: $30\% \cdot 20\% + (100\%-30\%) \cdot 70\% = 55\%$.
\paragraph{b)}
Angebot und Nachfrage regieren hier.
Die Nachfrage der Frauen ist deutlich größer als das Angebot der Universität, womit grundsätzlich die meisten nicht ihren Wunschstudiengang erhalten.

Dem entgegen bewerben sich die meisten Männer bei einem Studiengang, dessen Angebot größer ist, wodurch insgesamt mehr angenommen werden.
\fi
\ifnum\ZettelZehn=\True
\sect{10}{14. Juni 2016}{21. Juni 2016}
\fi
\ifnum\ZettelElf=\True
\sect{11}{21. Juni 2016}{28. Juni 2016}
\fi
\ifnum\ZettelZwoelf=\True
\sect{12}{28. Juni 2016}{5. Juli 2016}
\fi

\end{document}