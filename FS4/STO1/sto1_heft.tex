\documentclass[twoside]{article}
\usepackage[utf8]{inputenc}
\usepackage[T1]{fontenc}
\usepackage[ngerman]{babel}
\usepackage[margin=2.5cm]{geometry}

% import math packages
\usepackage{amsmath}
\usepackage{amsfonts}
\usepackage{amssymb}
\usepackage{amsthm}
% contradiction lightning
\usepackage{stmaryrd}
% algorithms and pseudo code
\usepackage{algorithmic}
\usepackage{algorithm}
%\usepackage{clrscode3e}
% formatting and layout
\usepackage{color}
\usepackage{fancyhdr}
% references and links
\usepackage{hyperref}
\usepackage{lastpage}
% settings
\usepackage{perpage}
\MakePerPage{footnote}
% custom text colors
\definecolor{pblue}{rgb}{0.13,0.13,1}
\definecolor{pgreen}{rgb}{0,0.5,0}
% macro commands
%% requires color package and the custom colors defined here 
\newcommand{\todo}[1]{
	\addcontentsline{toc}{subsubsection}{TODO: #1}
	\textcolor{pgreen}{\texttt{\-\\ \-\\//TODO: #1\-\\ \-\\}}
}
\newcommand{\sect}[3]{	%	custom section (level 1)
	\newpage
	\addcontentsline{toc}{section}{Zettel #1 (#2)}
	\section*{Zettel Nr. #1 (Ausgabe: #2, Abgabe: #3)}
	\label{sec:#1}
	\fancyfoot[LO,RE]{Abgabe: #3}
}
\newcommand{\sus}[1]{
	\addcontentsline{toc}{subsection}{Hausübung #1}
	\subsection*{Hausübung #1}
	\label{ssec:#1}
}
\newcommand{\susP}[2]{
	\sus{#1}
	\points{#2}
}
\newcommand{\sss}[1]{
	\addcontentsline{toc}{subsubsection}{Teilaufgabe #1}
	\subsubsection*{Teilaufgabe #1}
	\label{sssec:#1}
}
\newcommand{\points}[1]{
	\begin{flushright}
		\begin{Large}
			[~~~~\string| ~#1~]
		\end{Large}
	\end{flushright}
}
%% custom short command for blue text
\newcommand{\tcb}[1]{
	\textcolor{blue}{#1}
}
%% custom short command for green text
\newcommand{\tcg}[1]{
	\textcolor{green}{#1}
}
% tikz
\usepackage{tikz}
\usetikzlibrary{%
	arrows,			% arrow types
	automata,		% states and transitions
	positioning,	% position shifting
	shapes,			% circles and stuff for Venn diagrams
	backgrounds		% coloring and stuff
}
\usepackage{pgf}

%#+-------------------------------------------------+#%
%#+						VARIABLEN			     	+#%
%#+-------------------------------------------------+#%

\begin{document}

%% Fach-Daten
\newcommand{\fachname}{Stochastik 1 für Studierende der Informatik}
\newcommand{\fachnummer}{MATH3-Inf}
\newcommand{\veranstaltungsnummer}{65-832}
\newcommand{\stinegruppe}{Übungsgruppe 2}
\newcommand{\termin}{Dienstag, 14.15 - 15.00\\Geom 431}
%% Gruppenmitglied 1
\newcommand{\memOneName}{Utz Pöhlmann}
\newcommand{\memOneMail}{4poehlma@informatik.uni-hamburg.de}
\newcommand{\memOneNr}{6663579}
%% Gruppenmitglied 2
\newcommand{\memTwoName}{Louis Kobras}
\newcommand{\memTwoMail}{4kobras@informatik.uni-hamburg.de}
\newcommand{\memTwoNr}{6658699}
%% Gruppenmitglied 3
\newcommand{\memThreeName}{Felix Gebauer}
\newcommand{\memThreeMail}{4gebauer@informatik.uni-hamburg.de}
\newcommand{\memThreeNr}{6671660}
%% Gruppenmitglied 4
\newcommand{\memFourName}{}
\newcommand{\memFourMail}{}
\newcommand{\memFourNr}{}
%% Datum
\newcommand{\datum}{\today\\}
%% leere Seite
\newcommand{\leereSeite}{
	\newpage
	\thispagestyle{empty}
	\mbox{}
	\newpage
}






%#+-------------------------------------------------+#%
%#+						FORMATIERUNG		     	+#%
%#+-------------------------------------------------+#%

\newcommand{\fach}{
	\begin{Huge}
		\fachname\\
	\end{Huge}
	\begin{LARGE}
		Modul: \fachnummer\\
		Veranstaltung: \veranstaltungsnummer\\
	\end{LARGE}
}

\newcommand{\gruppe}{
	\begin{LARGE}
		\stinegruppe\\
	\end{LARGE}
	\begin{Large}
		\termin\\
	\end{Large}
}

\newcommand{\memberOfGroup}[3]{
	\begin{center}
		\begin{Large}
			#1
		\end{Large}\-\\
		#2\-\\
		#3\-\\
	\end{center}
	\vspace{.5cm}
}
\newcommand{\datumf}{
	\begin{Large}
		\datum\-\\
	\end{Large}
}

% setting up footers and headers
%% setting page style so that footers and headers can be used
\pagestyle{fancy}
%% overwrite default values
\fancyhead{}
\fancyfoot{}
%% first group member in upper left corner
\lhead{\memOneName\-\\\memOneNr}
%% second group member in the upper center
\chead{\memTwoName\-\\\memTwoNr}
%% third group member in upper right corner
\rhead{\memThreeName\-\\\memThreeNr}
%% lower left corner (empty)
\lfoot{}
%% subject name in lower center
\cfoot{\fachname}
%% lower right corner (empty)
\rfoot{}
%% sets the page number to appear in either the lower left or lower right corner,
%% depending on whether the page number is even or odd, in the format "page X of Y"
\fancyfoot[LE,RO]{Seite \thepage ~von \pageref{LastPage}}
%% thin separation line over the footer
\renewcommand{\footrulewidth}{0.4pt}



%#+-------------------------------------------------+#%
%#+						DECKBLATT  			     	+#%
%#+-------------------------------------------------+#%

\thispagestyle{empty}
\-\vspace{0.5cm}
\begin{center}
	\fach
	\vspace{1.5cm}
	\gruppe
	\vspace{1.5cm}
	% group members
	\memberOfGroup{\memOneName}{\memOneMail}{\memOneNr}
	\memberOfGroup{\memTwoName}{\memTwoMail}{\memTwoNr}
	\memberOfGroup{\memThreeName}{\memThreeMail}{\memThreeNr}
	\memberOfGroup{\memFourName}{\memFourMail}{\memFourNr}
	% 1 cm to next element
	\vspace{1cm}
	\datumf
	\vspace{1cm}
	
	\textbf{Punkte für die Hausübungen:}\\
	\vspace{1cm}
	\begin{tabular}{c|c|c|c}
	~1.1~&~1.2~&~1.3~&~$\Sigma$~	\\	\hline
		 &	   &	 &
	\end{tabular}
	
\end{center}
\newpage






























%#+-------------------------------------------------+#%
%#+						INHALT  			     	+#%
%#+-------------------------------------------------+#%
\leereSeite
\thispagestyle{empty}
\tableofcontents
\newpage
\pagenumbering{arabic}

%#+-------------------------------------------------+#%
%#+						ZETTEL 1					+#%
%#+-------------------------------------------------+#%
\sect{1}{05. April 2016}{12. April 2016}
%%%%%%%%%%%%%%%%%%%%%%%%%%%%%
%
%%%% Aufgabe 1
%
%%%%%%%%%%%%%%%%%%%%%%%%%%%%%
\susP{1.1}{8}
(Zufallsexperimente, 2+2+2+2 Punkte).
Handelt es sich in den folgenden Situationen um Zufallsexperimente?
Begründen Sie ihre Antwort.
\begin{enumerate}
	%%%%% Aufgabe 1a ---------------
	\item[a)] Sonnenaktivität am 07.04.2016 um 10:00.\\
	%%%%% Bearbeitung 1a ----------
	\textbf{Kein Zufallsexperiment}, da die Situation nicht rekonstruierbar ist.
	
	%%%%% Aufgabe 1b ---------------
	\item[b)] Verkehrssituation am Schlump Donnerstags 10:00.\\
	%%%%% Bearbeitung 1b ----------
	\textbf{Zufallsexperiment}, da die Situation rekonstruiert und dementsprechend das Experiment beliebig oft wiederholt werden kann.
	
	%%%%% Aufgabe 1c ---------------
	\item[c)] Gleichzeitiger Wurf von drei fairen Würfeln, Beobachtung der Augensumme.\\
	%%%%% Bearbeitung 1c ----------
	\textbf{Zufallsexperiment}, da die Situation rekonstruiert und dementsprechend das Experiment beliebig oft wiederholt werden kann.
	
	%%%%% Aufgabe 1d ---------------
	\item[d)] Lebenszeit römischer Kaiser nach Inthronisierung.\\
	%%%%% Bearbeitung 1d ----------
	\textbf{Kein Zufallsexperiment}, da die Umstände für jeden Kaiser eindeutig sind und dementsprechend die Situation nicht rekonstruierbar ist.
	
	
\end{enumerate}


%%%%%%%%%%%%%%%%%%%%%%%%%%%%%
%
%%%% Aufgabe 2
%
%%%%%%%%%%%%%%%%%%%%%%%%%%%%%
\susP{1.2}{12}
(Zufall in der Praxis, 6+6 Punkte).
Stellen Sie in den beiden folgenden Situationen dar, an welchen Stellen sich Zufallseinflüsse auswirken.
Geben Sie darüber hinaus kurz an, welche Ziele (ggf. aus Sicht der unterschiedlichen Parteien) erreicht werden sollen.
\begin{enumerate}
	%%%%% Aufgabe 2a ---------------
	\item[a)] Ein Flughafen hat eine Landebahn, die Flugzeuge müssen beim Landen einen gewissen zeitlichen Abstand halten (vorangegangenes Flugzeug muss die Landebahn verlassen und sich weit genug entfernt haben, Wirbelschleppen müssen \dq verflogen\dq  sein, \dots).
	Nach Flugplan kommen die Flugzeuge gleichmäßig an, der Abstand zwischen zwei Ankünften ist im Mittel etwas größer als der notwendige zeitliche Abstand zwischen zwei Landungen.
	Ist eine Landung für ein Flugzeug noch nicht möglich, da die Landebahn noch nicht wieder freigegeben ist, muss das Flugzeug eine Warteschleife fliegen.\\
	\vspace{.5cm}\\
	%%%%% Bearbeitung 2a ----------
	Das primäre Zufallsereigniss ist die Verspätung von Fliegern\footnote{siehe analog dazu Beispiel 1.3.1 im Skript, Version 05.04.2016}.
	Zufallsereignisse können sich insofern auf den Flughafen auswirken, dass Leute verspätet in ihren Flieger steigen können, ankommende Passagiere ihr Gepäck nicht zeitnah erhalten, da sich das Gepäckstück in einem anderen, verzufallten Flugzeug befinden kann, oder Passagiere ihren Anschluss verpassen.
	Eine zufällige Bombendrohung kann zu einer kompletten Evakuierung des Flughafens verleiten, wodurch der Betrieb vollständig zum Stillstand kommt.
	Zufällig kann es sein, dass zwei anfliegende Flugzeuge nicht richtig koordiniert werden und es zu einer (beinahe-)Kollision kommt.
	Der Gegenfall ist, dass ein Flugzeug zufällig fälschlicherweise in die Warteschlange eingereiht wird.
	
	Die Ziele der Passagiere sind, dass sie ihr Gepäck erhalten und ihre Anschlussgelegenheiten rechtzeitig erreichen.
	
	Die Ziele der Flughafenbetreiber sind ein reibungsloser Betrieb und zufriedene Passagiere.
	%%%%% Aufgabe 2b ---------------
	\item[b)]
	Die Universität Hamburg betreibt das System STiNE zur Vorlesungsplanung, -information und -unter-stützung auf einem Server.
	Von Studierenden und Dozenten kommen Anfragen an und werden bearbeitet.
	Wird eine Anfrage nicht nach einer bestimmten Zeit erfolgreich bearbeitet, so wird sie unerfolgreich abgelehnt.\\
	\vspace{.5cm}\\
	%%%%% Bearbeitung 2b ----------
	Das primäre Zufallsereignis ist die Anfrage einer Person an das System\footnote{analog Skript, Bsp. 1.3.2, V. 05.04.2016}.
	Hierbei gibt es drei Zufallsgrößen: Art, Umfang und Anzahl der Anfragen.
	Die Art der Anfrage bindet unterschiedlich viele Ressourcen gleichzeitig, während der Umfang die Dauer der Ressourcenbindung beeinflusst.
	Die Anzahl der Anfragen gibt an, wie viele Anfragen das System gleichzeitig bearbeiten können muss.
	Ein weiteres Zufallsereignis ist der Zeitpunkt der Anfrage, welcher in einem Zeitraum der Unerreichbarkeit bzw. Nichtverfügbarkeit des Systems liegen kann.
	
	Ziel des Betreibers ist ein sinnvolles LoS\footnote{Skript, Bsp. 1.3.2, V. 05.04.2016}.
	
	Ziel des Service-Nutzers ist eine angemessene Bearbeitungszeit des Systems sowie dass er die Anfrage nicht mehrfach stellen muss, bis sie erfolgreich bearbeitet wird.
	
	
\end{enumerate}
%%%%%%%%%%%%%%%%%%%%%%%%%%%%%
%
%%%% Aufgabe 3
%
%%%%%%%%%%%%%%%%%%%%%%%%%%%%%
\susP{1.3}{5}
(Mengenoperationen und Venn-Diagramme, 1+1+1+2 Punkte).
In einem Venn-Diagramm werden die Grundmengen symbolisch durch geometrische Objekte, meistens Kreise oder Ellipsen dargestellt.
Die Resultate betrachteter Mengenverknüpfungen werden
dann farblich oder durch Markierung hervorgehoben.
Beispielsweise veranschaulicht das folgende Diagramm den Schnitt $A \cap B$ zweier Mengen $A$ und $B$.
\def\firstcircle{(0,0) circle (1.5cm)}
\def\secondcircle{(1,2) circle (1.5cm)}
\def\thirdcircle{(2,0) circle (1.5cm)}
\begin{center}
	\begin{tikzpicture}
		\draw \firstcircle node {$A$};
		\draw \thirdcircle node {$B$};
		\begin{scope}[fill opacity=0.5]
			\clip \firstcircle;
			\fill[gray] \thirdcircle;
		\end{scope}
	\end{tikzpicture}
\end{center}
Zeichnen Sie die entsprechenden Diagramme für die folgenden Operationen.
\begin{enumerate}
	%%%%% Aufgabe 3a ---------------
	\item[a)] $A \setminus B = \{x \in A: x \not \in B\}$ für zwei Mengen $A,B$.\\
	%%%%% Bearbeitung 3a ----------
	\begin{tikzpicture}
	\begin{scope}[shift={(0cm,0cm)}]
        \begin{scope}[even odd rule]% first circle without the second
            \clip \thirdcircle (-3,-3) rectangle (3,3);
        	\fill[yellow] \firstcircle;
        \end{scope}
        \draw \firstcircle node {$A$};
        \draw \thirdcircle node {$B$};
    \end{scope}
    \end{tikzpicture}
	
	
	%%%%% Aufgabe 3b ---------------
	\item[b)]$A \Delta B := (A \setminus B) \cup (B \setminus A)$ für zwei Mengen $A,B$.
	($A \Delta B$ heißt auch \textit{symmetrische Mengendifferenz}.)\\
	%%%%% Bearbeitung 3b ----------
	\begin{tikzpicture}
	\begin{scope}[shift={(0cm,0cm)}]
        \begin{scope}[even odd rule]% first circle without the second
            \clip \thirdcircle (-3,-3) rectangle (3,3);
        	\fill[red] \firstcircle;
        \end{scope}
        \begin{scope}[even odd rule]% first circle without the second
            \clip \firstcircle (-3,-3) rectangle (5,3);
        	\fill[red] \thirdcircle;
        \end{scope}
        \draw \firstcircle node {$A$};
        \draw \thirdcircle node {$B$};
    \end{scope}
    \end{tikzpicture}
	
	
	%%%%% Aufgabe 3c ---------------
	\item[c)]$A \cap B \cap C$ für drei Mengen $A,B,C$.\\
	%%%%% Bearbeitung 3c ----------
	\begin{tikzpicture}
		\begin{scope}
    		\clip \firstcircle;
    		\clip \secondcircle;
			\fill[green] \thirdcircle;
		\end{scope}
		\draw \firstcircle node {$A$};
		\draw \secondcircle node {$B$};
		\draw \thirdcircle node {$C$};
	\end{tikzpicture}
	
	
	%%%%% Aufgabe 3d ---------------
	\item[d)]$(A \cap B) \cup (A \cap C) \cup (B \cap C)$ für drei Mengen $A,B,C$.\\
	%%%%% Bearbeitung 3d ----------
	\begin{tikzpicture}
		\draw \firstcircle node {$A$};
		\draw \secondcircle node[above] {$B$};
		\draw \thirdcircle node[below] {$C$};
		\begin{scope}[fill opacity=0.5]
			\clip \firstcircle;
			\fill[blue] \thirdcircle;
		\end{scope}
		\begin{scope}[fill opacity=0.5]
			\clip \firstcircle;
			\fill[blue] \secondcircle;
		\end{scope}
		\begin{scope}[fill opacity=0.5]
			\clip \secondcircle;
			\fill[blue] \thirdcircle;
		\end{scope}
	\end{tikzpicture}
\end{enumerate}

%#+-------------------------------------------------+#%
%#+						ZETTEL 2					+#%
%#+-------------------------------------------------+#%
\sect{2}{12. April 2016}{19. April 2016}
\susP{2.1}{6}
(Ereignisse und Mengen, 3+3 Punkte).
Es sei $\Omega \neq \emptyset$ eine Ergebnismenge, außerdem seen $A,B,C \subset \Omega$ Ereignisse.
\begin{enumerate}
	%%%%% Aufgabe 1a ----------
	\item[a)] Beschreiben Sie das Ereignis $A \cup (B \cap C)$ verbal.\\
	\vspace{.5cm}\\ 
	%%%%% Bearbeitung 1a ----------
	Es tritt das Ereignis $A$ ein oder die Ereignisse $B$ und $C$.
	
	%%%%% Aufgabe 1b ----------
	\item[b)]Beschreiben Sie das Ereignis \dq Höchstens zwei der Ereignisse $A,B,C$ treten ein\dq ~mengentheoretisch.
	%%%%% Bearbeitung 1b ----------
	\[
		(A \cup B \cup C) \setminus (A \cap B \cap C)
	\]
\end{enumerate}

\susP{2.2}{4}
(Warten auf Zahl, 2+2 Punkte).
In einem Zufallsexperiment wird eine Münze solange geworfen, bis zum ersten Mal \dq Zahl\dq ~erscheint, die möglichen Ausgänge sind die natürlichen Zahlen, d.h. $\mathbb{N}$ ist die Ergebnismenge.
\begin{enumerate}
	%%%%% Aufgabe 2a ------------
	\item[a)]
	Stellen Sie das Ereignis \dq Der erste Wurf, bei dem Zahl erscheint, hat eine ungerade Nummer\dq ~als Menge dar.\\\vspace{.5cm}\\
	%%%%% Bearbeitung 2a ------------
	\[
		\{k, k \in \mathbb{N}, 2 \not | k\}
	\]
	$k$ ist die Anzahl der Würfe, nach denen zum ersten Mal \textit{Zahl} erscheint.
	%%%%% Aufgabe 2b ------------	
	\item[b)] Stellen Sie das Ereignis \dq Spätestens nach 10 Würfen ist einmal Zahl erschienen\dq ~als Menge dar.\\\vspace{.5cm}\\
	%%%%% Bearbeitung 2b ------------
	\[
		\{k, k \in \mathbb{N}, k \leq 10\}
	\]
	$k$ ist die Anzahl der Würfe, nach denen zum ersten Mal \textit{Zahl} erscheint.
\end{enumerate}

\susP{2.3}{7}
(Wahrscheinlichkeitsmaße, 2+5 Punkte).
Über $\Omega=\{1,2,3\}$ soll ein Wahrscheinlichkeitsmaß $P$ definiert werden.
\begin{enumerate}
	%%%%% Aufgabe 3a ------------
	\item[a)] Vervollständigen Sie die folgende Tabelle so, dass $P$ ein Wahrscheinlichkeitsmaß wird.\\\vspace{.5cm}\\
	\begin{center}\begin{tabular}{|c|c|c|c|c|c|c|c|c|}\hline
		$A$		&	$\emptyset$	&	$\{1\}$			&	$\{2\}$				&	$\{3\}$				&	$\{1,2\}$			&	$\{1,3\}$		&	$\{2,3\}$			&	$\{1,2,3\}$		\\ \hline
		$P(A)$	&	\tcg{$0$}	&	$\frac{1}{3}$	&	\tcb{$\frac{1}{2}$}	&	\tcb{$\frac{1}{6}$}	&	\tcb{$\frac{5}{6}$}	&	$\frac{1}{2}$	&	\tcb{$\frac{2}{3}$}	&	\tcg{$1$}		\\ \hline
	\end{tabular}\end{center}
	$P(\emptyset)=0$ und $P(\Omega)=1$ sind nach Definition gegeben.
	
	Es gilt $ \frac{1}{2} = P(\{1,3\}) = P(\{1\} \cup \{3\}) = P(\{1\}) + P(\{3\}) - P(\{1\} \cap \{3\}) = \frac{1}{3} + x - 0 \Rightarrow P(\{3\}) = \frac{1}{6}$.
	
	Desweiteren kann über $P(\Omega)=P(\{1,2,3\})=P(\{1\} \cap \{2\} \cap \{3\})=1$ der Wert von $P(\{2\})$ folgendermaßen ermittelt werden: $1 = P(\Omega) = P(\{1\} \cap \{2\} \cap \{3\}) = P(\{1\} \cap \{3\} \cap \{2\}) = P((\{1\} \cap \{3\}) \cap \{2\}) = P(\{1,3\}) + P(\{2\}) - P((\{1\} \cap \{3\}) \cup \{2\}) = \frac{1}{2} + x \Rightarrow x = P(\{2\}) = \frac{1}{2}$.
	
	Über den Additionssatz gilt dann $\frac{1}{3} + \frac{1}{2} (- 0) = P(\{1\}) + P(\{2\}) - P(\{1\} \cap \{2\}) = P(\{1\} \cap \{2\}) = P(\{1,2\}) = \frac{5}{6}$.
	
	Analog dazu $\frac{1}{2} + \frac{1}{6} (- 0) = P(\{2\}) + P(\{3\}) - P(\{1\} \cap \{3\}) = P(\{2\} \cap \{3\}) = P(\{2,3\}) = \frac{2}{3}$.
	
	%%%%% Aufgabe 3b ------------
	\item[b)] In einer anderen Situation kennen Sie über $\Omega = \{1,2,3\}$ nur Angaben zu $P(\{1\})$ und $P(\{2,3\})$.
	Begründen Sie, warum diese Information nicht ausreicht, um $P:2^{\Omega} \rightarrow \mathbb{R}$ eindeutig festzulegen.\\\vspace{.5cm}\\
	%%%%% Bearbeitung 3b -----------
	$P$ ist nicht eindeutig festlegbar, da keine eindeutige Aussage über $P(\{2\})$ und $P(\{3\})$ getroffen werden kann; es ist lediglich deren Vereinigung bekannt.
	Ohne definitive Werte für $P(\{2\})$ und $P(\{3\})$ lassen sich deren Vereinigungen mit $P(\{1\})$ nicht berechnen, weswegen die Tabelle nicht vervollständigt werden kann.
	
	
	Diese Information reicht nicht aus, um P eindeutig festzulegen, da man nur 2 Informationen gegeben hat, die weder eine Schnittmenge haben, um auf eine dritte schließen zu können, noch jedes bis auf ein Elementarereignis gegeben ist, um auf das letze schließen zu können.
	Somit sind keine weiteren Informationen extrahierbar.
	
	
\end{enumerate}

\susP{2.4}{8}
(Rechnen mit Wahrscheinlichkeiten, 3+5 Punkte).
Es werden zwei faire Würfel geworfen, dabei werden die folgenden Ereignisse betrachtet.
\begin{itemize}
	\item $A$ sei das Ereignis \dq Pasch gewürfelt\dq ~, d.h. beide Würfel zeigen die gleiche Augenzahl.
	Es gilt $P(A)=\frac{6}{36}=\frac{1}{6}$.
	\item $B$ sei das Ereignis \dq Maximum der Augenzahlen ist $\leq 3$, es gilt $P(B) = \frac{9}{36} = \frac{1}{4}$.
	\item $C$ sei das Ereignis \dq Augensumme 7 gewürfelt\dq ~, es gilt $P(C) = \frac{6}{36} = \frac{1}{6}$.
	\item $D$ sei das Ereignis \dq Augensumme 11 gewürfelt\dq ~, es gilt $P(D) = \frac{2}{36} = \frac{1}{18}$.
\end{itemize}
\begin{enumerate}
	\item[a)] Es gilt außerdem $P(A \cap B) = \frac{3}{36} = \frac{1}{12}$.
	Nutzen Sie diese Information und den Additionssatz, um $P(A \cup B)$ zu berechnen.\\\vspace{.5cm}\\
	Additionssatz: $P(A \cup B) = P(A) + P(B) - P(A \cap B)$.
	\[
		P(A \cup B) = P(A) + P(B) - P(A \cap B) = \frac{6}{36} + \frac{9}{36} - \frac{3}{36} = \frac{12}{36} = \frac{1}{3}
	\]
	\item[b)] Begründen Sie, dass $A,C,D$ paarweise disjunkt sind.
	Berechnen Sie anschließend $P(A \cup C \cup D)$.\\\vspace{.5cm}\\
	Da bei $C$ und $D$ unterschiedliche Summen gefordert sind, kann es nicht ein Ergebnis geben, welches auf beide Szenarien zutrifft.
	Die Summen, die in $C$ und $D$ gefordert werden, sind beide ungerade.
	Bei $A$ wird ein Pasch gefordert, d.h. beide Würfel zeigen die Augenzahl $i$.
	Demnach gilt für die Augensumme $i+i=2i$, welches eine gerade Zahl ist, womit sie weder den Wert 7 noch den Wert 11 annehmen kann.
	\begin{align*}
		P(A \cup C \cup D)	=& P(A) + P(C) + P(D) - (P(A \cap C) + P(A \cap D) + P(C \cap D)) + P(A \cap C \cap D)\\
							=& \frac{6}{36} + \frac{6}{36} + \frac{2}{36} - (0 + 0 + 0) + 0 = \frac{14}{36} = \frac{7}{18}
	\end{align*}
\end{enumerate}

\end{document}