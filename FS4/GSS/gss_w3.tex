\documentclass{article}
\usepackage[utf8]{inputenc}
\usepackage[T1]{fontenc}
\usepackage[ngerman]{babel}
\usepackage[margin=2.5cm]{geometry}

% those damn graphs that vsis wants
\usepackage{blockgraph}
% import math packages
\usepackage{amsmath}
\usepackage{amsfonts}
\usepackage{amssymb}
\usepackage{amsthm}
% contradiction lightning
\usepackage{stmaryrd}
% algorithms and pseudo code
\usepackage{algorithmic}
\usepackage{algorithm}
%\usepackage{clrscode3e}
% formatting and layout
\usepackage{color}
% settings
\usepackage{perpage}
\MakePerPage{footnote}
% custom text colors
\definecolor{pblue}{rgb}{0.13,0.13,1}
\definecolor{pgreen}{rgb}{0,0.5,0}
% header and footer
\usepackage{fancyhdr}
% filler text
\usepackage{lipsum}
% macro commands
%% requires color package and the custom colors defined here 
\newcommand{\todo}[1]{
	\addcontentsline{toc}{subsubsection}{TODO: #1}
	\textcolor{pgreen}{\texttt{\-\\ \-\\//TODO: #1\-\\ \-\\}}
}
\newcommand{\sect}[3]{	%	custom section (level 1)
	\newpage
	\section*{Zettel Nr. #1 (Ausgabe: #2, Abgabe: #3)}
}
\newcommand{\sus}[1]{	%	custom section (level 2)
	\subsection*{#1}
}
\newcommand{\sss}[1]{	%	custom section (level 3)
	\subsubsection*{Teilaufgabe #1}
}
% tikz
\usepackage{tikz}
\usetikzlibrary{arrows,automata,positioning}
\usepackage{pgf}

%#+-------------------------------------------------+#%
%#+						VARIABLEN			     	+#%
%#+-------------------------------------------------+#%

\begin{document}

%% Fach-Daten
\newcommand{\fachname}{Grundlagen der Systemsoftware}
\newcommand{\fachnummer}{InfB-GSS}
\newcommand{\veranstaltungsnummer}{64-091}
\newcommand{\stinegruppe}{$ $}
\newcommand{\termin}{$ $}
%% Gruppenmitglied 1
\newcommand{\memOneName}{Utz Pöhlmann}
\newcommand{\memOneMail}{4poehlma@informatik.uni-hamburg.de}
\newcommand{\memOneNr}{6663579}
%% Gruppenmitglied 2
\newcommand{\memTwoName}{Louis Kobras}
\newcommand{\memTwoMail}{4kobras@informatik.uni-hamburg.de}
\newcommand{\memTwoNr}{6658699}
%% Gruppenmitglied 3
\newcommand{\memThreeName}{Marius Widmann}
\newcommand{\memThreeMail}{4widmann@informatik.uni-hamburg.de}
\newcommand{\memThreeNr}{6714203}
%% Datum
\newcommand{\datum}{\today\\}






%#+-------------------------------------------------+#%
%#+						FORMATIERUNG		     	+#%
%#+-------------------------------------------------+#%


\newcommand{\fach}{
	\begin{Huge}
		\fachname\\
	\end{Huge}
	\begin{LARGE}
		Modul: \fachnummer\\
		Veranstaltung: \veranstaltungsnummer\\
	\end{LARGE}
}

\newcommand{\gruppe}{
	\begin{LARGE}
		\stinegruppe\-\\
	\end{LARGE}
	\begin{Large}
		\termin\\
	\end{Large}
}

\newcommand{\memberOfGroup}[3]{
	\begin{center}
		\begin{Large}
			#1
		\end{Large}\\
		#2\\
		#3\\
	\end{center}
	\vspace{.5cm}
}

\newcommand{\datumf}{
	\begin{Large}
		\datum\-\\
	\end{Large}
}

% default header and footer formatting overwrite
\fancyhead{}
\fancyfoot{}
\newcommand{\grafik}[1]{
\begin{figure}[!h]
	\begin{center}\resizebox{.9\textwidth}{!}{#1}\end{center}
\end{figure}
}


%#+-------------------------------------------------+#%
%#+						DECKBLATT  			     	+#%
%#+-------------------------------------------------+#%

\thispagestyle{empty}
%\-\vspace{0.5cm}
\begin{center}
	\fach
	\vspace{1.5cm}
	\gruppe
	\vspace{1.5cm}
	% group members
	\memberOfGroup{\memOneName}{\memOneMail}{\memOneNr}
	\memberOfGroup{\memTwoName}{\memTwoMail}{\memTwoNr}
	\memberOfGroup{\memThreeName}{\memThreeMail}{\memThreeNr}
	% 1 cm to next element
	\vspace{1cm}
	\datumf
	
\end{center}
\newpage

% formatting commands for the rest of the document
\pagenumbering{arabic}
\pagestyle{fancy}	%	allows for headers and footers
\lhead{\memOneName}	
\chead{\memTwoName}
\rhead{\memThreeName}
\lfoot{\today}
\rfoot{\thepage}


%#+-------------------------------------------------+#%
%#+						ZETTEL 1					+#%
%#+-------------------------------------------------+#%
\sect{3}{09. Mai 2016}{25. Mai 2016}
\textit{\textbf{Scaling aus dem \texttt{Blockgraph}-Package hat rumgesponnen. Deswegen scaled floating image, deswegen kleine Labels. Sorry :'( (Zur Not reinzoomen, ist ja eine Vektorgrafik)}}
\sus{Scheduling-Algorithmen}
\sss{a}
\grafik{
\begin{blockgraph}{28}{3}{1}
	% \bglabelxx erzeut Beschriftung der X-Achse an bestimmter Position
	%% x axis labels
	\foreach \n in {0,...,28}{\bglabelxx{\n}}
	
	%% y axis labels
	\bglabely{0}{Prozess}
	\bglabely{1}{Queue-Platz 1}
	\bglabely{2}{Queue-Platz 2}

	% \bgblock erzeugt Block innerhalb des Graphen. Parameter:
	%    Y-Position (z.B. CPU), optional
	%    Beginn auf der X-Achse
	%    Ende auf der X-Achse
	%    Beschriftung
	%% processes
	\bgblock[0]{1}{7}{$P_1$}
	\bgblock[0]{8}{9}{$P_3$}
	\bgblock[0]{10}{12}{$P_4$}
	\bgblock[0]{13}{18}{$P_2$}
	\bgblock[0]{19}{28}{$P_5$}
	
	%% load times
	\bgemptysingleblock[0]{0}
	\bgemptysingleblock[0]{7}
	\bgemptysingleblock[0]{9}
	\bgemptysingleblock[0]{12}
	\bgemptysingleblock[0]{18}
	
	%% Queue
	\bgblock[1]{0}{1}{$P_1$}
	\bgblock[1]{5}{12}{$P_2$}
	\bgblock[2]{6}{7}{$P_3$}
	\bgblock[2]{7}{9}{$P_4$}
	\bgblock[2]{9}{12}{$P_5$}
	\bgblock[1]{12}{18}{$P_5$}

\end{blockgraph}
}
\sss{b}
\grafik{
	\begin{blockgraph}{38}{5}{1}
		%% x axis labels
		\foreach \n in {0,...,38}{\bglabelxx{\n}}
		
		%% y axis labels
		\bglabely{0}{Prozess}
		\bglabely{1}{Queue-Platz 1}
		\bglabely{2}{Queue-Platz 2}	
		\bglabely{3}{Queue-Platz 3}	
		\bglabely{4}{Queue-Platz 4}	
		
		%% processes
		\bgblock[0]{1}{3}{$P_1$}
		\bgblock[0]{6}{8}{$P_1$}
		\bgblock[0]{9}{11}{$P_2$}
		\bgblock[0]{12}{13}{$P_3$}
		\bgblock[0]{14}{16}{$P_4$}
		\bgblock[0]{17}{19}{$P_1$}
		\bgblock[0]{20}{22}{$P_5$}
		\bgblock[0]{23}{25}{$P_2$}
		\bgblock[0]{26}{28}{$P_5$}
		\bgblock[0]{29}{30}{$P_2$}
		\bgblock[0]{31}{33}{$P_5$}
		\bgblock[0]{36}{38}{$P_5$}
		
		%% load times
		\bgemptysingleblock[0]{0}
		\bgemptysingleblock[0]{5}
		\bgemptysingleblock[0]{8}
		\bgemptysingleblock[0]{11}
		\bgemptysingleblock[0]{13}
		\bgemptysingleblock[0]{16}
		\bgemptysingleblock[0]{19}
		\bgemptysingleblock[0]{22}
		\bgemptysingleblock[0]{25}
		\bgemptysingleblock[0]{28}
		\bgemptysingleblock[0]{30}
		\bgemptysingleblock[0]{35}
		
		%% Queue - Slot 1
		\bgblock[1]{0}{1}{$P_1$}
		\bgblock[1]{3}{5}{$P_1$}
		\bgblock[1]{6}{8}{$P_2$}
		\bgblock[1]{8}{11}{$P_3$}
		\bgblock[1]{11}{13}{$P_4$}
		\bgblock[1]{13}{16}{$P_1$}
		\bgblock[1]{16}{19}{$P_5$}
		\bgblock[1]{19}{22}{$P_2$}
		\bgblock[1]{22}{25}{$P_5$}
		\bgblock[1]{25}{28}{$P_2$}
		\bgblock[1]{28}{30}{$P_5$}
		\bgblock[1]{33}{35}{$P_5$}
		
		%% Queue - Slot 2
		\bgblock[2]{5}{6}{$P_2$}
		\bgblock[2]{6}{8}{$P_3$}
		\bgblock[2]{8}{11}{$P_4$}
		\bgblock[2]{11}{13}{$P_1$}
		\bgblock[2]{13}{16}{$P_5$}
		\bgblock[2]{16}{19}{$P_2$}
		
		%% Queue - Slot 3
		\bgblock[3]{7}{8}{$P_4$}
		\bgblock[3]{8}{11}{$P_1$}
		\bgblock[3]{11}{13}{$P_5$}
		\bgblock[3]{13}{16}{$P_2$}
		
		%% Queue - Slot 4
		\bgblock[4]{9}{11}{$P_5$}
		\bgblock[4]{11}{13}{$P_2$}
		
	\end{blockgraph}
}

\sus{Echtzeit \& Mehrprozessor-Scheduling}
\sss{a}
Auftrag A1 benötigt alle 4 Ticks 1 Tick zum Rechnen.\\
Auftrag A2 benötigt alle 7 Ticks 3 Ticks zum Rechnen.\\
Auftrag A3 benötigt alle 3 Ticks 1 Tick zum Rechnen.\\
Bringt man dies alles auf einen Nenner:\\
Auftrag A1 benötigt alle 84 Ticks 21 Ticks zum Rechnen.\\
Auftrag A2 benötigt alle 84 Ticks 36 Ticks zum Rechnen.\\
Auftrag A3 benötigt alle 84 Ticks 28 Ticks zum Rechnen.\\
21+36+28=85\\
85 Ticks benötigte Zeit > 84 Ticks zur Verfügung stehende Zeit\\
Somit gibt es keinen „guten“ Zeitplan.
\sss{b.ii}
\grafik{
	\begin{blockgraph}{25}{5}{1}
		%% x axis labels
		\foreach \n in {0,...,25}{\bglabelxx{\n}}

		%% y axis labels
		\bglabely{0}{Prozess}
		\bglabely{1}{Queue-Platz 1}
		\bglabely{2}{Queue-Platz 2}	
		\bglabely{3}{Queue-Platz 3}
		\bglabely{4}{Überschrittene Deadline}
		
		%% processes
		\bgblock[0]{0}{1}{$B_2$}
		\bgblock[0]{1}{2}{$B_4$}
		\bgblock[0]{2}{4}{$B_3$}
		\bgblock[0]{4}{5}{$B_4$}
		\bgblock[0]{5}{8}{$B_1$}
		\bgblock[0]{8}{9}{$B_4$}
		\bgblock[0]{9}{11}{$B_3$}
		\bgblock[0]{11}{12}{$B_2$}
		\bgblock[0]{12}{13}{$B_4$}
		\bgblock[0]{13}{16}{$B_1$}
		\bgblock[0]{16}{17}{$B_4$}
		\bgblock[0]{17}{20}{$B_1$}
		\bgblock[0]{20}{21}{$B_4$}
		\bgblock[0]{21}{23}{$B_3$}
		\bgblock[0]{23}{25}{$B_1$}
		
		%% Queue - Slot 1
		\bgblock[1]{0}{1}{$B_4$}
		\bgblock[1]{7}{13}{$B_1$}
		\bgblock[1]{14}{17}{$B_1$}
		\bgblock[1]{18}{21}{$B_3$}
		\bgblock[1]{21}{23}{$B_1$}
		\bgblock[1]{24}{25}{$B_4$}
		
		%% Queue - Slot 2
		\bgblock[2]{0}{2}{$B_3$}
		
		%% Queue - Slot 3
		\bgblock[3]{0}{5}{$B_1$}
		
		%% Überschrittene Deadline
		\bgblock[4]{7}{8}{$B_1$}
		\bgblock[4]{14}{16}{$B_1$}
		
	\end{blockgraph}
}\newpage
\sss{c}
\grafik{
	\begin{blockgraph}{13}{5}{1}
		\foreach \n in {0,...,13}{\bglabelxx{\n}}
		
		\bglabely{0}{CPU 1}
		\bglabely{1}{CPU 2}
		\bglabely{2}{CPU 3}
		\bglabely{3}{CPU 4}
		\bglabely{4}{Queue}
		
		%% CPU 1
		\bgblock[0]{0}{4}{$P_1$}
		\bgblock[0]{4}{12}{$P_5$}
		%% CPU 2
		\bgblock[1]{2}{7}{$P_3$}
		\bgblock[1]{7}{12}{$P_6$}
		%% CPU 3
		\bgblock[2]{2}{8}{$P_2$}
		\bgblock[2]{9}{13}{$P_7$}
		%% CPU 4
		\bgblock[3]{3}{11}{$P_4$}
		%% Queue
		\bgblock[4]{3}{4}{$P_5$}
		\bgblock[4]{5}{7}{$P_6$}
		
	\end{blockgraph}
	
}

\sus{Prioritätsinversion}
\sss{a}
\textit{Anmerkung:} Die x-Achsenabschnitte sind in 10er-Schritten zu lesen.
\grafik{
	\begin{blockgraph}{17}{4}{1}
		\foreach \n in {0,...,17}{\bglabelxx{\n}}
		\bglabely{0}{Prozessor}
		\bglabely{1}{Semaphor(B,M)}
		\bglabely{2}{Queue-Slot 1}
		\bglabely{3}{Queue-Slot 2}
	\end{blockgraph}
}

\end{document}