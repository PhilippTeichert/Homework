\documentclass{article}
\usepackage[utf8]{inputenc}
\usepackage[T1]{fontenc}
\usepackage[ngerman]{babel}
\usepackage[margin=2.5cm]{geometry}

% those damn graphs that vsis wants
\usepackage{blockgraph}
% import math packages
\usepackage{amsmath}
\usepackage{amsfonts}
\usepackage{amssymb}
\usepackage{amsthm}
% contradiction lightning
\usepackage{stmaryrd}
% algorithms and pseudo code
\usepackage{algorithmic}
\usepackage{algorithm}
%\usepackage{clrscode3e}
% formatting and layout
\usepackage{color}
% settings
\usepackage{perpage}
\MakePerPage{footnote}
% custom text colors
\definecolor{pblue}{rgb}{0.13,0.13,1}
\definecolor{pgreen}{rgb}{0,0.5,0}
% header and footer
\usepackage{fancyhdr}
% filler text
\usepackage{lipsum}
% macro commands
%% requires color package and the custom colors defined here 
\newcommand{\todo}[1]{
	\addcontentsline{toc}{subsubsection}{TODO: #1}
	\textcolor{pgreen}{\texttt{\-\\ \-\\//TODO: #1\-\\ \-\\}}
}
\newcommand{\sect}[3]{	%	custom section (level 1)
	\newpage
	\section*{Zettel Nr. #1 (Ausgabe: #2, Abgabe: #3)}
}
\newcommand{\sus}[1]{	%	custom section (level 2)
	\subsection*{#1}
}
\newcommand{\sss}[1]{	%	custom section (level 3)
	\subsubsection*{Teilaufgabe #1}
}
% tikz
\usepackage{tikz}
\usetikzlibrary{arrows,automata,positioning}
\usepackage{pgf}

%#+-------------------------------------------------+#%
%#+						VARIABLEN			     	+#%
%#+-------------------------------------------------+#%

\begin{document}

%% Fach-Daten
\newcommand{\fachname}{Grundlagen der Systemsoftware}
\newcommand{\fachnummer}{InfB-GSS}
\newcommand{\veranstaltungsnummer}{64-091}
\newcommand{\stinegruppe}{$ $}
\newcommand{\termin}{$ $}
%% Gruppenmitglied 1
\newcommand{\memOneName}{Utz Pöhlmann}
\newcommand{\memOneMail}{4poehlma@informatik.uni-hamburg.de}
\newcommand{\memOneNr}{6663579}
%% Gruppenmitglied 2
\newcommand{\memTwoName}{Louis Kobras}
\newcommand{\memTwoMail}{4kobras@informatik.uni-hamburg.de}
\newcommand{\memTwoNr}{6658699}
%% Gruppenmitglied 3
\newcommand{\memThreeName}{Marius Widmann}
\newcommand{\memThreeMail}{4widmann@informatik.uni-hamburg.de}
\newcommand{\memThreeNr}{6714203}
%% Datum
\newcommand{\datum}{\today\\}






%#+-------------------------------------------------+#%
%#+						FORMATIERUNG		     	+#%
%#+-------------------------------------------------+#%


\newcommand{\fach}{
	\begin{Huge}
		\fachname\\
	\end{Huge}
	\begin{LARGE}
		Modul: \fachnummer\\
		Veranstaltung: \veranstaltungsnummer\\
	\end{LARGE}
}

\newcommand{\gruppe}{
	\begin{LARGE}
		\stinegruppe\-\\
	\end{LARGE}
	\begin{Large}
		\termin\\
	\end{Large}
}

\newcommand{\memberOfGroup}[3]{
	\begin{center}
		\begin{Large}
			#1
		\end{Large}\\
		#2\\
		#3\\
	\end{center}
	\vspace{.5cm}
}

\newcommand{\datumf}{
	\begin{Large}
		\datum\-\\
	\end{Large}
}

% default header and footer formatting overwrite
\fancyhead{}
\fancyfoot{}


%#+-------------------------------------------------+#%
%#+						DECKBLATT  			     	+#%
%#+-------------------------------------------------+#%

\thispagestyle{empty}
%\-\vspace{0.5cm}
\begin{center}
	\fach
	\vspace{1.5cm}
	\gruppe
	\vspace{1.5cm}
	% group members
	\memberOfGroup{\memOneName}{\memOneMail}{\memOneNr}
	\memberOfGroup{\memTwoName}{\memTwoMail}{\memTwoNr}
	\memberOfGroup{\memThreeName}{\memThreeMail}{\memThreeNr}
	% 1 cm to next element
	\vspace{1cm}
	\datumf
	
\end{center}
\newpage

% formatting commands for the rest of the document
\pagenumbering{arabic}
\pagestyle{fancy}	%	allows for headers and footers
\lhead{\memOneName}	
\chead{\memTwoName}
\rhead{\memThreeName}
\lfoot{\today}
\rfoot{\thepage}


%#+-------------------------------------------------+#%
%#+						ZETTEL 1					+#%
%#+-------------------------------------------------+#%
\sect{3}{09. Mai 2016}{25. Mai 2016}
\textit{\textbf{Scaling aus dem \texttt{Blockgraph}-Package hat rumgesponnen. Deswegen scaled floating image, deswegen kleine Labels. Sorry :'(}}
\sus{Scheduling-Algorithmen}
\sss{a}
\begin{figure}[!h]
\resizebox{.9\linewidth}{!}{
\begin{blockgraph}{27}{3}{1}
	% \bglabelxx erzeut Beschriftung der X-Achse an bestimmter Position
	%% x axis labels
	\bglabelxx{0}
	\bglabelxx{5}
	\bglabelxx{10}
	\bglabelxx{15}
	\bglabelxx{20}
	\bglabelxx{25}
	
	%% y axis labels
	\bglabely{0}{Prozess}
	\bglabely{1}{Queue-Platz 1}
	\bglabely{2}{Queue-Platz 2}

	% \bgblock erzeugt Block innerhalb des Graphen. Parameter:
	%    Y-Position (z.B. CPU), optional
	%    Beginn auf der X-Achse
	%    Ende auf der X-Achse
	%    Beschriftung
	%% processes
	\bgblock[0]{0}{7}{$P_1$}
	\bgblock[0]{8}{9}{$P_3$}
	\bgblock[0]{10}{12}{$P_4$}
	\bgblock[0]{13}{18}{$P_2$}
	\bgblock[0]{19}{27}{$P_5$}
	
	%% load times
	\bgemptysingleblock[0]{7}
	\bgemptysingleblock[0]{9}
	\bgemptysingleblock[0]{12}
	\bgemptysingleblock[0]{18}
	
	%% Queue
	\bgblock[1]{0}{1}{$P_1$}
	\bgblock[1]{5}{12}{$P_2$}
	\bgblock[2]{6}{7}{$P_3$}
	\bgblock[2]{7}{9}{$P_4$}
	\bgblock[2]{9}{12}{$P_5$}
	\bgblock[1]{12}{18}{$P_5$}

\end{blockgraph}
}
\end{figure}
\sss{b}


\sus{Echtzeit \& Mehrprozessor-Scheduling}
\sss{a}
\sss{b}
\sss{c}


\sus{Prioritätsinversion}
\sss{a}

\end{document}