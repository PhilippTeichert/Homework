\documentclass{article}
\usepackage[utf8]{inputenc}
\usepackage[T1]{fontenc}
\usepackage[ngerman]{babel}
\usepackage[margin=2.5cm]{geometry}

% import math packages
\usepackage{amsmath}
\usepackage{amsfonts}
\usepackage{amssymb}
\usepackage{amsthm}
% contradiction lightning
\usepackage{stmaryrd}
% algorithms and pseudo code
\usepackage{algorithmic}
\usepackage{algorithm}
%\usepackage{clrscode3e}
% formatting and layout
\usepackage{color}
% settings
\usepackage{footnote}
\usepackage{perpage}
\MakePerPage{footnote}
% custom text colors
\definecolor{pblue}{rgb}{0.13,0.13,1}
\definecolor{pgreen}{rgb}{0,0.5,0}
% header and footer
\usepackage{fancyhdr}
% filler text
\usepackage{lipsum}
% macro commands
%% requires color package and the custom colors defined here 
\newcommand{\todo}[1]{
	\addcontentsline{toc}{subsubsection}{TODO: #1}
	\textcolor{pgreen}{\texttt{\-\\ \-\\//TODO: #1\-\\ \-\\}}
}
\newcommand{\sect}[3]{	%	custom section (level 1)
	\newpage
	\addcontentsline{toc}{section}{Zettel #1 (#2)}
	\section*{Zettel Nr. #1 (Ausgabe: #2, Abgabe: #3)}
	\label{sec:#1}
}
\newcommand{\ttt}[1]{\texttt{#1}}

% tikz
\usepackage{tikz}
\usetikzlibrary{arrows,automata,positioning}
\usepackage{pgf}

%#+-------------------------------------------------+#%
%#+						VARIABLEN			     	+#%
%#+-------------------------------------------------+#%

\begin{document}

%% Fach-Daten
\newcommand{\fachname}{Grundlagen der Systemsoftware}
\newcommand{\fachnummer}{InfB-GSS}
\newcommand{\veranstaltungsnummer}{64-091}
\newcommand{\stinegruppe}{$ $}
\newcommand{\termin}{$ $}
%% Gruppenmitglied 1
\newcommand{\memOneName}{Utz Pöhlmann}
\newcommand{\memOneMail}{4poehlma@informatik.uni-hamburg.de}
\newcommand{\memOneNr}{6663579}
%% Gruppenmitglied 2
\newcommand{\memTwoName}{Louis Kobras}
\newcommand{\memTwoMail}{4kobras@informatik.uni-hamburg.de}
\newcommand{\memTwoNr}{6658699}
%% Gruppenmitglied 3
\newcommand{\memThreeName}{Marius Widmann}
\newcommand{\memThreeMail}{4widmann@informatik.uni-hamburg.de}
\newcommand{\memThreeNr}{6714203}
%% Datum
\newcommand{\datum}{\today\\}






%#+-------------------------------------------------+#%
%#+						FORMATIERUNG		     	+#%
%#+-------------------------------------------------+#%


\newcommand{\fach}{
	\begin{Huge}
		\fachname\\
	\end{Huge}
	\begin{LARGE}
		Modul: \fachnummer\\
		Veranstaltung: \veranstaltungsnummer\\
	\end{LARGE}
}

\newcommand{\gruppe}{
	\begin{LARGE}
		\stinegruppe\-\\
	\end{LARGE}
	\begin{Large}
		\termin\\
	\end{Large}
}

\newcommand{\memberOfGroup}[3]{
	\begin{center}
		\begin{Large}
			#1
		\end{Large}\\
		#2\\
		#3\\
	\end{center}
	\vspace{.5cm}
}

\newcommand{\datumf}{
	\begin{Large}
		\datum\-\\
	\end{Large}
}

% default header and footer formatting overwrite
\fancyhead{}
\fancyfoot{}


%#+-------------------------------------------------+#%
%#+						DECKBLATT  			     	+#%
%#+-------------------------------------------------+#%

\thispagestyle{empty}
%\-\vspace{0.5cm}
\begin{center}
	\fach
	\vspace{1.5cm}
	\gruppe
	\vspace{1.5cm}
	% group members
	\memberOfGroup{\memOneName}{\memOneMail}{\memOneNr}
	\memberOfGroup{\memTwoName}{\memTwoMail}{\memTwoNr}
	\memberOfGroup{\memThreeName}{\memThreeMail}{\memThreeNr}
	% 1 cm to next element
	\vspace{1cm}
	\datumf
	
\end{center}
\newpage

% formatting commands for the rest of the document
\pagenumbering{arabic}
\pagestyle{fancy}	%	allows for headers and footers
\lhead{\memOneName}	
\chead{\memTwoName}
\rhead{\memThreeName}
\lfoot{\today}
\rfoot{\thepage}


%#+-------------------------------------------------+#%
%#+						ZETTEL 1					+#%
%#+-------------------------------------------------+#%
\sect{6}{27. Juni 2016}{06. Juli 2016}
\setcounter{section}{6}
\subsection{Zentrale Begriffe der Kryptographie}
\stepcounter{subsubsection}
\subsubsection{Schlüsselaustausch (Pflicht; 2 Punkte)}
Für $n$ Personen gibt es $\binom{n}{2}$ Paare.
\paragraph{symmetrisch:}
Bei $n$ Personen muss jeder seinen Schlüssel an $n-1$ Personen weitergeben.
Es gibt also $n$ Schlüssel, die $n-1$ mal weitergegeben werden, also $n*(n-1)=n^2-n$ Tauschaktionen.
Falls die Kommunikation in paarweise beide Richtungen stets mit dem gleichen Schlüssel stattfindet, bleiben $\frac{n^2-n}{2}=\binom{n}{2}$ Schlüsseltauschaktionen.
So viele Schlüssel muss es auch geben.
\paragraph{asymmetrisch:}
Jede Person muss ein Schlüsselpaar generieren, einen \ttt{private key} und einen \ttt{public key}.
Der Public Key wird per Broadcast oder als automatisierter Mailanhang verschickt, somit ensteht für jeden \ttt{private key}-Halter genau eine Aktion betreffs Schlüsselweitergabe.
Dies macht bei $n$ Personen $n$ 'Tausch'-Aktionen, bei $2n$ generierten Schlüsseln.
\subsubsection{Hybride Kryptosysteme (Pflicht; 3 Punkte)}
\paragraph{Umstände:}
Hybride Kryptosysteme eignen sich bei großen Nachrichten, da symmetrische Verschlüsselung um mehrere Zehnerpotenzen schneller arbeiten als asymmetrische Verschlüsselungen.
Durch die asymmetrische Verschlüsselung des vergleichsweise kurzen Keys (i.d.R. 128-256 Bit) wird trotzdem die erhöhte Sicherheit gewährleistet, falls eine dritte Partei den übermittelten Schlüssel erhält (dieser kann durch die Asymmetrie nicht entschlüsselt werden außer vom legitimen Empfänger).
\paragraph{Detail-Verfahren:}
\begin{enumerate}
	\item sie verschlüsselt die Nachricht $N$ symmetrisch mit dem von ihr erzeugtem Schlüssel $S$
	\item $S$ wird asymmetrisch mit Bobs public Key $K_p^B$ verschlüsselt
	\item Alice übermittelt ($K_p^B(S)$, $S(N)$) an Bob
	\item Bob entschlüsselt $K_p^B(S)$ mit $K_s^B$
	\item Bob entschlüsselt $S(N)$ mit $S$ symmetrisch
\end{enumerate}
\paragraph{Nachricht:}
Folgt aus eben: $N' = \{(K_p^B(S),S(N))\}$, also die symmetrisch verschlüsselte ursprüngliche Nachricht sowie der asymmetrisch verschlüsselte Key.
\subsection{Parkhaus}
\stepcounter{subsubsection}
\subsubsection{Sicherheitsanalyse (Pflicht; 4 Punkte)}
\paragraph{Schwächen:}
Der links aufgedruckte Code ist für jeden Zweck jeweils immer identisch. (s. die Zahl darüber 32, 34, 36): Steht eine 32 darüber, wurde das Ticket vom Kino bearbeitet, bei 34 und bei 36 vom Händler.
Der zweite Code von links ist immer identisch.
\paragraph{Angreifermodell:}
\begin{center}
	\begin{savenotes}	% to display footnotes after the table has been processed
		\begin{tabular}{rp{.8\textwidth}}
			\textbf{Rolle}			&	Benutzer des Parkhauses, jedoch kein Kunde im Kino oder beim Einzelhändler	\\
			\textbf{Verbreitung}	&	kann nur sein eigenes Ticket einsehen und hebt auch keine Tickets zum Verlgeich auf oder macht Fotografien o.Ä.\footnote{
											Diese Einschätzung basiert darauf, dass das Angreifermodell denjenigen Angreifer darstelllen soll, gegen den das System noch geschützt ist.
											Sollte der Angreifer mehrere Tickets vorliegen haben, kann er herausfinden, was wir oben mit den Präfizes herausgefunden haben, und sich so den gewünschten Präfix vorne auf sein eigenes Ticket drucken.
											Dadurch wäre das System gebrochen.
											Dieser Angreifer ist vom Angreifermodell also nicht abgedeckt.
										}
										Er probiert also nacheinander alle möglichen Barcodes durch.
										
										Ebenso kann er nicht selber Tickets editieren oder bearbeiten.	\\
			\textbf{Verhalten}		&	aktiv: liest veränderte Tickets am Automaten ein	\\
			\textbf{Ressourcen}		&	beschränkt: nicht genügend Rechenkapazität, um den Algorithmus zu knacken;
										Er hat außer über die Barcodescanner keinen Zugriff auf das System, insbesondere stehen ihm keine weiteren Schnittstellen zur Verfügung
		\end{tabular}
	\end{savenotes}
\end{center}
\subsubsection{Umsetzung mit kryptographischen Techniken (Pflicht; 4 Punkte)}


\subsection{Authentifizierungsprotokolle}
\stepcounter{subsubsection}
\subsubsection{Authentifikationssystem auf Basis indeterministischer symmetrischer Verschlüsselung (Pflicht; 2 Punkte)}
\subsubsection{Challenge-Response-Authentifizierung (Pflicht; 2 Punkte) }


\stepcounter{subsection}
\subsection{RSA-Verfahren}
\stepcounter{subsubsection}
\subsubsection{Anwendung (Pflicht; 6 Punkte)}


\end{document}