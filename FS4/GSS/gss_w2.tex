\documentclass{article}
\usepackage[utf8]{inputenc}
\usepackage[T1]{fontenc}
\usepackage[ngerman]{babel}
\usepackage[margin=2.5cm]{geometry}

% import math packages
\usepackage{amsmath}
\usepackage{amsfonts}
\usepackage{amssymb}
\usepackage{amsthm}
% contradiction lightning
\usepackage{stmaryrd}
% algorithms and pseudo code
\usepackage{algorithmic}
\usepackage{algorithm}
\usepackage{clrscode3e}
% formatting and layout
\usepackage{color}
% settings
\usepackage{perpage}
\MakePerPage{footnote}
% custom text colors
\definecolor{pblue}{rgb}{0.13,0.13,1}
\definecolor{pgreen}{rgb}{0,0.5,0}
% header and footer
\usepackage{fancyhdr}
% filler text
\usepackage{lipsum}
% macro commands
%% requires color package and the custom colors defined here 
\newcommand{\todo}[1]{
	\addcontentsline{toc}{subsubsection}{TODO: #1}
	\textcolor{pgreen}{\texttt{\-\\ \-\\//TODO: #1\-\\ \-\\}}
}
\newcommand{\sect}[3]{	%	custom section (level 1)
	\newpage
	\addcontentsline{toc}{section}{Zettel #1 (#2)}
	\section*{Zettel Nr. #1 (Ausgabe: #2, Abgabe: #3)}
	\label{sec:#1}
}
\newcommand{\sus}[1]{	%	custom section (level 2)
	\addcontentsline{toc}{subsection}{Übungsaufgabe #1}
	\subsection*{Übungsaufgabe #1}
	\label{ssec:#1}
}
\newcommand{\susP}[2]{
	\sus{#1}
	\points{#2}
}
\newcommand{\sss}[1]{	%	custom section (level 3)
	\addcontentsline{toc}{subsubsection}{Aufgabe #1}
	\subsubsection*{Aufgabe #1}
	\label{sssec:#1}
}
\newcommand{\points}[1]{	%	adds a box where the points' worth of a task is written
	\begin{flushright}
		\begin{Large}
			[~~~~\string| ~#1~]
		\end{Large}
	\end{flushright}
}
% tikz
\usepackage{tikz}
\usetikzlibrary{arrows,automata,positioning}
\usepackage{pgf}



%#+-------------------------------------------------+#%
%#+						VARIABLEN			     	+#%
%#+-------------------------------------------------+#%

\begin{document}

%% Fach-Daten
\newcommand{\fachname}{Grundlagen der Systemsoftware}
\newcommand{\fachnummer}{InfB-GSS}
\newcommand{\veranstaltungsnummer}{64-091}
\newcommand{\stinegruppe}{}
\newcommand{\termin}{Mittwoch, 10-12}
%% Gruppenmitglied 1
\newcommand{\memOneName}{Utz Pöhlmann}
\newcommand{\memOneMail}{4poehlma@informatik.uni-hamburg.de}
\newcommand{\memOneNr}{6663579}
%% Gruppenmitglied 2
\newcommand{\memTwoName}{Louis Kobras}
\newcommand{\memTwoMail}{4kobras@informatik.uni-hamburg.de}
\newcommand{\memTwoNr}{6658699}
%% Gruppenmitglied 3
\newcommand{\memThreeName}{Marius Widmann}
\newcommand{\memThreeMail}{4widmann@informatik.uni-hamburg.de}
\newcommand{\memThreeNr}{6714203}
%% Datum
\newcommand{\datum}{\today\\}






%#+-------------------------------------------------+#%
%#+						FORMATIERUNG		     	+#%
%#+-------------------------------------------------+#%


\newcommand{\fach}{
	\begin{Huge}
		\fachname\\
	\end{Huge}
	\begin{LARGE}
		Modul: \fachnummer\\
		Veranstaltung: \veranstaltungsnummer\\
	\end{LARGE}
}

\newcommand{\gruppe}{
	\begin{LARGE}
		\stinegruppe\-\\
	\end{LARGE}
	\begin{Large}
		\termin\\
	\end{Large}
}

\newcommand{\memberOfGroup}[3]{
	\begin{center}
		\begin{Large}
			#1
		\end{Large}\\
		#2\\
		#3\\
	\end{center}
	\vspace{.5cm}
}

\newcommand{\datumf}{
	\begin{Large}
		\datum\-\\
	\end{Large}
}

% default header and footer formatting overwrite
\fancyhead{}
\fancyfoot{}


%#+-------------------------------------------------+#%
%#+						DECKBLATT  			     	+#%
%#+-------------------------------------------------+#%

\thispagestyle{empty}
%\-\vspace{0.5cm}
\begin{center}
	\fach
	\vspace{1.5cm}
	\gruppe
	\vspace{1.5cm}
	% group members
	\memberOfGroup{\memOneName}{\memOneMail}{\memOneNr}
	\memberOfGroup{\memTwoName}{\memTwoMail}{\memTwoNr}
	\memberOfGroup{\memThreeName}{\memThreeMail}{\memThreeNr}
	% 1 cm to next element
	\vspace{1cm}
	\datumf
	
\end{center}
\newpage

% formatting commands for the rest of the document
\pagenumbering{arabic}
\pagestyle{fancy}	%	allows for headers and footers
\lhead{\memOneName}	% first two group members
\chead{\memTwoName}			% third group member
\rhead{\memThreeName}		% last two group members
\lfoot{\today}
\rfoot{\thepage}


%#+-------------------------------------------------+#%
%#+						ZETTEL 1					+#%
%#+-------------------------------------------------+#%
\sect{2}{25. April 2016}{04. Mai 2016}

\susP{1: Grundlagen von Betriebssystemen}{12}
\sss{a)}
\paragraph{Ressourcen-Manager.}
Abstraktionsschicht, um die Hardwarekomponenten wie zum Beispiel Laufwerke, Prozessoren, Arbeitsspeicher und andere I/O-Geräte für Anwendungen und/oder Prozesse zur Verfügung zu stellen.
Allgemein gesprochen gibt es zwei Arten der Ressourcenteilung: Space und Time Multiplexing.
\paragraph{Erweitertes Maschinenkonzept / Extended Machine}
Hierbei handelt es sich um das Kernel- und BIOS-Schnittstellenmanagement.

\sss{b)}
\begin{itemize}
	\item Ressourcen-Manager:
		\begin{itemize}
			\item Scheduling
			\item Festplattenzugriffsverwaltung
		\end{itemize}
	\item Extended Machine:
		\begin{itemize}
			\item Laden und Verwalten von Gerätetreibern
			\item Verwalten von I/O-Ereignissen
			\item Bereitstellung von Schnittstellen
		\end{itemize}
\end{itemize}

\susP{2: Prozesse und Threads}{12}
\sss{a)}
\paragraph{Programm.}
Anweisungen in einer Programmiersprache oder Maschinensprache, die (prozedural) ausgeführt werden.
\paragraph{Prozess.}
Ein Prozess ist ein Programm im aktiven Ausführungszustand (im Prozessor).
\paragraph{Thread.}
Ein Ausführungtsstrang eines nebenläufigen Programmes.
Ein Thread ist ein Programmabschnitt der gleichzeitig mit anderen Threads ausgeführt wird.
\paragraph{Abgrenzung.}
Ein Programm ist ausführbarer Code im Speicher.

Als Prozess bezeichnet man den Referenzwert, mit dem auf ein gerade ausgeführtes Programm zugegriffen werden kann.
Ein Prozess hat einen eigenen Speicherbereich.

Ein Thread ist ein eine Reihenfolge, nach der ein Programm im Prozess abgearbeitet wird. Es können mehrere Threads gleichzeitig aktiv sein.
Threads, die zum selben Prozess gehören, teilen sich einen Speicherbereich.

\sss{d)}
\begin{enumerate}
	\item[X]	rechenbereit
	\item[Y]	rechnend
	\item[Z]	blockiert
	\item[a]	Prozess wird initialisiert
	\item[b]	der Prozess wird ausgewählt
	\item[c]	der Prozess wird blockiert
	\item[d]	Blockierungsgrund ist nicht mehr vorhanden
	\item[e]	Prozessbeendigung oder kritischer Fehler
	\item[f]	ein anderer Prozess wird ausgewählt
\end{enumerate}
\susP{3: n-Adressmaschinen}{16}
\sss{a)}
\begin{codebox}
	\Procname{\texttt{2-Adressmaschine}}
	\li MOVE C1, A1 := C1 = A1
	\li ADD C1, A2 := C1 = C1 + A2
	\li DIV C1, A3 := C1 = C1 / A3
	\li MOVE C2, B1 := C2 = B1
	\li SUB C2, B2 := C2 = C2 - B2
	\li DIV C2, B3 := C2 = C2 / B3
	\li ADD C1, C2 := C1 = C1 + C2
\end{codebox}
Das Ergebnis steht in \texttt{C1}.
\begin{itemize}
	\item Anzahl der Leseaufträge: 5*2 + 2*1 (2*move) = 12
	\item Anzahl der Schreibaufträge: 4 mal C1 ü 3 mal C2 = 7
	\item 1 Speicherzugriff benötigt 1ZE (Zeit Einheit)
	\item Berechnungszeit: (12ZE +7ZE) + 0,05ZE * 7 = 19,35ZE
\end{itemize}

\end{document}