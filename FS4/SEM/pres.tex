\documentclass[%
	handout,
	compressed
]{beamer}

\usepackage[utf8]{inputenc}
\usepackage[ngerman]{babel}
\usepackage[T1]{fontenc}

\usepackage{graphicx}

\usetheme[compressed]{Dresden}
\usecolortheme{beaver}


\author{Louis Kobras}
\title{[Arbeitstitel:]Task-Scheduling in verteilten Softwareentwicklungsprojekten}
\renewcommand{\date}{03. Juni 2016}

\begin{document}

\begin{frame}
	\maketitle
	\vspace{-0.2cm}
	\begin{center}
		Seminar Konzepte verteilter Softwareentwicklung
		\begin{figure}[h]
			\includegraphics[scale=.2]{img/uhh-logo}
		\end{figure}
	\end{center}
\end{frame}

\section{Präamble}
	\subsection{Der Autor}
		\begin{frame}
			\frametitle{Whoami}
			\begin{itemize}
				\item 21 Jahre
				\item B.Sc. SSE FS4
				\item Interessen: Game Development\pause
				\item mag: Starcraft, Stargate, Schokolade
				\item mag nicht: früh aufstehen\pause
				\item Motivation für das Seminar: Softwareentwicklung ist cool
				\item Motivation für das Thema: Algorithmen und Spieltheorie sind cool
			\end{itemize}
		\end{frame}
	\subsection{Gliederung}
		\begin{frame}
			\frametitle{Aufbau}
			\tableofcontents
		\end{frame}

\section{Motivation}
	\subsection{}
		\begin{frame}
			\frametitle{Problem: Scheduling}
			\begin{itemize}
				\item wichtig für geordneten Projektablauf\pause
				\item wichtig für effizienten Projektablauf\pause
				\item interessant aus algorithmischer Sicht\pause
				\item Paradebeispiel für Aspekte der Spieltheorie
			\end{itemize}
		\end{frame}
		\begin{frame}
			\frametitle{Warum nicht Doodle (oder Ähnliches)?}
			\begin{itemize}
				\item Mangel an Übersichtlichkeit
				\item Persönliche Motivation
				\item Nichtoptimierung
				\item Einzelläufigkeit
			\end{itemize}
		\end{frame}
\section{Hauptteil}
	\subsection{Theoretischer Ansatz}	
		\begin{frame}
			\frametitle{Lösung einzelner Teilprobleme}
			\begin{itemize}
				\item Dependency-Resolving durch TopoSort
				\item Zeitplanung durch NP-schwere Probleme
			\end{itemize}
		\end{frame}
		\begin{frame}
				\frametitle{Approximation durch NP-schwere Probleme}
				Auswahl (möglicherweise) passender Probleme:
				\begin{itemize}
					\item Interval Scheduling
					\item Load Balancing
					\item Maximierungsprobleme wie Max-Sat
				\end{itemize}
		\end{frame}
	\subsection{Praktischer Ansatz}
		\begin{frame}
			\frametitle{Zerlegung in Teilprobleme}
			\pause
			\begin{itemize}
				\item Input
					\begin{itemize}
						\item Bearbeitbare Datenstruktur
					\end{itemize}\pause
				\item Magic
					\begin{itemize}
						\item Zerlegung in Teilprobleme
						\item Reduktion jedes Teilproblems
						\item Nutzen von bekannten Algorithmen zur Lösung der Teilprobleme
						\item Rekonstruktion eines Ergebnisses
					\end{itemize}\pause
				\item Output
					\begin{itemize}
						\item hübsche graphische Darstellung (Gantt-Diagramm)
					\end{itemize}
			\end{itemize}
		\end{frame}
		\begin{frame}
			\frametitle{Lösungsansatz}
			\begin{itemize}
				\item mithilfe der gewonnen Kenntnisse Algorithmus schreiben
				\item mithilfe des Algorithmus ein Beispiel berechnen
				\item \textit{sehr} praktischer Ansatz: ein Stück Software implementieren
			\end{itemize}
		\end{frame}
	\subsection{Aus der Spieltheorie}
		\begin{frame}
			\frametitle{Bewertung einer Lösung}
			Begriffsdefinitionen aus der Spieltheorie:
			\begin{itemize}
				\item1 \textbf{Nash-Equilibrium.}		Keiner will einseitig von seiner Strategie abweichen.\pause
				\item \textbf{Pareto-Optimum.}			Keiner kann sich verbessern, ohne dass ein anderer sich verschlechtert.\pause
				\item \textbf{Auszahlungsmaximiert.}	$\lnot \exists$ Möglichkeit, dass eine Partei mehr Gewinn machen kann.
			\end{itemize}
		\end{frame}
		\begin{frame}
			todo: Errechneten Schedule auf o.g. Begriffe untersuchen\\
			\-\\
			Plan: Alle errechneten Schedules erfüllen o.g. Begriffe, sofern sie innerhalb des Projektramens liegen.
		\end{frame}
\section{Schluss}
		\begin{frame}
			\frametitle{Echtweltbezug}
			todo...
		\end{frame}
\section{Quellen}
		\begin{frame}
			\begin{itemize}
				\item Prof. Andreaes VKO-Unterlagen
				\item Franks AuD-Unterlagen
				\item evtl. Franks und Daniels FGI$n$ Unterlagen
				\item Diekmann, Andreas: \textit{Spieltheorie: Einführung, Beispiele, Experimente}, Rowohlt Taschenbuch Verlag; Auflage: 4 (1. April 2009)

			\end{itemize}
		\end{frame}

\end{document}