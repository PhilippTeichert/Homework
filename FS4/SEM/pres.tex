\documentclass{beamer}

\usepackage[utf8]{inputenc}
\usepackage[ngerman]{babel}
\usepackage[T1]{fontenc}

\usepackage{graphicx}

\usetheme[compressed]{Dresden}
\usecolortheme{beaver}


\author{Louis Kobras}
\title{Arbeitsplan-Scheduling in verteilten Software-Entwicklungsprojekten}
\renewcommand{\date}{03. Juni 2016}

\begin{document}

\begin{frame}
	\maketitle
	\vspace{-0.2cm}
	\begin{center}
		Seminar Konzepte verteilter Softwareentwicklung
		\begin{figure}[h]
			\includegraphics[scale=.2]{img/uhh-logo}
		\end{figure}
	\end{center}
\end{frame}

\section{Präamble}
	\subsection{Der Autor}
		\begin{frame}
			\frametitle{Whoami}
		\end{frame}
	\subsection{Gliederung}
		\begin{frame}
			\frametitle{Aufbau}
			\tableofcontents
		\end{frame}

\section{Einleitung}
		\begin{frame}
			\frametitle{Problem: Scheduling}
		\end{frame}
\section{Hauptteil}
		\begin{frame}
			\frametitle{Warum nicht Doodle (oder Ähnliches)?}
		\end{frame}
	\subsection{Theoretischer Ansatz}	
		\begin{frame}
				\frametitle{Approximation durch NP-schwere Probleme}
		\end{frame}
	\subsection{Praktischer Ansatz}
		\begin{frame}
			\frametitle{Lösungsansatz}
		\end{frame}
	\subsection{Aus der Spieltheorie}
		\begin{frame}
			\frametitle{Bewertung einer Lösung}
		\end{frame}
\section{Schluss}
		\begin{frame}
			\frametitle{Echtweltbezug}
		\end{frame}

\end{document}