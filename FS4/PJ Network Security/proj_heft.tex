\documentclass{article}
\usepackage[utf8]{inputenc}
\usepackage[T1]{fontenc}
\usepackage[ngerman]{babel}
\usepackage[margin=2.5cm]{geometry}
\usepackage{hyperref}

% import math packages
\usepackage{amsmath}
\usepackage{amsfonts}
\usepackage{amssymb}
\usepackage{amsthm}
% contradiction lightning
\usepackage{stmaryrd}
% algorithms and pseudo code
\usepackage{algorithmic}
\usepackage{algorithm}
%\usepackage{clrscode3e}
% formatting and layout
\usepackage{color}
% settings
\usepackage{perpage}
\MakePerPage{footnote}
% quotation marks
\usepackage[
    left = ``,%
    right = '',%
    leftsub = `,%
    rightsub = '%
]{dirtytalk}
% custom text colors
\definecolor{pblue}{rgb}{0.13,0.13,1}
\definecolor{pgreen}{rgb}{0,0.5,0}
% macro commands
%% requires color package and the custom colors defined here 
\newcommand{\todo}[1]{
	\addcontentsline{toc}{subsubsection}{TODO: #1}
	\textcolor{pgreen}{\texttt{\-\\ \-\\//TODO: #1\-\\ \-\\}}
}
\newcommand{\sect}[3]{	%	custom section (level 1)
	\newpage
	\addcontentsline{toc}{section}{Zettel #1 (#2)}
	\section*{Zettel Nr. #1 (Ausgabe: #2, Abgabe: #3)}
	\label{sec:#1}
}
\newcommand{\sus}[1]{
	\addcontentsline{toc}{subsection}{Aufgabe #1}
	\subsection*{Übungsaufgabe #1}
	\label{ssec:#1}
}
\newcommand{\sss}[1]{
	\addcontentsline{toc}{subsubsection}{Teilaufgabe #1}
	\subsubsection*{Aufgabe #1}
	\label{sssec:#1}
}
\newcommand{\points}[1]{
	\begin{flushright}
		\begin{Large}
			[~~~~\string| ~#1~]
		\end{Large}
	\end{flushright}
}
\newcommand{\command}[1]{\say{\texttt{#1}}}
% tikz
\usepackage{tikz}
\usetikzlibrary{arrows,automata,positioning}
\usepackage{pgf}

%#+-------------------------------------------------+#%
%#+						VARIABLEN			     	+#%
%#+-------------------------------------------------+#%

\begin{document}

%% Fach-Daten
\newcommand{\fachname}{Projekt Network Security}
\newcommand{\fachnummer}{InfB-Proj}
\newcommand{\veranstaltungsnummer}{64-185}
\newcommand{\stinegruppe}{$ $}
\newcommand{\termin}{Donnerstag, 12.00 - 18.00\\F-027}
%% Gruppenmitglied 1
\newcommand{\memOneName}{Utz Pöhlmann}
\newcommand{\memOneMail}{4poehlma@informatik.uni-hamburg.de}
\newcommand{\memOneNr}{6663579}
%% Gruppenmitglied 2
\newcommand{\memTwoName}{Louis Kobras}
\newcommand{\memTwoMail}{4kobras@informatik.uni-hamburg.de}
\newcommand{\memTwoNr}{6658699}
%% Gruppenmitglied 3
\newcommand{\memThreeName}{}
\newcommand{\memThreeMail}{}
\newcommand{\memThreeNr}{}
%% Gruppenmitglied 4
\newcommand{\memFourName}{}
\newcommand{\memFourMail}{}
\newcommand{\memFourNr}{}
%% Datum
\newcommand{\datum}{\today\\}






%#+-------------------------------------------------+#%
%#+						FORMATIERUNG		     	+#%
%#+-------------------------------------------------+#%

\newcommand{\fach}{
	\begin{Huge}
		\fachname\\
	\end{Huge}
	\begin{LARGE}
		Modul: \fachnummer\\
		Veranstaltung: \veranstaltungsnummer\\
	\end{LARGE}
}

\newcommand{\gruppe}{
	\begin{LARGE}
		\stinegruppe\\
	\end{LARGE}
	\begin{Large}
		\termin\\
	\end{Large}
}

\newcommand{\memberOfGroup}[3]{
	\begin{center}
		\begin{Large}
			#1
		\end{Large}\\
		#2\\
		#3\\
	\end{center}
	\vspace{.5cm}
}
\newcommand{\datumf}{
	\begin{Large}
		\datum\-\\
	\end{Large}
}



%#+-------------------------------------------------+#%
%#+						DECKBLATT  			     	+#%
%#+-------------------------------------------------+#%

\thispagestyle{empty}
\-\vspace{0.5cm}
\begin{center}
	\fach
	\vspace{1.5cm}
	\gruppe
	\vspace{1.5cm}
	% group members
	\memberOfGroup{\memOneName}{\memOneMail}{\memOneNr}
	\memberOfGroup{\memTwoName}{\memTwoMail}{\memTwoNr}
	\memberOfGroup{$ $}{$ $}{$ $}
	\memberOfGroup{$ $}{$ $}{$ $}
	% 1 cm to next element
	\vspace{1cm}
	\datumf
	\vspace{1cm}
	
	\textbf{Punkte für den Hausaufgabenteil:}\\
	\vspace{1cm}
	\begin{tabular}{c|c}
	~7.1~&~$\Sigma$~	\\	\hline
		 &
	\end{tabular}
	
\end{center}
\newpage






























%#+-------------------------------------------------+#%
%#+						INHALT  			     	+#%
%#+-------------------------------------------------+#%
\thispagestyle{empty}
\tableofcontents
\newpage
\pagenumbering{arabic}

%#+-------------------------------------------------+#%
%#+						ZETTEL 1					+#%
%#+-------------------------------------------------+#%
\sect{1}{07. April 2016}{14. April 2016}
\sus{1.1: Hilfe zu Befehlen}
Beim Aufruf von \command{man ls} im Terminal wird eine Liste von Optionen und Parametern angegeben, die mit dem Befehl \command{ls} verwendet werden können.
Die \textit{man page} muss durch drücken der q-Taste verlassen werden.
Wird statt \command{man ls} \command{ls help} eingegeben, so wird der komplette Hilfetext ins Terminal gedruckt und anschließend direkt der Prompt wieder angegeben.

Der Befehl \command{script} startet eine Wrapper-Shell (Default: Bourne Shell, wenn SHELL Parameter nicht gesetzt ist) und zeichnet alle I/O-Streams in der beim Aufruf angegebenen Datei auf.
Die Wrapper-Shell kann mit Ctrl+D (oder exit)  beendet werden.
Die Formatierung ist für den neuen Nutzer bzw. auf den ersten Blick ein wenig strange, dafür enthält die Datei alle relevanten Informationen.
Dies ist bei \command{man script} unter BUGS vermerkt, und zwar dass \command{script} alles in den log file schreibt, inklusive line feeds und Backspaces.
"This is not what the naive user expects."\footnote{Aus der manual-Seite von \texttt{script}}
Der Befehl kann in soweit helfen, dass der komplette Shell-Dialog aufgezeichnet wird.

\sus{1.2: Benutzerkonten und -Verwaltung}
Neuen user angelegt mit \command{sudo adduser <username>} \cite{1}, wobei für \texttt{<username>} in diesem Fall \texttt{labmate} eingesetzt wird.
Nach Eingabe von sudo ist die Authentifizierung mit dem eigenen Passwort erforderlich.
Ist dies geschehen, so wird man aufgefordert, zunächst das Passwort und dann weitere persönliche Daten für \texttt{labmate} einzugeben (Passwort \texttt{laborratte}).

Die Benutzergruppen von \texttt{labmate} werden mit \command{groups labmate} angezeigt.
Output:
\begin{itemize}\item \texttt{labmate : labmate}\footnote{\texttt{labmate} ist derzeit nur in der Gruppe, die genau ihn selber enthält und genauso heißt wie er}\end{itemize}

Die neue Gruppe \texttt{labortests} wird erstellt mit dem Befehl \command{sudo addgroup labortests} \cite[1].
\texttt{labmate} wird der Gruppe \texttt{labortests} mit dem Befehl \command{sudo adduser labmate labortests}\cite{1} zugewiesen.
Alternativ kann dazu der Befehl \command{sudo usermod -aG labortests labmate} verwendet werden\cite{1}.
\texttt{usermod} ist ein Befehl zur Benutzerverwaltung und -manipulation, welcher sicherstellt, dass Manipulation der Nutzerdaten keine laufenden Prozesse beeinflusst\cite{7}.

Um \texttt{labmate} zu erlauben, \texttt{sudo} zu benutzen, muss er der entsprechenden Gruppe namens \texttt{admin} (seit 12.04 \texttt{sudo}) hinzugefügt werden: \command{sudo adduser labmate admin} \cite{2}.



\sus{1.3: Datei- und Rechteverwaltung}
Das Wechseln des Benutzers erfolgt mit \command{su <username>} \cite{3}.
Dabei muss man das Passwort des neuen Nutzers eingeben.

Das Wechseln in das home-Verzeichnis erfolgt (unabhängig vom Nutzer) mit \command{cd /home}.
Der aktuelle Pfad wird mit \command{pwd} angezeigt.
Das neue Verzeichnis wird mit \command{mkdir <pathname>} angelegt (hierbei ist wiederum \command{sudo} vonnöten).
Wechseln in den neuen Ordner mit \command{cd labreports}.
Anlegen neuer Dateien erfolgt mit \command{sudo touch bericht1.txt}.
Geöffnet wird die Datei mit \command{sudo pico bericht1.txt}.
Eingeben folgender Zeichen erfolgt: \textit{hkgfhk}.
Speichern mit \command{Ctrl+O}, Beenden mit \command{Ctrl+X}.

Das Verändern der Zugriffsrechte erfolgt mithilfe der Befehle \command{chgrp} \cite{4} und \command{chmod} \cite{5}.
Zunächst muss mit \command{chgrp labortests beispiel1.txt} die Gruppe, der die Datei gehört, auf \command{labortests} gesetzt werden (sonst gehört die Datei der Gruppe, die nur den Eigentümer enthält).
Danach können mit der Oktal-Variante von \command{chmod} \cite{6} die Zugriffsrechte derart gesetzt werden, dass Eigentümer und Gruppe Lese- und Schreibzugriff, sonst jedoch kein Zugriff möglich ist.
Dies entspricht dem Befehl \command{chmod 660 beispiel1.txt}.
Die Ziffer 6 steht hierbei für einen Lese- und Schreibzugriff und die Ziffer 0 steht für Keine Zugriffsrechte.
Die erste Stelle ist die Einstufung für den Eigentümer, die zweite Stelle ist die Einstufung für die Gruppe, die dritte Stelle ist die Einstufung für andere Nutzer.
(\textit{Anmerkung: Dieser spezifische Fall ist bei \cite{6} als eines der Anwendungsbeispiele gelistet.})

Der Befehl \texttt{wget} lädt eine angegebene Datei herunter.
Verwendung: \command{wget <URL>}.
Die Datei wird dabei in das aktuelle Arbeitsverzeichnis heruntergeladen.


\sus{1.4: Administration und Aktualisierung}
\texttt{apt-get} ist der Paket-Manager für Ubuntu und Debian-basierende Systeme (das Stück Kernel, welches für die Installation und Verwaltung von Software-Paketen verantwortlich ist).
\command{apt-get upgrade} aktualisiert sämtliche derzeit installierten Pakete.
Der Parameter \texttt{-y} kann übergeben werden, um die Installation zu automatisieren (den Prompt automatisch mit 'Ja' zu beantworten).
\command{apt-get upgdate} aktualisiert die systeminterne Liste von Paketquellen\footnote{Eine Reihe von Servern, von denen die installierte Software heruntergeladen wurde. Die Liste umfasst die offiziellen Ubuntu-Server sowie Adressen, die vom Benutzer hinzugefügt wurden}.
Auch hier kann der Parameter \texttt{-y} übergeben werden für den gleichen Effekt wie bei \texttt{upgrade}.





\begin{thebibliography}{1}
\addcontentsline{toc}{section}{Literatur}
\bibitem{1} https://help.ubuntu.com/community/AddUsersHowto
\bibitem{2} https://help.ubuntu.com/community/RootSudo
\bibitem{3} http://www.namhuy.net/44/add-delete-and-switch-user-in-ubuntu-by-command-lines.html
\bibitem{4} https://wiki.ubuntuusers.de/chgrp/
\bibitem{5} https://wiki.ubuntuusers.de/chmod/
\bibitem{6} https://wiki.ubuntuusers.de/chmod/\string#Oktal-Modus
\bibitem{7} http://linux.die.net/man/8/usermod
\end{thebibliography}

\end{document}