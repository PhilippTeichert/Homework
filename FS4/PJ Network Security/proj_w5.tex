\documentclass[twoside]{article}

\usepackage[ngerman]{babel}
\usepackage[utf8]{inputenc}
\usepackage[T1]{fontenc}

\usepackage{fancyhdr}

\usepackage[margin=2.54cm]{geometry}

\usepackage{listings}

\usepackage{xcolor}

\usepackage{graphicx}

\usepackage{hyperref}

\newcommand{\say}[1]{%
	``#1''%
}
\newcommand{\ttt}[1]{%
	\texttt{#1}%
}
\newcommand{\mref}[1]{[\nameref{#1} (S. \pageref{#1})]}
\newcommand{\todo}[1]{\textcolor{red}{\begin{Huge}
	\begin{center}
		\textbf{TODO: #1}
	\end{center}
\end{Huge}}}

%%%%%%%%%%%%%%%%%%%%%%%%%%%
% Source code inclusion
\definecolor{pblue}{rgb}{0.13,0.13,1}
\definecolor{pgreen}{rgb}{0,0.5,0}
\lstset{ %
language=Java,   							% choose the language of the code
basicstyle=\small\ttfamily,  				% the size of the fonts that are used for the code
numbers=left,                   			% where to put the line-numbers
numbersep=5pt,                  			% how far the line-numbers are from the code
backgroundcolor=\color{light-light-gray},   % choose the background color. You must add
frame=lrtb,           						% adds a frame around the code
tabsize=4,          						% sets default tabsize to 2 spaces
captionpos=b,           					% sets the caption-position to bottom
breaklines=true,        					% sets automatic line breaking
xleftmargin=1.5cm,							% space from the left paper edge
commentstyle=\color{pgreen},
keywordstyle=\color{pblue},
literate=%
    {Ö}{{\"O}}1
    {Ä}{{\"A}}1
    {Ü}{{\"U}}1
    {ß}{{\ss}}1
    {ü}{{\"u}}1
    {ä}{{\"a}}1
    {ö}{{\"o}}1
    {~}{{\textasciitilde}}1
}
\renewcommand{\lstlistingname}{Code}
\definecolor{light-light-gray}{gray}{0.95}


\begin{document}
\pagestyle{fancy}
\fancyhead{}
\fancyfoot{}
\fancyhead[L]{Louis Kobras\\6658699}
\fancyhead[C]{Abgabe:\\23. Juni 2016}
\fancyhead[R]{Utz Pöhlmann\\6663579}
\fancyfoot[RE,LO]{Seite \thepage}

\begin{center}
\begin{Huge}
\textbf{SVS Bachelor-Projekt Network Security}
\end{Huge}\\\-\\
\begin{Large}
\textbf{Blatt 5: Beschreibung der Experimentierumgebung}
\end{Large}\\\-\\
\begin{minipage}[t]{0.48\textwidth}
\begin{center}\textbf{
	Louis Kobras\\
	6658699}
\end{center}
\end{minipage}
\begin{minipage}[t]{0.48\textwidth}
\begin{center}\textbf{
	Utz Pöhlmann\\
	6663579}
\end{center}
\end{minipage}
\end{center}

\section{Netzwerkeinstellungen}
\label{sec:network-settings}
	\setcounter{subsection}{1}
	\subsection{}
	\label{ssec:1.2}
	ClientVM:\\
	\begin{center}\begin{tabular}{|p{.3\textwidth}|p{.3\textwidth}|p{.3\textwidth}|}\hline
		\textbf{IP-Adresse} (\ttt{ifconfig -a}): 192.168.254.44&
		\textbf{Standard-Gateway} (\ttt{route -n}): 192.168.254.2&
		\textbf{DNS-Nameserver} (\ttt{nslookup ubuntu.com}): 10.1.1.1\\\hline
	\end{tabular}\end{center}
	RouterVM:\\
	\begin{center}\begin{tabular}{|c|c|}\hline
		eth0	&	eth1	\\\hline
		172.16.137.222	&	192.168.254.2\\\hline
	\end{tabular}\end{center}
	ServerVM:\\
	IP-Adresse der Server-VM: 172.16.137.144
\section{Absichern eines Einzelplatzrechners mit iptables (ClientVM)}
\label{sec:securing-workplace}
	\subsection{}
	\label{ssec:2.1}
	Anzeigen der Firewall-Regeln mit \ttt{sudo iptables -L}; alle Regeln löschen mit \ttt{sudo iptables -F}\footnote{löscht alle Regeln nacheinander}; OpenSSH-Server nach Paketquellen-Update via apt-get installiert (automatisch gestartet).
	\subsection{}
	\label{ssec:2.2}
	Regelwerk siehe \mref{2.2-1}.\\
	\ttt{iptables} säubern mit \ttt{sudo iptables -F}, einladen der Regeln aus einem Textfile mit \ttt{sudo iptables-restore < \~/iptable} (Dateiinhalt im Anhang ebenda).
	\subsection{}
	\label{ssec:2.3}
	\begin{itemize}
		\item SSH-Verbindungsversuch von RouterVM mit \ttt{sudo ssh 192.168.254.44} erfolgreich
		\item SSH-Verbindungsversuch in die andere Richtung nicht erfolgreich (Connection refused)
		\item hosten eines Servers mit \ttt{netcat -l 5555} erfolgreich, Verbindung (\ttt{sudo netcat 192.168.254.44 5555}) erwartungsgemäß fehlgeschlagen
		\item Firefox ist bei DROP schneller als bei REJECT
	\end{itemize}
	\subsection{}
	\label{ssec:2.4}
	Dynamische Regeln vgl. \mref{2.4-1}.\\
	Man muss nicht jeden Port und jedes Protokoll einzelnd abdecken.
	Stateful Filter sind effizienter, da sie sich nur die Paket-Header ansehen.
\section{Absichern eines Netzwerks (RouterVM)}
\label{sec:securing-network}
	\subsection{}
	\label{ssec:3.1}
	Der Aufruf bedeutet (nach \cite{nat}): \say{Maskiere alles, was an eth0 ausgeht}.\\
	Es wird die Adressumsetzung (NAT) aktiviert und die Schnittstelle markiert (\cite{ubuntu:nat}).\\
	Source: 192.168.254.0; Maske: 24
	\subsection{}
	\label{ssec:3.2}
	Die Client-VM kann die Server-VM anpingen; umgekehrt geht dies nicht.\\
	\textit{Vermutung:} Die Client-VM ist von außen nicht direkt ansprechbar, da sie hinter der RouterVM versteckt ist.
	\subsection{}
	\label{ssec:3.3}
	Regelsatz im Anhang unter \mref{3.3-1}\\
	\textbf{ACHTUNG:} Funktioniert nicht!
	Ab hier alle Angaben theoretische Überlegungen
	\subsection{}
	\label{ssec:3.4}
	Folgender Eintrag in der iptable *filter an Stelle [0] sollte den SSH-Tunnel zulassen: \ttt{-A FORWARD -d 172.16.137.144 -p tcp --port 22 -j ACCEPT}
	\subsection{}
	\label{ssec:3.5}
	Folgende Regeln sollte die Aufgabe erfüllen:\\
	\-~~~~\ttt{iptables -A PREROUTING -t nat -i eth0 -p tcp --dport 5022 -j DNAT --to 192.168.254.44:22}\\
	\-~~~~\ttt{iptables -A FORWARD -p tcp -d 192.168.254.44 --dport 22 -j ACCEPT}\\
	Zusätzlich muss der öffentliche Port mithilfe von netcat geöffnet werden: \ttt{nc -l 5022}
	\subsection{}
	\begin{itemize}
		\item Zuweisen der IP 172.16.137.42 mit \ttt{ifconfig eth0:1 172.16.137.42 netmask 255.255.255.0}
		\item PREROUTING-Regel: \ttt{-A PREROUTING -i eth0 -j DNAT {-}{-}to 192.168.254.44}
		\item FORWARD-Regel: \ttt{-A FORWARD -d 192.168.254.44 -j ACCEPT}
		\item Login von der Server-VM mit \ttt{ssh user@172.16.137.42 -L 2000:172.16.137.42:22}
		\item Testweise Client-VM von der Server-VM aus neugestartet
	\end{itemize}
	\label{ssec:3.6}

\section{SSH-Tunnel}
\label{sec:ssh-tunnel}
	\subsection{}
	\label{ssec:4.1}
	iptables-Regeln vgl. \mref{a4-1}.
	\subsection{}
	\label{ssec:4.2}
	Tunnelerzeugung mit \ttt{ssh user@172.16.137.42 -L 2000:172.16.137.42:80}\\
	Beobachtung mit Wireshark ergibt TCP-Pakete, die an SSH weitergeleitet werden.
	Auslesen ist nicht möglich, Gesprächspartner stimmen überein.
	\subsection{}
	\label{ssec:4.3}
	Es ist erforderlich, den Zielserver zu kennen, sowie lokal einen Port zu öffnen und einen freien Port auf dem Server zu wissen.\\
	Als Alternative bietet sich Dynamic Forwarding an (ssh-Aufruf um \ttt{-D} erweitern) \cite{ubuntu:ssh}.\\
	Dem Browser muss mitgeteilt werden, einen Proxy zu verwenden (HOWTO: \cite{ubuntu:sshdyn}).
	\subsection{}
	\label{ssec:4.4}
	Aufbauen einer Reverse-Verbindung von der Client-VM zur Server-VM: \ttt{ssh -R 5555:localhost:22 user@172.16.137.144}\\
	Rücktunneln von der Server-VM: \ttt{ssh localhost -p 5555} \cite{reverse-ssh}
	
\section{OpenVPN}
\label{sec:openvpn}
	\subsection{}
	\label{ssec:5.1}
	Konfiguration vgl. \mref{a5-1}
	\subsection{}
	\label{ssec:5.2}
	Key wird erzeugt durch \ttt{openvpn --genkey --secret static.key}.
	Key liegt dann in \$pwd/static.key in ASCII.
	\subsection{}
	\label{ssec:5.3}
	Konfiguration vgl. \mref{a5-3}. \textbf{Verbindung schlägt fehl.}
	\subsection{}
	\label{ssec:5.4}
	Scheitern an OpenVPN-Server-Konfiguration und der Verbindungsherstellung.
	\subsection{}.
	\label{ssec:5.5}
	\subsection{}.
	\label{ssec:5.6}

\section{HTTP-Tunnel}
\label{sec:http-tunnel}
	\subsection{}
	\label{ssec:6.1}
	Einzelne Regel: \ttt{-A FORWARD -i eth1 -p tcp -m multiport ! --dports 80,53 -j DROP}
	\subsection{}.
	\label{ssec:6.2}
	\subsection{}
	\label{ssec:6.3}
	Konfiguration vgl. \mref{a6-3}.
	\subsection{}.
	\label{ssec:6.4}
	\subsection{}.
	\label{ssec:6.5}
	\subsection{}.
	\label{ssec:6.6}


\begin{thebibliography}{1}
\bibitem{nat}			\url{www.netfilter.org/documentation/HOWTO/de/NAT-HOWTO-6.html}
\bibitem{ubuntu:nat}	\url{https://wiki.ubuntuusers.de/Router/}
\bibitem{ubuntu:ssh}	\url{https://help.ubuntu.com/community/SSH/OpenSSH/PortForwarding}
\bibitem{ubuntu:sshdyn}	\url{https://help.ubuntu.com/community/SSH/OpenSSH/PortForwarding\#Dynamic\_Port\_Forwarding}
\bibitem{reverse-ssh}	\url{https://howtoforge.com/reverse-ssh-tunneling}
\end{thebibliography}
\newpage
\section*{ANHANG}
\label{sec:app}
	\subsection*{2.2: ClientVM-Filterregeln}
	\label{2.2-1}
	\todo{textfile}
	\subsection*{2.4: ClientVM Stateful Filtering}
	\label{2.4-1}
	\todo{textfile}
	\subsection*{3.3: Filterregeln}
	\label{3.3-1}
	\begin{lstlisting}
iptables -t filter -A FORWARD -d 10.1.1.2/32 -j DROP
iptables -t filter -A FORWARD -d 10.0.0.0/8 -j DROP
iptables -t filter -A FORWARD -p udp --dport 53 --sport 53 -J ACCEPT
iptables -t filter -A FORWARD -i eth1 -m state --state NEW -j ACCEPT
iptables -t filter -A FORWARD -m state --state ESTABLISHED,RELATED -j ACCEPT
iptables -t filter -A FORWARD -p tcp -m multiport ! --ports 80,443,8080 -j DROP
	\end{lstlisting}
	\subsection*{4.1: SSH-Ausgang}
	\label{a4-1}
	\todo{textfile}
	\subsection*{5.1}
	\label{a5-1}
	\todo{textfile}
	\subsection*{5.3}
	\label{a5-3}
	\todo{textfile}
	\subsection*{6.3}
	\label{a6-3}
	\todo{textfile}
\end{document}