\documentclass[twoside]{article}

\usepackage[ngerman]{babel}
\usepackage[utf8]{inputenc}
\usepackage[T1]{fontenc}

\usepackage{fancyhdr}

\usepackage[margin=2.5cm]{geometry}

\usepackage{listings}

\usepackage{xcolor}

\usepackage{hyperref}

\newcommand{\say}[1]{%
	''#1''%
}
\newcommand{\ttt}[1]{%
	\texttt{#1}%
}
\newcommand{\mref}[1]{[\nameref{#1} (S. \pageref{#1})]}
\newcommand{\todo}[1]{\textcolor{red}{\begin{Huge}
	\begin{center}
		\textbf{TODO: #1}
	\end{center}
\end{Huge}}}

%%%%%%%%%%%%%%%%%%%%%%%%%%%
% Source code inclusion
\definecolor{pblue}{rgb}{0.13,0.13,1}
\definecolor{pgreen}{rgb}{0,0.5,0}
\lstset{ %
language=Python,   							% choose the language of the code
basicstyle=\small\ttfamily,  				% the size of the fonts that are used for the code
numbers=left,                   			% where to put the line-numbers
numbersep=5pt,                  			% how far the line-numbers are from the code
backgroundcolor=\color{light-light-gray},   % choose the background color. You must add
frame=lrtb,           						% adds a frame around the code
tabsize=4,          						% sets default tabsize to 2 spaces
captionpos=b,           					% sets the caption-position to bottom
breaklines=true,        					% sets automatic line breaking
xleftmargin=1.5cm,							% space from the left paper edge
commentstyle=\color{pgreen},
keywordstyle=\color{pblue},
literate=%
    {Ö}{{\"O}}1
    {Ä}{{\"A}}1
    {Ü}{{\"U}}1
    {ß}{{\ss}}1
    {ü}{{\"u}}1
    {ä}{{\"a}}1
    {ö}{{\"o}}1
    {~}{{\textasciitilde}}1
}
\renewcommand{\lstlistingname}{Code}
\definecolor{light-light-gray}{gray}{0.95}


\begin{document}
\pagestyle{fancy}
\fancyhead{}
\fancyfoot{}
\fancyhead[L]{Louis Kobras\\6658699}
\fancyhead[R]{Utz Pöhlmann\\6663579}
\fancyfoot[RE,LO]{Seite \thepage}

\begin{center}
\begin{Huge}
\textbf{SVS Bachelor-Projekt Network Security}
\end{Huge}\\\-\\
\begin{Large}
\textbf{Blatt 6: Kryptographie}
\end{Large}\\\-\\
\begin{minipage}[t]{0.48\textwidth}
\begin{center}\textbf{
	Louis Kobras\\
	6658699}
\end{center}
\end{minipage}
\begin{minipage}[t]{0.48\textwidth}
\begin{center}\textbf{
	Utz Pöhlmann\\
	6663579}
\end{center}
\end{minipage}
\end{center}


\section{Absicherung des TCP-Chats mit SSL}
	\label{sec:tcp}
		\textit{Anmerkung:} Bei dieser Aufgabe haben wir uns Hilfe von \ttt{SS16G06} (Andre, Katharina) geholt, da wir nach wie vor keine funktionierende Chat-Implementierung hatten.
		\begin{itemize}
			\item Erstellen von Keys und Zertifikaten nach \cite{keytool}
			\item I/O-Dialog vgl. \mref{keytool-dialog}
			\item Sourcecode für Client und Server siehe ebenda.
		\end{itemize}
		Es wird ein Zertifikat für den Public Key des Servers benötigt.
		Dieses wurde mit \ttt{keytool} erstellt.
		
		Es wird ein Zertifikat für die Verbindung des Clients zum Server benötigt.
		Dieses wurde mit \ttt{openssl} erstellt.

\section{CAs und Webserver-Zertifikate}
	\label{sec:zertifikate}
	\stepcounter{subsection}
	\subsection{Selbstsignierte Zertifikate}
		\label{ssec:self-signed}
		Es wurde in mehreren Läufen die Fallstudie durchgearbeitet.
		Und zwar mehrmals und sowohl zusammen als auch einzeln und unabhängig voneinander.
		Mit dem Ergebnis, dass der Apache2-Server nicht funktioniert.
		Die Gruppe neben uns, denen wir inzwsichen bestimmt mega auf den Keks gehen und denen ich als Wiedergutmachung ein Eis mitgebracht habe, konnte uns leider auch nicht helfen.
		Unsere certs und pems und reqs und keys und csrs wurden alle ordnungsgemäß erstellt und Schritt für Schritt, Wort für Wort nach der Fallstudie erzeugt und bearbeitet.
		Es ist kaputt.
		Selbst Reboots und Reinstallationen helfen nicht.
		Folglich funktionieren Aufgabe 2.1ff nicht.
		Mal wieder.
		Sind wir echt so blöd oder liegt vielleicht ein Fehler auf unserer Maschine vor?
	\subsection{HTTPS-Weiterleitung}
		\label{ssec:https}
		folgefehlend.
	\subsection{sslstrip}
		\label{ssec:sslstrip}
		\textit{sslstrip} wurde nach \cite{sslstrip} installiert und gestartet.\\
		Die Verbindung wurde am \say{(svs.informatik.uni-hamburg.de)} in der Logdatei erkannt. (IP: 134.100.15.55)\\
		Die Browsereinstellungen wurde unter \ttt{Edit $\rightarrow$ Preferences $\rightarrow$ Advanced $\rightarrow$ Network $\rightarrow$ Connection $\rightarrow$ Settings...} auf \ttt{localhost} und Port \ttt{8080} gesetzt.
		Zudem wurden die \ttt{No Proxy for}-Einstellungen entfernt.\\
		Der Inhalt der Datei: vgl. \mref{ssllog}
		Die Lösung aus Aufgabe 2.3 ist somit definitiv ein Plus an Sicherheit.\\
		Der Sinn von HSTS ist, sich vor sog. \say{downgrade Attacks} zu schützen.
		Hierbei wird der Client dazu gezwungen, statt einer \say{modernen} sicheren Verbindung eine \say{alte} unsichere Verbindung aufzubauen. (Bsp.: HTTP statt HTTPS)
		Ein weiterer Nutzen ist, \say{Session Hijacking} zu unterbinden.
		Hier wird ein Authentifikationscookie abgefangen und so ein Man-In-The-Middle-Angriff gestartet.
		Da HSTS Webservern erlaubt, auf Browser den Zwang einer sicheren Verbindung (via HTTPS) auszuüben, sind alle Server, die diese Möglichkeit nutzen, auch sicher vor SSL-Stripping-Angriffen.
		
\section{Unsichere selbstentwickelte Verschlüsselungsalgorithmen}
	\label{sec:encrypt}
	\subsection{BaziCrypt}
		\label{ssec:bazi}
		Quellcode siehe \mref{bazicrypt}.
		
		Das Programm nimmt die zu entschlüsselnden Dateien als Parameter entgegen:
		\begin{verbatim}
$ python bazidec.py n01.txt.enc n02.txt.enc n03.txt.enc
message 1: Hallo Peter. Endlich koennen wir geheim kommunizieren! Bis bald, Max
message 2: Hi Max! Super, Sicherheitsbewusstsein ist ja extrem wichtig! Schoene Gruesse, P
eter.
message 3: Hi Peter, hast du einen Geheimtipp fuer ein gutes Buch fuer mich? Gruss, Max

Process finished with exit code 0
		\end{verbatim}
	\subsection{AdvaziCrypt - Denksport}
		\label{ssec:advazi}
Beim PKCS7-Padding wird jede zu verschlüsselnde Nachicht mit Länge $L_i$ auf eine konstante Länge $L_{max}$ aufgestockt.

Die Pading-Bytes sind allerdings nicht 0-Bytes, wie bei Bazi-Crypt, sondern werden aus der Differenz von $L_{max}$ und $L_i$ berechnet.

Beispiel:

$L_max - L_i = 1$

Pading-Bytes: 0x01

$L_max - L_i = 2$

Pading-Bytes: 0x0202

$L_max - L_i = 3$

Pading-Bytes: 0x030303

$L_max - L_i = 4$

Pading-Bytes: 0x04040404

$L_max - L_i = 5$

Pading-Bytes: 0x0505050505

usw.

Es gibt insgesamt $0xFF = 16 \cdot 16 = 2^4 \cdot 2^4 = 2^8 = 256$ verschiedene mögliche Padding-Bytes, sofern $L_{max}$ lang genug gewählt wurde.

	\subsection{AdvaziCrypt - Angriff implementieren}
		\label{ssec:advazi2}
		Quellcode siehe \mref{advazicrypt}.
		
		Auch AdvaziDec nimmt die Dateien als Parameter entgegen.
		Leider wird derzeit bei einem Aufruf von AdvaziDec auch BaziDec ausgeführt, weswegen zuerst alle Nachrichten einmal (fehlerhaft) mit BaziDec entschlüsselt ausgedruckt werden, bevor die korrekte Übersetzung erfolgt.
		\begin{verbatim}
$ python advazidec.py n04.txt.enc n05.txt.enc n06.txt.enc           
message 1: Hi Max, natuerlich: Kryptologie von A. Beutelspacher ist super. Gruss Peter
message 2: Hi Peter, worum geht es in dem Buch? Ciao, Max.
message 3: Hi Max, das ist ein super Buch, das viele Krypto-Themen abdeckt. Gruss Peter

Process finished with exit code 0
		\end{verbatim}
		Die BaziDec-Nachrichten wurden der Leserlichkeit halber ausgelassen.
		Zusätzlich ist zu sagen, dass eigentlich die Nachricht aufgrund der Natur von AdvaziCrypt mit einem beliebigen Zeichen aufgefüllt wird, bis die Standardlänge erreicht ist.
		Da \LaTeX ~diese Zeichen u.U. nicht codieren kann, wurden auch sie hier weggelassen.
		
		\paragraph{Fazit:} Max und Peter freuen sich über (pseudo)sichere Kommunikation.
		Vielleicht sollten sie das Buch tatsächlich mal lesen, über das sie reden, dann fällt ihnen auf, wie leicht Bazi und Advazi zu knacken sind.
\section{EasyAES}
	\label{sec:easyaes}
	Quellcode für diese Aufgabe vgl. \mref{AES}.
	
	Leider hatten wir keine Zeit mehr, den Code vernünftig auszuprobieren.
	Wir haben nur begrenzt viel Rechenkapazität (mehr als 2GB RAM ist nicht drin) und damit hat es immer noch zu lange gedauert.
	Wir hatten bei dem Brute-Force Angriff auf Zettel 2 Aufgabe 2.2 ja auch schon Laufzeitschwierigkeiten...
	
	Zum Starten die Klasse \ttt{AES.java} aufrufen.
\section{Timing-Angriff auf Passwörter (Bonusaufgabe)}
	\label{sec:timing}
		Quellcode für einen Timing-Angriff in Anlehnung an die gegebene Methode vgl. \mref{timing}.
		Eine vertrauenswürdige Quelle hat uns gesagt, dass dieser Quellcode einen vollständigen Timing-Angriff durchführt.




\begin{thebibliography}{1}
\bibitem{sslstrip}	\url{https://moxie.org/software/sslstrip/}
\bibitem{keytool}	\url{http://alvinalexander.com/java/java-keytool-keystore-certificates}
\end{thebibliography}

\newpage
\section*{ANHANG}
	\label{sec:anhang}
	\subsection*{keytool-Dialog}
		\label{keytool-dialog}
		\subsubsection*{IO-Dialog}
		\begin{lstlisting}[language=Bash]
pwd
# $(pwd) -> /home/apollo
# generate directory
mkdir PJNS
mkdir SSL-Server
cd PJNS/SSL-Server/
pwd
# $(pwd) -> /home/apollo/PJNS/SSL-Server
# generate keystore
keytool -genkey -alias server -keystore privateKeys.store
#### password: 'foobar'
#### first and last name: 'Louis Kobras'
#### orga unit: 'SVS'
#### organization: 'Uni Hamburg'
#### City: 'Hamburg'
#### State: 'Hamburg'
#### two-letter country code: 'DE'
#### correct: 'yes'
#### key password: '123xyz'
# generate temporary certificate
keytool -export -alias server -file cert.crt -keystore privateKeys.store 
#### keystore password: 'foobar'
# import certificate in public keystore
keytool -import -alias public-server -file cert.crt -keystore publicKeys.store
#### store password: 'barfoo'
#### do i trust this?: 'yes'
# create openssl stuff
## server stuff
openssl genrsa -out server-ca.key 4096
openssl req -new -x509 -days 365 -key server-ca.key -out server-ca.crt
#### country code: 'DE'
#### state: 'Hamburg'
#### locale: 'Hamburg'
#### organization: 'Uni Hamburg'
#### unit: 'SVS'
#### common name: 'localhost'
#### email: '4kobras@inf...'
## client stuff
openssl genrsa -out client.key 4096
openssl -req -new -key client.key -out client.crt
openssl req -new -key client.key -out client.crt
# sign certificate
openssl x509 -req -days 365 -in client.crt -CA server-ca.crt -CAkey server-ca.key -set_serial 01 -out client-cert.crt
# switch into java workspace and open connection
cd /home/apollo/IntelliJProjects/FS4/src/PJNS/B06/A1/
openssl s_client -connect localhost:1337 -cert ~/PJNS/SSL-Server/client-cert.crt -key ~/PJNS/SSL-Server/client.key -CAfile ~/PJNS/SSL-Server/server-ca.crt
# start server program and connect to it with the client program
		\end{lstlisting}
		\subsubsection*{Server.java}
		\begin{lstlisting}[language=Java]
package PJNS.B06.A1;

import java.awt.*;
import java.awt.event.*;
import javax.net.ssl.*;
import javax.net.ServerSocketFactory;
import javax.swing.*;
import java.util.*;
import java.util.stream.Collectors;
import java.util.concurrent.ConcurrentHashMap;
import java.io.IOException;

public class Server extends JFrame
{

    private static final int PORT = 1337;
    private final JTextArea textArea = new JTextArea();
    private SSLServerSocket server = null;
    private final Map<String, Client> clients = new ConcurrentHashMap<>();

    public Server() throws HeadlessException
    {
        super();
        final JPanel panel = new JPanel();
        panel.setLayout(new BorderLayout());
        panel.setBackground(Color.white);
        getContentPane().add(panel);
        // UI auf deutsch weil ist benutzerfreundlicher in Deutschland
        panel.add("North", new JLabel("Empfangener Text:"));
        panel.add("Center", textArea);
    }

    private void listenSocket()
    {
        try
        {
            final ServerSocketFactory socketFactory = SSLServerSocketFactory.getDefault();
            server = (SSLServerSocket) socketFactory.createServerSocket(PORT);
            server.setNeedClientAuth(true);
        }
        catch (IOException e)
        {
            // Fehlermeldungen auf Englisch weil ist Profi
            System.out.println("Could not listen on port " + PORT);
            System.exit(- 1);
        }
        while (true)
        {
            accept();
        }
    }

    private void accept()
    {
        Client w;
        try
        {
            w = new Client((SSLSocket) server.accept(), textArea);
            Thread t = new Thread(w);
            t.start();
        }
        catch (IOException e)
        {
            System.out.println("Accept failed: " + PORT);
            System.exit(- 1);
        }
    }

    protected void finalize() throws Throwable
    {
        super.finalize();
        try
        {
            server.close();
        }
        catch (IOException e)
        {
            System.out.println("Could not close socket");
            System.exit(- 1);
        }
    }

    public void registerClient(final String clientName, final Client client)
    {
        clients.put(clientName, client);
    }

    protected boolean hasClient(final String clientName)
    {
        return clients.containsKey(clientName);
    }

    protected void broadcast(final String message)
    {
        final java.util.List<Client> clients = this.clients.entrySet()
                                                       .stream()
                                                       .map(Map.Entry::getValue)
                                                       .collect(Collectors.toList());
        for (final Client client : clients)
        {
            client.sendMessage(message);
        }
    }

    public static void main(final String... args)
    {
        System.setProperty("javax.net.ssl.keyStore", "/Users/apollo/PJNS/SSL-Server/privateKeys.store");
        System.setProperty("javax.net.ssl.keyStorePassword", "foobar");
        System.setProperty("javax.net.ssl.trustStore", "/Users/apollo/PJNS/SSL-Server/publicKeys.store");
        System.setProperty("javax.net.ssl.trustStorePassword", "barfoo");
        Server socketServer = new Server();
        socketServer.setTitle("Fancy SSL PJNS.B06.A1.Server");
        WindowListener windowListener = new WindowAdapter()
        {
            public void windowClosing(WindowEvent e)
            {
                System.exit(0);
            }
        };
        socketServer.addWindowListener(windowListener);
        socketServer.pack();
        socketServer.setVisible(true);
        socketServer.listenSocket();

    }

}
		\end{lstlisting}
		\subsubsection*{Client.java}
		\begin{lstlisting}
package PJNS.B06.A1;

import java.io.*;
import java.util.logging.*;
import javax.swing.*;
import javax.naming.ldap.*;
import javax.naming.InvalidNameException;
import javax.net.ssl.*;
import javax.security.cert.X509Certificate;

public class Client implements Runnable
{

    private static final Logger LOGGER = Logger.getLogger("PJNS.B06.A1.Client");
    private SSLSocket _sslSocket;
    private JTextArea _text;
    private PrintWriter _pwOut = null;
    private String _client;
    private Server _server = new Server();

    public Client(SSLSocket socket, JTextArea textArea)
    {
        _sslSocket = socket;
        _text = textArea;
    }

    @Override
    public void run()
    {
        BufferedReader brIn;
        final SSLSession sslSession = _sslSocket.getSession();
        if (sslSession.isValid())
        {
            try
            {
                final InputStream is = _sslSocket.getInputStream();
                brIn = new BufferedReader(new InputStreamReader(is));
                _pwOut = new PrintWriter(_sslSocket.getOutputStream(), true);

                getUsername(sslSession);
                stuff(brIn);
            }
            catch (IOException | InvalidNameException e)
            {
                logError(e);
            }
        }
    }

    /**
     * SOURCE: https://stackoverflow.com/questions/2914521/how-to-extract-cn-from-x509certificate-in-java
     */
    private void getUsername(SSLSession session) throws InvalidNameException
    {
        try
        {
            final X509Certificate[] x509Certificates = session.getPeerCertificateChain();
            final String name = x509Certificates[0].getSubjectDN().getName();
            final LdapName lname = new LdapName(name);
            for (final Rdn round : lname.getRdns())
            {
                if (round.getType().equalsIgnoreCase("CN"))
                {
                    _client = (String) round.getValue();
                    logInfo("Username is: " + _client);
                }
            }
        }
        catch (SSLPeerUnverifiedException e) {
            logError(e);
        }
    }

    private void stuff(BufferedReader brIn) throws IOException
    {
        String message;
        if (_client != null && ! _client.isEmpty())
        {
            if (! _server.hasClient(_client))
            {
                _server.registerClient(_client, this);
                while (true)
                {
                    message = brIn.readLine();
                    if (message != null)
                    {
                        broadcast(message);
                    }
                    else
                    {
                        break;
                    }
                }
            }
            else
            {
                logInfo("PJNS.B06.A1.Client already registered");
            }
        }
    }

    /**
     * Log an Info message
     *
     * @param s Info to log
     */
    private void logInfo(String s)
    {
        LOGGER.log(Level.INFO, s);
    }

    /**
     * Log an Error or an Exception as Severe
     *
     * @param e Error/Exception to log
     */
    private void logError(Exception e)
    {
        LOGGER.log(Level.SEVERE, "1", e);
        System.exit(1);
    }

    /**
     * Broadcasts a message to both yourself and the server
     *
     * @param msg the message to broadcast
     */
    private void broadcast(String msg)
    {
        _text.append(msg + "\n");
        _server.broadcast(msg);
    }

    /**
     * Sends a message to the output stream
     *
     * @param msg the message to send
     */
    public void sendMessage(final String msg)
    {
        _pwOut.println(msg);
    }
}
		\end{lstlisting}
	\subsection*{ssllog}
		\label{ssllog}
		\textit{Anmerkung:} Aufgrund der Länge der Zeilen wurden nachträglich manuell Zeilenumbrüche eingefügt.
\begin{lstlisting}
2016-06-23 15:45:58,171 POST Data (safebrowsing.clients.google.com):
goog-malware-shavar;a:239451-244530:s:234828-234868,234872-234874,234876-234888,234890-234895,
    234897-234901,234903-234904,234906,234910-234915,234917-234927,
    234929-235103,235105-235168,235170-235176,235178-235198,235200-235256,
    235258-235274,235276-235307,235309-235473,235477,235479-235485,
    235487-235807,235809-235974,235976-236189,236191-236363,236365-236366,
    236368-237764,237766-238163,238165-238181,238183-238196,238198-239044,
    239046-239996:mac
goog-phish-shavar;a:448570-450992:s:268799-268858,268860-269092,269096-269182,
    269184-269212,269214-269222,269224-269265,269267-269291,269293-269295,
    269297-269311,269313,269316-269347,269349-269351,269353-269356,
    269359-269360,269362-269368,269370,269390-269392,269408-269513,
    269515-269518,269521-269575,269577-269606,269608-269615,269617,
    269619-269628,269630-269694,269696-269734,269736-269843,269845,
    269847-269902,269904-269926,269928-269937,269939-269995,269997-270023,
    270025-270092,270094-270115,270117-270151,270153-270163,270165-270190,
    270192-270205,270207-270234,270236,270238-270275,270277-270374,
    270376-270402,270404-271027,271030-271036,271038-271049,271051-271118,
    271120-271194,271196-271212,271214-271227,271229-271391,271393-271394,
    271398-271434,271436-271442,271444-271449,271452-271472,271475-271526,
    271528-271595:mac

2016-06-23 15:47:15,606 SECURE POST Data (svs.informatik.uni-hamburg.de):
username=admin&password=password

\end{lstlisting}
	\subsection*{Knacking BaziCrypt}
		\label{bazicrypt}
		\begin{lstlisting}
import sys


def task_three_point_one(msg):
    """
    task_three_point_two function of the program. takes a HEX-encrypted message and decrypts it
    :param msg: the message to decrypt
    :return: the decrypted message
    """
    # find the key that was used to encrypt
    """
    absuses the nature of the encryption method used by taking the last 10 chars of the message, which are the key
    """
    key = msg[-20:]
    return get_message(key, msg)


def get_message(key, msg):
    # decrypt the message using the found key
    """
        due to the nature of the encryption method used,t he message, being symmetrically encrypted,
        can be decrypted by XORing the message with the encryption key.
        So, bring the key to the same length as the message and xor them
        """
    key *= 10  # takes advantage of the fact that each key had a length of 10 chars
    # and each message had a length of 100 chars, setting the length factor to 10
    msg = xor(msg, key)
    msg = get_chars(msg)
    return ''.join(msg)


def xor(first_string, second_string):
    """
    given two HEX strings, uses the XOR function between them by converting them to lists of integers
    :param first_string:
    :param second_string:
    :return: the result of XOR
    """
    # convert the first HEX string to integers
    first_list = []
    for c in first_string:
        b = int(c, 16)
        first_list.append(b)
    # converts the second HEX string to integers
    second_list = []
    for c in second_string:
        b = int(c, 16)
        second_list.append(b)
    result = []
    # XORs the lists
    for i in range(0, len(first_list)):
        j = first_list[i] ^ second_list[i]
        j = hex(j)  # converts the XORd integer to a HEX
        j = j[2:]  # cut the HEX '0x' notation given by the built-in function cast hex(1)
        result.append(j)
    result = ''.join(result)  # joins the XORd list to a string
    return result


def get_chars(lst):
    """
    given a list of HEX values, this function returns the char values of each position
    :param lst: a list of hex values
    :return: a list of char values as integer
    """
    res = []
    for i in xrange(0, len(lst) - 1, 2):
        a = int(lst[i], 16)  # get first place as int
        a *= 16  # multiply first place by 16
        b = int(lst[i + 1], 16)  # get second place as int
        b += a  # add values together to get ASCII char code
        res.append(chr(b))
    return res


for j in range(1, len(sys.argv)):
    f = open(sys.argv[j], 'r')
    f = f.read()
    f = f.encode('hex')
    f = task_three_point_one(f)
    print "message %d: %s" % (j, f)

		\end{lstlisting}
	\subsection*{advazicrypt-Angriff}
		\label{advazicrypt}
		\begin{lstlisting}
from bazidec import xor
from bazidec import get_message


def key(i):
    if i < 256:
        appendix = hex(i)[2:]
        if len(appendix) < 2:
            appendix = '0%s' % appendix
    dec = '0x%s' % (appendix * i)
    return dec


def task_three_point_three(msg, padding):
    """
    task_three_point_two function of the program. takes a HEX-encrypted message and decrypts it
    :param msg: the message to decrypt
    :return: the decrypted message
    """
    # find the key that was used to encrypt
    """
    absuses the nature of the encryption method used by taking the last 10 chars of the message, which are the key
    """
    message_bytes = msg[-20:]
    padding_bytes = padding[-20:]
    key = xor(message_bytes, padding_bytes)
    return get_message(key, msg)


min_range = 10
max_range = 100
"""
We can't get any intel about the first part of the key if the padding is smaller than the key length
with only part of the key, we can only decrypt parts of the message in periodic intervals
thus we start with a length of 10, which is required for a complete key
the max range is the length of each message (all of them are 100 bytes long)
to improve runtime, switch the for loops (the answers then have to be sorted manually)
"""
msgs = []
for j in range(1, len(sys.argv)):
    for key_iterator in range(min_range, max_range):
        padding_bytes = key(key_iterator)[2:]
        msg = open(sys.argv[j], 'r').read().encode('hex')
        str = task_three_point_three(msg, padding_bytes)
        if str.__contains__("Max") or str.__contains__("Peter"):
            print "message %d: %s" % (j, str)

		\end{lstlisting}
		
	\subsection*{EasyAES}
		\label{AES}
		\subsubsection*{SchluesselVersuchZwei.java}
		\begin{lstlisting}[language=Java]
package B06.A4;

import java.util.ArrayList;

public class SchluesselVersuchZwei
{
    private ArrayList<Key> key;

    public SchluesselVersuchZwei()
    {
        key = generiereAlleSchluessel();
        System.out.println(key); //TODO: wegen debug
    }
    
    public static void main(String[] args)
    {
        new SchluesselVersuchZwei();
    }

    public ArrayList<Key> getKeys()
    {
        return key;
    }

    private ArrayList<Key> generiereAlleSchluessel()
    {
        ArrayList<Key> schluesselListe = new ArrayList<Key>();
        Key schluessel = new Key();
        schluesselListe.add(schluessel);//generiere schlüssel für 0 nichtNullBytes
        ArrayList<Key> schluesselListe1 = generiereSchluessel1NNB();//generiere schlüssel für 1 nichtNullByte
        ArrayList<Key> schluesselListe2 = generiereSchluessel2NNB();//generiere schlüssel für 2 nichtNullBytes
        schluesselListe.addAll(schluesselListe1);
        schluesselListe.addAll(schluesselListe2);
        return schluesselListe;
    }

    private ArrayList<Key> generiereSchluessel1NNB()
    {
        ArrayList<Key> schluesselListe = new ArrayList<Key>();
        for (int index = 0; index < 16; index++) //Für jedes Feld (0-15)
        {
            Key schluessel = new Key();
            for (int wert = 1; wert < 256; wert++) //Für jeden Schlüsselwert (0x00 - 0xff)
            {
                schluessel.setKey(index, wert); //Gib Ergebnis
                schluesselListe.add(schluessel);
            }

        }
        return schluesselListe;
    }

    private ArrayList<Key> generiereSchluessel2NNB()
    {
        ArrayList<Key> schluesselListe = new ArrayList<Key>();
        for (int index1 = 0; index1 < 16; index1++) //Für jedes Feld (0-14)
        {
            for (int wert1 = 1; wert1 < 256; wert1++) //Für jeden Schlüsselwert (0x00 - 0xff)
            {
                for (int index2 = index1+1; index2 < 16; index2++) //Für jedes Feld (0-15)
                {
                    for (int wert2 = 1; wert2 < 256; wert2++) //Für jeden Schlüsselwert (0x00 - 0xff)
                    {
                        Key schluessel = new Key();
                        schluessel.setKey(index1, wert1); //Gib Ergebnis
                        schluessel.setKey(index2, wert2); //Gib Ergebnis
                        schluesselListe.add(schluessel);
                    }

                }
            }

        }
        
        return schluesselListe;
    }

}
		\end{lstlisting}
		\subsubsection*{Key.java}
		\begin{lstlisting}[language=Java]
package B06.A4;

public class Key
{

    private Hex[] _key;

    public Key()
    {
        _key = new Hex[16];
        for (int index = 0; index < 16; index++)
        {
            _key[index] = new Hex(0);
        }
    }
    
    public static void main(String args[])
    {
        int[] werte = new int[16];
        for(int i=0; i<werte.length; i++){
            werte[i] = i;
        }
        System.out.println(new Key(werte).toString());
    }

    public Key(int[] werte)
    {
        assert werte.length == 16;
        _key = new Hex[16];
        for (int index = 0; index < 16; index++)
        {
            _key[index] = new Hex(werte[index]);
        }
    }

    public void setKey(int index, int wert)
    {
        assert index < 16;
        assert index > -1;
        assert wert < 256;
        assert wert > -1;
        _key[index] = new Hex(wert);
    }

    public Hex getWert(int index)
    {
        assert index < 16;
        assert index > -1;
        return _key[index];
    }

    @Override
    public String toString()
    {
        StringBuilder sb = new StringBuilder();
        for (int index = 0; index < _key.length-1; index++)
   {
            sb.append(_key[index]);
            sb.append(":");
        }
        sb.append(_key[_key.length-1]);

        return sb.toString();
    }

}
		\end{lstlisting}
		\subsubsection*{Hex.java}
		\begin{lstlisting}[language=Java]
package B06.A4;

public class Hex
{
    int _wert;
    /**
     * Es wird davon ausgegangen dass nur Werte zwischen 0 und 255 eigegeben werden.
     * 
     * @param wert
     */
    public Hex(int wert){
        assert wert < 256;
        assert wert > -1;
        _wert = wert;
    }
    
    public static void main(String args[])
    {
        System.out.println(new Hex(42).toString());
    }
    
    @Override
    public String toString(){
        if(_wert < 16) return "0".concat(Integer.toHexString(_wert));
        else return Integer.toHexString(_wert);
    }
}
		\end{lstlisting}
		\subsubsection*{AES.java}
		\begin{lstlisting}[language=Java]
package B06.A4;

import java.security.MessageDigest;
import java.util.ArrayList;
import java.util.Arrays;

import javax.crypto.Cipher;
import javax.crypto.spec.SecretKeySpec;

import sun.misc.BASE64Decoder;
import sun.misc.BASE64Encoder;

/**
 * Eine Klasse, die eine doppelte AES-Verschlüsselung durch einen Meet-In-The-Middle-Angriff knackt
 * gegeben einen Klartext und einen verschluesselten Schluesseltext
 * 
 * 
 * @author Alexander Gräsel, Louis Kobras, Utz Pöhlmann
 * Quelle: http://blog.axxg.de/java-aes-verschluesselung-mit-beispiel/
 */
public class AES
{

    private static final String _klartext = "Verschluesselung"; //Der Klartext
    private final static String _schluesseltext = "be393d39ca4e18f41fa9d88a9d47a574"; //Der doppelt verschlüsselte Klartext
    private SchluesselVersuchZwei _generator;

    /**
     * Versucht, die beiden Schlüssel E_k1 und E_k2 heauszufinden mit
     * _schluesseltext = E_k2(E_k1(_klartext))
     * und scheibt sie auf die Konsole
     * 
     * @param args
     * @throws Exception
     */
    public static void main(String[] args) throws Exception
    {
        AES aes = new AES();
        ArrayList<String> gefundeneSchluessel = aes.findeSchluessel(_klartext,
                _schluesseltext);
        System.out.println(gefundeneSchluessel);
    }

    /**
     * Versucht, die beiden Schlüssel E_k1 und E_k2 heauszufinden mit
     * _schluesseltext = E_k2(E_k1(_klartext))
     * durch einen Meet-In-The-Middle-Angriff
     * 
     * @param klartext
     * @param schluesseltext
     * @return eine ArrayList von Typ String mit folgenden Elementen:
     * 		"Folgende Schluessel wurden gefunden:"
     * 		"E_k1: " + schluessel1.toString()
     * 		"E_k2: " + schluessel2.toString()
     * @throws Exception 
     */
    public ArrayList<String> findeSchluessel(String klartext,
            String schluesselText)
    {
        String textStufeEins1;
        String textStufeEins2;
        ArrayList<String> ergebnis = new ArrayList<String>();
        try
        {
            ArrayList<SecretKeySpec> schluesselListe = generiereSchluessel();
            System.out.println(schluesselListe);
            for (SecretKeySpec schluesselEnc : schluesselListe)
            {
                textStufeEins1 = verschluesseln(klartext, schluesselEnc);
                for (SecretKeySpec schluesselDec : schluesselListe)
                {
                    textStufeEins2 = entschluesseln(schluesselText,
                            schluesselDec);
                    System.out.println(textStufeEins1 + ":" + textStufeEins2);
                    if (textStufeEins1.equals(textStufeEins2))
                    {
                        ergebnis.add("Folgende Schluessel wurden gefunden:");
                        ergebnis.add("E_k1: " + schluesselEnc.toString());
                        ergebnis.add("E_k2: " + schluesselDec.toString());
                    }
                }
            }
        }
        catch (Exception e)
        {
            System.out.println(e.getMessage());
        }
        return ergebnis;
    }

    /**
     * Generiert alle möglichen Schlüssel, auf die die Vorgaben (16 Byte, maximal an 2 Stellen keine 0-Bytes) zutreffen
     * 
     *  @return die Schluessel Liste als Liste von SecretKeySpec
     * @throws Exception 
     */
    private ArrayList<SecretKeySpec> generiereSchluessel() throws Exception
    {
        ArrayList<SecretKeySpec> raffinierteSchluesselListe = new ArrayList<SecretKeySpec>();
        _generator = new SchluesselVersuchZwei();
        ArrayList<Key> schluesselListe = _generator.getKeys();
        for (Key key : schluesselListe)
        {
            SecretKeySpec schluessel = bereiteSchluesselVor(key);
            raffinierteSchluesselListe.add(schluessel);
        }
        return raffinierteSchluesselListe;
    }

    /**
     * Bringt einen Schluessel in ein Format mit dem das Programm weiter arbeiten kann (SecretKeySpec)
     * 
     * @param schluessel
     * @return der fertige Schluessel als SecretKeySpec
     * @throws Exception 
     */
    private SecretKeySpec bereiteSchluesselVor(Key schluessel) throws Exception
    {
        // Das Passwort bzw der Schluesseltext
        String schluesselString = schluessel.toString();
        // byte-Array erzeugen
        byte[] key = (schluesselString).getBytes("UTF-8");
        // aus dem Array einen Hash-Wert erzeugen mit MD5 oder SHA
        MessageDigest sha = MessageDigest.getInstance("SHA-256");
        key = sha.digest(key);
        // nur die ersten 128 bit nutzen
        key = Arrays.copyOf(key, 16);
        // der fertige Schluessel
        return new SecretKeySpec(key, "AES");
    }

    /**
     * Verschluesselt einen Klatext nach AES-Verfahren
     * 
     * @param klartext
     * @return der verschluesselte Klartext
     * @throws Exception 
     */
    private String verschluesseln(String klartext, SecretKeySpec schluessel)
            throws Exception
    {
        // Verschluesseln
        Cipher cipher = Cipher.getInstance("AES");
        cipher.init(Cipher.ENCRYPT_MODE, schluessel);
        byte[] encrypted = cipher.doFinal(klartext.getBytes());

        // bytes zu Base64-String konvertieren (dient der Lesbarkeit)
        BASE64Encoder myEncoder = new BASE64Encoder();
        return myEncoder.encode(encrypted);
    }

    /**
     * Entschluesselt einen Klatext nach AES-Verfahren
     * 
     * @param schluesselText
     * @return der entschluesselte schluesselText
     * @throws Exception 
     */
    private String entschluesseln(String schluesselText,
            SecretKeySpec schluessel) throws Exception
    {
        // BASE64 String zu Byte-Array konvertieren
        BASE64Decoder myDecoder2 = new BASE64Decoder();
        byte[] crypted2 = myDecoder2.decodeBuffer(schluesselText);

        // Entschluesseln
        Cipher cipher2 = Cipher.getInstance("AES");
        cipher2.init(Cipher.DECRYPT_MODE, schluessel);
        byte[] cipherData2 = cipher2.doFinal(crypted2);
        return new String(cipherData2);
    }

}
		\end{lstlisting}
		
	\subsection*{Timing-Angriff}
		\label{timing}
		\textit{Anmerkung:} String \textunderscore symbole enthält nicht die Sonderzeichen,
		\begin{enumerate}
			\item weil \LaTeX nicht alle codieren kann und
			\item weil die Aufgabe schwammig gestellt ist.
					Ich habe eine deutsche Tastatur, kann aber, weil Linux, deutlich mehr Sonderzeichen über AltGr erreichen als ein Windows Keyboard.
					Außerdem kann man, sobald ein NumBlock angeschlossen ist, faktisch jedes ASCII-Zeichen über Tastatur erreichen.
		\end{enumerate}
\begin{lstlisting}[language=Java]
public class Timer
{
	private final char[] _passwort = "abcdefgh".toCharArray();
	private boolean _gefunden = false;
	private int _passwortLaenge;
	private final char[] _symbole = "abcdefghijklmnopqrstuvwxyz1234567890"
    	.toCharArray();
	private final int _maxPasswortLaenge = 20;

	public static void main(String[] args)
	{
    	Timer timer = new Timer();
    	timer._passwortLaenge = timer.findePasswortLaenge();
    	System.out.println(timer._passwortLaenge);
    	System.out.println(timer.findePasswort());
	}

	/**
 	* Findet das Passwort. Period.
 	* @return Das Passwort.
 	* @requirements: Die Passwortlaenge wurde bestimmt.
 	*/
	private String findePasswort()
	{
    	String laufPasswort = "";
    	String bisherigesPasswort = "";
    	long[] zeiten = new long[_passwortLaenge];
    	long startZeit;
    	long endZeit;
    	do
    	{
        	// Zaehlt einen Zaehler bis zur bestimmen Passwortlaenge
        	for (int laenge = 0; laenge < _passwortLaenge; laenge++)
        	{
            	// Zaehlt einen Zaehler bis zur Laenge des Eingabealphabets
            	for (int zaehler = 0; zaehler < _symbole.length; zaehler++)
            	{
                	// haengt das aktuelle Laufsymbol an das Passwort
                	laufPasswort = bisherigesPasswort + _symbole[zaehler];
                	startZeit = System.nanoTime();
                	// Hier wird das Passwort geprueft
                	_gefunden = passwordCompare(laufPasswort.toCharArray(), _passwort);
                	endZeit = System.nanoTime();
                	// Berechnet Zeitdifferenz fuer aktuelles Passwort
                	zeiten[zaehler] = endZeit - startZeit;
            	}
            	// haengt an das bisherige Passwort dasjenige Symbol an, fuer welches die groesste Zeit gebraucht wurde
            	bisherigesPasswort += _symbole[gibIndexVonMaximum(zeiten)];
        	}
    	}
    	while (!_gefunden);
    	return bisherigesPasswort;
	}

	/**
 	* Bestimmt die Laenge eines Passworts
 	* @return die Laenge eines Passworts
 	*/
	private int findePasswortLaenge()
	{
    	String passwort = "";
    	long[] zeiten = new long[_maxPasswortLaenge];
    	long startZeit;
    	long endZeit;
    	for (int zaehler = 1; zaehler < _maxPasswortLaenge; zaehler++)
    	{
        	passwort.concat("a");
        	startZeit = System.nanoTime();
//        	System.out.println("Startzeit: " + startZeit);
        	// prueft, ob das Passwort korrekt ist
        	passwordCompare(passwort.toCharArray(), _passwort);
        	endZeit = System.nanoTime();
//        	System.out.println("Endzeit: " + endZeit);
        	zeiten[zaehler] = endZeit - startZeit;
        	System.out.println("Zeitdifferenz: " + zeiten[zaehler]);
    	}
    	return gibIndexVonMaximum(zeiten);
	}

	/**
 	* Gibt den Index des hoechsten Wertes zurueck
 	* @param array Eingabe, das nach dem hoechsten Wert durchsucht werden soll
 	* @return den Index des hoechsten Wertes
 	*/
	private int gibIndexVonMaximum(long[] array)
	{
    	int index = 0;
    	for (int i = 0; i < array.length; i++)
    	{
        	if (array[i] > array[index])
        	{
            	index = i;
        	}
    	}
    	return index;
	}

	/**
 	* Vergleicht zwei Char-Arrays
 	* @param a erstes Array
 	* @param b zweites Array
 	* @return true, wenn sie gleich sind; false sonst
 	*/
	boolean passwordCompare(char[] a, char[] b)
	{
    	int i;
    	if (a.length != b.length) return false;
    	for (i = 0; i < a.length && a[i] == b[i]; i++)
        	;
    	return i == a.length;
	}
}
\end{lstlisting}
\end{document}