\documentclass[twoside]{article}
\usepackage[utf8]{inputenc}
\usepackage[T1]{fontenc}
\usepackage[ngerman]{babel}
\usepackage[margin=2.5cm]{geometry}
\usepackage{hyperref}
\usepackage{lastpage}
\usepackage{lipsum}

% import math packages
\usepackage{amsmath}
\usepackage{amsfonts}
\usepackage{amssymb}
\usepackage{amsthm}
% contradiction lightning
\usepackage{stmaryrd}
% algorithms and pseudo code
\usepackage{algorithmic}
\usepackage{algorithm}
% formatting and layout
\usepackage{color}
\usepackage{fancyhdr}
% settings
\usepackage{perpage}
\MakePerPage{footnote}
% quotation marks
\usepackage[
    left = ``,%
    right = '',%
    leftsub = `,%
    rightsub = '%
]{dirtytalk}
% custom text colors
\definecolor{pblue}{rgb}{0.13,0.13,1}
\definecolor{pgreen}{rgb}{0,0.5,0}
% macro commands
%% requires color package and the custom colors defined here 
\newcommand{\todo}[1]{
	% add TODO to table of contents
	\addcontentsline{toc}{subsubsection}{TODO: #1}
	\textcolor{pgreen}{ % color TODO green
		\texttt{ % make TODO have fancy font
			\-\\ \-\\//TODO: #1\-\\ \-\\
		}
	}
}
\newcommand{\sect}[4]{	%	custom section (level 1)
	\newpage
	\addcontentsline{toc}{section}{Zettel #1 (#2)}
	\section*{Blatt Nr. #1 (Ausgabe: #2, Abgabe: #3)}
	\begin{center}
		\begin{LARGE}
			#4
		\end{LARGE}
	\end{center}
	\label{sec:#1}
}
\newcommand{\sus}[1]{
	\addcontentsline{toc}{subsection}{Aufgabe #1}
	\subsection*{Übungsaufgabe #1}
	\label{ssec:#1}
}
\newcommand{\sss}[1]{
	\addcontentsline{toc}{subsubsection}{Teilaufgabe #1}
	\subsubsection*{Aufgabe #1}
	\label{sssec:#1}
}
\newcommand{\points}[1]{
	\begin{flushright}
		\begin{Large}
			[~~~~\string| ~#1~]
		\end{Large}
	\end{flushright}
}
\newcommand{\tit}[1]{\textit{#1}}
\newcommand{\ttt}[1]{\texttt{#1}}


\newcommand{\command}[1]{\say{\texttt{#1}}}
% tikz
\usepackage{tikz}
\usetikzlibrary{arrows,automata,positioning}
\usepackage{pgf}

%#+-------------------------------------------------+#%
%#+						VARIABLEN			     	+#%
%#+-------------------------------------------------+#%

\begin{document}

%% Fach-Daten
\newcommand{\fachname}{Projekt Network Security}
\newcommand{\fachnummer}{InfB-Proj}
\newcommand{\veranstaltungsnummer}{64-185}
\newcommand{\stinegruppe}{$ $}
\newcommand{\termin}{Donnerstag, 12.00 - 18.00\\F-027}
%% Gruppenmitglied 1
\newcommand{\memOneName}{Utz Pöhlmann}
\newcommand{\memOneMail}{4poehlma@informatik.uni-hamburg.de}
\newcommand{\memOneNr}{6663579}
%% Gruppenmitglied 2
\newcommand{\memTwoName}{Louis Kobras}
\newcommand{\memTwoMail}{4kobras@informatik.uni-hamburg.de}
\newcommand{\memTwoNr}{6658699}
%% Gruppenmitglied 3
\newcommand{\memThreeName}{}
\newcommand{\memThreeMail}{}
\newcommand{\memThreeNr}{}
%% Gruppenmitglied 4
\newcommand{\memFourName}{}
\newcommand{\memFourMail}{}
\newcommand{\memFourNr}{}
%% Datum
\newcommand{\datum}{\today\\}






%#+-------------------------------------------------+#%
%#+						FORMATIERUNG		     	+#%
%#+-------------------------------------------------+#%

\newcommand{\fach}{
	\begin{Huge}
		\fachname\\
	\end{Huge}
	\begin{LARGE}
		Modul: \fachnummer\\
		Veranstaltung: \veranstaltungsnummer\\
	\end{LARGE}
}

\newcommand{\gruppe}{
	\begin{LARGE}
		\stinegruppe\\
	\end{LARGE}
	\begin{Large}
		\termin\\
	\end{Large}
}

\newcommand{\memberOfGroup}[3]{
	\begin{center}
		\begin{Large}
			#1
		\end{Large}\-\\
		#2\-\\
		#3\-\\
	\end{center}
	\vspace{.5cm}
}
\newcommand{\datumf}{
	\begin{Large}
		\datum\-\\
	\end{Large}
}


% setting up footers and headers
%% setting page style so that footers and headers can be used
\pagestyle{fancy}
%% overwrite default values
\fancyhead{}
\fancyfoot{}
%% first group member in upper left corner
\lhead{\memOneName\-\\\memOneNr}
%% upper center (empty)
\chead{04. Mai 2016\\}
%% second group member in upper right corner
\rhead{\memTwoName\-\\\memTwoNr}
%% lower left corner (empty)
\lfoot{}
%% subject name in lower center
\cfoot{\fachname}
%% lower right corner (empty)
\rfoot{}
%% sets the page number to appear in either the lower left or lower right corner,
%% depending on whether the page number is even or odd, in the format "page X of Y"
\fancyfoot[LE,RO]{Seite \thepage ~von \pageref{LastPage}}
%% thin separation line over the footer
\renewcommand{\footrulewidth}{0.4pt}


%#+-------------------------------------------------+#%
%#+						DECKBLATT  			     	+#%
%#+-------------------------------------------------+#%

%\thispagestyle{empty}
%\-\vspace{0.5cm}
%\begin{center}
%	\fach
%	\vspace{1.5cm}
%	\gruppe
%	\vspace{1.5cm}
%	% group members
%	\memberOfGroup{\memOneName}{\memOneMail}{\memOneNr}
%	\memberOfGroup{\memTwoName}{\memTwoMail}{\memTwoNr}
%	\memberOfGroup{}{}{}
%	\memberOfGroup{}{}{}
%	% 1 cm to next element
%	\vspace{1cm}
%	\datumf
%	
%\end{center}
%\newpage



\sect{2}{14. April 2016}{04. Mai 2016}{Kennwortsicherheit}
\pagenumbering{arabic}
























%#+-------------------------------------------------+#%
%#+						ZETTEL 1					+#%
%#+-------------------------------------------------+#%

\sus{1. Sicherheit lokaler Rechner}
\sss{1.1 Zugriff auf /etc/passwd und /etc/shadow des Webservers}
Überblick über die VM.
\begin{itemize}
	\item \tit{Blatt2-Admin-PC.vmwarevm} wurde aus \tit{/home/vmware} nach \tit{/home/ss16g07/vmware} kopiert
	\item wurde über \tit{\underline{F}ile -> \underline{O}pen...} importiert (die virtuelle Festplatte wurde eingelesen)
	\item VM wurde gestartet; beim Boot wurde danach gefragt, ob die VM kopiert oder verschoben wurde; nach Aufgabe wurde \say{kopiert} ausgewählt
\end{itemize}
Booten von der CD
\begin{itemize}
	\item \tit{grml}-iso wurde aus \tit{/home/vmware} nach \tit{/home/ss16g07/vmware} kopiert
	\item Neues Image wurde in den VM-Einstellungen in das CD-Laufwerk eingelegt
	\item Ebenfalls unter den VM-Einstellungen wurde das CD-Laufwerk verbunden
	\item Während des Bootens der VM wurde das BIOS aufgerufen, um sicherzustellen, dass von der CD gebootet wird
\end{itemize}
Einlesen und durchsuchen der Root-Partition
\begin{itemize}
	\item durch drücken von \tit{d} und \tit{Enter} wurde das deutscha tastaturlayout ausgewählt
	\item mit \tit{mount -r /dev/sda1} wurde die FEstplatte gemountet
	\item unter \tit{/etc/fstab} wurde herausgefunden, dass das Verzeichnis nun unter \tit{/mnt/sda1} zugreifbar ist
	\item die Dateien \tit{passwd} und \tit{shadow} wurden mit \tit{cat \$Dateiname} geöfnnet:
	\begin{itemize}
		\item \tit{passwd} enthält Einträge der folgenden Form: \cite{1}
		\begin{itemize}
			\item \tit{\$Nutzername}:x\footnote{Das \tit{x} indiziert, dass ein verschlüsseltes Passwort für diesen Nutzer existiert}:\tit{\$Nutzer ID}:\tit{\$Gruppen ID}:\tit{\$Nutzer ID Info}:\tit{\$home Verzeichnis}:\tit{\$Shell}
		\end{itemize}
		\item \tit{shadow} enthält Einträge der folgenden Form: \cite{2}
		\begin{itemize}
			\item \tit{\$Nutzername}:\tit{\$verschlüsseltes Password}:\tit{\$Tag der letzen Passwortänderung}\footnote{in Tagen seit dem 1. Jan 1970}:\tit{\$minimaler Zeitabstand zwischen Passwortänderungen}:\tit{\$maximaler Zeitabstand zwischen Passwortänderungen}:\tit{\$Warnungszeitraum für auslaufende Passwörter}:\tit{\$Zeit nach der ein Password ausläuft nach Inaktivität des Accounts}:\tit{\$Zeit die seit der Inaktivität des Accounts vergangen ist}:
		\end{itemize}
	\end{itemize}
	\item es gibt die Benutzer \tit{webadmin} und \tit{georg}
	\item durch Eingabe von \tit{cat group|grep \$Benutzername} wurden die Gruppen der Nutzer herausgefunden:
	\begin{itemize}
		\item georg
		\begin{itemize}
			\item admin
			\item georg
		\end{itemize}
		\item webadmin
		\begin{itemize}
			\item adm
			\item dialout
			\item cdrom
			\item plugdev
			\item lpadmin
			\item webadmin
			\item sambashare
		\end{itemize}
	\end{itemize}
\end{itemize}
\sss{1.2 Auslesen von Kennwörter}
\begin{itemize}
	\item salting: Hinzufügen einer zufälligen Zeichenkette (\say{salt})
	\item hashing: Umrechnung der Daten in Hash-Werte\footnote{Werte fester Länge, typischerweise hexadezimal codiert \cite{3}}
\end{itemize}
Installieren und Verwendung von \tit{john}
\begin{itemize}
	\item John wurde installiert mit \tit{apt-get install john}, es konnte jedoch nicht authentifiziert werden
	\item Einfaches Ausführen von \tit{john} zeigt die Hilfe-Seite
	\item Eingabe des Befehls \tit{john --incremental --users=webadmin /mnt/sda1/etc/shadow}, um das Passwort von \tit{webadmin} im \tit{incremental}-Mode zu ermitteln
	\item Nach 5 Minuten wurde eine manuelle Terminierung durchgeführt
\end{itemize}
Wörterbuchangriff
\begin{itemize}
	\item es wurde in das home Verzeichnis navigiert damit wieder Schreibzugriff besteht
	\item mit \tit{wget http:download.openwall.net/pub/wordlists/all.gz} wurde ein Wörterbuch runtergeladen
	\item durch \tit{gunzip all.gz} wurde das Wörterbuch entpackt
	\item und mit \tit{john --wordlist=all --users=webadmin /mnt/sda1/etc/shadow} der Angriff gestartet
	\item nach 21.01 sec war das Passwort herausgefunden: mockingbird
\end{itemize}
\sss{1.3 Setzen von neuen Kennwörtern}
\begin{itemize}
	\item Das Passwort von georg ist nicht ohne weiteres ermittelbar, weil es wahrscheinlich nicht im Wörterbuch steht
	\item zum unmounten von sda1 wurde \tit{umount /dev/sda1} eingegeben
	\item zum erneuten mounten wurde \tit{mount -w /dev/sda1} eingegeben
	\item zum Ändern des root Verzeichnsises mit shell Wechsel wurde \tit{chroot /mnt/sda1/ /bin/sh} eigegeben
	\item nun wurde das Passwort von georg auf \tit{1} gesetzt: \tit{passwd georg 1}
	\item es wurde das sytem durch \tit{exit} gefolgt von \tit{shutdown -r now} neu gestartet und sich als georg eingeloggt
\end{itemize}



\sus{2. Sichere Speicherung von Kennwörtern}
\sss{2.1 Angriffe mit Hashdatenbanken und Rainbow-Tables}
\begin{itemize}
	\item es wurde in das home Verzeichnis von webadmin navigiert durch \tit{cd /home/webadmin/}
	\item Wechseln in das Unterverzeichnis \tit{Rainbowtables/rcracki}
	\item Ausführung von \tit{rcracki} mit \tit{./rcracki <table-path> -l <password-file>}
	\item es konnten nicht alle Passwörter ermittelt werden.Vermutlich weil nicht alle Passwörter in der benutzten RainbowTable codiert waren
	\item eigene Programme sind immer gut, weil man weiß, was sie können, dementprecehnd dauert es aber auch lange, viel Umfang einzbauen
	\item diese Speicherung würde (da jedes Passwort einen Hash der Länge 128 Bit\cite{wikimd5} generiert) $\Sigma_{i=1}^{7} 128^i$ Bit $\hat{\approx}$ 71 Terabyte verbrauchen, während eine der gegebenen Rainbowtables nur ca. 40 MB groß ist
\end{itemize}
\sss{2.2 Eigener Passwort-Cracker}
\sss{2.3 Eigene Kennwort-Speicherfunktion in Java}

\sus{3. Forensische Wiederherstellung von Kennwörtern}

\sus{4. Unsicherer Umgang mit Passwörtern in Java}









\begin{thebibliography}{1}
\bibitem{1} http://www.cyberciti.biz/faq/understanding-etcpasswd-file-format/
\bibitem{2} http://www.cyberciti.biz/faq/understanding-etcshadow-file/
\bibitem{3} http://zeitstempel.hauke-laging.de/hashinfo.php
\bibitem{wikimd5} http://de.wikipedia.org/wiki/Message-Digest\textunderscore Algorithm\textunderscore 5
\end{thebibliography}

\end{document}