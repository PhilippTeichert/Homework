\documentclass[twoside]{article}

\usepackage[ngerman]{babel}
\usepackage[utf8]{inputenc}
\usepackage[T1]{fontenc}

\usepackage{fancyhdr}

\usepackage[margin=2.54cm]{geometry}

\usepackage{listings}

\usepackage{xcolor}

\usepackage{hyperref}

\newcommand{\say}[1]{%
	``#1''%
}
\newcommand{\ttt}[1]{%
	\texttt{#1}%
}

%%%%%%%%%%%%%%%%%%%%%%%%%%%
% Source code inclusion
\definecolor{pblue}{rgb}{0.13,0.13,1}
\definecolor{pgreen}{rgb}{0,0.5,0}
\lstset{ %
language=Java,   							% choose the language of the code
basicstyle=\small\ttfamily,  				% the size of the fonts that are used for the code
numbers=left,                   			% where to put the line-numbers
numbersep=5pt,                  			% how far the line-numbers are from the code
backgroundcolor=\color{light-light-gray},   % choose the background color. You must add
frame=lrtb,           						% adds a frame around the code
tabsize=4,          						% sets default tabsize to 2 spaces
captionpos=b,           					% sets the caption-position to bottom
breaklines=true,        					% sets automatic line breaking
xleftmargin=1.5cm,							% space from the left paper edge
commentstyle=\color{pgreen},
keywordstyle=\color{pblue},
literate=%
    {Ö}{{\"O}}1
    {Ä}{{\"A}}1
    {Ü}{{\"U}}1
    {ß}{{\ss}}1
    {ü}{{\"u}}1
    {ä}{{\"a}}1
    {ö}{{\"o}}1
    {~}{{\textasciitilde}}1
}
\renewcommand{\lstlistingname}{Code}
\definecolor{light-light-gray}{gray}{0.95}


\begin{document}
\pagestyle{fancy}
\fancyhead{}
\fancyfoot{}
\fancyhead[L]{Louis Kobras\\6658699}
\fancyhead[R]{Utz Pöhlmann\\6663579}
\fancyfoot[RE,LO]{Seite \thepage}

\begin{center}
\begin{Huge}
\textbf{SVS Bachelor-Projekt Network Security}
\end{Huge}\\\-\\
\begin{Large}
\textbf{Blatt 3: Datenkommunikation}
\end{Large}\\\-\\
\begin{minipage}[t]{0.48\textwidth}
\begin{center}\textbf{
	Louis Kobras\\
	6658699}
\end{center}
\end{minipage}
\begin{minipage}[t]{0.48\textwidth}
\begin{center}\textbf{
	Utz Pöhlmann\\
	6663579}
\end{center}
\end{minipage}
\end{center}

\section{HTTP}
\subsection{[local]}
Funktioniert so nicht, da die Website über das HTTPS-Protokoll läuft, welches SSL erfodert, welches wiederum nicht von telnet unterstützt wird.
Alternativ kann OpenSSL verwendet werden.\\
\\
Es konnte ein Stylesheet ausgelesen werden; (s. \cite{svscss}).\\
Kommentar: Zeilenumbrüche schaden nicht.


\section{SMPT (Mail Spoofing)}
\subsection{[local]}
\begin{itemize}
	\item Verbinden zum Mailserver mit \ttt{telnet mailhost.informatik.uni-hamburg.de 25} (25 ist der Port des Servers)
	\item Eingabe folgender Befehle:
	\begin{itemize}
		\item \ttt{EHLO svs-labwall.informatik.uni-hamburg.de}
		\item \ttt{MAIL FROM: <svsg07@informatik.uni-hamburg.de>}
		\item \ttt{RCPT TO: <4kobras@informatik.uni-hamburg.de>}
		\item \ttt{DATA}
	\end{itemize}
	\item folgender Mail-Text: \ttt{``Hier könnte Ihr Inhalt stehen''}
	\item Quelltextvergleich mit ``echter'' Mail ergab fehlende Konfigurationsinformationen
	\item Mangel nicht offensichtlich, kann durch Ergänzung der obigen Eingabe zwischen \ttt{DATA} und dem Mail-Text angepasst werden
	\item Fehlende Informationen:
	\begin{itemize}
		\item \ttt{MIME-Version: 1.0}
		\item \ttt{Content-Type: text/plain;\\
				\-~~~~~~~~charset=ISO-8895-1;\\
				\-~~~~~~~~DelSp="Yes";\\
				\-~~~~~~~~format="flowed"}
		\item \ttt{Content-Disposition: inline}
		\item \ttt{Content-Transfer-Encoding: 7bit}
		\item \ttt{User-Agent: Internet Messaging Program (IMP) H3 (4.1.5)}
	\end{itemize}
	\item Absender \ttt{gmail.com} ebenfalls möglich; folgender Eintrag ist zu modifizieren:
	\begin{itemize}
		\item \ttt{EHLO google.com}
	\end{itemize}
	\item Fehlerfall: Shell terminiert \ttt{telnet}-Ausführung mit dem Kommentar \ttt{``I can break rules, too. Goodbye.''}
\end{itemize}



\section{License Server (DNS-Spoofing)}
\subsection{[local]}
Protokoll:
\begin{enumerate}
	\item Key als User-Input
	\item Übermitteln des Keys an den Server
	\item Rückgabe vom Server, ob Key gültig oder nicht (\ttt{SERIAL\textunderscore VALID=0} bzw. \ttt{SERIAL\textunderscore VALID=1})
	\item[4a] Wenn gültig, Dank für Kauf
	\item[4b] Wenn nicht gültig, FBI ist unterwegs
\end{enumerate}
\subsection{}
\begin{itemize}
	\item Verhindern der Kommunikation der Software mit dem echten Auth-Server
	\item Geschehen durch Erweitern des Hosts um \ttt{127.0.0.1    license-server.svslab} in \ttt{/etc/hosts}
	\item Herunterladen der Java-Klasse \ttt{TCPClient.java} (\cite{javatcp})
	\item Manipulieren des Servers: \ttt{ServerSocket} auf svslab-Port (1337) gesetzt
	\item Manipulieren des Servers: Rückgabe des Servers auf statisch \ttt{\say{SERIAL\textunderscore VALID=1}} gesetzt
	\item $\Longrightarrow$ alle Keys gültig, unabhängig von Eingabe
\end{itemize} 
\subsection{}
Es gibt zwei anmerkbare Mängel.
\begin{enumerate}
	\item Es sollte nicht angegeben werden, ob die Serial-Länge korrekt ist.
	\item Es könnte mithilfe einer eindeutigen Signatur o.Ä. eine Abfrage an den Server eingebunden werden, ob er \say{echt} ist (gehasht).
\end{enumerate}


\section{License Server (Brute-Force-Angriff)}
\subsection{[local]}
Das Programm funktioniert an sich, wenn man aber an den Server sendet, kriegt man (scheinbar nach Zufall) entweder \say{invalid command} oder \say{invalid length} zurück, bei Eingabe von \ttt{serial=abcdefgh} ($a,b,c,d,e,f,g,h \in \{0,1,...,9\}$).

Wir baten zwei Gruppen neben uns um Hilfe, jedoch konnten diese uns auch nicht weiterhelfen bzw. haben keinen Fehler in unserem Programm gefunden.

Als Ausgangspunkt wurde die Java-Klasse \ttt{TCPClient.java} (source: \cite{javatcp}) genommen.

Einige gültige Keys:\\
\begin{center}
\begin{minipage}[t]{0.23\textwidth}
	\begin{itemize}
		\item 03133700
		\item 06264700
		\item 09401100
		\item 15668500
	\end{itemize}
\end{minipage}
\begin{minipage}[t]{0.23\textwidth}
	\begin{itemize}
		\item 18802200
		\item 21935900
		\item 25069600
		\item 31337000
	\end{itemize}
\end{minipage}
\begin{minipage}[t]{0.23\textwidth}
	\begin{itemize}
		\item 47005500
		\item 59540300
		\item 62674000
		\item 72075100
	\end{itemize}
\end{minipage}
\begin{minipage}[t]{0.23\textwidth}
	\begin{itemize}
		\item 87743600
		\item 90877300
		\item 94011000
		\item 97144700
	\end{itemize}
\end{minipage}
\end{center}

\subsection{}
Möglichkeiten, sich zu verteidigen, enthalten, sind jedoch nicht beschränkt auf:
\begin{itemize}
	\item Sperren des Absenders der Auth-Anfrage nach $n$ Fehlversuchen (Unterbrechen von Brute-Force-Attacken)\footnote{Je nach Art der Sperrung ist dies lediglich eine Bremse; wird z.B. nur die IP gesperrt, kann diese resettet werden, um wieder Zugang zu erlangen.}
	\item Prüfung der IP bzw. Prüfsumme, ob Empfänger und Absender korrekt sind (Zurechenbarkeit)
	\item Limitieren der Eingabe auf $k$ pro Minute (Verlangsamen von Brute-Force-Attacken)
\end{itemize}
\subsection{[local]}
Alle gefundenen gültigen Schlüssel sind durch 100 teilbar, liegen also in der Form $xxxxxx00$ vor.
Ist ein Schlüssel außderdem durch 1000 teilbar, also in der Form $xxxxx000$, so ist ebenso ein Schlüssel der Form $0xxxxx00$ gültig, der die gleiche Ziffernfolge anstelle der $x$ enthält.


\section{Implementieren eines TCP-Chats}
\subsection{[local]}
Es werden zwei URLs in Fragmenten gesendet.
Zusammengesetzt sehen sie folgendermaßen aus:
\begin{itemize}
	\item $[$URL1$]$ \texttt{http://www.oracle.com/technetwork/java/socket-140484.html}
	\item $[$URL2$]$ \texttt{http://code.google.com/p/example-of-servlet/source/browse/trunk/}\\\texttt{src/javaLanguage/basic/?r=56\#basic\%2Fsocket}
\end{itemize}
Da UDP ein unzuverlässiges Protokoll ist, war es wie erwartet nötig, einige Zeit zu warten, bis alle Fragmente empfangen wurden (jeweils 4).
\subsection{}
\subsection{}
\subsection{}
\subsection{}

\begin{thebibliography}{1}
\bibitem{svscss}	\url{https://www.inf.uni-hamburg.de/assets/application-11e3b49e605\\ff8ba1f01d275bd36850edfdfc1fbbb8c22e55fae1baf643a00d0.css}
\bibitem{javatcp}	\url{https://systembash.com/a-simple-java-tcp-server-and-tcp-client/}
\end{thebibliography}
\newpage
\section*{Anhang 1: Quelltext zu Aufgabe 3}
\subsubsection*{Klasse GeneratorTool.java}
\begin{lstlisting}
import java.util.ArrayList;

public class GeneratorTool {
	// Halterung für die Walzen
	private ArrayList<Integer> _walzen;
	// Speicher für das aktuelle Passwort
	private String _passwort;
	// Halterung für alle gültigen Symbole//
	private ArrayList<String> _charListe;
	// Liste aller gültigen Symbole
	private final String _symbols;
	// Walze, die zur Zeit die letzte ist, welche bearbeitet wird.
	private int _aktuelleWalze;

	/**
	 * Konstruktor. Probiert automatisch alle Passwörter durch
	 */
	public GeneratorTool() {
		_walzen = new ArrayList<Integer>();
		setupWalzenListe();
		_passwort = "";
		_symbols = "0123456789";
		_charListe = new ArrayList<String>();
		setupCharListe();
		_aktuelleWalze = 0;
	} // end Konstruktor

	/**
	 * Iteriert über die Walzen, bis die Abbruchbedingung erfüllt ist oder
	 * alle Werte ausprobiert wurden
	 */
	public void findeEinPasswort() {
		_aktuelleWalze = 7;
		_passwort = "";
		if (_aktuelleWalze != _walzen.size()) {
			tick(_walzen.get(_aktuelleWalze));
			// System.out.println(_passwort);
		}
	} // end findePasswort()

	/**
	 * Dreht die gegebene Walze um ein Feld weiter. Bei Überlauf wird die Walze
	 * zurückgesetzt und die nächste Walze rekursiv aufgerufen. Anschließend
	 * wird das zum aktuellen Walzenstand gehörende Passwort generiert.
	 * 
	 * @param walze
	 *            der Stand der aktuellen Walze
	 * @param walzenIndex
	 *            der Index der aktuellen Walze (wichtig zum Ändern des Wertes
	 *            im Walzenarray)
	 */
	private void tick(int walze) {
		walze += 1;
		_walzen.set(_aktuelleWalze, walze); // setzt den Wert der Walze auch im
		// Array
		if (walze == _symbols.length()) {
			_aktuelleWalze = rolleWalze(_aktuelleWalze); // setzt die nächste
			// Walze weiter
			// walze = _walzen.get(walzenIndex);
			if (_aktuelleWalze < _walzen.size()) {
				walze = 0; // setzt die aktuelle Walze zurück
				_walzen.set(_aktuelleWalze, walze); // setzt den Wert der Walze
				// auch im Array
			}
		}
		if (_aktuelleWalze < _walzen.size()) {
			for (int w : _walzen) {
				if (w >= 0) {
					_passwort += getSymbol(w);
				}
			}
		}
	} // end tick(int)

	/**
	 * Wird aufgerufen, sobald eine Walte komplett durchgedreht hat. Setzt die
	 * nächste Walze einen Index weiter. Methode: Walze 1 dreht durch. Walze 1
	 * wird wieder auf Wert 1 gesetzt. Walze 2 dreht durch. Walze 2 wird wieder
	 * auf Wert 1 gesetzt und Walze 1 wird einen Wert weiter gesetzt. Walte 2
	 * dreht durch. usw.
	 * 
	 * @param walzenIndex
	 *            der Index der aktuell durch gedrehten Walze
	 * @return der Index der neuen aktuellen Walze
	 */
	private int rolleWalze(int walzenIndex) {
		if (walzenIndex == 0) {
			if (_walzen.get(walzenIndex) >= _symbols.length() - 1) {
				_walzen.set(walzenIndex, 0);
				return walzenIndex + 1; // gehe zur nächsten Walze
			} else {
				_walzen.set(walzenIndex, _walzen.get(walzenIndex) + 1);
				return walzenIndex;
			}
		} else if (0 < walzenIndex && walzenIndex < _walzen.size()) {
			if (_walzen.get(walzenIndex) >= _symbols.length() - 1) {
				_walzen.set(walzenIndex, 0);
				return rolleWalze(walzenIndex - 1) + 1; // setze letzte Walze +1
			} else {
				_walzen.set(walzenIndex, _walzen.get(walzenIndex) + 1);
				return walzenIndex;
			}
		}
		return -1; // Hier kommt das Programm nie an
	}

	/**
	 * Holt ein Symbol aus der Symbolliste
	 * 
	 * @param index
	 *            der Index des Symbols
	 * @return das Symbol
	 */
	private String getSymbol(int index) {
		return "" + _symbols.charAt(index);
	} // end getSymbol()

	/**
	 * setzt die gültigen Symbole in einer Liste auf
	 */
	private void setupCharListe() {
		for (int i = 0; i < _symbols.length(); i++) {
			_charListe.add(i, _symbols.substring(i, i + 1));
		}
	} // end setupCharListe()

	/**
	 * setzt die Walzen auf und sortiert sie in einer Liste
	 */
	private void setupWalzenListe() {
		int walze0 = 0;
		int walze1 = 0;
		int walze2 = 0;
		int walze3 = 0;
		int walze4 = 0;
		int walze5 = 0;
		int walze6 = 0;
		int walze7 = -1;
		_walzen.add(0, walze0);
		_walzen.add(1, walze1);
		_walzen.add(2, walze2);
		_walzen.add(3, walze3);
		_walzen.add(4, walze4);
		_walzen.add(5, walze5);
		_walzen.add(6, walze6);
		_walzen.add(7, walze7);
	} // end setupWalzenListe()

	/**
	 * gettermethode für _passwort
	 * 
	 * @return _passwort
	 */
	public String getPasswort() {
		return _passwort;
	} // end get_passwort()

} // end class
\end{lstlisting}
\subsubsection*{Klasse TCPClient.java}
\begin{lstlisting}
import java.io.BufferedReader;
import java.io.DataOutputStream;
import java.io.InputStreamReader;
import java.net.Socket;
import java.util.Scanner;

class TCPClient {

	public static boolean _keyGefunden;

	public static void main(String argv[]) throws Exception {
		GeneratorTool gt = new GeneratorTool();
		_keyGefunden = false;

		String sentence;
		String modifiedSentence;
		Socket clientSocket = new Socket("localhost", 1337);
		do {
		DataOutputStream outToServer = new DataOutputStream(
				clientSocket.getOutputStream());
		BufferedReader inFromServer = new BufferedReader(new InputStreamReader(
				clientSocket.getInputStream()));

			//gt.findeEinPasswort();
			//sentence = "serial=".concat(gt.getPasswort());
			sentence = "help";
			System.out.println(sentence);
			//outToServer.writeBytes(sentence);
			outToServer.writeBytes(sentence + '\n');
			//outToServer.writeBytes(new Scanner(System.in).nextLine());
			try{Thread.sleep(500);}catch(Exception e){}
			modifiedSentence = inFromServer.readLine();
			if (modifiedSentence.contains("SERIAL_VALID=1")
					|| sentence.equals(""))
				_keyGefunden = true;
			System.out.println("FROM SERVER: " + modifiedSentence);
		} while (!_keyGefunden);
		clientSocket.close();
	}
}
\end{lstlisting}
\subsubsection*{Klasse TCPServer.java}
\begin{lstlisting}
import java.io.*;
import java.net.*;

class TCPServer
{
   public static void main(String argv[]) throws Exception
      {
         String clientSentence;
         String capitalizedSentence;
         ServerSocket welcomeSocket = new ServerSocket(1337);

         while(true)
         {
            Socket connectionSocket = welcomeSocket.accept();
            BufferedReader inFromClient =
               new BufferedReader(new InputStreamReader(connectionSocket.getInputStream()));
            DataOutputStream outToClient = new DataOutputStream(connectionSocket.getOutputStream());
            clientSentence = inFromClient.readLine();
            System.out.println("Received: " + clientSentence);
            capitalizedSentence = clientSentence.toUpperCase() + '\n';
            outToClient.writeBytes("SERIAL_VALID=1");
         }
      }
}
\end{lstlisting}
\section*{Anhang 2: Quelltext zu Aufgabe 4}
\subsubsection*{Klasse GeneratorTool.java}
wie in Aufgabe 3
\subsubsection*{Klasse TCPClient.java}
wie in Aufgabe 3
\section*{Anhang 3: Quelltext zu Aufgabe 5}
\subsection*{Aufgabe 5.1}
\subsubsection*{Klasse UDPClient.java}
\begin{lstlisting}
package blatt03.fuenf.eins;

import java.net.DatagramPacket;
import java.net.DatagramSocket;

public class UDPClient {
	public static void main(String args[]) throws Exception {
		while(true){
		DatagramSocket clientSocket = new DatagramSocket(9999);
		byte[] receiveData = new byte[1024];
		DatagramPacket receivePacket = new DatagramPacket(receiveData,
				receiveData.length);
		clientSocket.receive(receivePacket);
		String modifiedSentence = new String(receivePacket.getData());
		System.out.println("FROM SERVER:" + modifiedSentence + "\n");
		clientSocket.close();
	}}
}
\end{lstlisting}
\subsection*{Aufgabe 5.2}
\subsubsection*{Klasse ClientWorker.java}
\begin{lstlisting}
package blatt03.fuenf.zwei;

import java.awt.event.ActionEvent;
import java.io.BufferedReader;
import java.io.IOException;
import java.io.InputStreamReader;
import java.io.PrintWriter;
import java.net.Socket;
import java.net.UnknownHostException;

import javax.swing.JTextArea;

class ClientWorker implements Runnable {
	  private Socket _client;
	  private JTextArea _textArea;
		private SocketClientUI _ui;
		private Socket _socket;
		private BufferedReader _in = null;
		private PrintWriter _out = null;

	//Constructor
	  ClientWorker(Socket client, JTextArea textArea) {
	    _client = client;
	    _textArea = textArea;
	  }
	  
	  public void listenSocket(){
			
			//Create socket connection
			   try{
			     _socket = new Socket("kq6py", 4444);
			     _out = new PrintWriter(_socket.getOutputStream(), 
			                 true);
			     _in = new BufferedReader(new InputStreamReader(
			                _socket.getInputStream()));
			   } catch (UnknownHostException e) {
			     System.out.println("Unknown host: kq6py");
			     System.exit(1);
			   } catch  (IOException e) {
			     System.out.println("No I/O");
			     System.exit(1);
			   }
			}
		
		public void actionPerformed(ActionEvent event){
			   Object source = event.getSource();

			   if(source == _ui.getButton()){
			//Send data over socket
			      String text = _ui.getTextArea().getText();
			      _out.println(text);
			      _ui.getTextArea().setText(new String(""));
			      _out.println(text);
			   }
			//Receive text from server
			   try{
			     String line = _in.readLine();
			     System.out.println("Text received: " + line);
			   } catch (IOException e){
			     System.out.println("Read failed");
			     System.exit(1);
			   }
			} 

	  public void run(){
	    String line;
	    BufferedReader in = null;
	    PrintWriter out = null;
	    try{
	      in = new BufferedReader(new 
	        InputStreamReader(_client.getInputStream()));
	      out = new 
	        PrintWriter(_client.getOutputStream(), true);
	    } catch (IOException e) {
	      System.out.println("in or out failed");
	      System.exit(-1);
	    }

	    while(true){
	      try{
	        line = in.readLine();
	//Send data back to client
	        out.println(line);
	//Append data to text area
	        appendText(line);
	       }catch (IOException e) {
	        System.out.println("Read failed");
	        System.exit(-1);
	       }
	    }
	  }
	  
	  public synchronized void appendText(String line){
		    _textArea.append(line);
		  }
	  
	  protected void finalize(){
		//Objects created in run method are finalized when
		//program terminates and thread exits
		     try{
		        _socket.close();
		    } catch (IOException e) {
		        System.out.println("Could not close socket");
		        System.exit(-1);
		    }
		  }
	}
\end{lstlisting}
\subsubsection*{Klasse Server.java}
\begin{lstlisting}
package blatt03.fuenf.zwei;

import java.awt.event.ActionEvent;
import java.io.BufferedReader;
import java.io.IOException;
import java.io.InputStreamReader;
import java.io.PrintWriter;
import java.net.ServerSocket;
import java.net.Socket;

public class Server {

	private SocketServerUI _ui;
	private String _line;

	public static void main(String args[]) {
		new Server().listenSocket();
	}

	public Server() {
		_ui = new SocketServerUI();

	}

	public void listenSocket() {
		ServerSocket server = null;
		Socket client = null;
		System.out.println("1");
		try {
			server = new ServerSocket(4444);
		} catch (IOException e) {
			System.out.println("Could not listen on port 4444");
			System.exit(-1);
		}
		System.out.println("2");
		try {
			client = server.accept();
		} catch (IOException e) {
			System.out.println("Accept failed: 4444");
			System.exit(-1);
		}
		System.out.println("3");
		try {
			new BufferedReader(new InputStreamReader(
					client.getInputStream()));
			new PrintWriter(client.getOutputStream(), true);
		} catch (IOException e) {
			System.out.println("Read failed");
			System.exit(-1);
		}
		System.out.println("4");

		while(true){
		    ClientWorker w;
		    //server.accept returns a client connection
			      System.out.println("5");
			      w = new ClientWorker(client, _ui.getTextArea());
			      Thread t = new Thread((Runnable) w);
			      t.start();
			      try{Thread.sleep(1000);}catch(Exception e){}
		  }
	}

	public void actionPerformed(ActionEvent event) {
		Object source = event.getSource();

		if (source == _ui.getButton()) {
			_ui.getTextArea().setText(_line);
		}
	}
}
\end{lstlisting}
\subsubsection*{Klasse SocketClient.java}
\begin{lstlisting}
package blatt03.fuenf.zwei;

import java.awt.event.ActionEvent;
import java.io.BufferedReader;
import java.io.IOException;
import java.io.InputStreamReader;
import java.io.PrintWriter;
import java.net.Socket;
import java.net.UnknownHostException;

public class SocketClient {

	private SocketServerUI _ui;
	private Socket _socket;
	private BufferedReader _in = null;
	private PrintWriter _out = null;

	public static void main(String args[]) {
		new SocketClient().listenSocket();
	}

	public SocketClient() {
		_ui = new SocketServerUI();

	}
	
	public void listenSocket(){
		
		//Create socket connection
		   try{
		     _socket = new Socket("kq6py", 4321);
		     _out = new PrintWriter(_socket.getOutputStream(), 
		                 true);
		     _in = new BufferedReader(new InputStreamReader(
		                _socket.getInputStream()));
		   } catch (UnknownHostException e) {
		     System.out.println("Unknown host: kq6py");
		     System.exit(1);
		   } catch  (IOException e) {
		     System.out.println("No I/O");
		     System.exit(1);
		   }
		}
	
	public void actionPerformed(ActionEvent event){
		   Object source = event.getSource();

		   if(source == _ui.getButton()){
		//Send data over socket
		      String text = _ui.getTextArea().getText();
		      _out.println(text);
		      _ui.getTextArea().setText(new String(""));
		      _out.println(text);
		   }
		//Receive text from server
		   try{
		     String line = _in.readLine();
		     System.out.println("Text received: " + line);
		   } catch (IOException e){
		     System.out.println("Read failed");
		     System.exit(1);
		   }
		}  
}
\end{lstlisting}
\subsubsection*{Klasse SocketClientUI.java}
\begin{lstlisting}
package blatt03.fuenf.zwei;

import java.awt.BorderLayout;

import javax.swing.JButton;
import javax.swing.JFrame;
import javax.swing.JTextArea;

public class SocketClientUI {
	private JFrame _frame;
	private JTextArea _text;
	private JButton _button;
	public SocketClientUI(){
		_frame = new JFrame();
		_text = new JTextArea();
		_button = new JButton("Senden");
		_frame.setDefaultCloseOperation(JFrame.EXIT_ON_CLOSE);
		
		
		
		_frame.setLayout(new BorderLayout());
		_frame.add(_text, BorderLayout.CENTER);
		_frame.add(_button, BorderLayout.SOUTH);
		_frame.pack();
		_frame.setVisible(true);
	}
	public JTextArea getTextArea() {
		return _text;
		
	}
	public JButton getButton() {
		return _button;
		
	}
}
\end{lstlisting}
\subsubsection*{Klasse SocketServer.java}
\begin{lstlisting}
package blatt03.fuenf.zwei;

import java.awt.event.ActionEvent;
import java.io.BufferedReader;
import java.io.IOException;
import java.io.InputStreamReader;
import java.io.PrintWriter;
import java.net.ServerSocket;
import java.net.Socket;

public class SocketServer {

	private SocketServerUI _ui;
	private String _line;

	public static void main(String args[]) {
		new SocketServer().listenSocket();
	}

	public SocketServer() {
		_ui = new SocketServerUI();

	}

	public void listenSocket() {
		ServerSocket server = null;
		Socket client = null;
		BufferedReader in = null;
		PrintWriter out = null;
		
		try {
			server = new ServerSocket(4321);
		} catch (IOException e) {
			System.out.println("Could not listen on port 4321");
			System.exit(-1);
		}
		try {
			client = server.accept();
		} catch (IOException e) {
			System.out.println("Accept failed: 4321");
			System.exit(-1);
		}
		try {
			in = new BufferedReader(new InputStreamReader(
					client.getInputStream()));
			out = new PrintWriter(client.getOutputStream(), true);
		} catch (IOException e) {
			System.out.println("Read failed");
			System.exit(-1);
		}

		while (true) {
			try {
				_line = in.readLine();
				// Send data back to client
				out.println(_line);
			} catch (IOException e) {
				System.out.println("Read failed");
				System.exit(-1);
			}
		}
	}

	public void actionPerformed(ActionEvent event) {
		Object source = event.getSource();

		if (source == _ui.getButton()) {
			_ui.getTextArea().setText(_line);
		}
	}
}
\end{lstlisting}
\subsubsection*{Klasse SocketServerUI.java}
\begin{lstlisting}
package blatt03.fuenf.zwei;

import java.awt.BorderLayout;

import javax.swing.JButton;
import javax.swing.JFrame;
import javax.swing.JTextArea;

public class SocketServerUI {

	private JFrame _frame;
	private JTextArea _text;
	private JButton _button;
	public SocketServerUI(){
		_frame = new JFrame();
		_text = new JTextArea();
		_button = new JButton("Empfangen");
		_frame.setDefaultCloseOperation(JFrame.EXIT_ON_CLOSE);
		
		
		
		_frame.setLayout(new BorderLayout());
		_frame.add(_text, BorderLayout.CENTER);
		_frame.add(_button, BorderLayout.SOUTH);
		_frame.pack();
		_frame.setVisible(true);
	}
	public JTextArea getTextArea() {
		return _text;
		
	}
	public JButton getButton() {
		return _button;
		
	}
	
	public void setText(String text){
		_text.setText(_text.getText() + "\n" + text);
		_frame.pack();
	}
}
\end{lstlisting}
\end{document}