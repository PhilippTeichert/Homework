\documentclass[twoside]{article}

\usepackage[ngerman]{babel}
\usepackage[utf8]{inputenc}
\usepackage[T1]{fontenc}

\usepackage{fancyhdr}

\usepackage[margin=2.54cm]{geometry}

\newcommand{\say}[1]{%
	``#1''%
}
\newcommand{\ttt}[1]{%
	\texttt{#1}%
}


\begin{document}
\pagestyle{fancy}
\fancyhead{}
\fancyfoot{}
\fancyhead[L]{Louis Kobras\\6658699}
\fancyhead[R]{Utz Pöhlmann\\6663579}
\fancyfoot[RE,LO]{Seite \thepage}

\begin{center}
\begin{Huge}
\textbf{SVS Bachelor-Projekt Network Security}
\end{Huge}\\\-\\
\begin{Large}
\textbf{Blatt 3: Datenkommunikation}
\end{Large}\\\-\\
\begin{minipage}[t]{0.48\textwidth}
\begin{center}\textbf{
	Louis Kobras\\
	6658699}
\end{center}
\end{minipage}
\begin{minipage}[t]{0.48\textwidth}
\begin{center}\textbf{
	Utz Pöhlmann\\
	6663579}
\end{center}
\end{minipage}
\end{center}

\section{HTTP}
\subsection{}
\begin{itemize}
	\item hat nicht auf Pings reagiert (VM)
	\item RealOS: Ping an IP 134.100.56.130 erfolgreich, \ttt{telnet} fehlgeschlagen
\end{itemize}



\section{SMPT (Mail Spoofing)}
\subsection{}
\# zwei Sätze zum Topic\\
\# Input Protokoll\\
\# Kommentar erste Mail vs. echte Mail\\
\# Modifikation bei zweiter Mail



\section{License Server (DNS-Spoofing)}
\subsection{}
Protokoll:
\begin{enumerate}
	\item Key als User-Input
	\item Übermitteln des Keys an den Server
	\item Rückgabe vom Server, ob Key gültig oder nicht (\ttt{SERIAL\textunderscore VALID=0} bzw. \ttt{SERIAL\textunderscore VALID=1})
	\item[4a] Wenn gültig, Dank für Kauf
	\item[4b] Wenn nicht gültig, FBI ist unterwegs
\end{enumerate}
\subsection{}
\begin{itemize}
	\item Verhindern der Kommunikation der Software mit dem echten Auth-Server
	\item Geschehen durch Erweitern des Hosts um \ttt{127.0.0.1    license-server.svslab} in \ttt{/etc/hosts}
	\item Herunterladen der Java-Klasse \ttt{TCPClient.java}\footnote{https://systembash.com/a-simple-java-tcp-server-and-tcp-client/}
	\item Manipulieren des Servers: \ttt{ServerSocket} auf svslab-Port (1337) gesetzt
	\item Manipulieren des Servers: Rückgabe des Servers auf statisch \ttt{\say{SERIAL\textunderscore VALID=1}} gesetzt
	\item $\Longrightarrow$ alle Keys gültig, unabhängig von Eingabe
\end{itemize} 
\subsection{}
Es gibt zwei anmerkbare Mängel.
\begin{enumerate}
	\item Es sollte nicht angegeben werden, ob die Serial-Länge korrekt ist.
	\item Es könnte mithilfe einer eindeutigen Signatur o.Ä. eine Abfrage an den Server eingebunden werden, ob er \say{echt} ist (gehasht).
\end{enumerate}


\section{License Server (Brute-Force-Angriff)}
\subsection{}
Das Programm funktioniert an sich, wenn man aber an den Server sendet, kriegt man (scheinbar nach Zufall) entweder \say{invalid command} oder \say{invalid length} zurück, bei Eingabe von \ttt{serial=abcedfgh} ($a,b,c,d,e,f,g,h \in \{0,1,...,9\}$).

Wir baten zwei Gruppen neben uns um Hilfe, jedoch konnten diese uns auch nicht weiterhelfen bzw. haben keinen Fehler in unserem Programm gefunden.

Als Ausgangspunkt wurde die Java-Klasse \ttt{TCPClient.java} von [todo:link] genommen.

gültige Keys:
\begin{itemize}
	\item 90877300
\end{itemize}
\subsection{}
Möglichkeiten, sich zu verteidigen, enthalten, sind jedoch nicht beschränkt auf:
\begin{itemize}
	\item Sperren des Absenders der Auth-Anfrage nach $n$ Fehlversuchen (Unterbrechen von Brute-Force-Attacken)\footnote{Je nach Art der Sperrung ist dies lediglich eine Bremse; wird z.B. nur die IP gesperrt, kann diese resettet werden, um wieder Zugang zu erlangen.}
	\item Prüfung der IP bzw. Prüfsumme, ob Empfänger und Absender korrekt sind (Zurechenbarkeit)
	\item Limitieren der Eingabe auf $k$ pro Minute (Verlangsamen von Brute-Force-Attacken)
\end{itemize}
\subsection{}



\section{Implementieren eines TCP-Chats}
\subsection{}
\subsection{}
\subsection{}
\subsection{}
\subsection{}


\end{document}