\documentclass{article}
\usepackage[utf8]{inputenc}
\usepackage[T1]{fontenc}
\usepackage[ngerman]{babel}
\usepackage[margin=2.5cm]{geometry}

% import math packages
\usepackage{amsmath}
\usepackage{amsfonts}
\usepackage{amssymb}
\usepackage{amsthm}
% contradiction lightning
\usepackage{stmaryrd}

\usepackage{perpage}
\MakePerPage{footnote}

\begin{document}
\thispagestyle{empty}
\-\vspace{2cm}
\begin{center}
\begin{Huge}
Algorithmen und Datenstrukturen
\end{Huge}\\
\vspace{2cm}
\begin{LARGE}
Übungsgruppe 14
\end{LARGE}\\
\vspace{2cm}
\begin{Large}
Utz Pöhlmann
\end{Large}\\
4poehlma@informatik.uni-hamburg.de\\
6663579\\
\vspace{1cm}
\begin{Large}
Louis Kobras
\end{Large}\\
4kobras@informatik.uni-hamburg.de\\
6658699\\
\vspace{1cm}
\begin{Large}
Paul Testa
\end{Large}\\
paul.testa@gmx.de\\
6251548\\
\vspace{2.5cm}
\today\\
\vspace{2.5cm}
\textbf{Punkte für den Hausaufgabenteil:}\\
\vspace{1cm}
\begin{tabular}{c|c|c|c|c|c}
~~~~&~~~~&~~~~&~~~~&~~~~&$\Sigma$	\\	\hline
	&	 &	  &	   &	&

\end{tabular}
\end{center}

\newpage
\pagenumbering{arabic}
\section{Zettel vom 14.-16. Oktober // Abgabe: N/A}
\subsection{Präsenzaufgabe 1.1}
%---------------%
%	Aufgabe		%
%---------------%
Wiederholen Sie die \textit{O}-Notation und die verwandten Notationen.
Wie sind die einzelnen Mengen definiert?
Was bedeutet es, wenn $f \in O(g)$ gilt, was wenn $f \in \Theta (g)$ gilt und so weiter?\\
\vspace{1cm}
%---------------%
%	Bearbeitung	%
%---------------%
\begin{equation*}
	\begin{array}{llll}
		O(g(n)): 		& f(n) \in O(g(n))		&\Leftrightarrow \exists c \in \mathbb{R}^+ \exists n_0 \in \mathbb{N} \forall n >= n_0 : &\|f(n)\| <= c \cdot \|g(n)\|\\
		o(g(n)): 		& f(n) \in o(g(n))		&\Leftrightarrow \forall c \in \mathbb{R}^+ \exists n_0 \in \mathbb{N} \forall n >= n_0 : &\|f(n)\| <= c \cdot \|g(n)\|\\
		\Omega(g(n)):	& f(n) \in \Omega(g(n)) &\Leftrightarrow \exists c \in \mathbb{R}^+ \exists n_0 \in \mathbb{N} \forall n >= n_0 : &\|f(n)\| >= c \cdot \|g(n)\|\\
		\omega(g(n)):	& f(n) \in \omega(g(n)) &\Leftrightarrow \forall c \in \mathbb{R}^+ \exists n_0 \in \mathbb{N} \forall n >= n_0 : &\|f(n)\| >= c \cdot \|g(n)\|\\
		\Theta(g(n)):	& f(n) \in \Theta(g(n)) &\Leftrightarrow \exists c_1, c_2 \in \mathbb{R}^+ \exists n_0 \in \mathbb{N} \forall n >= n_0 : &c_1 \cdot \|g(n)\| <= \|f(n)\| <= c_2 \cdot \|g(n)\|
	\end{array}
\end{equation*}
\subsection{Präsenzaufgabe 1.2}
%---------------%
%	Aufgabe		%
%---------------%
Beweisen Sie:
\begin{itemize}
	\item $n^2+3n-5 \in O(n^2)$
	\item $n^2-2n \in \Theta(n^2)$
	\item $n! \in O((n+1)!)$
\end{itemize}
Gilt im letzten Fall auch $n! \in o((n+1)!)$?\\
\vspace{1cm}
%---------------%
%	Bearbeitung	%
%---------------%
\begin{equation*}
\begin{array}{rl}
	f(n) \in O(g(n)) &\Leftrightarrow lim_{n\rightarrow \infty} \frac{f(n)}{g(n)} < \infty\\
	f(n) &= n^2+3n-5\\
	g(n) &= n^2\\
	\frac{f(n)}{g(n)} &= \frac{n^2+3n-5}{n^2}
	\vspace{0.5cm}\\
	lim_{n \rightarrow \infty} \frac{n^2+3n-5}{n^2} &= lim_{n \rightarrow \infty} 1+\frac{3}{n}-\frac{5}{n^2}\\
		&=1+\frac{3}{\infty}-\frac{5}{\infty^2}\\
		&=1+0+0\\
		&=1 < \infty \Rightarrow f(n) \in O(g(n))
\end{array}
\end{equation*}
\begin{flushright}
$\square$
\end{flushright}
\vspace{1cm}
\begin{equation*}
\begin{array}{rl}
	c_1, c_2 \in \mathbb{R}^+,n_0 \in \mathbb{N} \forall n >= n_0:	& c_1 \cdot n^2<= n^2-2n <= c_2 \cdot n^2\\
		\Leftrightarrow & c_1 <= 1-\frac{1}{n} <= c_2
\end{array}
\end{equation*}
Dies ist erfüllbar ab $n_0 >= 2$ , da für $n=1$ im mittleren Ausdruck 0 herauskommt und $c_1$ größer als 0, aber kleiner als der mittlere Ausdruck sein muss.
Ist $n >= 2$, so kommt im mittleren Ausdruck $0,5$ heraus, für $c_1$ lässt sich ein beliebiger Wert aus $\string]0;0.5\string[$ wählen, sei es an dieser Stelle $\frac{1}{4}$.
Als Obergrenze für $c_2$ lässt sich jeder Wert größer oder gleich 1 wählen, da der mittlere Ausdruck nicht größer als 1 werden kann und somit die Bedingung des "kleiner gleich" sofort erfüllt ist.\\
Somit wird als Ergebnis für die Belegung gewählt: $c_1 = \frac{1}{4}; c_2 = 1; n_0 = 2$.
Mit dieser Belegung gilt $n^2-2n \in \Theta(n^2)$
\begin{flushright}
$\square$
\end{flushright}
\vspace{1cm}
\begin{equation*}
\begin{array}{rl}
	f(n) \in O(g(n)) &\Leftrightarrow lim_{n\rightarrow \infty} \frac{f(n)}{g(n)} < \infty\\
	f(n) &= n!\\
	g(n) &= (n+1)! = (n+1) \cdot n!
	\vspace{0.5cm}\\
	lim_{n\rightarrow \infty} \frac{n!}{(n+1) \cdot n!} &= lim_{n \rightarrow \infty} \frac{1}{n+1}\\
		&= \frac{1}{\infty}\\
		&= 0 < \infty \Rightarrow f(n) \in O(g(n))
\end{array}
\end{equation*}
Da die Bedingung für $o(g(n))$ ist, dass der Quotient nicht nur kleiner unendlich, sondern gleich null ist, was hier wie oben gezeigt gegeben ist, gilt auch $n! \in o((n+1)!)$.
\begin{flushright}
$\square$
\end{flushright}
\subsection{Präsenzaufgabe 1.3}
%---------------%
%	Aufgabe		%
%---------------%
Beweisen oder widerlegen Sie:
\begin{enumerate}
	\item $ f(n),g(n) \in O(h(n)) \Rightarrow f(n)+g(n) \in O(h(n)) $
	\item $ f(n),g(n) \in O(h(n)) \Rightarrow f(n) \cdot g(n) \in O(h(n)) $
\end{enumerate}
\vspace{1cm}
%---------------%
%	Bearbeitung	%
%---------------%
\[
	\exists c_1 \in \mathbb{R}^+ \exists n_{0_1} \in \mathbb{N} \forall n >= n_{0_1} : \|f(n)\| <= c_1 \cdot \|h(n)\|
\]
\[
	\exists c21 \in \mathbb{R}^+ \exists n_{0_2} \in \mathbb{N} \forall n >= n_{0_2} : \|g(n)\| <= c_2 \cdot \|h(n)\|
\]
\[
	n_0 = max(n_{0_1},n_{0_2})
\]
\[
	\|f(n)+g(n)\| <= c_1\cdot\|h(n)\|+c_2\cdot\|h(n)\| <= (c_1+c_2)\cdot\|h(n)\|
\]
\vspace{0.5cm}\\
Seien $f(n)$ und $g(n)$ Polynome zweiten Grades sowie $h(n)$ ein Polynom dritten Grades.
Dann sind sowohl $f(n)$ als auch $g(n)$ durch die \textit{limes}-Bedingung in $O(h(n))$.
Das Produkt zweier Polynome zweiten Grades ist allerdings ein Polynom vierten Grades, sodass gilt:
\begin{equation*}
	\lim_{n\rightarrow\infty}\frac{n^2 \cdot n^2}{n^3}=\lim_{n\rightarrow\infty}\frac{n^4}{n^3}=\lim_{n\rightarrow\infty}n=\infty
\end{equation*}
Damit ist das Produkt der Polynome nicht mehr in $O(h(n))$, da die \textit{limes}-Bedingung, nach der der Quotient der Polynome für \textit{n} gegen Unendlich kleiner als Unendlich sein zu hat, nicht erfüllt ist.
Damit ist (2) widerlegt.
\begin{flushright}
$\square$
\end{flushright}
\newpage
\section{Zettel vom 15.10. // Abgabe: 26.10.}
\subsection{Übungsaufgabe 2.1}
\begin{flushright}
\begin{Large}
[~~~~\string| ~~2~]
\end{Large}
\end{flushright}
%---------------%
%	Aufgabe		%
%---------------%
Begründen Sie formal, warum folgende Größenabschätzungen gelten bzw. nicht gelten:
\begin{enumerate}
	\item $3n^3-6n+20 \in O(n^3)$
	\item $n^2 \cdot \operatorname{log} n \in O(n^3) \cap \Omega(n^2)$
\end{enumerate}
\vspace{1cm}
%---------------%
%	Bearbeitung	%
%---------------%
\subsubsection{}
\[
	3n^3-6n+20 \in O(n^3) \Leftrightarrow lim_{n\rightarrow\infty}\frac{3n^3-6n+20}{n^3} < \infty
\]
\[
	lim_{n\rightarrow\infty}\frac{3n^3-6n+20}{n^3} = lim_{n\rightarrow\infty} \frac{3n^3}{n^3}-\frac{6n}{n^3}+\frac{20}{n^3}=lim_{n\rightarrow\infty}3-\frac{6}{n^2}+\frac{20}{n^3}=3-0+0<\infty
\]
\[
	\Rightarrow 3n^3-6n+20 \in O(n^3)  ~~~~~~~~\square
\]
\subsubsection{}
\[
	n^2 \cdot \operatorname{log} n \in O(n^3) \cap \Omega(n^2) \Leftrightarrow lim_{n\rightarrow\infty} \frac{n^2 \cdot \operatorname{log} n}{n^3} < \infty \land lim_{n\rightarrow\infty} \frac{n^2 \cdot \operatorname{log} n}{n^2} > 0
\]
\[
	\frac{n^2 \cdot \operatorname{log} n}{n^2} = \frac{1 \cdot \operatorname{log} n}{1} = \operatorname{log} n > 0~ \forall n > 1 \Rightarrow n^2 \cdot \operatorname{log} n \in \Omega(n^2)
\]
\[
	lim_{n\rightarrow\infty} \frac{n^2 \cdot \operatorname{log} n}{n^3} = lim_{n\rightarrow\infty} \frac{\operatorname{log} n}{n} \overset{\text{l'H}}{=} lim_{n\rightarrow\infty}\frac{1}{n} \cdot \frac{1}{1} = lim_{n\rightarrow\infty} \frac{1}{n} = \frac{1}{\infty} = 0 \Rightarrow n^2\cdot\operatorname{log} n \in  O(n^3)
\]
\[
	\Rightarrow n^2 \cdot \operatorname{log} n \in O(n^3) \cap \Omega(n^2)~~~~\square
\]
\subsection{Übungsaufgabe 2.2}
\begin{flushright}
\begin{Large}
[~~~~\string| ~~4~]
\end{Large}
\end{flushright}
%---------------%
%	Aufgabe		%
%---------------%
Ordnen Sie die folgenden Funktionen nach ihrem Wachstumsgrad in aufsteigender Reihenfolge, d.h. folgt eine Funktion $g(n)$ einer Funktion $f(n)$, so soll $f(n) \in O(g(n))$ gelten.
\[
	n, \operatorname{log} n, n^2, n^{\frac{1}{2}}, \sqrt{n}^3, 2^n, \operatorname{ln} n, 1000
\]
Mit log ist hier der Logarithmus zur Basis 2, mit ln der natürliche Logarithmus (Basis \textit{e}) gemeint.
Begründen Sie stets Ihre Aussage.
Zwei Funktionen $f(n)$ und $g(n)$ befinden sich ferner in der selben Äquivalenzklasse, wenn $f(n) \in \Theta (g(n))$ gilt.
Geben Sie an, welche Funktionen sich in derselben Äquivalenzklasse befinden und begründen Sie auch hier ihre Aussage.\\
\vspace{1cm}\\
%---------------%
%	Bearbeitung	%
%---------------%
Die bearbeitete Menge wird i.F. als $M_F$ bezeichnet.
Die Menge, die gerade alle Elemente von $M_F$ in aufsteigend sortierter Reihenfolge enthält, wird als $M_F'$ bezeichnet.\\
$M_F$ wird mit \textsc{InsertionSort} in $M_F'$ hineinsortiert.\\ \vspace{0.5cm}
Sei $e \in M_F$.
Für $e$ wird das Element $1000$ gewählt.
Da $\string|M_F'\string|$ leer ist, muss $1000$ nicht weiter geprüft werden.
\begin{flushright}
	$M_F' = \{1000\}$
\end{flushright}
$e$ wird nun über $M_F$ iteriert, bis $M_F' = Sorted(M_F)$. \vspace{0.5cm}\\
$e = n$
\begin{equation*}
	\begin{array}{llll}
		f(n) = n	\\
		g(n) = 1000	&	\underset{n\rightarrow\infty}{lim}	\frac{n}{1000}	&=	\infty	\Rightarrow n > 1000 &\Rightarrow n \not \in O(1000)
	\end{array}
\end{equation*}
\begin{flushright}
	$M_F' = \{1000,n\}$
\end{flushright}
\vspace{0.6cm}
$e = \operatorname{log} n$
\begin{equation*}
	\begin{array}{llll}
		.
	\end{array}
\end{equation*}
\begin{flushright}
	$M_F' = \{1000,\operatorname{log}n,n\}$
\end{flushright}
\vspace{0.6cm}
$e = 4$
\begin{equation*}
	\begin{array}{llll}
		.
	\end{array}
\end{equation*}
\begin{flushright}
	$M_F' = \{4,1000,\operatorname{log}n,n\}$
\end{flushright}
\vspace{0.6cm}
$e = n^2$
\begin{equation*}
	\begin{array}{llll}
		.
	\end{array}
\end{equation*}
\begin{flushright}
	$M_F' = \{4,1000,\operatorname{log}n,n,n^2\}$
\end{flushright}
\vspace{0.6cm}
$e = n^{\frac{1}{2}}$
\begin{equation*}
	\begin{array}{llll}
		.
	\end{array}
\end{equation*}
\begin{flushright}
	$M_F' = \{4,1000,n^{\frac{1}{2}},\operatorname{log}n,n,n^2\}$
\end{flushright}
\vspace{0.6cm}
$e = \sqrt{n}^3$
\begin{equation*}
	\begin{array}{llll}
		.
	\end{array}
\end{equation*}
\begin{flushright}
	$M_F' = \{4,1000,n^{\frac{1}{2}},\operatorname{log}n,n,\sqrt{n}^3,n^2\}$
\end{flushright}
\vspace{0.6cm}
$e = 2^n$
\begin{equation*}
	\begin{array}{llll}
		.
	\end{array}
\end{equation*}
\begin{flushright}
	$M_F' = \{4,1000,n^{\frac{1}{2}},\operatorname{log}n,n,\sqrt{n}^3,n^2,2^n\}$
\end{flushright}
\vspace{0.6cm}
$e = \operatorname{ln}n$
\begin{equation*}
	\begin{array}{llll}
		.
	\end{array}
\end{equation*}
\begin{flushright}
	$M_F' = \{4,1000,\operatorname{ln} n, \operatorname{log} n, n^{\frac{1}{2}}, n, \sqrt{n}^3, n^2, 2^n\}$
\end{flushright}
In der selben Äquivalenzklasse befinden sich zum einen $4$ und $1000$ und zum anderen $log(n)$ und $ln(n)$. Die restlichen Werte sind jeweils alleine in ihrer Äquivalenzklasse.
\\
\subsection{Übungsaufgabe 2.3}
\begin{flushright}
\begin{Large}
[~~~~\string| ~~2~]
\end{Large}
\end{flushright}
%---------------%
%	Aufgabe		%
%---------------%
Beweisen oder widerlegen Sie:
\[
	f(n),g(n) \in O(h(n)) \Rightarrow f(n)\cdot g(n) \in O((h(n))^2)
\]
\vspace{1cm}\\
%---------------%
%	Bearbeitung	%
%---------------%
Für diesen Beweis wird der Beweis des dritten Satzes der Summen- und Produkteigenschaften der O-Notation\footnote{vgl. Vorlesung, Foliensatz 1 (14.10.), S.33} zu Hilfe genommen:
\begin{proof}
Sei $f\in O(h_1)$ und $g\in O(h_2)$, dann gibt es ein $c$, $n_0$, so dass $f(n) \leq c \cdot h_1(n) \forall n \geq n_0$ und ebenso $c', n_0'$, so dass $g(n') \leq c' \cdot h_2(n') \forall n' \geq n_0'$.
Daraus folgt $f(n'')\cdot g(n'') \leq c \cdot c' \cdot h_1(n'') \cdot h_2(n'') \forall n'' \geq max(n_0, n_0')$, also $f \cdot g \in O(h_1 \cdot h_2)$.
\end{proof}
Setzt man nun $h_1, h_2 = h$ folgt daraus für den letzten Ausdruck des Beweises $f(n) \cdot g(n) \in O(h(n) \cdot h(n)) \Rightarrow f(n) \cdot g(n) \in O((h(n))^2)$.
\subsection{Übungsaufgabe 2.4}
\begin{flushright}
\begin{Large}
[~~~~\string| ~~8~]
\end{Large}
\end{flushright}
%---------------%
%	Aufgabe		%
%---------------%
Seien
\begin{enumerate}
	\item 
	\[
		T(n) :=  \begin{cases}
					\begin{array}{ll}
						0, & \text{für }n=0\\
						3 \cdot T(n-1)+2, &\text{sonst}
					\end{array}
				\end{cases}
	\]
	\item
	\[
		S(n) := \begin{cases}
					\begin{array}{ll}
						c,			& \text{für }n=1\\
						16 \cdot S(\frac{n}{4})+n^2, & \text{sonst}
					\end{array}
				\end{cases}
	\]
\end{enumerate}
Rekurrenzgleichungen (\textit{c} ist dabei eine Konstante).
\vspace{0.2cm}\\
Bestimmen Sie wie in der Vorlesung jeweils die Größenordnung der Funktion $T: \mathbb{N} \rightarrow \mathbb{N}$ einmals mittels der (a) Substitutionsmethode und einmal mittes des (b) Mastertheorems.
Ihre Ergebnisse sollten zumindest hinsichtlich der O-Notation gleich sein, so dass Sie etwaige Rechenfehler entdecken können!
Führen Sie bei (a) auch den Induktionsbeweis, der in der Vorlesung übersprungen wurde!
\vspace{1cm}\\
%---------------%
%	Bearbeitung	%
%---------------%
\begin{enumerate}
    \item[1.]
    \begin{enumerate}
        \item [a)]
        \[
            \begin{array}{lll}
                T(n)~&=~3 \cdot T(n-1)+2 &\\
                &=~3*(3*T(n-2)+2)+2~&=~3^2*T(n-2)+3^2-1\\
                &=~3^2*(3*T(n-3)+2)+8~&=~3^3*T(n-3)+3^3-1\\
                &=~...&\\
                &=~3^k*T(n-k)+3^k-1
            \end{array}
        \]
        \\
        Wir kommen auf eine sinnvolle Verallgemeinerung der Formel.\\
        Beweis der Formel durch vollständige Induktion:\\\\
        Induktionsanfang: $T(0)$ gilt nach Definiton.\\
        Induktionsschritt: Sei $n \in \mathbb{N}$ (s. Aufgabenstellung). Wir nehmen an, dass $T(n)$ gilt (Induktionsannahme) und zeigen $T(n+1)$. Es gilt\\
        \[
            \begin{array}{ll}
                T(n)~&=~3*T(n-1)+2\\
                T(n+1)~&=~3*T(n+1-1)+2\\
                &=~3*T(n)+2
                \\\\
                T(n)~&=~3^k*T(n-k)+3^k-1\\
                T(n+1)~&=~3^k*T(n+1-k)+3^k-1
            \end{array}
        \]
        Das zeigt $T(n+1)$.\\
        Damit sind der Induktionsanfang und der Induktionsschritt bewiesen. Es folgt, dass $T(n)$ für alle $n \in \mathbb{N}$ gilt.\\
        \\
        Da die Rekursion bei $T(0)~=~0$, also $n-k~=~0$ abbricht, wird mit $k~=~n$ weiter gerechnet.\\
        \[
            \begin{array}{ll}
                T(n)~&=~3^k*T(n-k)+3^k-1\\
                &=~3^n*T(n-n)+3^n-1\\
                &=~3^n*T(0)+3^n-1\\
                &=~3^n*0+3^n-1\\
                &=~3^n-1 \in \Theta (3^n)
            \end{array}
        \]
        \item [b)]
        Das Mastertheorem ist auf Aufgabe 1. nicht anwendbar, da die Form\\
        \[
		T(n) :=  \begin{cases}
					\begin{array}{ll}
						c, & \text{falls }n=1\\
						a \cdot T(\frac{n}{b})+f(n), &\text{falls }n>1
					\end{array}
				\end{cases}
    	\]
    	\\
    	bei\\
    	\[
		T(n) :=  \begin{cases}
					\begin{array}{ll}
						0, & \text{für }n=0\\
						3 \cdot T(n-1)+2, &\text{sonst}
					\end{array}
				\end{cases}
    	\]
    	\\
    	nicht eingehalten wurde.
    \end{enumerate}
    \item[2.]
    \begin{enumerate}
        \item [a)]
        \[
            \begin{array}{ll}
                S(n)~&=~16 \cdot S(\frac{n}{4})+n^2\\
                &=~16 \cdot S(\frac{16 \cdot S(\frac{n}{4})+n^2}{4})+n^2\\
                &=~16 \cdot S(\frac{16 \cdot S(\frac{16 \cdot S(\frac{n}{4})+n^2}{4})+n^2}{4})+n^2\\
                &=~16 \cdot S(\frac{16 \cdot S(\frac{16 \cdot S(\frac{16 \cdot S(\frac{n}{4})+n^2}{4})+n^2}{4})+n^2}{4})+n^2\\
                &=~16 \cdot S(\frac{16 \cdot S(\frac{16 \cdot S(\frac{16 \cdot S(\frac{16 \cdot S(\frac{n}{4})+n^2}{4})+n^2}{4})+n^2}{4})+n^2}{4})+n^2
            \end{array}
        \]
        \\
        Keine sinnvolle Vereinfachung erkennbar. $=>$ Substitutionsmethode nicht anwendbar.
        \item [b)]
        Die Form\\
        \[
		S(n) :=  \begin{cases}
					\begin{array}{ll}
						c, & \text{falls }n=1\\
						a \cdot T(\frac{n}{b})+f(n), &\text{falls }n>1
					\end{array}
				\end{cases}
    	\]
    	\\
    	ist bei\\
    	\[
		S(n) := \begin{cases}
					\begin{array}{ll}
						c,			& \text{für }n=1\\
						16 \cdot S(\frac{n}{4})+n^2, & \text{sonst}
					\end{array}
				\end{cases}
    	\]
    	\\
    	eingehalten. Das Mastertheorem ist daher anwendbar.\\
    	\\
    	\begin{enumerate}
    	    \item[I.]
    	    $S(n) \in \Theta (n^{log_b(a)})$, falls $f(n) \in O(n^{log_b(a)- \epsilon})$ für ein $\epsilon > 0$.\\
    	    \\
    	    \[
    	        \begin{array}{ll}
    	            f(n) &\in O(n^{log_b(a)- \epsilon})\\
    	            n^2 &\in O(n^{log_4(16)- \epsilon})\\
    	            n^2 &\in O(n^{2- \epsilon})
    	        \end{array}
    	    \]
    	    \\
    	    Hierfür kann kein $\epsilon$ gefunden werden. Daher gilt diese Aussage nicht\\
    	    \item[II.]
    	    $S(n) \in \Theta (n^{log_b(a)} \cdot log_2(n))$, falls $f(n) \in \Theta (n^{log_b(a)})$.\\
    	    \[
    	        \begin{array}{ll}
    	            f(n) &\in O(n^{log_b(a)})\\
    	            n^2 &\in O(n^{log_4(16)})\\
    	            n^2 &\in O(n^2)
    	        \end{array}
    	    \]
    	    \\
    	    Dies stimmt, daher gilt diese Aussage.
    	    \item[III.]
    	    $S(n) \in \Theta (f(n))$, falls $f(n) \in \Omega (n^{log_b(a)+ \epsilon})$ für ein $\epsilon > 0$ \textbf{und} $a \cdot f(\frac{n}{b}) \leq \delta \cdot f(n)$ für ein $\delta < 1$ und große $n$.\\
    	    \\
    	    \[
    	        \begin{array}{ll}
    	            f(n) &\in \Omega (n^{log_b(a)+ \epsilon})\\
    	            n^2 &\in \Omega (n^{log_4(16)+ \epsilon})\\
    	            n^2 &\in \Omega (n^{2+ \epsilon})
    	        \end{array}
    	    \]
    	    \\
    	    Dies stimmt für alle $\epsilon \geq 0$, also auch für mindestens ein $\epsilon > 0$.
    	    \[
    	        \begin{array}{lll}
    	            a \cdot f(\frac{n}{b}) \leq \delta \cdot f(n)~~~~~~~~~~~~~~~~~~~~~~~&\text{\textbackslash einsetzen}\\
    	            16 \cdot (\frac{n}{4})^2 \leq \delta \cdot n^2 &\text{\textbackslash $\sqrt()$}\\
    	            4 \cdot \frac{n}{4} \leq \sqrt{\delta} \cdot n\\
    	            n \leq \sqrt{\delta} \cdot n &\text{\textbackslash :n    (n ist immer positiv, da $n \in \mathbb{N}$, s. Aufgabenstellung)}\\
    	            1 \leq \sqrt{\delta} &\text{\textbackslash $()^2$}\\
    	            1 \leq \delta
    	        \end{array}
    	    \]
    	    \\
    	    Damit ist $\delta \geq 1$ und nicht, wie benötigt, $\delta < 1$. Daher gilt diese Aussage nicht.
    	    \\\\
    	    Da nur II. gilt, gilt $S(n) \in \Theta (n^{log_b(a)} \cdot log_2(n))$, also $S(n) \in \Theta (n^2 \cdot log_2(n))$.
    	\end{enumerate}
    \end{enumerate}
\end{enumerate}
\end{document}