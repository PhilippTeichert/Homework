\documentclass{article}
\usepackage[utf8]{inputenc}
\usepackage[T1]{fontenc}
\usepackage[ngerman]{babel}
\usepackage[margin=2.5cm]{geometry}

\usepackage{amsmath}
\usepackage{amssymb}

\usepackage{perpage}
\MakePerPage{footnote}

\begin{document}
\thispagestyle{empty}
\-\vspace{2cm}
\begin{center}
\begin{Huge}
Algorithmen und Datenstrukturen
\end{Huge}\\
\vspace{2cm}
\begin{LARGE}
Übungsgruppe 14
\end{LARGE}\\
\vspace{2cm}
\begin{Large}
Utz Pöhlmann
\end{Large}\\
4poehlma@informatik.uni-hamburg.de\\
6663579\\
\vspace{1cm}
\begin{Large}
Louis Kobras
\end{Large}\\
4kobras@informatik.uni-hamburg.de\\
6658699\\
\vspace{3cm}
\today\\
\vspace{3cm}
\textbf{Punkte:}\\
\vspace{1cm}
\begin{tabular}{c|c|c|c|c|c}
~~~~&~~~~&~~~~&~~~~&~~~~&$\Sigma$	\\	\hline
	&	 &	  &	   &	&

\end{tabular}
\end{center}

\newpage
\pagenumbering{arabic}
\section{Präsenzteil}
\subsection{Präsenzaufgabe 1.1}
%---------------%
%	Aufgabe		%
%---------------%
Wiederholen Sie die \textit{O}-Notation und die verwandten Notationen.
Wie sind die einzelnen Mengen definiert?
Was bedeutet es, wenn $f \in O(g)$ gilt, was wenn $f \in \Theta (g)$ gilt und so weiter?\\
\vspace{1cm}
%---------------%
%	Bearbeitung	%
%---------------%
\begin{equation*}
	\begin{array}{llll}
		O(g(n)): 		& f(n) \in O(g(n))		&\Leftrightarrow \exists c \in \mathbb{R}^+ \exists n_0 \in \mathbb{N} \forall n >= n_0 : &\|f(n)\| <= c \cdot \|g(n)\|\\
		o(g(n)): 		& f(n) \in o(g(n))		&\Leftrightarrow \forall c \in \mathbb{R}^+ \exists n_0 \in \mathbb{N} \forall n >= n_0 : &\|f(n)\| <= c \cdot \|g(n)\|\\
		\Omega(g(n)):	& f(n) \in \Omega(g(n)) &\Leftrightarrow \exists c \in \mathbb{R}^+ \exists n_0 \in \mathbb{N} \forall n >= n_0 : &\|f(n)\| >= c \cdot \|g(n)\|\\
		\omega(g(n)):	& f(n) \in \omega(g(n)) &\Leftrightarrow \forall c \in \mathbb{R}^+ \exists n_0 \in \mathbb{N} \forall n >= n_0 : &\|f(n)\| >= c \cdot \|g(n)\|\\
		\Theta(g(n)):	& f(n) \in \Theta(g(n)) &\Leftrightarrow \exists c_1, c_2 \in \mathbb{R}^+ \exists n_0 \in \mathbb{N} \forall n >= n_0 : &c_1 \cdot \|g(n)\| <= \|f(n)\| <= c_2 \cdot \|g(n)\|
	\end{array}
\end{equation*}
\subsection{Präsenzaufgabe 1.2}
%---------------%
%	Aufgabe		%
%---------------%
Beweisen Sie:
\begin{itemize}
	\item $n^2+3n-5 \in O(n^2)$
	\item $n^2-2n \in \Theta(n^2)$
	\item $n! \in O((n+1)!)$
\end{itemize}
Gilt im letzten Fall auch $n! \in o((n+1)!)$?\\
\vspace{1cm}
%---------------%
%	Bearbeitung	%
%---------------%
\begin{equation*}
\begin{array}{rl}
	f(n) \in O(g(n)) &\Leftrightarrow lim_{n\rightarrow \infty} \frac{f(n)}{g(n)} < \infty\\
	f(n) &= n^2+3n-5\\
	g(n) &= n^2\\
	\frac{f(n)}{g(n)} &= \frac{n^2+3n-5}{n^2}
	\vspace{0.5cm}\\
	lim_{n \rightarrow \infty} \frac{n^2+3n-5}{n^2} &= lim_{n \rightarrow \infty} 1+\frac{3}{n}-\frac{5}{n^2}\\
		&=1+\frac{3}{\infty}-\frac{5}{\infty^2}\\
		&=1+0+0\\
		&=1 < \infty \Rightarrow f(n) \in O(g(n))
\end{array}
\end{equation*}
\begin{flushright}
$\square$
\end{flushright}
\vspace{1cm}
\begin{equation*}
\begin{array}{rl}
	c_1, c_2 \in \mathbb{R}^+,n_0 \in \mathbb{N} \forall n >= n_0:	& c_1 \cdot n^2<= n^2-2n <= c_2 \cdot n^2\\
		\Leftrightarrow & c_1 <= 1-\frac{1}{n} <= c_2
\end{array}
\end{equation*}
Dies ist erfüllbar ab $n_0 >= 2$ , da für $n=1$ im mittleren Ausdruck 0 herauskommt und $c_1$ größer als 0, aber kleiner als der mittlere Ausdruck sein muss.
Ist $n >= 2$, so kommt im mittleren Ausdruck $0,5$ heraus, für $c_1$ lässt sich ein beliebiger Wert aus $\string]0;0.5\string[$ wählen, sei es an dieser Stelle $\frac{1}{4}$.
Als Obergrenze für $c_2$ lässt sich jeder Wert größer oder gleich 1 wählen, da der mittlere Ausdruck nicht größer als 1 werden kann und somit die Bedingung des "kleiner gleich" sofort erfüllt ist.\\
Somit wird als Ergebnis für die Belegung gewählt: $c_1 = \frac{1}{4}; c_2 = 1; n_0 = 2$.
Mit dieser Belegung gilt $n^2-2n \in \Theta(n^2)$\\
\begin{flushright}
$\square$
\end{flushright}
\vspace{1cm}
\begin{equation*}
\begin{array}{rl}
	f(n) \in O(g(n)) &\Leftrightarrow lim_{n\rightarrow \infty} \frac{f(n)}{g(n)} < \infty\\
	f(n) &= n!\\
	g(n) &= (n+1)! = n \cdot n!
	\vspace{0.5cm}\\
	lim_{n\rightarrow \infty} \frac{n!}{n \cdot n!} &= lim_{n \rightarrow \infty} \frac{1}{n}\\
		&= \frac{1}{\infty}\\
		&= 0 < \infty \Rightarrow f(n) \in O(g(n))
\end{array}
\end{equation*}
Da die Bedingung für $o(g(n))$ ist, dass der Quotient nicht nur kleiner unendlich, sondern gleich null ist, was hier wie oben gezeigt gegeben ist, gilt auch $n! \in o((n+1)!)$.\\
\begin{flushright}
$\square$
\end{flushright}
\subsection{Präsenzaufgabe 1.3}
%---------------%
%	Aufgabe		%
%---------------%
Beweisen oder widerlegen Sie:
\begin{enumerate}
	\item $ f(n),g(n) \in O(h(n)) \Rightarrow f(n)+g(n) \in O(h(n)) $
	\item $ f(n),g(n) \in O(h(n)) \Rightarrow f(n) \cdot g(n) \in O(h(n)) $
\end{enumerate}
\vspace{2cm}
%---------------%
%	Bearbeitung	%
%---------------%
\end{document}