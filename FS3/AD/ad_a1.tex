\documentclass{article}
\usepackage[utf8]{inputenc}
\usepackage[T1]{fontenc}
\usepackage[ngerman]{babel}
\usepackage[margin=2.5cm]{geometry}

\usepackage{amsmath}
\usepackage{amssymb}

\usepackage{perpage}
\MakePerPage{footnote}

\begin{document}
\thispagestyle{empty}
\-\vspace{2cm}
\begin{center}
\begin{Huge}
Algorithmen und Datenstrukturen
\end{Huge}\\
\vspace{2cm}
\begin{LARGE}
Übungsgruppe 14
\end{LARGE}\\
\vspace{2cm}
\begin{Large}
Utz Pöhlmann
\end{Large}\\
4poehlma@informatik.uni-hamburg.de\\
6663579\\
\vspace{1cm}
\begin{Large}
Louis Kobras
\end{Large}\\
4kobras@informatik.uni-hamburg.de\\
6658699\\
\vspace{3cm}
\today\\
\vspace{3cm}
\textbf{Punkte:}\\
\vspace{1cm}
\begin{tabular}{c|c|c|c|c|c}
~~~~&~~~~&~~~~&~~~~&~~~~&$\Sigma$	\\	\hline
	&	 &	  &	   &	&

\end{tabular}
\end{center}

\newpage
\pagenumbering{arabic}
\section{Präsenzteil}
\subsection{Präsenzaufgabe 1.1}
%---------------%
%	Aufgabe		%
%---------------%
Wiederholen Sie die \textit{O}-Notation und die verwandten Notationen.
Wie sind die einzelnen Mengen definiert?
Was bedeutet es, wenn $f \in O(g)$ gilt, was wenn $f \in \Theta (g)$ gilt und so weiter?\\
\vspace{2cm}
%---------------%
%	Bearbeitung	%
%---------------%
\subsection{Präsenzaufgabe 1.2}
%---------------%
%	Aufgabe		%
%---------------%
Beweisen Sie:
\begin{itemize}
	\item $n^2+3n-5 \in O(n^2)$
	\item $n^2-2n \in \Theta(n^2)$
	\item $n! \in O((n+1)!)$
\end{itemize}
Gilt im letzten Fall auch $n! \in o((n+1)!)$?\\
\vspace{2cm}
%---------------%
%	Bearbeitung	%
%---------------%
\begin{equation}
\begin{array}{rl}
	f(n) \in O(g(n)) &\Leftrightarrow lim_{n\rightarrow \infty} \frac{f(n)}{g(n)} < \infty\\
	f(n) &= n^2+3n-5\\
	g(n) &= n^2\\
	\frac{f(n)}{g(n)} &= \frac{n^2+3n-5}{n^2}
	\vspace{1cm}\\
	lim_{n \rightarrow \infty} \frac{n^2+3n-5}{n^2} &= lim_{n \rightarrow \infty} 1+\frac{3}{n}-\frac{5}{n^2}\\
		&=1+\frac{3}{\infty}-\frac{5}{\infty^2}\\
		&=1+0+0\\
		&=1 < \infty \Rightarrow f(n) \in O(g(n))
\end{array}
\end{equation}
\subsection{Präsenzaufgabe 1.3}
%---------------%
%	Aufgabe		%
%---------------%
Beweisen oder widerlegen Sie:
\begin{enumerate}
	\item $ f(n),g(n) \in O(h(n)) \Rightarrow f(n)+g(n) \in O(h(n)) $
	\item $ f(n),g(n) \in O(h(n)) \Rightarrow f(n) \cdot g(n) \in O(h(n)) $
\end{enumerate}
\vspace{2cm}
%---------------%
%	Bearbeitung	%
%---------------%
\end{document}