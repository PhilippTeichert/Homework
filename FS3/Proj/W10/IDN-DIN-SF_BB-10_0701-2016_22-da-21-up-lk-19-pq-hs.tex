% following line for presentation mode
\documentclass{beamer}
%% following line for hand-in
%\documentclass[handout]{beamer}

% language and encoding settings
\usepackage[utf8]{inputenc}
\usepackage[T1]{fontenc}
\usepackage[ngerman]{babel}

% AMS Math packages
\usepackage{amsmath}
\usepackage{amssymb}
\usepackage{amsthm}
\usepackage{amsfonts}

% Presentation Layout
\usetheme{Madrid}
\title{Intense Doge Navigation}
\author{CS3 - Dev Team Alpha}

% formatting and layout
\usepackage{color}
\definecolor{pblue}{rgb}{0.13,0.13,1}
\definecolor{pgreen}{rgb}{0,0.5,0}
% macro commands
%% requires color package and the custom colors defined here 
\newcommand{\todo}[1]{
	\addcontentsline{toc}{subsubsection}{TODO: #1}
	\textcolor{pgreen}{\texttt{\-\\ \-\\//TODO: #1\-\\ \-\\}}
	}

%%%%%%%%%%%%%%%%%%%%%%%%%%%%%%%%%%%%
%	TITLE slide
%	Copy for following presentations
%	remember to adjust the hand-in date
%
\begin{document}
\frame{
	\thispagestyle{empty}
	\begin{center}
		\begin{Large}
			\textbf{Projektmanagement WS15/16}\\
		\end{Large}
		\vspace{0.3cm}
		\textbf{Woche 10}\\
		\vspace{0.5cm}
		\begin{tabular}{cc}
			\begin{tabular}{c}
				Louis Kobras\\
				\textbf{Matr.Nr:} 6658699\\
				\textbf{Email:} 4kobras@inf.
			\end{tabular}
			&
			\begin{tabular}{c}
				Utz Pöhlmann\\
				\textbf{Matr.Nr:} 6663579\\
				\textbf{Email:} 4poehlma@inf.
			\end{tabular}
		\\ \\
			\begin{tabular}{c}
				Hauke Stieler\\
				\textbf{Matr.Nr:} 6664494\\
				\textbf{Email:} 4stieler@inf.
			\end{tabular}
			&
			\begin{tabular}{c}
				Philipp Quach\\
					\textbf{Matr.Nr:} 6706421\\
				\textbf{Email:} 4quach@inf.
			\end{tabular}
		\end{tabular}
		\\ \vspace{0.35cm}
		\begin{tabular}{c}
			Dennis Alexy\\
			\textbf{Matr.Nr:} 6686188\\
			\textbf{Email:} 4alexy@inf.
		\end{tabular}
		\\ \-\\ \-\\
		Abgabedatum: 07. Januar 2016\\
	\end{center}
}

%%%%%%%%%%%%%%%%%%%%%%%%%%%%%%%%%
%	BEGINNING of presentation content

\frame{
	\frametitle{Change Requests}
	Folgende Möglichkeiten für Change Requests sind uns spontan eingefallen.\\
	\begin{enumerate}
		\item Wäre der Fall, dass die Entscheidung auf die Farbstreifen gefallen wäre, so könnte dem Kunden die Farbauswahl nicht gefallen.\\
		\textbf{AUSWIRKUNG:} Wir rufen den Maler noch mal an.
		\item Kunde will ein neues Feature\\
		\textbf{AUSWIRKUNG:} Dauert $X$ Personenstunden länger ($X \in \mathbb{R}^+$)
		\item Kunde will Erweiterung der Möglichkeiten für Ladeninhaber\\
		\textbf{AUSWIRKUNG:} Wenn vernünftige Projektstruktur, wenige. Dauert halt länger.
	\end{enumerate}


}
%%-------------------------------------------------------
\frame{
	\frametitle{define:Change Request}
	\begin{minipage}[t]{0.48\textwidth}
		\textbf{Infos:}\\
		\begin{itemize}
			\item Auftraggeber
			\item Autor des CR
			\item CR-spezifische Details
			\item geschätzter Ressourcenverbrauch
			\item Deadline
			\item Bindende Zusage des Auftraggebers
		\end{itemize}
	\end{minipage}
	\begin{minipage}[t]{0.48\textwidth}
		\textbf{Modus der Erfassung}
		\begin{itemize}
			\item Speicherung Digital und Analog
			\item Akte mit folgendem Inhalt:
				\begin{itemize}
					\item CR
					\item Eigene Notizen
					\item Nach Timestamp sortiert
					\item eindeutige ID
				\end{itemize}
			\item Akte mit abgelehnten CRs (kann sich jemand ransetzen, der grad nichts zu tun hat)
		\end{itemize}
	\end{minipage}



}
%%-------------------------------------------------------
\frame{
	\frametitle{define:RS}
	\begin{itemize}
		\item Mangels anderer Informationen wird 'RS' als Notation für 'Risikomanagement' interpretiert
		\item Notfallplan für häufige Risiken wird mit Vertretern der Stakeholder sowie dem Lenkungsausschuss nach Projektauftrag besprochen. (Murphy's Law)
		\item Bei vernünftiger Planung ist genügend Luft, um die meisten Situationen abzufangen.
	\end{itemize}


}
%%-------------------------------------------------------
\frame{
	\frametitle{Welche CRs annehmen?}
	\textbf{ANGENOMMEN:}\\
	\-~~~~Alle, sofern mit entsprechenden Ressourcenanforderungen unterschrieben\\
	\textbf{ABGELEHNT:}\\
	\-~~~~Alle, die nicht mit unserem Teil des Projektes zu tun haben\\
	\-~~~~Alle, bei denen uns die notwendigen Ressourcen nicht zugesprochen werden



}
%%-------------------------------------------------------




\end{document}