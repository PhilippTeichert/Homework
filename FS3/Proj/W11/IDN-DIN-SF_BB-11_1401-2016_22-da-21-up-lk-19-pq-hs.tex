% following line for presentation mode
\documentclass{beamer}
%% following line for hand-in
%\documentclass[handout]{beamer}

% language and encoding settings
\usepackage[utf8]{inputenc}
\usepackage[T1]{fontenc}
\usepackage[ngerman]{babel}

% AMS Math packages
\usepackage{amsmath}
\usepackage{amssymb}
\usepackage{amsthm}
\usepackage{amsfonts}

% Presentation Layout
\usetheme{Madrid}
\title{Intense Doge Navigation}
\author{CS3 - Dev Team Alpha}

% formatting and layout
\usepackage{color}
\definecolor{pblue}{rgb}{0.13,0.13,1}
\definecolor{pgreen}{rgb}{0,0.5,0}
% macro commands
%% requires color package and the custom colors defined here 
\newcommand{\todo}[1]{
	\addcontentsline{toc}{subsubsection}{TODO: #1}
	\textcolor{pgreen}{\texttt{\-\\ \-\\//TODO: #1\-\\ \-\\}}
	}

%%%%%%%%%%%%%%%%%%%%%%%%%%%%%%%%%%%%
%	TITLE slide
%	Copy for following presentations
%	remember to adjust the hand-in date
%
\begin{document}
\frame{
	\thispagestyle{empty}
	\begin{center}
		\begin{Large}
			\textbf{Projektmanagement WS15/16}\\
		\end{Large}
		\vspace{0.3cm}
		\textbf{Woche 11}\\
		\vspace{0.5cm}
		\begin{tabular}{cc}
			\begin{tabular}{c}
				Louis Kobras\\
				\textbf{Matr.Nr:} 6658699\\
				\textbf{Email:} 4kobras@inf.
			\end{tabular}
			&
			\begin{tabular}{c}
				Utz Pöhlmann\\
				\textbf{Matr.Nr:} 6663579\\
				\textbf{Email:} 4poehlma@inf.
			\end{tabular}
		\\ \\
			\begin{tabular}{c}
				Hauke Stieler\\
				\textbf{Matr.Nr:} 6664494\\
				\textbf{Email:} 4stieler@inf.
			\end{tabular}
			&
			\begin{tabular}{c}
				Philipp Quach\\
					\textbf{Matr.Nr:} 6706421\\
				\textbf{Email:} 4quach@inf.
			\end{tabular}
		\end{tabular}
		\\ \vspace{0.35cm}
		\begin{tabular}{c}
			Dennis Alexy\\
			\textbf{Matr.Nr:} 6686188\\
			\textbf{Email:} 4alexy@inf.
		\end{tabular}
		\\ \-\\ \-\\
		Abgabedatum: \today\\
	\end{center}
}

%%%%%%%%%%%%%%%%%%%%%%%%%%%%%%%%%
%	BEGINNING of presentation content

\frame{
	\frametitle{Abschluss}
	\begin{minipage}[t]{0.48\textwidth}
		\textbf{Art des Abschlusses:}\\
		\-~~~~Release Party\\
		\textbf{Methode:}\\
		\-~~~~Pizza, Jazz und Bier, für Gäste (Kunde o.Ä.) Sekt
	\end{minipage}
	\begin{minipage}[t]{0.48\textwidth}
		\textbf{Liefergegenstände:}
		\begin{itemize}
			\item Programm auf CD und USB-Stick
			\item Handbuch als PDF und einmal ausgedruckt
			\item API-Dokumentation
			\item Kopie der Projektakte
			\item angesammeltes IPR
			\item Abschlussbericht inkl. Wirtschaftlichkeitsbetrachtung
		\end{itemize}
	\end{minipage}


}
%%-------------------------------------------------------
\frame{
	\frametitle{Durchführung der Abnahme}
	\begin{itemize}
		\item Kunde wird zur Release Party eingeladen
		\item nach dem Sektempfang, aber vor der Pizza, bekommt er Stuff auf Bühne übergeben
		\item kurze Programmvorstellung
		\item Backstage wird abgerechnet
	\end{itemize}


}
%%-------------------------------------------------------
\frame{
	\frametitle{Lerneffekte}
	\begin{itemize}
		\item Erfahrungen im Umgang mit Dritten
		\item Erfahrungen im Abschätzen der Produktivität und Effizienz
		\item Erfahrungen im Abschätzen von Risiken und Problemen
		\item Erfahrungen mit Vor- und Nachteilen der verwendeten Infrastuktur
		\item Heftführung erleichtert WrapUp (Projektakte ordentlich halten)
	\end{itemize}


}
%%-------------------------------------------------------
\frame{
	\frametitle{Aktualisierung der Projektakte}
	\begin{itemize}
		\item keine weiteren Artefakte entstanden
		\item vorhandene Artefakte auf aktuellem Stand
		\item alle vorhandenen Artefakte bereits in Projektakte enthalten
		\item $\Rightarrow$ PA ist bereits aktuell
	\end{itemize}
	\-\vspace{0.5cm}\\
	\textbf{Aufgabe 2.2:}\\
	Einstimmig beschlossen, dass Aufgabe unverständlich bzw. so nicht beantwortbar ist


}
%%-------------------------------------------------------




\end{document}