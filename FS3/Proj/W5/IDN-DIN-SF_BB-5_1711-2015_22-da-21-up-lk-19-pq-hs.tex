% following line for presentation mode
\documentclass{beamer}
%% following line for hand-in
%\documentclass[handout]{beamer}

% language and encoding settings
\usepackage[utf8]{inputenc}
\usepackage[T1]{fontenc}
\usepackage[ngerman]{babel}

% AMS Math packages
\usepackage{amsmath}
\usepackage{amssymb}
\usepackage{amsthm}
\usepackage{amsfonts}

% Presentation Layout
\usetheme{Madrid}

%%%%%%%%%%%%%%%%%%%%%%%%%%%%%%%%%%%%
%	TITLE slide
%	Copy for following presentations
%	remember to adjust the hand-in date
%
\begin{document}
\frame{
	\thispagestyle{empty}
	\begin{center}
		\begin{Large}
			\textbf{Projektmanagement WS15/16}\\
		\end{Large}
		\vspace{0.3cm}
		\textbf{Woche 5}\\
		\vspace{0.5cm}
		\begin{tabular}{cc}
			\begin{tabular}{c}
				Louis Kobras\\
				\textbf{Matr.Nr:} 6658699\\
				\textbf{Email:} 4kobras@inf.
			\end{tabular}
			&
			\begin{tabular}{c}
				Utz Pöhlmann\\
				\textbf{Matr.Nr:} 6663579\\
				\textbf{Email:} 4poehlma@inf.
			\end{tabular}
		\\ \\
			\begin{tabular}{c}
				Hauke Stieler\\
				\textbf{Matr.Nr:} 6664494\\
				\textbf{Email:} 4stieler@inf.
			\end{tabular}
			&
			\begin{tabular}{c}
				Philipp Quach\\
					\textbf{Matr.Nr:} 6706421\\
				\textbf{Email:} 4quach@inf.
			\end{tabular}
		\end{tabular}
		\\ \vspace{0.35cm}
		\begin{tabular}{c}
			Dennis Alexy\\
			\textbf{Matr.Nr:} 6686188\\
			\textbf{Email:} 4alexy@inf.
		\end{tabular}
		\\ \-\\ \-\\
		Abgabedatum: 19. November 2015\\
	\end{center}
}

%%%%%%%%%%%%%%%%%%%%%%%%%%%%%%%%%
%	BEGINNING of presentation content

\frame{
	\frametitle{Aufwandsschätzung I: Was und warum?}
	\textbf{Schema:}\\
	Es wurde sich innerhalb der Gruppe auf ein relatives Schätzungsschema geeinigt, da für absolute Aufwandsschätzung die Arbeitsgruppe über zu wenig Erfahrung und Vergleichswerte verfügt.\\
	Dabei wurde jede zu schätzende Anforderung in \textit{Informationsbeschaffung}, \textit{Entwurf} und \textit{Implementation} aufgeteilt.
	Die Anforderung erhielt den höchsten Teilschätzwert.\\
	\pause
	\vspace{0.4cm}
	\textbf{Skala:}\\
	Maus < Quokka < Schaf < Elephant < Blauwal\\
	\vspace{0.4cm}
	\textbf{Begründung:}\\
	Es ist leicht, zu sagen, eine Maus sei kleiner als ein Quokka, jedoch schwer, zu sagen, ein Quokka ist um 1.6 Messeinheiten größer als eine Maus.


}
%%-------------------------------------------------------
\frame{
	\frametitle{Aufwandsschätzung II: Schätzbeispiel \textit{Anfahrt}}
	\pause
	\begin{itemize}
		\item \textbf{Informationsbeschaffung:}\\
			Geringer Aufwand (Zugangsstraßen beschaffbar mit aktuellem Straßenatlas oder Dienstleister wie \textit{Google Maps}).
			Abgeschätzte Dauer <1h.\\
			\textbf{Schätzung:} Maus\pause
		\item \textbf{Entwurf:}\\
			IDN muss lediglich eine Grafik mit umliegenden Straßen sowie eine Wegbeschreibung enthalten.
			Abgeschätzte Dauer <2h.\\
			\textbf{Schätzung:} Quokka\pause
		\item \textbf{Implementation:}\\
			Sind Grafik und schriftliche Beschreibung fertig, muss lediglich ein Menüpunkt eingefügt werden, der beides anzeigt.
			Abgeschätze Dauer <1h.\\
			\textbf{Schätzung:} Maus
	\end{itemize}


}
%%-------------------------------------------------------
\frame{
	\frametitle{Aufwandsschätzung III (nach PSP in Projektakte)}
	
	\begin{itemize}
		\item \textbf{ÖPNV:} Schaf
		\begin{itemize}
			\item \textbf{Informationen:} Maus
			\item \textbf{Entwurf:} Quokka
			\item \textbf{Implementation:} Schaf
		\end{itemize}
		\item \textbf{Gepäckverfolgung:} Schaf
		\begin{itemize}
			\item \textbf{Informationen:} Schaf
			\item \textbf{Entwurf:} Quokka
			\item \textbf{Implementation:} Schaf
		\end{itemize}
		\item \textbf{Fluginformationen:} Quokka
		\begin{itemize}
			\item \textbf{Informationen:} Maus
			\item \textbf{Entwurf:} Quokka
			\item \textbf{Implementation:} Quokka
		\end{itemize}
		\item \textbf{Anfahrt:} Quokka
		\begin{itemize}
			\item \textbf{Informationen:} Maus
			\item \textbf{Entwurf:} Quokka
			\item \textbf{Implementation:} Maus
		\end{itemize}
	\end{itemize}

}
%%-------------------------------------------------------
\frame{
	\frametitle{Aufwandsschätzung IV (nach PSP in Projektakte)}

	\begin{itemize}
		\item \textbf{Lageplan:} Schaf
		\begin{itemize}
			\item \textbf{Informationen:} Maus
			\item \textbf{Entwurf:} Quokka
			\item \textbf{Implementation:} Schaf
		\end{itemize}
		\item \textbf{Werbe-Schnittstelle:} Elephant
		\begin{itemize}
			\item \textbf{Informationen:} \texttt{Null}
			\item \textbf{Entwurf:} Quokka
			\item \textbf{Implementation:} Elephant
		\end{itemize}
		\item \textbf{Notfallsignalgeber:} Quokka
		\begin{itemize}
			\item \textbf{Informationen:} \texttt{Null}
			\item \textbf{Entwurf:} Quokka
			\item \textbf{Implementation:} Quokka
		\end{itemize}
		\item \textbf{Verwaltungs-Schnittstelle:} Schaf
		\begin{itemize}
			\item \textbf{Informationen:} \texttt{Null}
			\item \textbf{Entwurf:} Quokka
			\item \textbf{Implementation:} Schaf
		\end{itemize}
	\end{itemize}

}
%%-------------------------------------------------------




\end{document}