% following line for presentation mode
\documentclass{beamer}
%% following line for hand-in
%\documentclass[handout]{beamer}

% language and encoding settings
\usepackage[utf8]{inputenc}
\usepackage[T1]{fontenc}
\usepackage[ngerman]{babel}

\usepackage{hyperref}

% AMS Math packages
\usepackage{amsmath}
\usepackage{amssymb}
\usepackage{amsthm}
\usepackage{amsfonts}

% Presentation Layout
\usetheme{Madrid}
\title{Intense Doge Navigation}
\author{CS3 - Dev Team Alpha}

% formatting and layout
\usepackage{color}
\definecolor{pblue}{rgb}{0.13,0.13,1}
\definecolor{pgreen}{rgb}{0,0.5,0}
% macro commands
%% requires color package and the custom colors defined here 
\newcommand{\todo}[1]{
	\addcontentsline{toc}{subsubsection}{TODO: #1}
	\textcolor{pgreen}{\texttt{\-\\ \-\\//TODO: #1\-\\ \-\\}}
	}

%%%%%%%%%%%%%%%%%%%%%%%%%%%%%%%%%%%%
%	TITLE slide
%	Copy for following presentations
%	remember to adjust the hand-in date
%
\begin{document}
\frame{
	\thispagestyle{empty}
	\begin{center}
		\begin{Large}
			\textbf{Projektmanagement WS15/16}\\
		\end{Large}
		\vspace{0.3cm}
		\textbf{Woche 8}\\
		\vspace{0.5cm}
		\begin{tabular}{cc}
			\begin{tabular}{c}
				Louis Kobras\\
				\textbf{Matr.Nr:} 6658699\\
				\textbf{Email:} 4kobras@inf.
			\end{tabular}
			&
			\begin{tabular}{c}
				Utz Pöhlmann\\
				\textbf{Matr.Nr:} 6663579\\
				\textbf{Email:} 4poehlma@inf.
			\end{tabular}
		\\ \\
			\begin{tabular}{c}
				Hauke Stieler\\
				\textbf{Matr.Nr:} 6664494\\
				\textbf{Email:} 4stieler@inf.
			\end{tabular}
			&
			\begin{tabular}{c}
				Philipp Quach\\
					\textbf{Matr.Nr:} 6706421\\
				\textbf{Email:} 4quach@inf.
			\end{tabular}
		\end{tabular}
		\\ \vspace{0.35cm}
		\begin{tabular}{c}
			Dennis Alexy\\
			\textbf{Matr.Nr:} 6686188\\
			\textbf{Email:} 4alexy@inf.
		\end{tabular}
		\\ \-\\ \-\\
		Abgabedatum: 10. Dezember 2015\\
	\end{center}
}

%%%%%%%%%%%%%%%%%%%%%%%%%%%%%%%%%
%	BEGINNING of presentation content

\frame{
	\frametitle{Planung}
	\begin{itemize}
		\item bieten mehrere Informationsmöglichkeiten an\footnote{vgl. \href{https://github.com/4kobras/Homework/blob/master/FS3/Proj/Meetingsprotokolle/IDN-DIN-SF\_PrBl\_BB7\_22-da-21-up-lk-19-pq-hs\_1012-2015.wow}{\textcolor{pblue}{dieses} Meetingsprotokoll}}
		\item Spezifischer Informationsfluss nach Absprache
		\item halten schriftlich Wünsche des Informierten fest (vgl. \href{https://github.com/4kobras/Homework/blob/master/FS3/Proj/Projektakte/Kommunikationsplan.ods}{Projektakte - Kommunikationsplan})
	\end{itemize}


}
%%-------------------------------------------------------
\frame{
	\frametitle{Protokoll}
	\begin{itemize}
		\item Kommunikationsterminierung übersichtlich
		\item Meetings on demand
		\item Gesprächsprotokolle sind einzusehen unter \href{https://github.com/4kobras/Homework/tree/master/FS3/Proj/Meetingsprotokolle}{*\textcolor{pblue}{click me}*}.
	\end{itemize}


}
%%-------------------------------------------------------
\frame{
	\frametitle{Stand der Projektakte}
	Die Projektakte enthält derzeit:
	\begin{itemize}
		\item PSP
		\item Projektauftrag
		\item Project Canvas
		\item Meeting Schedule
		\item Kommunikationsplan
		\item Zeitplanübersicht
	\end{itemize}
	Gesprächs- und Meetingsprotokolle sind gesondert gelagert (\href{https://github.com/4kobras/Homework/tree/master/FS3/Proj/Meetingsprotokolle}{*\textcolor{pblue}{click me}*}).


}
%%-------------------------------------------------------
\frame{
	\frametitle{Demo}
	\begin{center}
		An dieser Stelle werden Kommunikationsplan und Meeting Schedule vorgestellt (anklicken für Verlinkung).\\
	\end{center}
	\href{https://github.com/4kobras/Homework/blob/master/FS3/Proj/Projektakte/Kommunikationsplan.ods}{Kommunikationsplan}\\
	\href{https://github.com/4kobras/Homework/blob/master/FS3/Proj/Projektakte/Meeting\%20Schedule.pdf}{Meeting Schedule}\\
	\vspace{1cm}
	Der Kommunikationsplan ist unvollständig, weil nicht mit allen Informationsempfängern besprochen.\\
	Das Meeting Schedule umfasst nur die regelmäßigen Meetings, nicht die on demand Treffen.


}
%%-------------------------------------------------------




\end{document}