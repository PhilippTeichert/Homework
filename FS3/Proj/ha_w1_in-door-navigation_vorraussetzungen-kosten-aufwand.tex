\documentclass{beamer}

% language and encoding settings
\usepackage[utf8]{inputenc}
\usepackage[T1]{fontenc}
\usepackage[ngerman]{babel}

% AMS Math packages
\usepackage{amsmath}
\usepackage{amssymb}
\usepackage{amsthm}
\usepackage{amsfonts}

% Presentation Layout
\usetheme{Madrid}

%%%%%%%%%%%%%%%%%%%%%%%%%%%%%%%%%%%%
%	TITLE slide
%	Copy for following presentations
%	remember to adjust the hand-in date
%
\begin{document}
\frame{
	\thispagestyle{empty}
	\begin{center}
		\begin{Large}
			\textbf{Projektmanagement WS15/16}\\
		\end{Large}
		\vspace{1cm}
		\begin{tabular}{cc}
			\begin{tabular}{c}
				Louis Kobras\\
				\textbf{Matr.Nr:} 6658699\\
				\textbf{Email:} 4kobras@inf.
			\end{tabular}
			&
			\begin{tabular}{c}
				Utz Pöhlmann\\
				\textbf{Matr.Nr:} 6663579\\
				\textbf{Email:} 4poehlma@inf.
			\end{tabular}
		\\ \\
			\begin{tabular}{c}
				Hauke Stieler\\
				\textbf{Matr.Nr:} 6664494\\
				\textbf{Email:} 4stieler@@inf.
			\end{tabular}
			&
			\begin{tabular}{c}
				Philipp Quach\\
					\textbf{Matr.Nr:} 6706421\\
				\textbf{Email:} 4quach@inf.
			\end{tabular}
		\\ \\
			\begin{tabular}{c}
				Melf Krüger\\
				\textbf{Matr.Nr:} 6652097\\
				\textbf{Email:} 4mkruege@inf.
			\end{tabular}
			&
			\begin{tabular}{c}
				Dennis Alexy\\
				\textbf{Matr.Nr:} 6686188\\
				\textbf{Email:} 4alexy@inf.
			\end{tabular}
		\end{tabular}
		\\ \-\\ \-\\
		Abgabedatum: 22. Oktober 2015\\
	\end{center}
}

%%%%%%%%%%%%%%%%%%%%%%%%%%%%%%%%%
%	BEGINNING of presentation content
\frame{
	\frametitle{Audio-Guide}
	\vspace{-3cm}
	\begin{center}
		\textbf{Idee:}\\
	\end{center}
	Tragbarer Audio-Guide, der als akkustisches Navi arbeitet\\ \vspace{0.5cm}
	\pause
	\begin{minipage}[t]{0.48\linewidth}
		\textbf{Aufwand:}\\
		\begin{itemize}
			\item .
		\end{itemize}
	\end{minipage}
	\pause
	\begin{minipage}[t]{0.48\linewidth}
		\textbf{Kosten:}\\
		\begin{itemize}
			\item .
		\end{itemize}
	\end{minipage}
}
\frame{
	\frametitle{Farbmarkierungen}
	\vspace{-3cm}
	\begin{center}
		\textbf{Idee:}\\
	\end{center}
	Farbliche Streifen, die die wichtigsten Wege entlanglaufen\\ \vspace{0.5cm}
	\pause
	\begin{minipage}[t]{0.48\linewidth}
		\textbf{Aufwand:}\\
		\begin{itemize}
			\item .
		\end{itemize}
	\end{minipage}
	\pause
	\begin{minipage}[t]{0.48\linewidth}
		\textbf{Kosten:}\\
		\begin{itemize}
			\item .
		\end{itemize}
	\end{minipage}
}
\frame{
	\frametitle{Standortanzeiger}
	\begin{center}
		\textbf{Idee:}\\
	\end{center}
	Tafeln mit einer Karte, die einem die aktuelle Position anzeigen\\
	\pause
	\begin{minipage}[t]{0.48\linewidth}
		\textbf{Aufwand:}\\
		\begin{itemize}
			\item .
		\end{itemize}
	\end{minipage}
	\pause
	\begin{minipage}[t]{0.48\linewidth}
		\textbf{Kosten:}\\
		\begin{itemize}
			\item .
		\end{itemize}
	\end{minipage}
}
\frame{
	\frametitle{Minimap-App}
	\begin{center}
		\textbf{Idee:}\\
	\end{center}
	App für Smartphones, die einem die Position auf einer Karte anzeigt\\
	\pause
	\begin{minipage}[t]{0.48\linewidth}
		\textbf{Aufwand:}\\
		\begin{itemize}
			\item .
		\end{itemize}
	\end{minipage}
	\pause
	\begin{minipage}[t]{0.48\linewidth}
		\textbf{Kosten:}\\
		\begin{itemize}
			\item .
		\end{itemize}
	\end{minipage}
}
\end{document}