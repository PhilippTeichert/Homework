\documentclass[ngerman]{gdb-aufgabenblatt}


\renewcommand{\Aufgabenblatt}{1}
\renewcommand{\Ausgabedatum}{Mi. 15.10.2014}
\renewcommand{\Abgabedatum}{Do. 31.10.2014}
\renewcommand{\Gruppe}{M�ller, Meyer, Schulze}
\renewcommand{\STiNEGruppe}{01}


\begin{document}


\section{Beispiel f�r ER-Diagramm}

\begin{center}
\begin{tikzpicture}

\node[entity] (e1) {E1};
\node[attribut] (e1-a1) [above left =5mm and 4mm of e1.north] {\underline{A1}} edge (e1);
\node[attribut] (e1-a2)  [above right=5mm and 4mm of e1.north] {A2} edge (e1);
\node[multivalentattribut] (e1-a3)  [right=5mm of e1] {A3} edge (e1);

\node[entity] (e2) [below left =1cm and 1mm of e1.south] {E1} edge [erbt] (e1);
\node[entity] (e3)      [below right=1cm and 1mm of e1.south] {E3} edge [erbt] (e1);
\node[entity] (e4) [right =7cm of e3] {E4};

\node[weakentity] (e5) [below =3cm of e3] {E5};
\node[attribut] (e5-a1)  [right=5mm of e5] {\dashuline{A1}} edge (e5);

\node[relationship] (r1) [right=2cm of e3] {R1};
\path (r1) edge node[at end,anchor=north west] {$[0;2]$} (e3);
\path (r1) edge node[at end,anchor=north east] {$[7;9]$} (e4);


\node[weakrelationship] (r2) [below=1cm of e3] {R2};
\path (r2) edge node[at end,anchor=north west] {$1$} (e3);
\draw[double distance=2pt] (r2) -- node[at end,anchor=south east] {$8$} (e5);



\end{tikzpicture}
\end{center}






\section{Beispiel f�r relationales Datenbankschema}

\begin{RMSchma}
Person(\soliduline{PID}, Name, Vorname, \dashuline{(HaustierName, HaustierRasse) $\rightarrow$ (Haustier.Name, Haustier.Rasse)})

Haustier(\soliduline{Name, Rasse}, \dashuline{Herrchen $\rightarrow$ Person.PID})
\end{RMSchma}






\section{Beispiel f�r Ausdruck der Relationenalgebra}

\begin{align*}
 &\umbenennung{Rasse}{Sorte}(\projektion{Rasse, Geschlecht}((Wolf\verbund{Wolf.WID=Haustier.HID} (\selektion{Name=\wert{Hasso}}Haustiere)) \natverbund Person))
\\  &=\{ \wert{Steppenwolf}, \wert{m} \}
\end{align*}




\newpage
\section{Beispiel f�r SQL-Anfrage}

\begin{verbatim}
SELECT 
  h.Name,
  h.Rasse
FROM 
  Haustier h,
  Person p
WHERE
  h.Herrchen = p.PID AND
  p.Vorname LIKE "P%"
\end{verbatim}








\section{Beispiel f�r Operatorbaum}

\begin{tikzpicture}
\node (Haustier) {Haustier};
\node (Wolf) [left=25mm of Haustier] {Wolf};
\node (join1) [above=20mm of $(Haustier)!.5!(Wolf)$] {$\verbund{Wolf.WID=Haustier.HID}$};
\node (selektion1) [above=of join1] {$\selektion{Name=\wert{Hasso}}$};
\node (projektion) [above=of selektion1] {$\projektion{Rasse}$};
\node (final) [above=of projektion] {};

\path (Haustier) edge node[smallr,near start,above right] {?? Tupel\\?? Attribute} (join1);
\path (Wolf) edge node[smalll,near start,above left] {?? Tupel\\?? Attribute} (join1);
\path (join1) edge node[smallr,near start,above left] {?? Tupel\\?? Attribute} (selektion1);
\path (selektion1) edge node[smallr,midway,left] {$??\cdot\frac{??}{??}=??$ Tupel\\?? Attribute} (projektion);
\path (projektion) edge node[smallr,midway,left] {$??$ Tupel\\1 Attribut} (final);
\end{tikzpicture}







\section{Beispiel f�r Tabelle mit Sperranforderungen}

\begin{tabular}{|p{2cm}|p{2cm}|p{2cm}|p{2cm}|p{1cm}|p{1cm}|p{1cm}|p{3cm}|}
\hline
Zeitschritt & T\ts{1} & T\ts{2} & T\ts{3} & x & y & z & Bemerkung\\
\hline
\hline
0 &  &  &  & NL & NL & NL & \\
\hline
1 & lock(x,X) &  &  & X\ts{1} & NL & NL & \\
\hline
2 & write(x) & lock(y,R) &  & X\ts{1} & R\ts{2} & NL & \\
\hline
3 &  &  &  &  &  &  & \\
\hline
4 &  &  &  &  &  &  & \\
\hline
5 &  &  &  &  &  &  & \\
\hline
\end{tabular}



\section{*Thema*}

*L�sung*






\end{document}