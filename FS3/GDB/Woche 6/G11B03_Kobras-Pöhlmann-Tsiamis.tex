\documentclass[ngerman]{gdb-aufgabenblatt}


\renewcommand{\Aufgabenblatt}{3}
\renewcommand{\Ausgabedatum}{Mi. 11.11.2015}
\renewcommand{\Abgabedatum}{Do. 27.11.2015}

% für is-a relation
\usetikzlibrary{shapes}


\begin{document}

\section{Informationsmodellierung}

%%%%%%%%%%%%%%%%%%%%%%%%%%%%%%%%%%%%%%%%
%%%%	TODO Grafik drehen
%%%%%%%%%%%%%%%%%%%%%%%%%%%%%%%%%%%%%%%%
\begin{center}
\begin{tikzpicture}
	%%%% entitäten
	\node[entity]		(person)												{Person};
	% is-a relation der personen
	\node[regular polygon,regular polygon sides=6,draw, ultra thick, fill=blue!20]
						(p-isa)			[left=0.5cm of person]					{is-a};
	\node[entity]		(lektor)		[left=1cm of p-isa]						{Lektor};
	\node[entity]		(ss)			[above=0.5cm of lektor]					{Schriftsteller};
	\node[entity]		(verleger)		[below=0.5cm of lektor]					{Verleger};
	%% verleger-relationen
	\node[relationship]	(rel-vl-vl)		[left=2cm of verleger]					{arbeitet für};
	%% lektor-relationen
	\node[relationship]	(rel-lk-buch)	[above left=2cm and 2cm of lektor]		{prüft};
	%% schriftsteller-relationen
	\node[relationship] (rel-ss-shg)	[above right=1cm and 1cm of ss]			{nimmt teil};
	\node[relationship]	(rel-ss-buch)	[above left=1cm and 1cm of ss]			{schreibt};
	
	\node[entity]		(buch)			[above left=1cm and 1cm of rel-ss-buch]	{Buch};
	%% buch-relationen
	\node[relationship]	(rel-buch-pv)	[left=2cm of buch]						{platziert};
	\node[entity]		(verlag)		[left=3cm of lektor]					{Verlag};
	\node[entity]		(shg)			[above right=1cm and 1cm of rel-ss-shg]	{SHG};
	\node[entity]		(pv)			[left=2cm of rel-buch-pv]				{Preisverleihung};
					
	
	%%%% attribute
	% personelle attribute
	\node[attribut]				(ps-vorname)	[right=1cm of person]				{Vorname}				edge (person);
	\node[attribut]				(ps-!svnr)		[above=0.5cm of ps-vorname]			{\underline{\#SVNr}}	edge (person);
	\node[attribut]				(ps-nachname)	[below=0.5cm of ps-vorname]			{Nachname}				edge (person);
	\node[attribut]				(ps-gbd)		[below=0.5cm of person]				{Geburtsdatum}			edge (person);
	% büchliche attribute
	\node[attribut]				(b-jahr)	[above=0.5cm of buch]					{Jahr}					edge	(buch);
	\node[attribut]				(b-titel)	[left=0.5cm of b-jahr]					{Titel}					edge	(buch);
	\node[attribut]				(b-!isbn)	[left=0.5cm of b-titel]					{\underline{\#ISBN}}	edge	(buch);
	\node[multivalentattribut]	(b-autoren)	[right=0.5cm of b-jahr]					{Autoren}				edge	(buch);
	\node[attribut]				(b-verlag)	[right=0.5cm of b-autoren]				{Verlag}				edge	(buch);
	% verlegte attribute
	\node[attribut]				(v-!name)	[above left=0.2cm and 0.5cm of verlag]	{\underline{Name}}		edge	(verlag);
	\node[attribut]				(v-adresse)	[below left=0.2cm and 0.5cm of verlag]	{Adresse}				edge	(verlag);
	\node[attribut]				(v-a-hnr)	[left=0.3cm of v-adresse]				{Haus-Nr.}				edge	(v-adresse);
	\node[attribut]				(v-a-str)	[above=0.3cm of v-a-hnr]				{Straße}				edge	(v-adresse);
	\node[attribut]				(v-a-plz)	[below=0.3cm of v-a-hnr]				{PLZ}					edge	(v-adresse);
	% pvg-attribute
	\node[attribut]				(p--name)	[left=0.4cm of pv]						{\underline{Name}}		edge	(pv);
	\node[attribut]				(p--jahr)	[above=0.2cm of p--name]				{\underline{Jahr}}		edge	(pv);
	\node[attribut]				(p-verans)	[below=0.2cm of p--name]				{Veranstalter}			edge	(pv);
	% shg-attribute
	\node[attribut]				(s-!name)	[above right=0.2cm and 0.5cm of shg]	{\underline{Name}}		edge	(shg);
	\node[attribut]				(s-ort)		[below right=0.2cm and 0.5cm of shg]	{Ort}					edge	(shg);
	% verleger-attribute
	\node[attribut]				(v-abt)		[below=0.3cm of verleger]				{Abteilung}				edge	(verleger);
	% pv-rel-attribute
	\node[attribut]				(pv-platz)	[below right=0.4cm and 0.4cm of rel-buch-pv]	{Platzierung}	edge	(rel-buch-pv);
	\node[attribut]				(pv-kat)	[below left=0.4cm and 0.4cm of rel-buch-pv]	{Kategorie}		edge	(rel-buch-pv);
	%% pfade
	\path
		(person)		edge						(p-isa)
		(p-isa)			edge						(ss)
		(p-isa)			edge						(lektor)
		(p-isa)			edge						(verleger)
		(ss)			edge	node	{$[0;*]$}	(rel-ss-shg)
		(rel-ss-shg)	edge	node	{$[5;20]$}	(shg)
		(buch)			edge	node	{$[0;*]$}	(rel-buch-pv)
		(rel-buch-pv)	edge	node	{$[3;20]$}	(pv)
	;
	
\end{tikzpicture}
\end{center}

%%%% notizen für 1
%Person
%	!#SVNr
%	Vorname
%	Nachname
%	Geburtsdatum
%
%Buch
%	!#ISBN
%	Titel
%	Erscheinungsjahr
%	#NrAutoren [1;*]
%	Verlag [1;1]
%	wird begutachtet [0;*]
%		Begutachtung findet an einem Datum statt
%		Begutachtung [1;1] Buches von [2;2] LektorInnen
%			einer ist Erstgutachter
%			zweiner ist Zweitgutachter
%	platziert bei Preisverleihung [0;*]
%		Platz
%		Kategorie
%
%Verlag
%	!Name
%	Adresse
%		Straße
%		Hausnr
%		PLZ
%	verlegt [0;*]
%	beschäftigt VerlegerInnen [1;*]
%	beschäftigt LektorInnen [0;*]
%
%Preisverleihung
%	.Name
%	.Jahr
%	Veranstalter
%	platziert Bücher [3;20]
%
%SHG
%	!Name
%	Versammlungsort
%	hat [5:20] SchriftstellerInnen als Mitglieder
%
%SchriftstellerIn (is-a Person)
%	Mitglied bei [0;*] SHGs
%	schreibt Buch [1;*]
%
%VerlegerIn (is-a Person)
%	arbeitet bei Verlag [1;1]
%	Abteilung
%
%LektorIn (is-a Person)
%	arbeitet bei Verlag [1;*]
%	begutachtet [0;*]


\section{Logischer Entwurf}



\section{Relationale Algebra und SQL}



\section{Algebraische Optimierung}
\begin{minipage}[t]{0.48\textwidth}
\begin{center}
\begin{tikzpicture}
	\node	(rennfahrerin)	{RennfahrerIn};
	\node	(platzierung)	[right=1cm of rennfahrerin]		{Platzierung};
	\node	(rennstall)		[right=1cm of platzierung]		{Rennstall};
	\node	(n-rennfahrer)	[below=-0.1cm of rennfahrerin]	{40};
	\node	(n-platzierung)	[below=-0.1cm of platzierung]	{400};
	\node	(n-rennstall)	[below=-0.1cm of rennstall]		{12};
	\node	(lower-joint)	[above right=1cm and 0.4cm of rennfahrerin]	{$\verbund{}$};
	\node	(upper-joint)	[above=2cm of platzierung]		{$\verbund{}$};
	\node	(lower-sigma)	[above=1cm of upper-joint]		{$\sigma$};
	\node	(upper-sigma)	[above=1cm of lower-sigma]		{$\sigma$};
	\node	(pi)			[above=1cm of upper-sigma]		{$\pi$};
	
	\path
		(rennfahrerin)	edge	(lower-joint)
		(platzierung)	edge	(lower-joint)
		(rennstall)		edge	(upper-joint)
		(lower-joint)	edge	node	{16.000}	(upper-joint)
		(upper-joint)	edge	node	{192.000}	(lower-sigma)
		(lower-sigma)	edge	node	{16.000}	(upper-sigma)
		(upper-sigma)	edge	node	{400}		(pi)
	;
	
	
\end{tikzpicture}
\end{center}


\end{minipage}
\begin{minipage}[t]{0.48\textwidth}
	\begin{tikzpicture}
	\node	(rennfahrerin)	{RennfahrerIn};
	\node	(platzierung)	[right=1cm of rennfahrerin]		{Platzierung};
	\node	(rennstall)		[right=1cm of platzierung]		{Rennstall};
	\node	(n-rennfahrer)	[below=-0.1cm of rennfahrerin]	{40};
	\node	(n-platzierung)	[below=-0.1cm of platzierung]	{400};
	\node	(n-rennstall)	[below=-0.1cm of rennstall]		{12};
	\node	(lower-joint)	[above right=1cm and 0.4cm of rennfahrerin]	{$\verbund{}$};
	\node	(upper-joint)	[above=2cm of platzierung]		{$\verbund{}$};
	\node	(lower-sigma)	[above=1cm of upper-joint]		{$\sigma$};
	\node	(upper-sigma)	[above=1cm of lower-sigma]		{$\sigma$};
	\node	(pi)			[above=1cm of upper-sigma]		{$\pi$};
	
	\path
		(rennfahrerin)	edge	(lower-joint)
		(platzierung)	edge	(lower-joint)
		(rennstall)		edge	(upper-joint)
		(lower-joint)	edge	node	{16.000}	(upper-joint)
		(upper-joint)	edge	node	{192.000}	(lower-sigma)
		(lower-sigma)	edge	node	{16.000}	(upper-sigma)
		(upper-sigma)	edge	node	{400}		(pi)
	;
	\end{tikzpicture}
\end{minipage}



\end{document}