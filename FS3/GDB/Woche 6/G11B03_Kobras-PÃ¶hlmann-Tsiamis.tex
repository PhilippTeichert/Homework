\documentclass[ngerman]{gdb-aufgabenblatt}


\renewcommand{\Aufgabenblatt}{3}
\renewcommand{\Ausgabedatum}{Mi. 11.11.2015}
\renewcommand{\Abgabedatum}{Do. 27.11.2015}
\renewcommand{\Gruppe}{Kobras, P�hlmann, Tsiamis}
\renewcommand{\STiNEGruppe}{11}

% f�r is-a relation
\usetikzlibrary{shapes}
% f�r rotieren des er-diagrammes
%\usepackage{graphicx}
%\usepackage{rotating}

\begin{document}

\section{Informationsmodellierung}


%\begin{sidewaysfigure}
\begin{center}
\begin{tikzpicture}
	%%%% entit�ten
	\node[entity]		(person)												{Person};
	% is-a relation der personen
	\node[regular polygon,regular polygon sides=6,draw, ultra thick, fill=blue!20]
						(p-isa)			[left=0.5cm of person]					{is-a};
	\node[entity]		(lektor)		[left=1cm of p-isa]						{Lektor};
	\node[entity]		(ss)			[above=0.5cm of lektor]					{Schriftsteller};
	\node[entity]		(verleger)		[below=0.5cm of lektor]					{Verleger};
	%% verleger-relationen
	\node[relationship]	(rel-vl-vl)		[left=2cm of verleger]					{arbeitet f�r};
	%% lektor-relationen
	\node[relationship]	(rel-lk-buch)	[above left=2cm and 2cm of lektor]		{pr�ft};
	%% schriftsteller-relationen
	\node[relationship] (rel-ss-shg)	[above right=1cm and 1cm of ss]			{nimmt teil};
	\node[relationship]	(rel-ss-buch)	[above left=1cm and 1cm of ss]			{schreibt};
	
	\node[entity]		(buch)			[above left=1cm and 1cm of rel-ss-buch]	{Buch};
	%% buch-relationen
	\node[relationship]	(rel-buch-pv)	[left=2cm of buch]						{platziert};
	\node[entity]		(verlag)		[left=3cm of lektor]					{Verlag};
	\node[entity]		(shg)			[above right=1cm and 1cm of rel-ss-shg]	{SHG};
	\node[entity]		(pv)			[left=2cm of rel-buch-pv]				{Preisverleihung};
					
	
	%%%% attribute
	% personelle attribute
	\node[attribut]				(ps-vorname)	[right=1cm of person]				{Vorname}				edge (person);
	\node[attribut]				(ps-!svnr)		[above=0.5cm of ps-vorname]			{\underline{\#SVNr}}	edge (person);
	\node[attribut]				(ps-nachname)	[below=0.5cm of ps-vorname]			{Nachname}				edge (person);
	\node[attribut]				(ps-gbd)		[below=0.5cm of person]				{Geburtsdatum}			edge (person);
	% b�chliche attribute
	\node[attribut]				(b-jahr)	[above=0.5cm of buch]					{Jahr}					edge	(buch);
	\node[attribut]				(b-titel)	[left=0.5cm of b-jahr]					{Titel}					edge	(buch);
	\node[attribut]				(b-!isbn)	[left=0.5cm of b-titel]					{\underline{\#ISBN}}	edge	(buch);
	\node[multivalentattribut]	(b-autoren)	[right=0.5cm of b-jahr]					{Autoren}				edge	(buch);
	\node[attribut]				(b-verlag)	[right=0.5cm of b-autoren]				{Verlag}				edge	(buch);
	% verlegte attribute
	\node[attribut]				(v-!name)	[above left=0.2cm and 0.5cm of verlag]	{\underline{Name}}		edge	(verlag);
	\node[attribut]				(v-adresse)	[below left=0.2cm and 0.5cm of verlag]	{Adresse}				edge	(verlag);
	\node[attribut]				(v-a-hnr)	[left=0.3cm of v-adresse]				{Haus-Nr.}				edge	(v-adresse);
	\node[attribut]				(v-a-str)	[above=0.3cm of v-a-hnr]				{Stra�e}				edge	(v-adresse);
	\node[attribut]				(v-a-plz)	[below=0.3cm of v-a-hnr]				{PLZ}					edge	(v-adresse);
	% pvg-attribute
	\node[attribut]				(p--name)	[left=0.4cm of pv]						{\underline{Name}}		edge	(pv);
	\node[attribut]				(p--jahr)	[above=0.2cm of p--name]				{\underline{Jahr}}		edge	(pv);
	\node[attribut]				(p-verans)	[below=0.2cm of p--name]				{Veranstalter}			edge	(pv);
	% shg-attribute
	\node[attribut]				(s-!name)	[above right=0.2cm and 0.5cm of shg]	{\underline{Name}}		edge	(shg);
	\node[attribut]				(s-ort)		[below right=0.2cm and 0.5cm of shg]	{Ort}					edge	(shg);
	% verleger-attribute
	\node[attribut]				(v-abt)		[below=0.3cm of verleger]				{Abteilung}				edge	(verleger);
	% pv-rel-attribute
	\node[attribut]				(pv-platz)	[below right=0.4cm and 0.4cm of rel-buch-pv]	{Platzierung}	edge	(rel-buch-pv);
	\node[attribut]				(pv-kat)	[below left=0.4cm and 0.4cm of rel-buch-pv]	{Kategorie}		edge	(rel-buch-pv);
	
%\path (r1) edge node[at end,anchor=north west] {$[0;2]$} (e3);
%\node[relationship] (r1) [right=2cm of e3] {R1};
	%%%% relationen
	%% relationsbl�cke
	
	%% pfaden
	\path
		(person)		edge						(p-isa)
		(p-isa)			edge						(ss)
		(p-isa)			edge						(lektor)
		(p-isa)			edge						(verleger)
		(ss)			edge	node	{$[0;*]$}	(rel-ss-shg)
		(rel-ss-shg)	edge	node	{$[5;20]$}	(shg)
		(buch)			edge	node	{$[0;*]$}	(rel-buch-pv)
		(rel-buch-pv)	edge	node	{$[3;20]$}	(pv)
	;
	
\end{tikzpicture}
\end{center}
%\end{sidewaysfigure}

%%%% notizen f�r 1
%Person
%	!#SVNr
%	Vorname
%	Nachname
%	Geburtsdatum
%
%Buch
%	!#ISBN
%	Titel
%	Erscheinungsjahr
%	#NrAutoren [1;*]
%	Verlag [1;1]
%	wird begutachtet [0;*]
%		Begutachtung findet an einem Datum statt
%		Begutachtung [1;1] Buches von [2;2] LektorInnen
%			einer ist Erstgutachter
%			zweiner ist Zweitgutachter
%	platziert bei Preisverleihung [0;*]
%		Platz
%		Kategorie
%
%Verlag
%	!Name
%	Adresse
%		Stra�e
%		Hausnr
%		PLZ
%	verlegt [0;*]
%	besch�ftigt VerlegerInnen [1;*]
%	besch�ftigt LektorInnen [0;*]
%
%Preisverleihung
%	.Name
%	.Jahr
%	Veranstalter
%	platziert B�cher [3;20]
%
%SHG
%	!Name
%	Versammlungsort
%	hat [5:20] SchriftstellerInnen als Mitglieder
%
%SchriftstellerIn (is-a Person)
%	Mitglied bei [0;*] SHGs
%	schreibt Buch [1;*]
%
%VerlegerIn (is-a Person)
%	arbeitet bei Verlag [1;1]
%	Abteilung
%
%LektorIn (is-a Person)
%	arbeitet bei Verlag [1;*]
%	begutachtet [0;*]


\section{Logischer Entwurf}



\section{Relationale Algebra und SQL}



\section{Algebraische Optimierung}

*L�sung*



%%%%%\section{Beispiel f�r ER-Diagramm}
%%%%%
%%%%%\begin{center}
%%%%%\begin{tikzpicture}
%%%%%
%%%%%\node[entity] (e1) {E1};
%%%%%\node[attribut] (e1-a1) [above left =5mm and 4mm of e1.north] {\underline{A1}} edge (e1);
%%%%%\node[attribut] (e1-a2)  [above right=5mm and 4mm of e1.north] {A2} edge (e1);
%%%%%\node[multivalentattribut] (e1-a3)  [right=5mm of e1] {A3} edge (e1);
%%%%%
%%%%%\node[entity] (e2) [below left =1cm and 1mm of e1.south] {E1} edge [erbt] (e1);
%%%%%\node[entity] (e3)      [below right=1cm and 1mm of e1.south] {E3} edge [erbt] (e1);
%%%%%\node[entity] (e4) [right =7cm of e3] {E4};
%%%%%
%%%%%\node[weakentity] (e5) [below =3cm of e3] {E5};
%%%%%\node[attribut] (e5-a1)  [right=5mm of e5] {\dashuline{A1}} edge (e5);
%%%%%
%%%%%\node[relationship] (r1) [right=2cm of e3] {R1};
%%%%%\path (r1) edge node[at end,anchor=north west] {$[0;2]$} (e3);
%%%%%\path (r1) edge node[at end,anchor=north east] {$[7;9]$} (e4);
%%%%%
%%%%%
%%%%%\node[weakrelationship] (r2) [below=1cm of e3] {R2};
%%%%%\path (r2) edge node[at end,anchor=north west] {$1$} (e3);
%%%%%\draw[double distance=2pt] (r2) -- node[at end,anchor=south east] {$8$} (e5);
%%%%%
%%%%%
%%%%%
%%%%%\end{tikzpicture}
%%%%%\end{center}
%%%%%
%%%%%
%%%%%\section{Beispiel f�r relationales Datenbankschema}
%%%%%
%%%%%\begin{RMSchma}
%%%%%Person(\soliduline{PID}, Name, Vorname, \dashuline{(HaustierName, HaustierRasse) $\rightarrow$ (Haustier.Name, Haustier.Rasse)})
%%%%%
%%%%%Haustier(\soliduline{Name, Rasse}, \dashuline{Herrchen $\rightarrow$ Person.PID})
%%%%%\end{RMSchma}
%%%%%
%%%%%
%%%%%
%%%%%
%%%%%
%%%%%
%%%%%\section{Beispiel f�r Ausdruck der Relationenalgebra}
%%%%%
%%%%%\begin{align*}
%%%%% &\umbenennung{Rasse}{Sorte}(\projektion{Rasse, Geschlecht}((Wolf\verbund{Wolf.WID=Haustier.HID} (\selektion{Name=\wert{Hasso}}Haustiere)) \natverbund Person))
%%%%%\\  &=\{ \wert{Steppenwolf}, \wert{m} \}
%%%%%\end{align*}
%%%%%
%%%%%
%%%%%
%%%%%
%%%%%\newpage
%%%%%\section{Beispiel f�r SQL-Anfrage}
%%%%%
%%%%%\begin{verbatim}
%%%%%SELECT 
%%%%%  h.Name,
%%%%%  h.Rasse
%%%%%FROM 
%%%%%  Haustier h,
%%%%%  Person p
%%%%%WHERE
%%%%%  h.Herrchen = p.PID AND
%%%%%  p.Vorname LIKE "P%"
%%%%%\end{verbatim}
%%%%%
%%%%%
%%%%%
%%%%%
%%%%%
%%%%%
%%%%%
%%%%%
%%%%%\section{Beispiel f�r Operatorbaum}
%%%%%
%%%%%\begin{tikzpicture}
%%%%%\node (Haustier) {Haustier};
%%%%%\node (Wolf) [left=25mm of Haustier] {Wolf};
%%%%%\node (join1) [above=20mm of $(Haustier)!.5!(Wolf)$] {$\verbund{Wolf.WID=Haustier.HID}$};
%%%%%\node (selektion1) [above=of join1] {$\selektion{Name=\wert{Hasso}}$};
%%%%%\node (projektion) [above=of selektion1] {$\projektion{Rasse}$};
%%%%%\node (final) [above=of projektion] {};
%%%%%
%%%%%\path (Haustier) edge node[smallr,near start,above right] {?? Tupel\\?? Attribute} (join1);
%%%%%\path (Wolf) edge node[smalll,near start,above left] {?? Tupel\\?? Attribute} (join1);
%%%%%\path (join1) edge node[smallr,near start,above left] {?? Tupel\\?? Attribute} (selektion1);
%%%%%\path (selektion1) edge node[smallr,midway,left] {$??\cdot\frac{??}{??}=??$ Tupel\\?? Attribute} (projektion);
%%%%%\path (projektion) edge node[smallr,midway,left] {$??$ Tupel\\1 Attribut} (final);
%%%%%\end{tikzpicture}
%%%%%
%%%%%
%%%%%
%%%%%
%%%%%
%%%%%
%%%%%
%%%%%\section{Beispiel f�r Tabelle mit Sperranforderungen}
%%%%%
%%%%%\begin{tabular}{|p{2cm}|p{2cm}|p{2cm}|p{2cm}|p{1cm}|p{1cm}|p{1cm}|p{3cm}|}
%%%%%\hline
%%%%%Zeitschritt & T\ts{1} & T\ts{2} & T\ts{3} & x & y & z & Bemerkung\\
%%%%%\hline
%%%%%\hline
%%%%%0 &  &  &  & NL & NL & NL & \\
%%%%%\hline
%%%%%1 & lock(x,X) &  &  & X\ts{1} & NL & NL & \\
%%%%%\hline
%%%%%2 & write(x) & lock(y,R) &  & X\ts{1} & R\ts{2} & NL & \\
%%%%%\hline
%%%%%3 &  &  &  &  &  &  & \\
%%%%%\hline
%%%%%4 &  &  &  &  &  &  & \\
%%%%%\hline
%%%%%5 &  &  &  &  &  &  & \\
%%%%%\hline
%%%%%\end{tabular}








\end{document}