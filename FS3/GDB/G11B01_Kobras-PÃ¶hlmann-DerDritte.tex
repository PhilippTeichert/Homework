\documentclass[ngerman]{gdb-aufgabenblatt}

\usepackage{enumerate}

\renewcommand{\Aufgabenblatt}{1}
\renewcommand{\Ausgabedatum}{Mi. 14.10.2015}
\renewcommand{\Abgabedatum}{Fr. 30.10.2015}
\renewcommand{\Gruppe}{Kobras, P�hlmann, DerDritte}
\renewcommand{\STiNEGruppe}{11}
\renewcommand{\Semester}{WS 2015/16}

\begin{document}


%------------------------------------------------
\section{Informationssysteme}
%------------------------------------------------


\begin{enumerate}[a)]
    \item
    \textbf{Charakterisierung:}\\
    Ein Informationssystem ist ein System aus Menschen und/oder Maschinen, die Informationen erzeugen und/oder benutzen und die durch Kommunikationsbeziehungen miteinander verbunden sind. (s. Vorlesung 1 Folie 26)\\
    Aufgaben eines rechnergest�tzten Informationssystems sind: (s. Vorlesung 1 Folie 26)
    \begin{enumerate}[1.]
        \item
        Erfassung von Informationen
        \item
        Speicherung von Informationen
        \item
        Transformation von Informationen
    \end{enumerate}
    
%------------------------------------------------

    \item
    \textbf{Datenunabh�ngigkeit:}
    \begin{enumerate}[1.]
        \item
        logische Datenunabh�ngigkeit (Struktur der Daten):\\
        Die Daten sollen unabh�ngig von den Anwendungen organisiert werden, sodass logische �nderungen der Daten keine Auswirkungen auf die Funktionsweise der Programme haben.
        \item
        physische Datenunabh�ngigkeit (Speicherort, -art):\\
        Die physische Organisation der Daten soll ohne Auswirkungen auf die logische Struktur ge�ndert werden k�nnen.
    \end{enumerate}
    
%------------------------------------------------

    \item
    \textbf{Beispiele:}\\
    Anwendungsbeispiele f�r Informationssysteme:
    \begin{enumerate}[1.]
        \item
        Beispiel 1.
        \item
        Beispiel 2.
        \item
        Beispiel 3.
    \end{enumerate}
    
\end{enumerate}


%------------------------------------------------
\section{Miniwelt}
%------------------------------------------------


\begin{enumerate}[a)]
    \item
    Relevante Elemente:
    \begin{enumerate}[1.]
        \item
        1.
    \end{enumerate}
    Relevante Vorg�nge:
    \begin{enumerate}[1.]
        \item
        1.
    \end{enumerate}
    
%------------------------------------------------

    \item
    Anforderungen an die Anwendung:
    \begin{enumerate}[1.]
        \item
        1.
    \end{enumerate}
\end{enumerate}


%------------------------------------------------
\section{Transaktionen}
%------------------------------------------------


Folgen durch einen Stromausfall zum Zeitpunkt A:
\begin{enumerate}[1.]
    \item
    1.
\end{enumerate}
Folgen durch einen Stromausfall zum Zeitpunkt B:
\begin{enumerate}[1.]
    \item
    1.
\end{enumerate}

%------------------------------------------------

Wie kann so etwas durch ein Datenbanksystem verhindert werden?
\begin{enumerate}[1.]
    \item
    1.
\end{enumerate}

%------------------------------------------------
\section{Warm-Up MySQL}
%------------------------------------------------


\begin{enumerate}[a)]
    \item
    Was ist passiert?
    \begin{enumerate}[1.]
        \item
        1.
    \end{enumerate}

%------------------------------------------------

    \item
    Was ist passiert?
    \begin{enumerate}[1.]
        \item
        1.
    \end{enumerate}
    
%------------------------------------------------

    \item
    Gemeinsamkeiten:    
    \begin{enumerate}[1.]
        \item
        1.
    \end{enumerate}
    
%------------------------------------------------

    Unterschiede:
    \begin{enumerate}[1.]
        \item
        1.
    \end{enumerate}
\end{enumerate}


\end{document}