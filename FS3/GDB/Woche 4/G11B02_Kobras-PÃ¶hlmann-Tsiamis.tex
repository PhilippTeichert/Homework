\documentclass[ngerman]{gdb-aufgabenblatt}


\renewcommand{\Aufgabenblatt}{2}
\renewcommand{\Ausgabedatum}{Mi. 28.10.2015}
\renewcommand{\Abgabedatum}{Fr. 13.11.2015}
\renewcommand{\Gruppe}{Kobras, P�hlmann, Tsiamis}
\renewcommand{\STiNEGruppe}{11}


\begin{document}
\section{Informationsmodellierung mit dem Entity-Relationship-Modell}
\subsection{Teilaufgabe a}
F�r das ER-Diagramm zu dieser Aufgabe siehe Anhang A.
\subsection{Teilaufgabe b}

\section{Informationsmodellierung}
\subsection{Teilaufgabe a}
Ein Fahrzeug ist eindeutig identifizierbar �ber seine KFZ-Kennung und hat einen Fahrzeugtyp.
Ein Fahrzeug ist auf eine Person gemeldet.
Eine Person kann beliebig viele Fahrzeuge anmelden.
Eine Person ist �ber eine eindeutige Sozialversicherungsnummer identifizierbar und hat einen Namen.
\subsection{Teilaufgabe b}
Eine Vorlesung hat eine eindeutige Vorlesungsnummer und einen Namen.
Bis zu f�nf Nachfolger bauen auf einer Vorlesung auf.
Eine Vorlesung baut auf beliebig vielen Vorg�ngern auf.
\subsection{Teilaufgabe c}
Eine Person hat einen eindeutigen Namen bestehend aus Vor- und Nachname sowie eine Adresse.
Ein Mann ist eine Person und hat Schuhe.
Eine Frau ist eine Person und hat mehrere Schuhe.
\subsection{Teilaufgabe d}
Eine Superkraft hat eine Beschreibung und eine eindeutige Bezeichnung.
Eine Superkraft kann von beliebig vielen Superhelden mit unterschiedlichen Einschr�nkungen genutzt werden.
Ein Superheld kann mehrere unterschiedlich stark begrenzte Superkr�fte haben.
Ein Superheld hat einen Namen und ein Einsatzgebiet sowie eine eindeutige Heldennummer.
\subsection{Teilaufgabe e}

\section{Schl�sselkandidaten}
\subsection{Teilaufgabe a}
\subsubsection*{Eindeutige Felder}
\begin{enumerate}
	\item \textit{Postleitzahl} ist im gegebenen Datensatz eindeutig.
	\item \textit{Telefonnummer} ist im gegebenen Datensatz eindeutig.
\end{enumerate}
\subsubsection*{Eindeutige Feldkombinationen}
\begin{enumerate}
	\item \textit{Geburtsdatum-Telefonnr.} ist in diesem Kontext eine eindeutige Kombination, da kein Tupel doppelt vorkommt und es, abgesehen von Zwillingen, unwahrscheinlich, wenn nicht unm�glich, ist, dass zwei Personen sowohl am gleichen Tag geboren wurden als auch die gleiche Telefonnummer besitzen.
	\item \textit{Nachname-1.Fach} ist eine Alternative zur in 1 genannten Kombination, allerdings eher st�ranf�llig, da es - wenn auch selten - vorkommen kann, dass zwei Personen mit gleichem Nachnamen die selbe Kombination an F�chern belegen.
\end{enumerate}
\subsubsection*{Warum \textit{Vorname-Hausnr}}
Die beiden Fridas wohnen zwar in unterschiedlichen Stra�en, besitzen jedoch die gleiche Hausnummer, und w�rden folglich f�lschlicherweise als gleich erkannt werden.
\subsection{Teilaufgabe b}
Da die Ziffern- Adressen- und Namensmengen begrenzt sind, ist es unvermeidbar, dass es zu Doppelungen kommt, zum Beispiel beim Namen.
Auch wenn Telefonnummern und Adressen zumindest teilweise unterschiedlich sind (Vorwahl/andere Stadt), so werden sie sich doch soweit ann�hren, dass eine Verwechslung wahrscheinlich wird.
Bei Zwillingen zum Beispiel ist alles bis auf den Vornamen identisch.
Bei Nachbarn ist alles bis auf Vor- und Nachnamen und Telefonnummer identisch.
Gleiche Vornamen oder Namenskombinationen sind m�glich und je nach Name recht wahrscheinlich.\\
Eine L�sungsm�glichkeit ist die Einf�hrung einer fortlaufend vergebenen Identifikationsnummer.
Die Nummer kann derart zusammengesetzt werden, dass sie eine Kombination aus einer Zahl, die die Universit�t/Hochschule repr�sentiert, dem Immatrikulationsjahr sowie einer fortlaufend vergebenen Nummer ist.
Diese kann derart vergeben werden, dass jeder Studierende genau dann eine Nummer erh�lt, wenn seine Applikation akzeptiert wurde.
Durch ein Anpassen des Annahmeverfahrens l�sst sich gew�hrleisten, dass eine Nummer nicht doppelt vergeben wird, wenn zwei Studenten zeitgleich akzeptiert werden.
Da Personenverzeichnisse auch Matrikel genannt werden, kann man diese Identifikationsnummer als 'Matrikelnummer' bezeichnen.





\end{document}