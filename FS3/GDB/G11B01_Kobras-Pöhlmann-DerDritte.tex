\documentclass[ngerman]{gdb-aufgabenblatt}

\usepackage{enumerate}

\renewcommand{\Aufgabenblatt}{1}
\renewcommand{\Ausgabedatum}{Mi. 14.10.2015}
\renewcommand{\Abgabedatum}{Fr. 30.10.2015}
\renewcommand{\Gruppe}{Kobras, P�hlmann, Tsiamis}
\renewcommand{\STiNEGruppe}{11}
\renewcommand{\Semester}{WS 2015/16}

\begin{document}


%------------------------------------------------
\section{Informationssysteme}
%------------------------------------------------


\begin{enumerate}[a)]
    \item
    \textbf{Charakterisierung:}\\
    Ein Informationssystem ist ein System aus Menschen und/oder Maschinen, die Informationen erzeugen und/oder benutzen und die durch Kommunikationsbeziehungen miteinander verbunden sind. (s. Vorlesung 1 Folie 26)\\
    Aufgaben eines rechnergest�tzten Informationssystems sind: (s. Vorlesung 1 Folie 26)
    \begin{enumerate}[1.]
        \item
        Erfassung von Informationen
        \item
        Speicherung von Informationen
        \item
        Transformation von Informationen
    \end{enumerate}
    
%------------------------------------------------

    \item
    \textbf{Datenunabh�ngigkeit:}
    \begin{enumerate}[1.]
        \item
        logische Datenunabh�ngigkeit (Struktur der Daten):\\
        Die Daten sollen unabh�ngig von den Anwendungen organisiert werden, sodass logische �nderungen der Daten keine Auswirkungen auf die Funktionsweise der Programme haben.
        \item
        physische Datenunabh�ngigkeit (Speicherort, -art):\\
        Die physische Organisation der Daten soll ohne Auswirkungen auf die logische Struktur ge�ndert werden k�nnen.
    \end{enumerate}
    
%------------------------------------------------

    \item
    \textbf{Beispiele:}\\
    Anwendungsbeispiele f�r Informationssysteme:
    \begin{enumerate}[1.]
        \item
        \textbf{Beispiel 1: US Airforce}\\
        Die USAF benutzt das sog. \textit{TBMCS (Theater Battle Management Core Systems)} zur Einsatzkoordinierung
        \item
        \textbf{Beispiel 2: Deutsche Polizei}\\
        Die Polizei verwendet ein System zum Erfassen von Personen- und Gegenstandsdaten zur Vereinfachung von Fahndungsaktionen
        \item
        \textbf{Beispiel 3: Medizin}\\
        In der Medizin werden IS verwendet, um Patienten- und Falldaten zu verwalten.
    \end{enumerate}
    
\end{enumerate}

\newpage
%------------------------------------------------
\section{Miniwelt}
%------------------------------------------------

\begin{enumerate}[a)]
    \item
    Relevante Elemente:\\ \vspace{0.3cm}\\
    \begin{minipage}[t]{0.32\textwidth}
    \textbf{Tippspieler}
        \begin{itemize}
        	\item \textbf{Eigenschaften:}
        		\begin{itemize}
        			\item ID
        			\item Vorname
        			\item Nachname
        		\end{itemize}
        	\item \textbf{Aktionen:}
        		\begin{itemize}
        			\item anmelden
        			\item erstellen einer Gemeinschaft
        			\item Tippen
        			\item Ergebnisse einsehen
        			\item Punktestand einsehen
        		\end{itemize}
        \end{itemize}
    \end{minipage}
    \begin{minipage}[t]{0.32\textwidth}
    \textbf{Gemeinschaft}
   		\begin{itemize}
   			\item \textbf{Eigenschaften:}
   				\begin{itemize}
   					\item ID
   					\item Ergebnisse
   					\item Wettbewerbe
   					\item Begegnungen
   					\item Mitspieler
   					\item Punktestand
   				\end{itemize}
   		\end{itemize}
    \end{minipage}
    \begin{minipage}[t]{0.32\textwidth}
    	\textbf{Tippspielgr�nder}
   		\begin{itemize}
   			\item \textbf{Eigenschaften:}
   				\begin{itemize}
   					\item Tippspieler
   				\end{itemize}
   			\item \textbf{Aktionen}
   				\begin{itemize}
   					\item Wettbewerbe anlegen
   					\item Begegnungen zum Wettbewerb hinzuf�gen
   					\item Ergebnisse eintragen
   					\item Punkte verteilen
   					\item Mitspieler hinzuf�gen oder entfernen
   				\end{itemize}
   		\end{itemize}
    \end{minipage}
    \vspace{0.5cm}
    
%------------------------------------------------

    \item Anforderungen an die Anwendung:
    \begin{enumerate}[1.]
        \item \textbf{Kontrolle �ber die operationalen Daten}\\
        Alle Daten k�nnen/m�ssen gemeinsam benutzt werden.\\ 
        Der Tippspielgr�nder kann auf alle Daten seiner Gemeinschaft zugreifen und sie ver�ndern.\\
        Ein Tippspieler kann personenbezogene Daten abfragen, jedoch nicht manipulieren.\\
        Die Gemeinschaft ist eine Datenbank mit Speicher-  und Zuordnungsfunktion.\\ 
        Objekte m�ssen eindeutig identifizierbar sein, sodass nicht ein Spieler zweimal in der selben Gemeinschaft vorhanden sein kann.
        \item \textbf{Kontrolle der Datenintegrit�t}\\
        Tippspielern muss der Zugang zu bestimmten Aktionen und Daten verwehrt werden.
    \end{enumerate}
\end{enumerate}


%------------------------------------------------
\section{Transaktionen}
%------------------------------------------------

St�rzt das System zum Zeitpunkt A ab, so passiert nichts, da die ge�nderte Daten nicht auf der Platte gespeichert werden.\\
St�rzt das System zum Zeitpunkt B, so werden die Daten vom Konto mit der ID 7 auf der Platte gespeichert.
Zwar wurde im Konto 7 das Geld �berwiesen und gespeichert, jedoch wurde bei Konto bei Konto 5 nicht der entsprechende Betrag abgebucht.
F�r dieses Problem gibt es zwei L�sungen.\\
Die erste L�sung ist ein \textit{Rollback-Verfahren}.
Als Rollback bezeichnet das "Zur�cksetzen" der einzelnen Verarbeitungsschritte einer Transaktion.
Das System wird dadurch vollst�ndig auf den Zustand vor dem Beginn der Transaktion zur�ckgef�hrt.\\
Die zweite L�sung ist \textit{Isolation}.
Mit Isolation bezeichnet man die Trennung von Transaktionen auf eine Weise, dass eine laufende
Transaktion nicht von einer parallel ablaufenden Transaktion durch �nderung der benutzten Daten
in einen undefinierten Zustand gebracht werden kann. 

%------------------------------------------------
\section{Warm-Up MySQL}
%------------------------------------------------


\begin{enumerate}[a)]
    \item Was ist passiert?
    \begin{enumerate}[1.]
        \item .
        \item .
        \item .
    \end{enumerate}

%------------------------------------------------

    \item
    Was ist passiert?
    \begin{enumerate}[1.]
        \item
        1.
    \end{enumerate}
    
%------------------------------------------------

    \item
    Gemeinsamkeiten:    
    \begin{enumerate}[1.]
        \item
        1.
    \end{enumerate}
    
%------------------------------------------------

    Unterschiede:
    \begin{enumerate}[1.]
        \item
        1.
    \end{enumerate}
\end{enumerate}


\end{document}