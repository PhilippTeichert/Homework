\documentclass[ngerman]{gdb-aufgabenblatt}


\renewcommand{\Aufgabenblatt}{4}
\renewcommand{\Ausgabedatum}{Mi. 25.11.2015}
\renewcommand{\Abgabedatum}{Fr. 11.12.2015}
\renewcommand{\Gruppe}{Kobras, P�hlmann, Tsiamis}
\renewcommand{\STiNEGruppe}{11}


\begin{document}
\section{Relationenalgebra}
\subsection{Teilaufgabe a}
$\projektion{Sorte} ( \selektion{Vorname = \wert{Horst}} (\projektion{PNR, Vorname} (Person)) \verbund{Person.PNR = Obst.Entdecker} \projektion{Sorte, Entdecker}(Obst))$
%---------------------------------------------------------------

\subsection{Teilaufgabe b}
$\projektion{Vorname, Nachname} (\projektion{PNR, Vorname, Nachname}(Person) \verbund{Person.PNR = Allergie.Person} \selektion{Symptom=\wert{Halskratzen}}(\projektion{Person, Symptom}(Allergie)))$

%---------------------------------------------------------------

\subsection{Teilaufgabe c}
$\projektion{Obst.Sorte, Person.Nachname}(\projektion{Sorte, Entdecker}(Obst) \verbund{Obst.Entdecker = Person.PNR}$\\$\projektion{PNR, Nachname}(Person) \verbund{Person.PNR = Allergie.Person} \projektion{Person}(\selektion{Symptom = \wert{W"urgreiz}}(\projektion{Person, Symptom}(Allergie))))$

%---------------------------------------------------------------
%
%---------------------------------------------------------------

\section{Schemadefinition}
\textbf{CREATE TABLE} 'Person'\\
('Vorname' VARCHAR(30) \textbf{PRIMARY KEY NOT NULL},\\
'Nachname' VARCHAR(30) \textbf{PRIMARY KEY NOT NULL},\\
'Geb.Datum' DATE \textbf{CHECK}('Geb.Datum' < CURRENT\_DATE \textbf{OR} 'Geb.Datum' \textbf{IS NULL}),\\
'Wohnort' VARCHAR(30) \textbf{NOT NULL});\\
\\
\textbf{CREATE TABLE} 'Vorstand'\\
('VID' INT \textbf{PRIMARY KEY NOT NULL},\\
'Sitz' VARCHAR(30) \textbf{UNIQUE NOT NULL});\\
\\
\textbf{CREATE TABLE} 'Verein'\\
('Name' VARCHAR(30) \textbf{PRIMARY KEY NOT NULL},\\
'Gr"undungsjahr' DATE);\\
\\
\textbf{CREATE TABLE} 'ist in'\\
('Person Vorname' VARCHAR(30) \textbf{PRIMARY KEY NOT NULL},\\
'Person Nachname' VARCHAR(30) \textbf{PRIMARY KEY NOT NULL},\\
'Vorstand VID' VARCHAR(30) \textbf{PRIMARY KEY NOT NULL},\\
'Amt' VARCHAR(30) \textbf{NOT NULL},\\
'von' DATE \textbf{NOT NULL},\\
'bis' DATE,\\
\textbf{CONSTRAINT} 'ist in Person Vorname'\\
\textbf{FOREIGN KEY}('Person Vorname') \textbf{REFERENCES} 'Person' ('Vorname'),\\
\textbf{CONSTRAINT} 'ist in Person Nachname'\\
\textbf{FOREIGN KEY}('Person Nachname') \textbf{REFERENCES} 'Person' ('Nachname'),\\
\textbf{CONSTRAINT} 'ist in Vorstand VID'\\
\textbf{FOREIGN KEY}('Vorstand VID') \textbf{REFERENCES} 'Vorstand' ('VID'));\\
\\
\textbf{ALTER TABLE} 'Vorstand'\\
\textbf{ADD} 'Verein Name' VARCHAR(30) \textbf{NOT NULL},\\
\textbf{ADD CONSTRAINT} 'von Verein' \textbf{FOREIGN KEY} ('Verein Name') \textbf{REFERENCES} 'Verein' ('Name');\\
\\
\textbf{ALTER TABLE} 'Verein'\\
\textbf{ADD} 'Person Vorname' VARCHAR(30) \textbf{NOT NULL},\\
\textbf{ADD CONSTRAINT} 'trainiert Person Vorname' \textbf{FOREIGN KEY} ('Person Vorname') \textbf{REFERENCES} 'Person' ('Vorname'),\\
\textbf{ADD} 'Person Nachname' VARCHAR(30) \textbf{NOT NULL},\\
\textbf{ADD CONSTRAINT} 'trainiert Person Nachname' \textbf{FOREIGN KEY} ('Person Nachname') \textbf{REFERENCES} 'Person' ('Nachname');\\
\\
\textbf{ALTER TABLE} 'Person'\\
\textbf{ADD} 'Verein Name' VARCHAR(30) \textbf{NOT NULL},\\
\textbf{ADD CONSTRAINT} 'Fan von Verein' \textbf{FOREIGN KEY} ('Verein Name') \textbf{REFERENCES} 'Verein' ('Name');\\
\\
\textbf{CREATE TABLE} 'spielt f"ur'\\
'Person Vorname' VARCHAR(30) \textbf{PRIMARY KEY NOT NULL},\\
'Person Nachname' VARCHAR(30) \textbf{PRIMARY KEY NOT NULL},\\
'Verein Name' VARCHAR(30) \textbf{PRIMARY KEY NOT NULL},\\
('Position' VARCHAR(30) \textbf{NOT NULL},\\
\textbf{CONSTRAINT} 'spielt f"ur Person Vorname' \textbf{FOREIGN KEY} ('Person Vorname') \textbf{REFERENCES} 'Person' ('Vorname'),\\
\textbf{CONSTRAINT} 'spielt f"ur Person Nachname' \textbf{FOREIGN KEY} ('Person Nachname') \textbf{REFERENCES} 'Person' ('Nachname'),\\
\textbf{CONSTRAINT} 'spielt f"ur Verein Name' \textbf{FOREIGN KEY} ('Verein Name') \textbf{REFERENCES} 'Verein' ('Name'));\\

%---------------------------------------------------------------
%
%---------------------------------------------------------------

\section{Datenmanipulation mit SQL}
\subsection{Teilaufgabe a}
\subsubsection{i}
\textbf{SELECT DISTINCT} RS.Name, \textbf{COUNT}(RF) \textbf{AS} Anzahl\\
\textbf{FROM} Rennstall RS, RennfahrerIn RF\\
\textbf{WHERE} RF.Rennstall = RS.RSID;

%------------------------
\subsubsection{ii}
\textbf{SELECT} RF.Vorname, RF.Nachname\\
\textbf{FROM} RennfahrerIn RF\\
\textbf{WHERE NOT IN}\\
(\textbf{SELECT} RF2.Vorname, RF2.Nachname\\
\textbf{FROM} RennfahrerIn RF2, Platzierung PL\\
\textbf{WHERE} RF2.RID = PL.RID);

%------------------------
\subsubsection{iii}
\textbf{SELECT} RF.Nachname, MIN(PL.Platz) \textbf{AS} BestePlatzierung\\
\textbf{FROM} RennfahrerIn RF, Platzierung PL\\
\textbf{WHERE} RF.RID = PL.RID\\
\textbf{ORDER BY} PL.Platz;

%------------------------
\subsubsection{iv}
\textbf{SELECT} RF.Nachname, MIN(PL.Platz) \textbf{AS} BestePlatzierung\\
\textbf{FROM}  RennfahrerIn RF, Platzierung PL\\
\textbf{ORDER BY} RF.Nachname;

%------------------------
\subsubsection{v}
\textbf{SELECT *}\\
\textbf{FROM} RennfahrerIn RF\\
\textbf{WHERE} RF.Geburt \textbf{BETWEEN} 1980-01-01 \textbf{AND} 1985-01-01\\
\textbf{AND} RF.Nachname = \_a\%;

%------------------------
\subsubsection{vi}
\textbf{SELECT} RS.TeamchefIn\\
\textbf{FROM} Rennstall RS, RennfahrerIn RF\\
\textbf{WHERE} RS.RSID = RF.Rennstall\\
\textbf{AND} RF.RID \textbf{IN}\\
(\textbf{SELECT} RF2.RID\\
\textbf{FROM} RennfahrerIn RF2, Platzierung PL\\
\textbf{WHERE} RF2.RID = PL.RID\\
\textbf{AND} PL.OID IN\\
(\textbf{SELECT} PL2.OID\\
\textbf{FROM} Platzierung PL2\\
\textbf{WHERE} \textbf{COUNT}(PL2.OID)> 5));

%---------------------------------------------------------------

\subsection{Teilaufgabe b}
\subsubsection{i}
\textbf{UPDATE} RennfahrerIn RF\\
\textbf{SET} RF.Wohnort = $\wert{Aston~Clinton~(United~Kingdom)}$\\
\textbf{WHERE} RF.RID = $\wert{8}$;

%------------------------
\subsubsection{ii}
\textbf{UPDATE} Rennstall RS\\
\textbf{SET} RS.Budget = RS.Budget + 50\\
\textbf{WHERE EXISTS}\\
(\textbf{SELECT *}\\
\textbf{FROM} RennfahrerIn RF, Platzierung PL\\
\textbf{WHERE} RS.RSID = RF.Rennstall\\
\textbf{AND} RF.RID = PL.RID\\
\textbf{GROUP BY} RS.RSID\\
\textbf{HAVING AVG}(PL.Platz) < 5);\\

%------------------------
\subsubsection{iii}
\textbf{INSERT INTO} Rennort (Name, Strecke)\\
\textbf{VALUES} ('Deutschland GP', 'Hockenheimring');\\
\\
Dies wird so nicht funktionieren, da der Prim�rschl�ssel, welcher nicht "NULL" sein darf, fehlt und somit "NULL" ist.

%------------------------
\subsubsection{iv}
\textbf{INSERT INTO} Platzierung\\
\textbf{VALUES} (\textbf{SELECT} PL.RID\\
\textbf{FROM} Platzierung PL, Rennort RO\\
\textbf{WHERE} RO.Name = $\wert{Brasilien~GP}$\\
\textbf{AND} RO.OID = PL.OID\\
\textbf{AND} PL.Platz = 1,\\
\textbf{SELECT} RO.OID\\
\textbf{FROM} Rennort RO\\
\textbf{WHERE} RO.Name = $\wert{Deutschland~GP}$,\\
1);\\
\\
Dies wird so nicht funktionieren, da der "Deutschland GP" nicht richtig initialisiert wurde.

%------------------------
\subsubsection{v}
\textbf{INSERT INTO} Platzierung\\
\textbf{VALUES} (\textbf{SELECT} RF.RID\\
\textbf{FROM} RennfahrerIn RF\\
\textbf{WHERE} RF.Vorname = $\wert{Felipe}$\\
\textbf{AND} RF.Nachname = $\wert{Massa}$,\\
\textbf{SELECT} RO.OID\\
\textbf{FROM} Rennort RO\\
\textbf{WHERE} Ro.Name = $\wert{Gro�britannien~GP}$,\\
5);

%---------------------------------------------------------------
%
%---------------------------------------------------------------

\section{Anfrageoptimierung}
\begin{tikzpicture}
\node (RS) {Rennstall RS};
\node (R) [left=25mm of RS] {RennfahrerIn R};
\node (P) [right=25mm of RS] {Platzierung P};
\node (RO) [right=25mm of P] {Rennort RO};
\node (join1) [above=20mm of $(R)!.5!(RS)$] {$\verbund{R.Rennstall = RS.RSID}$};
\node (join2) [above=20mm of $(join1)!.5!(P)$] {$\verbund{R.RID = P.RID}$};
\node (join3) [above=20mm of $(join2)!.5!(RO)$] {$\verbund{P.OID = RO.OID}$};
\node (selektion1) [above=of join3] {$\selektion{P.Platz~IN~(1, 3, 5, 7)}$};
\node (selektion2) [above=of selektion1] {$\selektion{RS.Budget~BETWEEN~150~AND~250}$};
\node (projektion) [above=of selektion2] {$\projektion{R.Vorname, R.Nachname, RO.Name}$};
\node (final) [above=of projektion] {};

\path (RS) edge node[smallr,near start,above right] {10 Tupel\\4 Attribute} (join1);
\path (R) edge node[smalll,near start,above left] {30 Tupel\\6 Attribute} (join1);
\path (P) edge node[smallr,near start,above right] {10.000 Tupel\\3 Attribute} (join2);
\path (join1) edge node[smalll,near start,above left] {30 Tupel\\9 Attribute} (join2);
\path (RO) edge node[smallr,near start,above right] {20 Tupel\\3 Attribute} (join3);
\path (join2) edge node[smalll,near start,above left] {10.000 Tupel\\11 Attribute} (join3);
\path (join3) edge node[smallr,near start,above left] {10.000 Tupel\\13 Attribute} (selektion1);
\path (selektion1) edge node[smallr,midway,left] {$10.000\cdot\frac{4}{20}=2.000$ Tupel\\13 Attribute} (selektion2);
\path (selektion2) edge node[smallr,midway,left] {$2.000\cdot\frac{1}{4}=500$ Tupel\\13 Attribute} (projektion);
\path (projektion) edge node[smallr,midway,left] {$500$ Tupel\\3 Attribute} (final);
\end{tikzpicture}
\newpage

Der optimierte Baum:\\

\begin{tikzpicture}
\node (RS) {Rennstall RS};
\node (R) [left=25mm of RS] {RennfahrerIn R};
\node (P) [right=25mm of RS] {Platzierung P};
\node (RO) [right=25mm of P] {Rennort RO};
\node (projektion1) [above=of RS] {$\projektion{RS.RSID, RS.Budget}$};
\node (projektion2) [above=of R] {$\projektion{R.Vorname, R.Nachname, R.Rennstall, R.RID}$};
\node (projektion3) [above=of RO] {$\projektion{RO.OID, RO.Name}$};
\node (selektion1) [above=of P] {$\selektion{P.Platz~IN~(1, 3, 5, 7)}$};
\node (selektion2) [above=of projektion1] {$\selektion{RS.Budget~BETWEEN~150~AND~250}$};
\node (projektion4) [above=of selektion2] {$\projektion{RS.RSID}$};
\node (join1) [above=20mm of $(projektion2)!.5!(projektion4)$] {$\verbund{R.Rennstall = RS.RSID}$};
\node (join2) [above=20mm of $(selektion1)!.5!(projektion3)$] {$\verbund{P.OID = RO.OID}$};
\node (projektion5) [above=of join1] {$\projektion{R.Vorname, R.Nachname, R.RID}$};
\node (projektion6) [above=of join2] {$\projektion{RO.Name, P.RID}$};
\node (join3) [above=20mm of $(projektion5)!.5!(projektion6)$] {$\verbund{R.RID = P.RID}$};
\node (projektion7) [above=of join3] {$\projektion{R.Vorname, R.Nachname, RO.Name}$};
\node (final) [above=of projektion7] {};

\path (RS) edge node[smallr,near start,above right] {10 Tupel\\4 Attribute} (projektion1);
\path (R) edge node[smalll,near start,above left] {30 Tupel\\6 Attribute} (projektion2);
\path (P) edge node[smallr,near start,above left] {10.000 Tupel\\3 Attribute} (selektion1);
\path (RO) edge node[smallr,near start,above right] {20 Tupel\\3 Attribute} (projektion3);
\path (projektion1) edge node[smallr,near start,above right] {10 Tupel\\2 Attribute} (selektion2);
\path (projektion2) edge node[smalll,near start,above left] {30 Tupel\\4 Attribute} (join1);
\path (selektion2) edge node[smallr,near start,above right] {$10\cdot\frac{1}{4}\approx 3$ Tupel\\2 Attribute} (projektion4);
\path (projektion4) edge node[smalll,near start,above right] {$\approx 3$ Tupel\\1 Attribute} (join1);
\path (projektion3) edge node[smallr,near start,above right] {20 Tupel\\2 Attribute} (join2);
\path (selektion1) edge node[smallr,near start,above left] {$10.000\cdot\frac{4}{20}=2.000$ Tupel\\3 Attribute} (join2);
\path (join2) edge node[smalll,near start,above right] {2.000 Tupel\\4 Attribute} (projektion6);
\path (join1) edge node[smalll,near start,above left] {$30\cdot\frac{1}{4}\approx 8$ Tupel\\4 Attribute} (projektion5);
\path (projektion5) edge node[smallr,near start,above left] {$\approx 8$ Tupel\\3 Attribute} (join3);
\path (projektion6) edge node[smallr,near start,above right] {2.000 Tupel\\2 Attribute} (join3);
\path (join3) edge node[smallr,midway,left] {$2.000\cdot\frac{1}{4}=500$ Tupel\\4 Attribute} (projektion7);
\path (projektion7) edge node[smallr,midway,left] {$500$ Tupel\\3 Attribute} (final);
\end{tikzpicture}
\\
Bewertung Suchbaum 1(der obere):\\
Er ist �bersichtlich.
Allerdings unbalanciert.
Und er rechnet mit sehr gro�en Tupeln.
Und er rechnet durchg�ngig mit gro�en Zahlen.
\\
\\
Bewertung von Suchbaum 2 (der untere):\\
Er ist un�bersichtlich.
Allerdings balancierter.
Und er rechnet mit kleinen Tupeln im Vergleich zum ersten.
Und er rechnet mit kleineren Zahlen als der erste.


%---------------------------------------------------------------
%
%---------------------------------------------------------------

\end{document}