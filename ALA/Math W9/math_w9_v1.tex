\documentclass[parskip,12pt,paper=a4,sffamily]{article}
%alternate documentclass:
%\documentclass[parskip,12pt,paper=a4,sffamily]{scrartl}
\usepackage[utf8]{inputenc}
\usepackage[ngerman]{babel}
\usepackage{lastpage}
\usepackage{color}   %May be necessary if you want to color links
\usepackage{hyperref}
% code snippets
\usepackage{listings}
% listing captions
\usepackage{caption}
% font: times new roman
\usepackage{times}
% tikz being tikz
\usepackage{tikz}
\usetikzlibrary{arrows,automata}
\usepackage{pgf}
% import math packages
\usepackage{amsmath}
\usepackage{amsfonts}
\usepackage{amssymb}
\usepackage{amsthm}
% contradiction lightning
\usepackage{stmaryrd}
% alignment options
\usepackage{ragged2e}
% page margins
\usepackage[margin=2.5cm]{geometry}

\definecolor{pblue}{rgb}{0.13,0.13,1}
\definecolor{pgreen}{rgb}{0,0.5,0}


\lstset{ %
language=Java,   							% choose the language of the code
basicstyle=\small\ttfamily,  				% the size of the fonts that are used for the code
numbers=left,                   			% where to put the line-numbers
numbersep=5pt,                  			% how far the line-numbers are from the code
backgroundcolor=\color{light-light-gray},   % choose the background color. You must add
frame=lrtb,           						% adds a frame around the code
tabsize=4,          						% sets default tabsize to 2 spaces
captionpos=b,           					% sets the caption-position to bottom
breaklines=true,        					% sets automatic line breaking
xleftmargin=1.5cm,							% space from the left paper edge
commentstyle=\color{pgreen},
keywordstyle=\color{pblue},
literate=%
    {Ö}{{\"O}}1
    {Ä}{{\"A}}1
    {Ü}{{\"U}}1
    {ß}{{\ss}}1
    {ü}{{\"u}}1
    {ä}{{\"a}}1
    {ö}{{\"o}}1
    {~}{{\textasciitilde}}1
}
\renewcommand{\lstlistingname}{Code}
\captionsetup[lstlisting]{font={footnotesize},margin=1.5cm,singlelinecheck=false } % removes "Listing 1: "
\definecolor{light-light-gray}{gray}{0.95}
\let\stdsection\section
\renewcommand\section{\stdsection}

% add line break for subtitle (size: large)
\title{Mathematik für Studierende der Informatik II\\
	Analysis und Lineare Algebra\\
    \-\\\large{
    	Abgabe der Hausaufgaben zum \today
    }
}
\author{~\\
	\Large{Louis Kobras}\\
	\large{6658699}\\ %Matrikelnummer; wenn nicht für Uni, auskommentieren
	\large{4kobras@informatik.uni-hamburg.de}\\
	\\
	\Large{Utz Pöhlmann}\\
	\large{6663579}\\ %Matrikelnummer; wenn nicht für Uni, auskommentieren
	\large{4poehlma@informatik.uni-hamburg.de}\\
	\\
	\Large{Jennifer Hartmann}\\
	\large{6706472}\\ %Matrikelnummer; wenn nicht für Uni, auskommentieren
	\large{fwuy089@studium.uni-hamburg.de}
}

% leave empty for no date on title page
% comment for auto-generated date
\date{\today}


\begin{document}
	\maketitle
	\newpage
	\newgeometry{top=2.5cm, bottom =2.5cm, left=2cm, right=3.5cm}
	\section*{Aufgabe 1}
	\label{sec:a1}
	\begin{flushright}
	\large{[~~~~/4]}
	\end{flushright}
	%-----AUFGABE-----%
	Es seien $x,y,a \in (0, \infty)$.
	Zeigen Sie
	\[ log_a \left(\frac{x}{y}\right) =log_a x-log_a y.\]
	\-\\\\\\
	%-----RECHNUNG-----%
	Anwendung der \textit{log}-Definition:
	\[
		\text{(1)}~\begin{cases}
			log_a(f)=x \Leftrightarrow a^x=f\\
			log_a(g)=y \Leftrightarrow a^y=g
		\end{cases}
	\]
	Bildung des Quotienten $\frac{f}{g}$:
	\[
		\frac{f}{g}=\frac{a^x}{a^y}=a^{x-y}\text{~~~~~~~~~~~~\textit{nach Präsenzaufgabe 8.1}}
	\]
	Wiederum Anwenden der \textit{log}-Definition:
	\[
		\frac{f}{g}=a^{x-y}~~~~\Leftrightarrow~~~~log_a\left(\frac{f}{g}\right)=x-y
	\]
	Einsetzen von (1):
	\[
		log_a\left(\frac{f}{g}\right)=x-y=log_a(f)-log_a(g)
	\]
	\qed
	%++++++++++++++++++++++++++++++++++++++++++++++++++++++++++%
	\section*{Aufgabe 2}
	\label{sec:a2}
	\begin{flushright}
	\large{[~~~~/4]}
	\end{flushright}
	%-----AUFGABE-----%
	Berechnen Sie die Ableitungen folgender Funktionen:
	\begin{itemize}
		\item[(a)] $f(x)=(2x^2+3x+1)^4$
		\item[(b)] $g(x)=\sqrt{2x^2+x-3}$
		\item[(c)] $h(x)=\frac{1}{x^2-1}$
	\end{itemize}
	\-\\\\\\
	%-----RECHNUNG-----%
	(a).
	\begin{align*}
		\frac{d}{dx}(2x^2+3x+1)^4=4 \cdot (2x^2+3x+1)^3 \cdot (4x+3)
	\end{align*}
	(b).
	\begin{align*}
		\sqrt{2x^2+x-3}&=(2x^2+x-3)^{\frac{1}{2}}\\
		\frac{d}{dx}(2x^2+x-3)^{\frac{1}{2}}&=\frac{1}{2} \cdot (2x^2+x-3)^{-\frac{1}{2}} \cdot (4x+1)=\frac{4x+1}{2 \cdot \sqrt{(2x^2+x-3)}}
	\end{align*}
	(c).
	\begin{align*}
		\frac{1}{x^2-1}&=(x^2-1)^{-1}\\
		\frac{d}{dx}(x^2-1)^{-1}&=-1 \cdot (x^2-1)^{-2} \cdot 2x=-\frac{2x}{(x^2-1)^2}
	\end{align*}
	%++++++++++++++++++++++++++++++++++++++++++++++++++++++++++%
	\section*{Aufgabe 3}
	\label{sec:a3}
	\begin{flushright}
	\large{[~~~~/4]}
	\end{flushright}
	%-----AUFGABE-----%
	Berechnen Sie die Ableitungen der folgenden Funktionen:
	\begin{itemize}
		\item[(a)] $f(x)=e^{x^2}$
		\item[(b)] $g(x)=x\cdot e^{x^2}$
		\item[(c)] $h(x)=\frac{\operatorname{ln}x}{x}$
	\end{itemize}
	\-\\\\\\
	%-----RECHNUNG-----%
	(a). \textit{(Kettenregel)}
	\begin{align*}
		e^{x^2}&=e^{(x^2)}\\
		\frac{d}{dx}e^{(x^2)}&=e^{(x^2)} \cdot 2x
	\end{align*}
	(b). \textit{(Kettenregel und Produktregel)}
	\begin{align*}
		x\cdot e^{x^2}&=x\cdot e^{(x^2)}\\
		\frac{d}{dx}x\cdot e^{(x^2)}&=1 \cdot e^{x^2} ~+~x \cdot e^{(x^2)} \cdot 2x=e^{(x^2)}\cdot(2x^2+1)
	\end{align*}
	(c). \textit{(Produktregel)}
	\begin{align*}
		\frac{\operatorname{ln}x}{x}&=\operatorname{ln}(x) \cdot x^{-1}\\
		\frac{d}{dx}\operatorname{ln}(x) \cdot x^{-1}&=\frac{1}{x} \cdot x^{-1}~+~\operatorname{ln}(x) \cdot -1 \cdot x^{-2}=\frac{1}{x^2}~-~\operatorname{ln}(x)\cdot\frac{1}{x^2}=\frac{1}{x^2}(1-\operatorname{ln}(x))
	\end{align*}
	%++++++++++++++++++++++++++++++++++++++++++++++++++++++++++%
	\section*{Aufgabe 4}
	\label{sec:a4}
	\begin{flushright}
	\large{[~~~~/4]}
	\end{flushright}
	%-----AUFGABE-----%
	Beweisen Sie die Quotientenregel mit Hilfe der Produktregel, der Kettenregel und der Tatsache
	\[
		\left( \frac{1}{x} \right)'=-\frac{1}{x^2}
	\]
	\-\\\\\\
	%-----RECHNUNG-----%
	%++++++++++++++++++++++++++++++++++++++++++++++++++++++++++%
	\section*{Aufgabe 5}
	\label{sec:a5}
	\begin{flushright}
	\large{[~~~~/4]}
	\end{flushright}
	%-----AUFGABE-----%
	Berechnen Sie die Ableitung der Funktion $x \mapsto \sqrt[3]{x}$ mit Hilfe der Umkehrregel.
	\-\\\\\\
	%-----RECHNUNG-----%
\end{document}