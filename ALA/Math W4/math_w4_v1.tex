\documentclass[parskip,12pt,paper=a4,sffamily]{article}
\usepackage[utf8]{inputenc}
\usepackage[ngerman]{babel}
\usepackage{lastpage}
\usepackage{color}   %May be necessary if you want to color links
\usepackage{hyperref}
% code snippets
\usepackage{listings}
% listing captions
\usepackage{caption}
\usepackage{times} % font
\usepackage{tikz}
% import math packages
\usepackage{amsmath}
\usepackage{amsfonts}
\usepackage{amssymb}
\usepackage{amsthm}
% contradiction lightning
\usepackage{stmaryrd}
% multiple authors
\usepackage{authblk}
% margin settings
\usepackage[margin=2.5cm]{geometry}

\definecolor{pblue}{rgb}{0.13,0.13,1}
\definecolor{pgreen}{rgb}{0,0.5,0}


\lstset{ %
language=Java,   							% choose the language of the code
basicstyle=\small\ttfamily,  				% the size of the fonts that are used for the code
numbers=left,                   			% where to put the line-numbers
numbersep=5pt,                  			% how far the line-numbers are from the code
backgroundcolor=\color{light-light-gray},   % choose the background color. You must add
frame=lrtb,           						% adds a frame around the code
tabsize=4,          						% sets default tabsize to 2 spaces
captionpos=b,           					% sets the caption-position to bottom
breaklines=true,        					% sets automatic line breaking
xleftmargin=1.5cm,							% space from the left paper edge
commentstyle=\color{pgreen},
keywordstyle=\color{pblue},
literate=%
    {Ö}{{\"O}}1
    {Ä}{{\"A}}1
    {Ü}{{\"U}}1
    {ß}{{\ss}}1
    {ü}{{\"u}}1
    {ä}{{\"a}}1
    {ö}{{\"o}}1
    {~}{{\textasciitilde}}1
}
\renewcommand{\lstlistingname}{Code}
\captionsetup[lstlisting]{font={footnotesize},margin=1.5cm,singlelinecheck=false } % removes "Listing 1: "
\definecolor{light-light-gray}{gray}{0.95}
\let\stdsection\section
\renewcommand\section{\stdsection}

% add line break for subtitle (size: large)
\title{Übungsaufgaben zum\\30. April 2015\\\large{
Analysis und Lineare Algebra: Mathematik für Informatiker II}}
\author{~\\
	\Large{Louis Kobras}\\\large{6658699}\\\large{4kobras@informatik.uni-hamburg.de}
	\\
	\Large{Utz Pöhlmann}\\\large{6663579}\\\large{4poehlma@informatik.uni-hamburg.de}
}
% leave empty for auto-generated date
\date{29.04.2015}


\begin{document}
	\maketitle
	\newpage
	\section*{Aufgabe 1}
	\label{sec:a1}
	Sei $V$ ein endlich erzeugter Vektorraum. Die Vektoren $v_1$,...,$v_r$ $\in$ $V$ bilden ein \textit{minimales Erzeugendensystem},
	falls sie ein Erzeugendensystem von $V$ bilden,
	aber $V$ nicht mehr von den Vektoren erzeugt wird, wenn man einen beliebigen der Vektoren entfernt.
	Zeigen Sie, dass die Vektoren $v_1$,...,$v_r$ genau dann eine Basis von $V$ bilden,
	wenn sie ein minimales Erzeugendensystem sind. \\
	\\
	\\
	Seien $v_1$,...,$v_r$ ein minimales Erzeugendensystem.
	Eine Basis ist ein Erzeugendensystem aus linear unabhängigen Vektoren mit $|Basis|=Dim(Vektorraum)$.
	Sie ist damit das kleinstmögliche Erzeugendensystem, denn sobald ein Vektor aus der Basis entfernt wird, erzeugt sie den Vektorraum nicht mehr.
	Damit ist die Definition von \textit{Basis} und \textit{minimalem Erzeugendensystem} identisch.\\
	${}~~~~\Rightarrow~~$Basis = minimales Erzeugendensystem.
	\section*{Aufgabe 2}
	\label{sec:a2}
	Bestimmen Sie eine Basis des von den Polynomen
	    $x^2~-~1$, 
	    $x^2~+~x$, 
	    $3x~+~1$ und
	    $x^2~-~x~+~1$
	erzeugten Unterraums von $\mathbb{R}[x]$. \\
	\\
	\\
	\begin{equation*}
    	\begin{array}{c|c}
    	    \begin{split}
    	        &~~~~~~~~~~~\begin{pmatrix}
    	            x^2 &   &    & - & 1 \\
    	            x^2 & + &  x & \\
    	                & + & 3x & + & 1 \\
    	            x^2 & - &  x & + & 1
    	        \end{pmatrix}
    	        \\ \\
    	        &\begin{array}{lcccl}
    	            I~~~~& 1 &  0 & -1 & \\
    	            II~~~& 1 &  1 &  0 &~~~~~~~~II-I \\
    	            III~~& 0 &  3 &  1 & \\
    	            IV~~~& 1 & -1 &  1 &~~~~~~~~IV-I
    	        \end{array}
    	        \\ \\
	            &\begin{array}{lcccl}
    	            I~~~~& 1 &  0 & -1 & \\
    	            II~~~& 0 &  1 &  1 & \\
    	            III~~& 0 &  3 &  1 &~~~~~~~~III-3\cdot II \\
    	            IV~~~& 0 & -1 &  2 &~~~~~~~~IV+II
    	        \end{array}
    	    \end{split}
    	    {}~~~~~&~~~~~{}
    	    \begin{split}
    	        &\begin{array}{lcccl}
    	            I~~~~& 1 & 0 & -1 & \\
    	            II~~~& 0 & 1 &  1 & \\
    	            III~~& 0 & 0 & -2 &~~~~~~~~III\cdot\left(-\frac{1}{2}\right) \\
    	            IV~~~& 0 & 0 &  3 &
    	        \end{array}
    	        \\ \\
    	        &\begin{array}{lcccl}
    	            I~~~~& 1 & 0 & -1 &~~~~~~~~I+III\\
    	            II~~~& 0 & 1 &  1 &~~~~~~~~II-III\\
    	            III~~& 0 & 0 &  1 & \\
    	            IV~~~& 0 & 0 &  3 &~~~~~~~~IV-3\cdot III
    	        \end{array}
    	        \\ \\
    	        &\begin{array}{lcccl}
    	            I~~~~& 1 & 0 & 0 & \\
    	            II~~~& 0 & 1 & 0 & \\
    	            III~~& 0 & 0 & 1 & \\
    	            IV~~~& 0 & 0 & 0 &
    	        \end{array}
	        \end{split}
	    \end{array}
	\end{equation*}
	\\
	\\
	\[
	    \Rightarrow~~\underline{\underline{Basis:~~x^2, x, 1}}
	\]
	\newpage
	\section*{Aufgabe 3}
	\label{sec:a3}
	Konstruieren Sie jeweils eine Basis der folgenden Unterräume von $\mathbb{R}^3$ bzw $\mathbb{R}^4$:\\
	(a)$~~U_1~=~\{(x,y,z)\in\mathbb{R}^3:~~x-y-z=0\}$\\
	(b)$~~U_2~=~\{(x,y,z,w)\in\mathbb{R}^4:x+3y+z=0,~x+y+w=0\}$\\
	\\
	\\
	(a)\\
    \[
	    x-y-z=0
	\]
	\[
	    \Rightarrow x-(y+z)=0
	\]
	\[
	    \Rightarrow x+(-1)(y+z)=0
	\]\\
	\[
	    Seien:~
	    v_1=\begin{pmatrix}
	            4\\3\\1
	        \end{pmatrix}
	    ;v_2=\begin{pmatrix}
	            5\\2\\3
            \end{pmatrix}
        ;v_3=\begin{pmatrix}
                9\\4\\5
            \end{pmatrix}
    \]\\\\
    \begin{equation*}
        \begin{array}{l|l}
            \begin{split}
                \begin{array}{l}
                    \begin{array}{lcccl}
                        I~~~~ & 4 & 3 & 1 & ~~~~~~~~I\cdot\left(\frac{1}{4}\right)\\
                        II~~~ & 5 & 2 & 3 & \\
                        III~~ & 9 & 4 & 5 &
                    \end{array}
                    \\ \\
                    \begin{array}{lcccl}
                        I~~~~ & 1 & \frac{3}{4} & \frac{1}{4} & \\
                        II~~~ & 5 & 2           & 3           & ~~~~~~~~II-5\cdot I \\
                        III~~ & 9 & 4           & 5           & ~~~~~~~~III-9\cdot I
                    \end{array}
                    \\ \\
                    \begin{array}{lcccl}
                        I~~~~ & 1 &  \frac{3}{4}  & \frac{1}{4}  & \\
                        II~~~ & 0 & -\frac{7}{4}  & \frac{7}{4}  &~~~~~~~~II\cdot\left(-\frac{4}{7}\right) \\
                        III~~ & 0 & -\frac{11}{4} & \frac{11}{4} &
                    \end{array}
                \end{array}
            \end{split}
            {}~~~~~&~~~~~{}
            \begin{split}
                \begin{array}{l}
                    \begin{array}{lcccl}
                        I~~~~ & 1 &  \frac{3}{4}  &  \frac{1}{4}  & \\
                        II~~~ & 0 & 1             & -1            & \\
                        III~~ & 0 & -\frac{11}{4} &  \frac{11}{4} & ~~~~~~~~III-\left(\frac{11}{4}\right)\cdot II
                    \end{array}
                    \\ \\
                    \begin{array}{lcccl}
                        I~~~~ & 1 & \frac{3}{4} &  \frac{1}{4} & ~~~~~~~~I-\frac{3}{4}\cdot II\\
                        II~~~ & 0 & 1           & -1           & \\
                        III~~ & 0 & 0           &  0           &
                    \end{array}
                    \\ \\
                    \begin{array}{lcccl}
                        I~~~~ & 1 & 0 &  1\\
                        II~~~ & 0 & 1 & -1\\
                        III~~ & 0 & 0 &  0
                    \end{array}
                \end{array}
            \end{split}
        \end{array}
	\end{equation*}
	\\
	\[
	    \underline{\underline{Basis:~(1,0,1),(0,1,-1)}}
	\]
	\newpage
	(b)
	\[x=-(3y+z)\]
	\[x+3y+z=x+y+w\]
	\[3y+z=y+w\]
	\[2y+z=w\]
	\[
	Seien:~~
	    v_1=\begin{pmatrix}
	        3\\-1\\0\\2
	    \end{pmatrix};
	    v_2=\begin{pmatrix}
	        3\\-2\\3\\-1
	    \end{pmatrix};
	    v_3=\begin{pmatrix}
	        -5\\1\\2\\4
	    \end{pmatrix};
	    v_4=\begin{pmatrix}
	        -1\\-1\\-2\\0
	    \end{pmatrix}
	\]
	\begin{equation*}
    	\begin{array}{c|c}
	        \begin{split}
	            \begin{array}{l}
    	            \begin{array}{lccccl}
    	                I~~~~ &  3 & -1 &  0 &  2 & ~~~~~~~~~I:3\\
    	                II~~~ &  3 & -2 &  3 & -1 &\\
    	                III~~ & -5 &  1 &  2 &  4 &\\
    	                IV~~~ & -1 & -1 & -2 &  0 &
    	            \end{array}
    	            \\ \\
    	            \begin{array}{lccccl}
    	                I~~~~ & 1 & -\frac{1}{3} & 0 & \frac{2}{3} \\
    	                II~~~ &  3 & -2 &  3 & -1 &~~~~~~~~II-3\cdot I\\
    	                III~~ & -5 &  1 &  2 &  4 &~~~~~~~~III+5\cdot I\\
    	                IV~~~ & -1 & -1 & -2 &  0 &~~~~~~~~IV+I
    	            \end{array}
    	            \\ \\
    	            \begin{array}{lccccl}
    	                I~~~~ & 1 & -\frac{1}{3} & 0 & \frac{2}{3} \\
    	                II~~~ & 0 & -1 & 3 & -3 & ~~~~~~~II\cdot(-1)\\
    	                III~~ & 0 &-\frac{2}{3} & 2 & \frac{22}{3} & \\
    	                IV~~~ & 0 & -\frac{4}{3} & -2 & \frac{2}{3} & \\  
    	            \end{array}
    	            \\ \\
    	            \begin{array}{lccccl}
    	                I~~~~ & 1 & -\frac{1}{3} & 0 & \frac{2}{3} \\
    	                II~~~ & 0 & 		   1 & -3 & 3 & \\
    	                III~~ & 0 & -\frac{2}{3} & 2 & \frac{22}{3} & ~~~~~~~~III+\frac{2}{3}\cdot II\\
    	                IV~~~ & 0 & -\frac{4}{3} & -2 & \frac{2}{3} &~~~~~~~~IV+\frac{4}{3}\cdot II \\ 
    	            \end{array}
    	            \\ \\
    	            \begin{array}{lccccl}
    	                I~~~~ & 1 & -\frac{1}{3} & 0 & \frac{2}{3} \\
    	                II~~~ & 0 & 1 & -3 & 3 & \\
    	                III~~ & 0 & 0 & 0 & \frac{28}{3} & ~~~~~~~tausche~mit~IV\\
    	                IV~~~ & 0 & 0 & 2 & \frac{11}{3} & ~~~~~~~tausche~mit~III
    	            \end{array}
    	            \\ \\
    	            \begin{array}{lccccl}
    	                I~~~~ & 1 & -\frac{1}{3} & 0 & \frac{2}{3} \\
    	                II~~~ & 0 & 1 & -3 & 3 & \\
    	                III~~ & 0 & 0 & 2 & \frac{11}{3} & ~~~~~~~~III:2\\
    	                IV~~~ & 0 & 0 & 0 & \frac{28}{3} & ~~~~~~~~IV\cdot\frac{3}{28}
    	            \end{array}
    	            \\ \\
    	            \begin{array}{lccccl}
    	                I~~~~ & 1 & -\frac{1}{3} & 0 & \frac{2}{3} & ~~~~~~~~I+\frac{1}{3}\cdot II\\
    	                II~~~ & 0 & 1 & -3 & 3 & \\
    	                III~~ & 0 & 0 & 1 & \frac{11}{6}& \\
    	                IV~~~ & 0 & 0 & 0 & 1 &
    	            \end{array}
	            \end{array}
	        \end{split}
	        {}~~~~~&~~~~~{}
	        \begin{split}
	            \begin{array}{l}
	                \begin{array}{lccccl}
    	                I~~~~ & 1 & 0 & -1 & \frac{5}{3}& ~~~~~~~~I-III\\
    	                II~~~ & 0 & 1 & -3 & 3 & \\
    	                III~~ & 0 & 0 & 1 & \frac{11}{6}& \\
    	                IV~~~ & 0 & 0 & 0 & 1 &
    	            \end{array}
    	            \\ \\
    	            \begin{array}{lccccl}
    	                I~~~~ & 1 & 0 & 0 & -\frac{1}{6} & ~~~~~~~~I+\frac{1}{6}\cdot IV\\
    	                II~~~ & 0 & 1 & -3 & 3 & \\
    	                III~~ & 0 & 0 & 1 & \frac{11}{6}& \\
    	                IV~~~ & 0 & 0 & 0 & 1 & 
    	            \end{array}
    	            \\ \\
    	            \begin{array}{lccccl}
    	                I~~~~ & 1 & 0 & 0 & 0 & \\
    	                II~~~ & 0 & 1 & -3 & 3 & ~~~~~~~~II+3\cdot III\\
    	                III~~ & 0 & 0 & 1 & \frac{11}{6}& \\
    	                IV~~~ & 0 & 0 & 0 & 1 & 
    	            \end{array}
    	            \\ \\
    	            \begin{array}{lccccl}
    	                I~~~~ & 1 & 0 & 0 & 0 & \\
    	                II~~~ & 0 & 1 & 0 & \frac{17}{2} & ~~~~~~~~II-\frac{2}{17}\cdot IV\\
    	                III~~ & 0 & 0 & 1 & \frac{11}{6}& \\
    	                IV~~~ & 0 & 0 & 0 & 1 &  
    	            \end{array}
    	            \\ \\
    	            \begin{array}{lccccl}
    	                I~~~~ & 1 & 0 & 0 & 0 & \\
    	                II~~~ & 0 & 1 & 0 & 0 & \\
    	                III~~ & 0 & 0 & 1 & \frac{11}{6} & ~~~~~~~~III-\frac{11}{6}\cdot IV \\
    	                IV~~~ & 0 & 0 & 0 & 1 & 
    	            \end{array}
    	            \\ \\
    	            \begin{array}{lccccl}
    	                I~~~~ & 1 & 0 & 0 & 0 & \\
    	                II~~~ & 0 & 1 & 0 & 0 & \\
    	                III~~ & 0 & 0 & 1 & 0 & \\
    	                IV~~~ & 0 & 0 & 0 & 1 & 
    	            \end{array}
    	            \\ \\
    	            \Rightarrow~~\underline{\underline{Basis:~(1,0,0,0),(0,1,0,0),}}\\\underline{\underline{(0,0,1,0),(0,0,0,1)}}
	            \end{array}
	        \end{split}
    	\end{array}
	\end{equation*}
	\newpage
	\section*{Aufgabe 4}
	\label{sec:a4}
	Bestimmen Sie eine Basis des von den Vektoren
    	$v_1=(-1,2,1,-1)$,
    	$v_2=(0,2,2,1)$,
    	$v_3=(0,0,2,-1)$ und
    	$v_4=(1,0,1,-3)$
	erzeugten Unterraums von $\mathbb{R}^4$. \\
	\\
	\\
	\begin{equation*}
    	\begin{array}{c|c}
    	    \begin{split}
        	    \begin{array}{l}
        	        \begin{array}{lccccl}
        	            I~~~~ & -1 & 2 & 1 & -1 &~~~~~~~~I\cdot\left(-1\right) \\
        	            II~~~ &  0 & 2 & 2 &  1 & \\
        	            III~~ &  0 & 0 & 2 & -1 & \\
        	            IV~~~ &  1 & 0 & 1 & -3 &
        	        \end{array}
        	        \\ \\
        	        \begin{array}{lccccl}
        	            I~~~~ & 1 & -2 & -1 &  1 & \\
        	            II~~~ & 0 &  2 & 2  &  1 & \\
        	            III~~ & 0 &  0 & 2  & -1 & \\
        	            IV~~~ & 1 &  0 & 1  & -3 &~~~~~~~~IV-I
        	        \end{array}
        	        \\ \\
        	        \begin{array}{lccccl}
        	            I~~~~ & 1 & -2 & -1 & 1 & \\
        	            II~~~ & 0 &  2 & 2 &  1 &~~~~~~~~II\cdot\left(\frac{1}{2}\right) \\
        	            III~~ & 0 &  0 & 2 & -1 & \\
        	            IV~~~ & 0 &  2 & 2 & -4 &
        	        \end{array}
        	        \\ \\
        	        \begin{array}{lccccl}
        	            I~~~~ & 1 & -2 & -1 &  1          & \\
        	            II~~~ & 0 &  1 &  1 & \frac{1}{2} & \\
        	            III~~ & 0 &  0 &  2 & -1          & \\
        	            IV~~~ & 0 &  2 &  2 & -4          &~~~~~~~~IV-2\cdot II
        	        \end{array}
    	        \end{array}
    	    \end{split}
    	    {}~~~~~&~~~~~{}
    	    \begin{split}
        	    \begin{array}{l}
        	        \begin{array}{lccccl}
        	            I~~~~ & 1 & -2 & -1 & 1           &~~~~~~~~I+2\cdot II \\
        	            II~~~ & 0 &  1 &  1 & \frac{1}{2} & \\ 
        	            III~~ & 0 &  0 & 2 & -1           &~~~~~~~~III:2 \\
        	            IV~~~ & 0 &  0 & 0 & -5           &~~~~~~~~IV\cdot\left(-\frac{1}{2}\right)
        	        \end{array}
        	        \\ \\
        	        \begin{array}{lccccl}
        	            I~~~~ & 1 & 0 & 1 &  2           &~~~~~~~~I-III\\ 
        	            II~~~ & 0 & 1 & 1 &  \frac{1}{2} &~~~~~~~~II-III\\  
        	            III~~ & 0 & 0 & 1 & -\frac{1}{2} &~~~~~~~~III+\frac{1}{2}IV\\
        	            IV~~~ & 0 & 0 & 0 &  1           & \\
        	        \end{array}
        	        \\ \\
        	        \begin{array}{lccccl}
        	            I~~~~ & 1 & 0 & 0 & \frac{5}{2} &~~~~~~~~I-\frac{5}{2}IV\\
        	            II~~~ & 0 & 1 & 0 & 1           &~~~~~~~~II-IV\\
        	            III~~ & 0 & 0 & 1 & 0           & \\
        	            IV~~~ & 0 & 0 & 0 & 1           &
    	            \end{array}
        	        \\ \\
        	        \begin{array}{lccccl}
        	            I~~~~ & 1 & 0 & 0 & 0 & \\
        	            II~~~ & 0 & 1 & 0 & 0 & \\
        	            III~~ & 0 & 0 & 1 & 0 & \\
        	            IV~~~ & 0 & 0 & 0 & 1 &
        	        \end{array}
    	        \end{array}
    	    \end{split}
    	\end{array}
	\end{equation*}
	\\ \\
	\[
	    \underline{\underline{Basis: (1,0,0,0),(0,1,0,0),(0,0,1,0),(0,0,0,1)}}
	\]
	\[\Rightarrow~~ \underline{\underline{erzeugen~ganz~\mathbb{R}^4}}\]
	\newpage
	\section*{Aufgabe 5}
	\label{sec:a5}
	Sei $t\in\mathbb{R}$.
	Bestimmen Sie die Dimension des von den Vektoren $v_1=(1,t,1), v_2=(2,2t,t)$ und $v_3=(-1,1,2t)$ erzeugten Untervektorraums $U_t$ von $\mathbb{R}^3$.\\
	\textit{Hinweis:} Die Dimension hängt von $t$ ab! \\
	\\
	\\
	\begin{equation*}	    
	        \begin{split}
	            \begin{array}{l}
	                \begin{array}{lcccl}
	                    I~~~~ &  1 &  t &  1 & \\
	                    II~~~ &  2 & 2t &  t &~~~~~~~~II-2\cdot I\\
	                    III~~ & -1 &  1 & 2t &~~~~~~~~III+I
	                \end{array}
	                \\ \\
	                \begin{array}{lcccl}
	                    I~~~~ &  1 & t   & 1    & \\
	                    II~~~ &  0 & 0   & t-2  & \\
	                    III~~ &  0 & t+1 & 2t+1 & 
	                \end{array}
	                \\
	                \\
	                Tausche~II~und~III.
	                \\
	                \\
	                \begin{array}{lcccl}
	                    I~~~~ & 1 & t   & 1    & \\
	                    II~~~ & 0 & t+1 & 2t+1 & \\
	                    III~~ & 0 & 0   & t-2  &
	                \end{array}
	                \\ \\
	              %  \begin{array}{lcccl}
	              %      I~~~~ & \\
	              %      II~~~ & \\
	              %      III~~ & 
	              %  \end{array}
	            \end{array}
	        \end{split}
	\end{equation*}
	Ist $t~=~2$, so lautet der letzte Vektor $v_3=(0,0,2-2)=(0,0,0)$.
	Somit ist für $t=2$ die Dimension des erzeugten Unterraums 2.\\
	Für alle anderen t ist die Diemnsion 3, da kein Vektor zu $(0,0,0)$ wird.
\end{document}