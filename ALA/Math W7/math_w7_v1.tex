\documentclass[parskip,12pt,paper=a4,sffamily]{article}
%alternate documentclass:
%\documentclass[parskip,12pt,paper=a4,sffamily]{scrartl}
\usepackage[utf8]{inputenc}
\usepackage[ngerman]{babel}
\usepackage{lastpage}
\usepackage{color}   %May be necessary if you want to color links
\usepackage{hyperref}
% code snippets
\usepackage{listings}
% listing captions
\usepackage{caption}
% font: times new roman
\usepackage{times}
% tikz being tikz
\usepackage{tikz}
\usetikzlibrary{arrows,automata}
\usepackage{pgf}
% import math packages
\usepackage{amsmath}
\usepackage{amsfonts}
\usepackage{amssymb}
\usepackage{amsthm}
% contradiction lightning
\usepackage{stmaryrd}
% alignment options
\usepackage{ragged2e}
% page margins
\usepackage[margin=2.5cm]{geometry}

\definecolor{pblue}{rgb}{0.13,0.13,1}
\definecolor{pgreen}{rgb}{0,0.5,0}


\lstset{ %
language=Java,   							% choose the language of the code
basicstyle=\small\ttfamily,  				% the size of the fonts that are used for the code
numbers=left,                   			% where to put the line-numbers
numbersep=5pt,                  			% how far the line-numbers are from the code
backgroundcolor=\color{light-light-gray},   % choose the background color. You must add
frame=lrtb,           						% adds a frame around the code
tabsize=4,          						% sets default tabsize to 2 spaces
captionpos=b,           					% sets the caption-position to bottom
breaklines=true,        					% sets automatic line breaking
xleftmargin=1.5cm,							% space from the left paper edge
commentstyle=\color{pgreen},
keywordstyle=\color{pblue},
literate=%
    {Ö}{{\"O}}1
    {Ä}{{\"A}}1
    {Ü}{{\"U}}1
    {ß}{{\ss}}1
    {ü}{{\"u}}1
    {ä}{{\"a}}1
    {ö}{{\"o}}1
    {~}{{\textasciitilde}}1
}
\renewcommand{\lstlistingname}{Code}
\captionsetup[lstlisting]{font={footnotesize},margin=1.5cm,singlelinecheck=false } % removes "Listing 1: "
\definecolor{light-light-gray}{gray}{0.95}
\let\stdsection\section
\renewcommand\section{\stdsection}

% add line break for subtitle (size: large)
\title{Hausaufgaben zum \today
    \\\large{
		Mathematik für Studierende der Informatik II\\
		(Analysis und Lineare Algebra)
    }
}
\author{~\\
	\Large{Louis Kobras}\\
	\large{6658699}\\ %Matrikelnummer; wenn nicht für Uni, auskommentieren
	\large{4kobras@informatik.uni-hamburg.de}\\
	\\
	\Large{Utz Pöhlmann}\\
	\large{6663579}\\ %Matrikelnummer; wenn nicht für Uni, auskommentieren
	\large{4poehlma@informatik.uni-hamburg.de}\\
	\\
	\Large{Jennifer Hartmann}\\
	\large{6706472}\\
	\large{fwuy089@studium.uni-hamburg.de}
}

% leave empty for no date on title page
% comment for auto-generated date
\date{\today}


\begin{document}
	\maketitle
	\newpage
	\section*{Aufgabe 1}
	\label{sec:a1}
	\subsection*{(a)}
	\label{ssec:a1.a}
	Untersuchen Sie die Menge
	\[M=\left\lbrace\frac{n+1}{m}:n,m\in\mathbb{N}\right\rbrace\]
	auf Beschränktheit nach oben und unten und bestimmen Sie gegebenenfalls Supremum und Infimum.\\
	\\
	\begin{equation*}
		\begin{split}
		\text{1. Fall: }&n=m=1\\
			&\frac{1+1}{1}=\frac{2}{1}=2\\
			\text{Sup}&\text{remum: }2\\
			\text{Inf}&\text{imum: }2\\
		\\
		\text{3. Fall: }&n=1,m\rightarrow+\infty\\
			&\frac{1+1}{m}=\frac{2}{m}\rightarrow 0\\
			\text{Sup}&\text{remum: }2\\
			\text{Inf}&\text{imum: }0^\text{+}
		\end{split}
		~~~~ ~~~~ ~~~~
		\begin{split}
		\text{2. Fall: }&n\rightarrow+\infty;m=1\\
			&\frac{n+1}{1}\rightarrow\frac{\infty}{1}\rightarrow+\infty\\
			\text{Sup}&\text{remum: }+\infty^\text{-}\\
			\text{Inf}&\text{imum: }2\\
		\\
		\text{4. Fall: }&n=m\rightarrow+\infty\\
			&\frac{n+1}{m}\rightarrow\infty\\
			\text{Sup}&\text{remum: }2\\
			\text{Inf}&\text{imum: }+\infty^\text{-}
		\end{split}
	\end{equation*}\\
	Somit ergibt sich, dass die Folge nach unten durch 0 beschränkt ist, wohingegen sie nach oben unbeschränkt ist.\\
	Supremum: $+\infty^\text{-}$\\
	Infimum: $0^\text{+}$
	\subsection*{(b)}
	\label{ssec:a1.b}
	Untersuchen Sie die Folge $(a_n)$ mit
	\[a_n=\frac{1}{3}-\frac{1}{2n}\]
	und bestimmen Sie gegebenenfalls den Grenzwert.\\
	\\
	\\
	Da $\frac{1}{3}$ konstant ist, ist die Folge allein von $\frac{1}{2n}$ abhängig.
	\[
		\begin{cases}
			\begin{split}
				n=1: & \frac{1}{2n}=\frac{1}{2}\\
				n\rightarrow+\infty: & \frac{1}{2n}\rightarrow 0
			\end{split}
		\end{cases}
	\]
	Somit sind die Beschränkungen der Folge
	\[
		\begin{cases}
			\begin{split}
				n=1: & \frac{1}{3}-\frac{1}{2}=-\frac{1}{6}\\
				n\rightarrow+\infty: & \frac{1}{2n}\rightarrow\frac{1}{3}
			\end{split}
		\end{cases}
	\]
	und der Grenzwert ist dementsprechend $\frac{1}{3}$.
	\section*{Aufgabe 2}
	\label{sec:a2}
	Untersuchen Sie das Konvergenzverhalten der Folge $(a_n)$ mit
	\[a_n=\left(2+\frac{1}{n}\right)^n\]
	und bestimmen Sie gegebenenfalls den Grenzwert.\\
	\\
	\\
	Es ist offensichtlich, dass $\frac{1}{n}$ gegen $0$  geht, je größer \textit{n} wird.
	Somit nähert sich der Ausdruck in der Klammer $2^\text{+}$ an.
	Da $2^\text{+}>2$ gilt $a_n>2^n$.
	$2^n$ ist nach oben nicht beschränkt, somit geht auch $a_n$ gegen $+\infty^\text{-}$ als Grenzwert.
	Als Infimum wird 3 festgestellt, da $n\geq 1\Rightarrow (2+\frac{1}{n})^n=(2+\frac{1}{1})^1=(2+1)^1=3^1=3$.
	\section*{Aufgabe 3}
	\label{sec:a3}
	Untersuchen Sie das Konvergenzverhalten der Folgen $(a_n)$ und $(b_n)$ mit
	\[a_n=\frac{3n^2-3n}{2n^2-1} \text{ und } b_n=\frac{3n^2-3n}{2n^3-1}\]
	und bestimmen Sie gegebenenfalls den Grenzwert.\\
	\\
	\\
	\textbf{Folge $a_n$:}\\
	\label{ssec:an}
	\textsc{Begrenzung:} Nach Satz 4.27.(4) im Skript ist der Grenzwert einer Folge, die in Form eines Quotienten vorliegt, eben der Quotient der Grenzwert der einzelnen Folgen im Zähler und im Nenner.
	\begin{align*}
		\underset{n\rightarrow+\infty}{lim}(3n^2-3n)&=n(3n-3)\\
		\underset{n\rightarrow+\infty}{lim}(2n^2-1)&=n\left(2n-\frac{1}{n}\right)
	\end{align*}
	Hier kürzt sich $n$ direkt raus, sodass sich als Folgen $(3n-3)$ und $\left(2n-\frac{1}{n}\right)$ ergeben.\\
	Da für $n\rightarrow+\infty$ der Ausdruck $\frac{1}{n}$ gegen 0 geht, ergibt sich für den Grenzwert von $a_n$ folgende vereinfachte Form:
	\[\frac{3n-3}{2n}\]
	Diese Form lässt sich weiter vereinfachen:
	\[\frac{3n-3}{2n}=\frac{3-\frac{3}{n}}{2}\]
	Auch hier wissen wir, dass $\frac{3}{n}$ gegen 0 geht.\\
	Folglich können wir schreiben:
	\[\underset{n\rightarrow+\infty}{lim}(a_n)=\underset{n\rightarrow+\infty}{lim}\frac{3n^2-3n}{2n^2-1}=\frac{3}{2}\]
	Dies war der Fall $n\rightarrow+\infty$.
	Es gilt noch den Fall $n=1$ zu beachten.
	Hier können wir einfach einsetzen:
	\[\frac{3n^2-3n}{2n^2-1}|n=1\Rightarrow \frac{3\cdot 1^2-3\cdot 1}{2\cdot 1^2-1}=\frac{3-3}{2-1}=\frac{0}{1}=0\]
	\-~\textsc{Konvergenz:} Nach Begrenzung geht $a_n$ gegen $1.5^\text{+}$\\
	\\
	\textbf{Folge $b_n$:}\\
	\label{ssec:bn}
	\textsc{Begrenzung:}
	Verfahren wie bei $a_n$ nach Satz 4.27.(4) im Skript.\\
	\[\underset{n\rightarrow+\infty}{lim}b_n~~~=\underset{n\rightarrow+\infty}{lim}~~~\frac{3n^2-3n}{2n^3-1}=\underset{n\rightarrow+\infty}{lim}~~~\frac{n(3n-3)}{n\left(2n^2-\frac{1}{n}\right)}=\underset{n\rightarrow+\infty}{lim}~~~\frac{3n-3}{2n^2}=\underset{n\rightarrow+\infty}{lim}~~~\frac{n\left(3-\frac{3}{n}\right)}{n(2n)}\]
	\[=\underset{n\rightarrow+\infty}{lim}~~~\frac{3-\frac{3}{n}}{2n}=\underset{n\rightarrow+\infty}{lim}~~~\frac{3}{2n}-\frac{\frac{3}{n}}{2n}=\underset{n\rightarrow+\infty}{lim}~~~0-\frac{3}{2n^2}=0\]
	Für $n=1$ ergibt sich wiederum 0.\\
	\-~\textsc{Konvergenz:} Nach Begrenzung konvergiert $b_n$ gegen $0$.\\
	\section*{Aufgabe 4}
	\label{sec:a4}
	Es sei $(a_n)$ eine konvergente Folge reeller Zahlen mit lim$_{n\rightarrow\infty}a_n=2$.
	Bestimmen Sie den Grenzwert der Folge $(b_n)$ mit
	\[b_n=(a^2_n-2)^2-3.\]\\
	\\
	Da $a_n\rightarrow 2$, können wir diesen Wert einfach in $b_n$ einsetzen und erhalten dann folgende Gleichung für den Grenzwert $b$ der Folge $b_n$:
	\[b=(2^2-2)^2-3=(4-2)^2-3=4-3=1\]
	\section*{Aufgabe 5}
	\label{sec:a5}
	\subsection*{(a)}
	\label{ssec:a5.a}
	Sei $(a_n)$ eine beschränkte Folge.
	Für jedes $n\in\mathbb{N}$ sei $b_n=\text{sup}\{a_m:m\ge n\}$.
	Zeigen Sie, dass $(b_n)$ konvergiert.\\
	Hinweis: Ist die Folge $(b_n)$ monoton? Ist sie beschränkt?
	\subsection*{(b)}
	\label{ssec:a5.b}
	Zeigen Sie, dass jede Cauchy-Folge reeller Zahlen konvergiert.\\
	Hinweis: Benutzen Sie (a), um einen Kandidaten für den Grenzwert der Cauchy-Folge zu finden.
\end{document}