\documentclass[parskip,12pt,paper=a4,sffamily]{article}
%alternate documentclass:
%\documentclass[parskip,12pt,paper=a4,sffamily]{scrartl}
\usepackage[utf8]{inputenc}
\usepackage[ngerman]{babel}
\usepackage{lastpage}
\usepackage{color}   %May be necessary if you want to color links
\usepackage{hyperref}
% code snippets
\usepackage{listings}
% listing captions
\usepackage{caption}
% tikz being tikz
\usepackage{tikz}
\usetikzlibrary{arrows,automata}
\usepackage{pgf}
% import math packages
\usepackage{amsmath}
\usepackage{amsfonts}
\usepackage{amssymb}
\usepackage{amsthm}
% contradiction lightning
\usepackage{stmaryrd}
% alignment options
\usepackage{ragged2e}
% page margins
\usepackage[margin=2.5cm]{geometry}

\definecolor{pblue}{rgb}{0.13,0.13,1}
\definecolor{pgreen}{rgb}{0,0.5,0}


\lstset{ %
language=Java,   							% choose the language of the code
basicstyle=\small\ttfamily,  				% the size of the fonts that are used for the code
numbers=left,                   			% where to put the line-numbers
numbersep=5pt,                  			% how far the line-numbers are from the code
backgroundcolor=\color{light-light-gray},   % choose the background color. You must add
frame=lrtb,           						% adds a frame around the code
tabsize=4,          						% sets default tabsize to 2 spaces
captionpos=b,           					% sets the caption-position to bottom
breaklines=true,        					% sets automatic line breaking
xleftmargin=1.5cm,							% space from the left paper edge
commentstyle=\color{pgreen},
keywordstyle=\color{pblue},
literate=%
    {Ö}{{\"O}}1
    {Ä}{{\"A}}1
    {Ü}{{\"U}}1
    {ß}{{\ss}}1
    {ü}{{\"u}}1
    {ä}{{\"a}}1
    {ö}{{\"o}}1
    {~}{{\textasciitilde}}1
}
\renewcommand{\lstlistingname}{Code}
\captionsetup[lstlisting]{font={footnotesize},margin=1.5cm,singlelinecheck=false } % removes "Listing 1: "
\definecolor{light-light-gray}{gray}{0.95}
\let\stdsection\section
\renewcommand\section{\stdsection}

% add line break for subtitle (size: large)
\title{Mathematik für Studierende der Informatik II\\
	Analysis und Lineare Algebra\\
    \-\\\large{
    	Abgabe der Hausaufgaben zum \today
    }
}
\author{~\\
	\Large{Louis Kobras}\\
	\large{6658699}\\ %Matrikelnummer; wenn nicht für Uni, auskommentieren
	\large{4kobras@informatik.uni-hamburg.de}\\
	\\
	\Large{Utz Pöhlmann}\\
	\large{6663579}\\ %Matrikelnummer; wenn nicht für Uni, auskommentieren
	\large{4poehlma@informatik.uni-hamburg.de}\\
	\\
	\Large{Jennifer Hartmann}\\
	\large{6706472}\\ %Matrikelnummer; wenn nicht für Uni, auskommentieren
	\large{fwuy089@studium.uni-hamburg.de}
}

% leave empty for no date on title page
% comment for auto-generated date
\date{\today}


\begin{document}
	\maketitle\thispagestyle{empty}
	\newpage
	\pagenumbering{arabic}
	\newgeometry{top=2.5cm, bottom =2.5cm, left=2cm, right=3.5cm}
	\section*{Aufgabe 1}
	\label{sec:a1}
	\begin{flushright}
	\large{[~~~~/4]}
	\end{flushright}
	%-----AUFGABE-----%
	Berechnen Sie die Ableitung der Funktion $f(x)=x^{x^2}$.
	\-\\\\\\
	%-----RECHNUNG-----%
	%++++++++++++++++++++++++++++++++++++++++++++++++++++++++++%
	\section*{Aufgabe 2}
	\label{sec:a2}
	\begin{flushright}
	\large{[~~~~/4]}
	\end{flushright}
	%-----AUFGABE-----%
	Berechnen Sie die Ableitungen:
	\begin{itemize}
		\item[(a)] $f(x)=x \cdot \operatorname{sin}5x$
		\item[(b)] $\frac{\operatorname{sin}x+\operatorname{cos}x}{\operatorname{cos}x}$
		\item[(c)] $ \operatorname{sin}(\operatorname{cos}(x-5)) $
		\item[(c)] $ (1-\operatorname{tan}\left(\frac{x}{2}\right))^{-2} $
	\end{itemize}
	\-\\\\\\
	%-----RECHNUNG-----%
	%++++++++++++++++++++++++++++++++++++++++++++++++++++++++++%
	\section*{Aufgabe 3}
	\label{sec:a3}
	\begin{flushright}
	\large{[~~~~/4]}
	\end{flushright}
	%-----AUFGABE-----%
	Finden Sie die Seitenlänge einer quaderförmigen Streichholzschachtel, die bei gegebenem
	Volumen von 45cm$^3$ die minimale Oberfläche hat, um den Materialverbrauch möglichst klein zu halten.
	Dabei soll eine der Seiten die Länge 5cm haben, damit die Streichhölzer hineinpassen.
	\-\\\\\\
	%-----RECHNUNG-----%
	%++++++++++++++++++++++++++++++++++++++++++++++++++++++++++%
	\section*{Aufgabe 4}
	\label{sec:a4}
	\begin{flushright}
	\large{[~~~~/4]}
	\end{flushright}
	%-----AUFGABE-----%
	Welches gleichschenklige Dreieck hat bei gegebenem Umfang \textit{U} die größte Fläche?
	\-\\\\\\
	%-----RECHNUNG-----%
	%++++++++++++++++++++++++++++++++++++++++++++++++++++++++++%
	\section*{Aufgabe 5}
	\label{sec:a5}
	\begin{flushright}
	\large{[~~~~/4]}
	\end{flushright}
	%-----AUFGABE-----%
	Zeigen Sie, dass die Graphen der Funktionen tan und cot keine horizontalen Tangenten haben.
	\-\\\\\\
	%-----RECHNUNG-----%
	Die Steigung einer horizontalen Tangente ist 0.\\
	Die Tangentensteigung wird durch die erste Ableitung derjenigen Funktion berechnet, die tangiert wird.
	\subsection*{Tangens}
	\label{ssec:tan}
	Berechnung der ersten Ableitung:
	\begin{equation*}
		\begin{array}{ll}
			f(x) &= \operatorname{x} \\
			f(x) &= \frac{\operatorname{sin}x}{\operatorname{cos}x} \\
			f'(x) &= \frac{\operatorname{cos}(x) \cdot \operatorname{cos}(x)~-~\operatorname{sin}(x) \cdot (-)\operatorname{sin}(x)}{\operatorname{cos}^2(x)} \\
			f'(x) &= \frac{\operatorname{cos}^2 x + \operatorname{sin}^2 x}{\operatorname{cos}^2 x} (\text{Nach Anwendung der Quotientenregel})			
		\end{array}
	\end{equation*}
	Diese Gleichung wird durch Anwendung des Kosinussatzes auf das rechtwinklige Dreieck des Einheitskreises, über welches die Kosinus-Funktion definiert ist, wobei durch die Eigenschaft der Rechtwinkligkeit der Satz von Pythagoras als Vereinfachung verwendet werden kann, weiter vereinfacht:
	\begin{equation*}
		\begin{array}{rl}
			\text{Satz des Pythagoras: } & a^2+b^2=c^2 \\
			a &:= \operatorname{cos}(x) \\
			b &:= \operatorname{sin}(x) \\
			c &:= 1 \text{, da der Radius des Einheitskreises (die Hypothenuse) 1 beträgt}
		\end{array}
	\end{equation*}
	Somit ergibt sich $a^2+b^2=\operatorname{cos}^2(x)+\operatorname{sin}^2(x)=1^2=1$ folgende Gleichung für die erste Ableitung der Tangensfunktion:
	\[
		f'(x)=\frac{1}{\operatorname{cos}^2 x}
	\]
	Damit die Steigung von $f(x)$ gleich 0 ist und somit eine horizontale Tangente vorliegt, muss der Funktionswert von $f'(x)$ gleich 0 sein. Wie zu erkennen ist, tritt dies nie ein, da $\frac{1}{\operatorname{cos}^2 (x)} \neq 0$.
	\subsection*{Kotangens}
	\label{ssec:cot}
	Verfahren wie beim Tangens.\\
	Bestimmung der Ableitung der \textit{cot}-Funktion:
	\begin{equation*}
		\begin{array}{lll}
			\frac{d}{dx}\operatorname{cot}(x) &= \frac{d}{dx}\frac{1}{\operatorname{tan}(x)}=\frac{d}{dx}\frac{\operatorname{cos}(x)}{\operatorname{sin}(x)}&\\
			&= \frac{((\operatorname{cos}(c))' \cdot \operatorname{sin}(x)~-~\operatorname{cos}(x) \cdot (\operatorname{sin}(x))'}{\operatorname{sin}^2(x)}~~~~ ~~~~&\text{ (Nach Quotientenregel)}\\
			&= \frac{-\operatorname{sin}(x) \cdot \operatorname{sin}(x)~-~\operatorname{cos}(x) \cdot \operatorname{cos}(x)}{\operatorname{sin}^2 (x)} &\\
			&= \frac{-\operatorname{sin}^2(x)~-~\operatorname{cos}^2(x)}{\operatorname{sin}^2(x)} &\\
			&= \frac{-(\operatorname{sin}^2(x)~+~\operatorname{cos}^2(x))}{\operatorname{sin}^2(x)}				&\text{ (Nach Anwenden des Satzes von Pythagoras)} \\
			&= \frac{-1}{\operatorname{sin}^2(x)}
		\end{array}
	\end{equation*}
	Wie eben ist auch hier wieder ersichtlich, dass die Ableitung der \textit{cot}-Funktion niemals gleich 0 werden kann, wodurch auch die \textit{cot}-Funktion keine Steigung von 0 und somit keine horizontale Tangente vorzuweisen hat.
\end{document}