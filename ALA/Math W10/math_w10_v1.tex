\documentclass[tikz,parskip,12pt,paper=a4,sffamily]{article}
%alternate documentclass:
%\documentclass[parskip,12pt,paper=a4,sffamily]{scrartl}
\usepackage[utf8]{inputenc}
\usepackage[ngerman]{babel}
\usepackage{lastpage}
\usepackage{color}   %May be necessary if you want to color links
\usepackage{hyperref}
% code snippets
\usepackage{listings}
% listing captions
\usepackage{caption}
% tikz being tikz
\usepackage{tikz}
\usetikzlibrary{hobby,arrows,automata,calc,angles,quotes}
\usepackage{pgf}
% import math packages
\usepackage{amsmath}
\usepackage{amsfonts}
\usepackage{amssymb}
\usepackage{amsthm}
% contradiction lightning
\usepackage{stmaryrd}
% alignment options
\usepackage{ragged2e}
% page margins
\usepackage[margin=2.5cm]{geometry}

\definecolor{pblue}{rgb}{0.13,0.13,1}
\definecolor{pgreen}{rgb}{0,0.5,0}


\lstset{ %
language=Java,   							% choose the language of the code
basicstyle=\small\ttfamily,  				% the size of the fonts that are used for the code
numbers=left,                   			% where to put the line-numbers
numbersep=5pt,                  			% how far the line-numbers are from the code
backgroundcolor=\color{light-light-gray},   % choose the background color. You must add
frame=lrtb,           						% adds a frame around the code
tabsize=4,          						% sets default tabsize to 2 spaces
captionpos=b,           					% sets the caption-position to bottom
breaklines=true,        					% sets automatic line breaking
xleftmargin=1.5cm,							% space from the left paper edge
commentstyle=\color{pgreen},
keywordstyle=\color{pblue},
literate=%
    {Ö}{{\"O}}1
    {Ä}{{\"A}}1
    {Ü}{{\"U}}1
    {ß}{{\ss}}1
    {ü}{{\"u}}1
    {ä}{{\"a}}1
    {ö}{{\"o}}1
    {~}{{\textasciitilde}}1
}
\renewcommand{\lstlistingname}{Code}
\captionsetup[lstlisting]{font={footnotesize},margin=1.5cm,singlelinecheck=false } % removes "Listing 1: "
\definecolor{light-light-gray}{gray}{0.95}
\let\stdsection\section
\renewcommand\section{\stdsection}

% add line break for subtitle (size: large)
\title{Mathematik für Studierende der Informatik II\\
	Analysis und Lineare Algebra\\
    \-\\\large{
    	Abgabe der Hausaufgaben zum \today
    }
}
\author{~\\
	\Large{Louis Kobras}\\
	\large{6658699}\\ %Matrikelnummer; wenn nicht für Uni, auskommentieren
	\large{4kobras@informatik.uni-hamburg.de}\\
	\\
	\Large{Utz Pöhlmann}\\
	\large{6663579}\\ %Matrikelnummer; wenn nicht für Uni, auskommentieren
	\large{4poehlma@informatik.uni-hamburg.de}\\
	\\
	\Large{Jennifer Hartmann}\\
	\large{6706472}\\ %Matrikelnummer; wenn nicht für Uni, auskommentieren
	\large{fwuy089@studium.uni-hamburg.de}
}

% leave empty for no date on title page
% comment for auto-generated date
\date{\today}


\begin{document}
	\maketitle\thispagestyle{empty}
	\newpage
	\pagenumbering{arabic}
	\newgeometry{top=2.5cm, bottom =2.5cm, left=2cm, right=3.5cm}
	\section*{Aufgabe 1}
	\label{sec:a1}
	\begin{flushright}
	\large{[~~~~/4]}
	\end{flushright}
	%-----AUFGABE-----%
	Berechnen Sie die Ableitung der Funktion $f(x)=x^{x^2}$.
	\-\\\\\\
	%-----RECHNUNG-----%
	\begin{equation*}
		\begin{array}{lll}
			\frac{d}{dx} x^{x^2}	&= \frac{d}{dx} e^{ln(x^{x^2})}													& \text{Ausdrücken als Exponent von }e						\\
									&= \frac{d}{dx} e^{x^2 ln(x)}													& \text{Anwenden der Potenzgesetze}							\\
									&= e^{x^2 ln(x)} \cdot \frac{d}{dx} x^2 ln(x)									& \text{Ableiten mithilfe der Kettenregel}					\\
									&= e^{x^2 ln(x)} \cdot \left( 2x \cdot ln(x)~+~x^2 \cdot \frac{1}{x} \right)	& \text{Ableiten des zweiten Faktors}						\\
									&= e^{x^2 ln(x)} \cdot (2x \cdot ln(x) + x)										& \text{Kürzen in der Klammer}								\\
									&= e^{x^2 ln(x)} \cdot (x + 2x \cdot ln(x))										& \text{Drehen der Summanden in der Klammer}				\\
									&= e^{ln(x^{x^2})} \cdot (x + 2x \cdot ln(x)									& \text{Anwenden der Potenzgesetze auf den ersten Faktor}	\\
									&= x^{x^2} (x+2x\cdot ln(x))													& \text{Auflösen des Exponenten von \textit{e}, fertige Ableitung}
		\end{array}
	\end{equation*}
	%++++++++++++++++++++++++++++++++++++++++++++++++++++++++++%
	\section*{Aufgabe 2}
	\label{sec:a2}
	\begin{flushright}
	\large{[~~~~/4]}
	\end{flushright}
	%-----AUFGABE-----%
	Berechnen Sie die Ableitungen:
	\begin{itemize}
		\item[(a)] $ f(x)=x \cdot \operatorname{sin}5x$
		\item[(b)] $ f(x)=\frac{\operatorname{sin}x+\operatorname{cos}x}{\operatorname{cos}x}$
		\item[(c)] $ g(x)=\operatorname{sin}(\operatorname{cos}(x-5)) $
		\item[(c)] $ h(x)=(1-\operatorname{tan}\left(\frac{x}{2}\right))^{-2} $
	\end{itemize}
	%-----RECHNUNG-----%
	\subsection*{(a)}
	\label{ssec:a2a}
	\begin{equation*}
		\begin{array}{lll}
			f(x)	&= x \cdot \operatorname{sin}(5x)										& {}					\\
			f'(x)	&= \frac{d}{dx} x \cdot \operatorname{sin}(5x)							& \text{Produktregel}	\\
					&= 1 \cdot \operatorname{sin}(5x)~+~x \cdot (\operatorname{sin}(5x))'	& \text{Kettenregel}	\\
					&= \operatorname{sin}(5x)~+~ 5 \cdot x \cdot \operatorname{cos}(5x)		& {}
		\end{array}
	\end{equation*}
	\subsection*{(b)}
	\label{ssec:a2b}
	\begin{equation*}
		\begin{array}{lll}
			f(x)	&= \frac{\operatorname{sin}(x) + \operatorname{cos}(x)}{\operatorname{cos}(x)}	& \begin{array}{l}	{sin}(x)+\operatorname{cos}(x):=u\\	\operatorname{cos}(x):=v	\end{array}		\\
			f'(x)	&= \frac{d}{dx}\frac{u}{v} = \frac{u'v~-~uv'}{v^2}								& \text{Quotientenregel}																			\\
			f'(x)	&= \frac{(\operatorname{sin}(x)+\operatorname{cos}(x))' \cdot \operatorname{cos}(x)		~-~		(\operatorname{sin}(x)+\operatorname{cos}(x)) \cdot \operatorname{cos}(x)'}{\operatorname{cos}^2x}	\\
			f'(x)	&= \frac{\operatorname{cos}^2(x)-\operatorname{sin}(x) \cdot \operatorname{cos}(x)		~-~		(-\operatorname{sin}(x))(\operatorname{sin}(x)+\operatorname{cos}(x))}{\operatorname{cos}^2x}		\\
			f'(x)	&= \frac{\operatorname{cos}^2(x)-\operatorname{sin}(x) \cdot \operatorname{cos}(x)		~+~		\operatorname{sin}^2x+\operatorname{sin}(x)\operatorname{cos}x}{\operatorname{cos}^2(x)}	&	\text{Verrechnen betragsgleicher} \\
					&																																															&	\text{vorzeichenunterschiedlicher Komponenten}\\
					&																																															&	\text{im Zähler}\\
			f'(x)	&= \frac{\operatorname{sin}^2x~+~\operatorname{cos}^2 x}{\operatorname{cos}^2 x}																											&	\text{Zusammenfassen (Begründung s. 5.2)} \\
			f'(x)	&= \frac{1}{\operatorname{cos}^2 x}
			
		\end{array}
	\end{equation*}
	\subsection*{(c)}
	\label{ssec:a2c}
	\begin{equation*}
		\begin{array}{lll}	
			g(x)	&=	\operatorname{sin}(\operatorname{cos(x-5)})		&	\operatorname{cos}(x-5):=u	\\
			g(x)	&=	\operatorname{sin}(u)							&	\text{Kettenregel}			\\
			g'(x)	&=	\operatorname{sin}(u)'															\\
			g'(x)	&=	\operatorname{cos}(u) \cdot u'					&	u:=\operatorname{cos}(x-5)	\\
			g'(x)	&=	\operatorname{cos}(\operatorname{cos}(x-5))	\cdot (-\operatorname{sin}(x-5)) \cdot (x-5)'	\\
			g'(x)	&=	- \operatorname{cos}(\operatorname{cos}(x-5))	\cdot \operatorname{sin}(x-5) \cdot 1 \\
			g'(x)	&=	- \operatorname{cos}(\operatorname{cos}(x-5))	\cdot \operatorname{sin}(x-5)
		\end{array}
	\end{equation*}
	\subsection*{(d)}
	\label{ssec:a2d}
	\begin{equation*}
		\begin{array}{lll}
			h(x)	&=	\left( 1-\operatorname{tan} \left( \frac{x}{2} \right) \right)^{-2}	&	\text{Umschreiben als Bruch}	\\
			h(x)	&=	\frac{1}{\left( 1-\operatorname{tan} \left( \frac{x}{2} \right) \right)^2}	&	\text{Quotientenregel: } \begin{array}{ll}
																																		u &:= 1~~~~\Rightarrow u'=0\\
																																		v &:= \left( 1-\operatorname{tan} \left( \frac{x}{2} \right) \right)^2
																																	\end{array} \\
			h'(x)	&=	\frac{d}{dx}\frac{u}{v}\\
			h'(x)	&=	\frac{u'v~-~uv'}{v^2}\\
			h'(x)	&=	\frac{0\cdot\left( 1-\operatorname{tan} \left( \frac{x}{2} \right) \right)^2~-~1 \cdot \left[\left( 1-\operatorname{tan} \left( \frac{x}{2} \right) \right)^2\right]'}{\left(\left( 1-\operatorname{tan} \left( \frac{x}{2} \right) \right)^2\right)^2} \\
			h'(x)	&=	\frac{-\left[\left( 1-\operatorname{tan} \left( \frac{x}{2} \right) \right)^2\right]'}{\left( 1-\operatorname{tan} \left( \frac{x}{2} \right) \right)^4}	\\
			h'(x)	&=	-\frac{2 \cdot \left( 1-\operatorname{tan} \left( \frac{x}{2} \right) \right) \cdot \frac{1}{-\operatorname{cos}^2 \left(\frac{x}{2}\right)} \cdot \frac{1}{2}}{\left( 1-\operatorname{tan} \left( \frac{x}{2} \right) \right)^4} \\
			h'(x)	&=	-\frac{\left( 1-\operatorname{tan} \left( \frac{x}{2} \right) \right) \cdot \frac{1}{-\operatorname{cos}\left(\frac{x}{2}\right)}}{\left( 1-\operatorname{tan} \left( \frac{x}{2} \right) \right)^4}	&	\text{Kürzen}\\
			h'(x)	&=	-\frac{\frac{1}{-\operatorname{cos}\left(\frac{x}{2}\right)}}{\left( 1-\operatorname{tan} \left( \frac{x}{2} \right) \right)^3}		&	\text{Auflösen des Doppelbruches}	\\
			h'(x)	&=	-\frac{1}{-\operatorname{cos}\left(\frac{x}{2}\right) \cdot \left( 1-\operatorname{tan} \left( \frac{x}{2} \right) \right)^3}		&	\text{Aufheben des doppelten Minus} \\
			h'(x)	&=	\frac{1}{\operatorname{cos}\left(\frac{x}{2}\right) \cdot \left( 1-\operatorname{tan} \left( \frac{x}{2} \right) \right)^3}
		\end{array}
	\end{equation*}
	%++++++++++++++++++++++++++++++++++++++++++++++++++++++++++%
	\section*{Aufgabe 3}
	\label{sec:a3}
	\begin{flushright}
	\large{[~~~~/4]}
	\end{flushright}
	%-----AUFGABE-----%
	Finden Sie die Seitenlänge einer quaderförmigen Streichholzschachtel, die bei gegebenem
	Volumen von 45cm$^3$ die minimale Oberfläche hat, um den Materialverbrauch möglichst klein zu halten.
	Dabei soll eine der Seiten die Länge 5cm haben, damit die Streichhölzer hineinpassen.
	\-\\\\\\
	%-----RECHNUNG-----%
	Ein Quader besitzt folgende Gleichungen:
	\[O=2ab+2ac+2bc\]
	und
	\[V=abc,\]
	wobei \textit{O} die Oberfläche ist, \textit{V} das Volumen und \textit{a, b, c} die Kanten.\\
	Mit den gegebenen Werten können wir sagen:
	\[V=45cm \land ( a=5cm \Leftrightarrow bc=9cm^2)\]
	Folglich kann man \textit{b} ausdrücken als:
	\[b=\frac{9cm^2}{c}\]
	Somit ergibt sich für \textit{O} eine rein von \textit{c} abhängige Gleichung:
	\begin{align*}
		O	&=2 \cdot 5cm \cdot \frac{9cm^2}{c}	~+~	2 \cdot 5cm \cdot c	~+~	2 \cdot \frac{9cm}{c} \cdot c \\
			&=\frac{90}{c}cm^2	~+~	c \cdot 10cm^2	~+~	18cm^2
	\end{align*}
	Diese Gleichung können wir als Funktion $O(c)$ behandeln.
	\begin{equation*}
		\begin{array}{lll}
			O(c)	&=	\frac{90}{c}+10c+18\\
			O'(c)	&=	\frac{d}{dx} 90 \cdot c^{-1}+10 \cdot c+18\\
			O'(c)	&=	-1 \cdot 90 \cdot c^{-2} + 10 \\
			O'(c)	&=	-\frac{90}{c^2}+10	&	\text{1. Ableitung für notwendige Bedingung}	\\
			O''(c)	&=	\frac{d}{dx} -90 \cdot c^{-2}+10 \\
			O''(c)	&=	-2 \cdot -90 c^{-3} \\
			O''(c)	&=	\frac{180}{c^3}		&	\text{2. Ableitung für hinreichende Bedingung}
		\end{array}
	\end{equation*}
	Da das Minimum der Oberfläche gesucht wird, wird die 1. Ableitung gleich 0 gesetzt.
	Das so erhaltene \textit{c} wird anschließend in die 2. Ableitung eingesetzt, um zu bestimmen, ob an der Stelle ein Minimum oder ein Maximum vorliegt.
	Ist ein Minimum bestimmt, so kann der für \textit{c} bestimmte Wert genutzt werden, um \textit{b} und die Oberfläche der Schachtel zu bestimmen.\\
	\\
	\underline{Notwendige Bedingung}:
	\begin{equation*}
		\begin{array}{lll}
			0		&=	-\frac{90}{c^2}+10	&	| -10 \\
			-10		&=	-\frac{90}{c^2}		&	| \cdot c^2 \\
			-10c^2	&=	-90					&	| :(-10)\\
			c^2		&=	-\frac{90}{-10}		&	\\
			c^2		&=	9					&	|\sqrt{()}\\
			c		&=	\pm 3
		\end{array}
	\end{equation*}
	Es ist anzumerken, dass $-3$ als Lösung ausgeschlossen werden kann, da sonst eine negative Kantenlänge vorläge, welche im mindestens bis zu vierdimensionalen Raum nicht vorkommen kann.\\
	\\
	\underline{Hinreichende Bedingung}:
	\begin{equation*}
		\begin{array}{llllllll}
			O'(3)	&=	&0	&\land	&O''(3)	&>	&0	&\Rightarrow	\underline{Minimum}\\
		\end{array}
	\end{equation*}
	\underline{\textit{c} in \textit{V}}:
	\[
		a = 5cm \land c = 3cm \land V = 45cm^3 \Leftrightarrow \underline{b = 3cm}
	\]
	\underline{	\textit{b, c} in \textit{O}}:
	\[O_{min}~=~2 \cdot [5 \cdot 3 ] cm^2~+~2 \cdot [5 \cdot 3]cm^2 ~+~ 2 \cdot [3 \cdot 3]cm^2 ~=~2 \cdot 15cm^2 ~+~2 \cdot 15 cm^2 ~+~2 \cdot 9cm^2 ~=~\underline{\underline{78 cm^2}}\]
	%++++++++++++++++++++++++++++++++++++++++++++++++++++++++++%
	\section*{Aufgabe 4}
	\label{sec:a4}
	\begin{flushright}
	\large{[~~~~/4]}
	\end{flushright}
	%-----AUFGABE-----%
	Welches gleichschenklige Dreieck hat bei gegebenem Umfang \textit{U} die größte Fläche?
	\-\\\\\\
	%-----RECHNUNG-----%
	\begingroup
	\centering
	\begin{tikzpicture}
		\coordinate [label=left:$A$]	(A) at (0,0);
		\coordinate [label=right:$B$]	(B) at (4,0);
		\coordinate [label=above:$C$]	(C)	at (2,5);
		\coordinate [label=below:$ $]	(D) at (2,0);
		
		\node [label=below:$b$] (X) at ($ (A)!.5!(B) $) {};
		\node [label=right:$a$](Y) at ($ (B)!.5!(C) $) {};
		\node [label=left:$a$](Z) at ($ (C)!.5!(A) $) {};
		\node [label=right:$h_b$](W) at ($ (D)!.3!(C) $) {};
		
		\draw (A) -- (B) -- (C) -- (A);
		\draw (D) -- (C);    
	\end{tikzpicture}\\
	\endgroup
	$h_b$ teilt $b$ in der Mitte, also auf der Länge $\frac{b}{2}$. $h_b$ steht rechtwinklig auf $b$.
	\begin{equation*}
		\begin{array}{llllllllll}
			&U &= 2a+b &\Leftrightarrow b=U-2a\\
			&A &= \frac{b \cdot h}{2} \\
			&h^2 + \left(\frac{b}{2}\right)^2&=a^2\\
			\Leftrightarrow& h^2&=a^2-\left(\frac{b}{2}\right)^2\\
			\Leftrightarrow& h&=\sqrt{a^2- \frac{b^2}{4}}&\Rightarrow h \geq 0\\
			\text{setze}&\text{\textit{h} in \textit{A} ein:}\\
			&A(a,b)&=\frac{b \cdot \sqrt{a^2-\frac{b^2}{4}}}{2}	\\
			\text{setze}&\text{\textit{b} in \textit{A} ein:}\\
			&A(a)&=\frac{(U-2a)\cdot\sqrt{a^2-\frac{(U-2a)^2}{4}}}{2}\\
			\Leftrightarrow&A(a)&=\frac{1}{2} \cdot (U-2a)\cdot\sqrt{a^2-\frac{(U-2a)^2}{4}}\\
			\Leftrightarrow&A(a)&=\frac{U-2a}{2} \cdot \sqrt{a^2-\frac{U^2-4aU+4a^2}{4}}\\
			\Leftrightarrow&A(a)&=\frac{U-2a}{2} \cdot \sqrt{\frac{4a^2-4U^2+4aU-4a^2}{4}}\\
			\Leftrightarrow&A(a)&=\frac{U-2a}{2} \cdot \sqrt{\frac{-U^2+4aU}{4}} \\
			\Leftrightarrow&A(a)&=\frac{U-2a}{2} \cdot \sqrt{\frac{(-1)\cdot (U^2-4aU)}{4}}\\
			\Leftrightarrow&A(a)&=\frac{U-2a}{2} \cdot \sqrt{-\frac{1}{4} \cdot (U^2-4aU)}\\
			\Leftrightarrow&A(a)&=\frac{1}{2} \cdot (U-2a) \cdot \sqrt{-\frac{U^2}{4}+aU} \\
		\end{array}
	\end{equation*}
	\begin{equation*}
		\begin{array}{ll}
			\text{Wird }-\frac{U^2}{4}\text{ negativ, so}\\
			\text{tritt eine irrationale Fläche auf.}\\
			aU=\frac{U^2}{4}\\
			a=\frac{U}{4}\\
			\Rightarrow -\frac{U^2}{4}+aU\geq 0
		\end{array}
		{}~~~~~{}
		\begin{array}{ll}
			\text{Wird }U-2a<0\text{, so wird die}\\
			\text{Fläche negativ}\\
			U=2a\\
			\frac{U}{2}=a\\
			\Rightarrow U-2a\geq 0
		\end{array}
	\end{equation*}
	Um eine reelle, positive Fläche zu erhalten, muss \textit{a} zwischen o.g. Werten liegen. Daraus ergibt sich der Definitionsbereich:
	\begin{equation*}
		\begin{array}{llllllll}
			\mathbb{D}=\{a\in\mathbb{R}&|\frac{U}{4}\leq a \leq \frac{U}{2}\}\\
			&	&\Rightarrow \text{Unser gesuchtes Maximum ist nicht nur lokal, sondern global}\\
			&A'(a)&=\frac{1}{2} \cdot (-2) \cdot \sqrt{-\frac{U^2}{4}+aU}~+~\frac{1}{2} \cdot (U-2a) \cdot \frac{1}{2 \cdot \sqrt{-\frac{U^2}{4}+aU}} \cdot U\\
			&\Leftrightarrow A'(a)&=\frac{U-2a}{2} \cdot \frac{U}{2 \cdot \sqrt{-\frac{U^2}{4}+aU}} - \sqrt{-\frac{U^2}{4}+aU}\\
			&\text{Gleich 0 setzen:}\\
			&&0&=\frac{U-2a}{2} \cdot \frac{U}{2 \cdot \sqrt{-\frac{U^2}{4}+aU}} - \sqrt{-\frac{U^2}{4}+aU}\\
			&\Leftrightarrow 0&=\frac{U^2-2aU}{4\sqrt{-\frac{U^2}{4}+aU}}-\frac{\left(\sqrt{-\frac{U^2}{4}+aU}\right)^2 \cdot 4}{4 \cdot \sqrt{-\frac{U^2}{4}+aU}}\\
			&\Leftrightarrow 0		&=U^2-2aU+\left(\frac{U^2}{4}-aU\right) \cdot 4\\
			&\Leftrightarrow 0		&=U^2-2aU+U^2-4aU\\
			&\Leftrightarrow	0	&=2U^2-6aU\\
			&\Leftrightarrow	0	&=2U(U-3a)\\
			&\Leftrightarrow	0	&=U-3a&|+3a\\
			&\Leftrightarrow 3a&=U
		\end{array}
	\end{equation*}
	$U=3a$ bedeutet im Klartext eingesetzt in $U=2a+b$, die wir schon kennen:
	\[2a+b=3a \Leftrightarrow b=a\]
	$\Rightarrow$ die maximale Flächehat ein gleichschenkliges Dreieck bei gegebenem Umfang, wenn es gleichseitig ist.
	%++++++++++++++++++++++++++++++++++++++++++++++++++++++++++%
	\section*{Aufgabe 5}
	\label{sec:a5}
	\begin{flushright}
	\large{[~~~~/4]}
	\end{flushright}
	%-----AUFGABE-----%
	Zeigen Sie, dass die Graphen der Funktionen tan und cot keine horizontalen Tangenten haben.
%	\-\\\\\\
	%-----RECHNUNG-----%
	Die Steigung einer horizontalen Tangente ist 0.\\
	Die Tangentensteigung wird durch die erste Ableitung derjenigen Funktion berechnet, die tangiert wird.
	\subsection*{Tangens}
	\label{ssec:tan}
	Berechnung der ersten Ableitung:
	\begin{equation*}
		\begin{array}{ll}
			f(x) &= \operatorname{x} \\
			f(x) &= \frac{\operatorname{sin}x}{\operatorname{cos}x} \\
			f'(x) &= \frac{\operatorname{cos}(x) \cdot \operatorname{cos}(x)~-~\operatorname{sin}(x) \cdot (-)\operatorname{sin}(x)}{\operatorname{cos}^2(x)} \\
			f'(x) &= \frac{\operatorname{cos}^2 x + \operatorname{sin}^2 x}{\operatorname{cos}^2 x} (\text{Nach Anwendung der Quotientenregel})			
		\end{array}
	\end{equation*}
	Diese Gleichung wird durch Anwendung des Kosinussatzes auf das rechtwinklige Dreieck des Einheitskreises, über welches die Kosinus-Funktion definiert ist, wobei durch die Eigenschaft der Rechtwinkligkeit der Satz von Pythagoras als Vereinfachung verwendet werden kann, weiter vereinfacht:
	\begin{equation*}
		\begin{array}{rl}
			\text{Satz des Pythagoras: } & a^2+b^2=c^2 \\
			a &:= \operatorname{cos}(x) \\
			b &:= \operatorname{sin}(x) \\
			c &:= 1 \text{, da der Radius des Einheitskreises (die Hypothenuse) 1 beträgt}
		\end{array}
	\end{equation*}
	Somit ergibt sich $a^2+b^2=\operatorname{cos}^2(x)+\operatorname{sin}^2(x)=1^2=1$ folgende Gleichung für die erste Ableitung der Tangensfunktion:
	\[
		f'(x)=\frac{1}{\operatorname{cos}^2 x}
	\]
	Damit die Steigung von $f(x)$ gleich 0 ist und somit eine horizontale Tangente vorliegt, muss der Funktionswert von $f'(x)$ gleich 0 sein. Wie zu erkennen ist, tritt dies nie ein, da $\frac{1}{\operatorname{cos}^2 (x)} \neq 0$:
	\[
		0=\frac{1}{\operatorname{cos}^2x}~~~~|\cdot\operatorname{cos}^2x~~~~\Rightarrow~~~~0 \neq 1 \lightning
	\]
	\subsection*{Kotangens}
	\label{ssec:cot}
	Verfahren wie beim Tangens.\\
	Bestimmung der Ableitung der \textit{cot}-Funktion:
	\begin{equation*}
		\begin{array}{lll}
			\frac{d}{dx}\operatorname{cot}(x) &= \frac{d}{dx}\frac{1}{\operatorname{tan}(x)}=\frac{d}{dx}\frac{\operatorname{cos}(x)}{\operatorname{sin}(x)}&\\
			&= \frac{((\operatorname{cos}(c))' \cdot \operatorname{sin}(x)~-~\operatorname{cos}(x) \cdot (\operatorname{sin}(x))'}{\operatorname{sin}^2(x)}~~~~ ~~~~&\text{ (Nach Quotientenregel)}\\
			&= \frac{-\operatorname{sin}(x) \cdot \operatorname{sin}(x)~-~\operatorname{cos}(x) \cdot \operatorname{cos}(x)}{\operatorname{sin}^2 (x)} &\\
			&= \frac{-\operatorname{sin}^2(x)~-~\operatorname{cos}^2(x)}{\operatorname{sin}^2(x)} &\\
			&= \frac{-(\operatorname{sin}^2(x)~+~\operatorname{cos}^2(x))}{\operatorname{sin}^2(x)}				&\text{ (Nach Anwenden des Satzes von Pythagoras)} \\
			&= \frac{-1}{\operatorname{sin}^2(x)}
		\end{array}
	\end{equation*}
	Wie eben ist auch hier wieder ersichtlich, dass die Ableitung der \textit{cot}-Funktion niemals gleich 0 werden kann, da
	\[
		0=\frac{-1}{\operatorname{sin}^2x}~~~~|\cdot\operatorname{sin}^2x~~~~\Rightarrow~~~~0 \neq -1 \lightning
	\]
	wodurch auch die \textit{cot}-Funktion keine Steigung von 0 und somit keine horizontale Tangente vorzuweisen hat.
\end{document}