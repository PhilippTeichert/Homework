\documentclass[parskip,12pt,paper=a4,sffamily]{article}
%alternate documentclass:
%\documentclass[parskip,12pt,paper=a4,sffamily]{scrartl}
\usepackage[utf8]{inputenc}
\usepackage[ngerman]{babel}
\usepackage{lastpage}
\usepackage{color}   %May be necessary if you want to color links
\usepackage{hyperref}
% code snippets
\usepackage{listings}
% listing captions
\usepackage{caption}
% font: times new roman
\usepackage{times}
% tikz being tikz
\usepackage{tikz}
\usetikzlibrary{arrows,automata}
\usepackage{pgf}
% import math packages
\usepackage{amsmath}
\usepackage{amsfonts}
\usepackage{amssymb}
\usepackage{amsthm}
% contradiction lightning
\usepackage{stmaryrd}
% alignment options
\usepackage{ragged2e}
% page margins
\usepackage[margin=2.5cm]{geometry}

\definecolor{pblue}{rgb}{0.13,0.13,1}
\definecolor{pgreen}{rgb}{0,0.5,0}


\lstset{ %
language=Java,   							% choose the language of the code
basicstyle=\small\ttfamily,  				% the size of the fonts that are used for the code
numbers=left,                   			% where to put the line-numbers
numbersep=5pt,                  			% how far the line-numbers are from the code
backgroundcolor=\color{light-light-gray},   % choose the background color. You must add
frame=lrtb,           						% adds a frame around the code
tabsize=4,          						% sets default tabsize to 2 spaces
captionpos=b,           					% sets the caption-position to bottom
breaklines=true,        					% sets automatic line breaking
xleftmargin=1.5cm,							% space from the left paper edge
commentstyle=\color{pgreen},
keywordstyle=\color{pblue},
literate=%
    {Ö}{{\"O}}1
    {Ä}{{\"A}}1
    {Ü}{{\"U}}1
    {ß}{{\ss}}1
    {ü}{{\"u}}1
    {ä}{{\"a}}1
    {ö}{{\"o}}1
    {~}{{\textasciitilde}}1
}
\renewcommand{\lstlistingname}{Code}
\captionsetup[lstlisting]{font={footnotesize},margin=1.5cm,singlelinecheck=false } % removes "Listing 1: "
\definecolor{light-light-gray}{gray}{0.95}
\let\stdsection\section
\renewcommand\section{\stdsection}

% add line break for subtitle (size: large)
\title{Hausaufgaben zum \today
    \\\large{
        Mathematik für Studierende der Informatik II\\
		(Analysis und Lineare Algebra)
    }
}
\author{~\\
	\Large{Louis Kobras}\\
	\large{6658699}\\ %Matrikelnummer; wenn nicht für Uni, auskommentieren
	\large{4kobras@informatik.uni-hamburg.de}\\
	\\
	\Large{Utz Pöhlmann}\\
	\large{6663579}\\ %Matrikelnummer; wenn nicht für Uni, auskommentieren
	\large{4poehlma@informatik.uni-hamburg.de}\\
	\\
	\Large{Jennifer Hartmann}\\
	\large{6706472}\\ %Matrikelnummer; wenn nicht für Uni, auskommentieren
	\large{fwuy089@studium.uni-hamburg.de}
}

% leave empty for no date on title page
% comment for auto-generated date
\date{\today}


\begin{document}
	\maketitle
	\newpage
	\section*{Aufgabe 1}
	\label{sec:a1}
	%-----AUFGABE-----%
	Berechnen Sie
	\[
		\underset{x\rightarrow 1}{\operatorname{lim}} \frac{x^2-x}{x^2-3x+2}.
	\]
	\\
	\\
	\\
	%-----RECHNUNG-----%
	%++++++++++++++++++++++++++++++++++++++++++++++++++++++++++%
	\section*{Aufgabe 2}
	\label{sec:a2}
	%-----AUFGABE-----%
	Bestimmen Sie alle $a\in\mathbb{R}$, so dass die im Folgenden definierte Funktion $f:\mathbb{R}\rightarrow\mathbb{R}$ stetig ist.
	\[
		f(x):=
		\begin{cases}
			8a+16x\text{, falls }x<2\\
			a^2(x+2)\text{, falls }x \ge 2
		\end{cases}
	\]
	%-----RECHNUNG-----%
	\\
	\\
	\\
	%++++++++++++++++++++++++++++++++++++++++++++++++++++++++++%
	\section*{Aufgabe 3}
	\label{sec:a3}
	%-----AUFGABE-----%
	Benutzen Sie die $\epsilon$-$\delta$-Definition der Stetigkeit, um zu zeigen, dass es keine reelle Zahl \textit{a} gibt, so dass die Funktion $f:\mathbb{R}\rightarrow\mathbb{R}$ mit der Definition
	\[
		f(x)=
		\begin{cases}
			\frac{1}{x}\text{, falls }\neq 0\text{ und}\\
			a\text{, sonst}
		\end{cases}
	\]
	an der Stelle 0 stetig ist.
	\\
	\\
	\\
	%-----RECHNUNG-----%
	%++++++++++++++++++++++++++++++++++++++++++++++++++++++++++%
	\section*{Aufgabe 4}
	\label{sec:a4}
	%-----AUFGABE-----%
	Weisen Sie mit Hilfe der $\epsilon$-$\delta$-Definition der Stetigkeit nach, dass das Produkt zweier stetiger Funktionen $f,g:\mathbb{R}\rightarrow\mathbb{R}$ stetig ist.
	\\
	\\
	\\
	%-----RECHNUNG-----%
	%++++++++++++++++++++++++++++++++++++++++++++++++++++++++++%
	\section*{Aufgabe 5}
	\label{sec:a5}
	%-----AUFGABE-----%
	Gegeben sei die Funktion $f:\mathbb{R}\rightarrow\mathbb{R}; x\rightarrow x^3$.
	Bestimmen Sie die Ableitung von \textit{f} an der Stelle $x_0\in\mathbb{R}$ als Grenzwert des Differenzquotienten.\\
	Hinweis: Im letzten Semester gab es Fragen, ob Polynomdivision etwas mit Ableitungen zu tun hat.
	In dieser Aufgabe kann man Polynomdivision benutzen, um die Ableitung zu berechnen.
	%-----RECHNUNG-----%
\end{document}