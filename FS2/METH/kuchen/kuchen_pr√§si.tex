\documentclass[pdf]{beamer}
\usepackage{amsmath}
\usepackage{amssymb}
\usepackage[utf8]{inputenc}
\usepackage[ngerman]{babel}
\usepackage{lastpage}
\usepackage{listings}
\usepackage{caption}
\usepackage{cancel}
\usepackage{color}
\usepackage{eurosym}
\usepackage{courier} \raggedright
\usepackage{nameref}
\usepackage{hyperref}
\usepackage{url}
\usepackage{tabularx}
\usepackage{titleref}
\usepackage{soul}
\usepackage{xcolor}
\usepackage{colortbl}

%Für [Seite]/[Anzahl-Seiten] in der unteren rechten ecke
%\newcommand*\oldmacro{}%
%\let\oldmacro\insertshorttitle%
%\renewcommand*\insertshorttitle
%{
% \oldmacro\hfill%
% \thepage\,/\,\pageref{LastPage}
%}
%\makeatletter
%\newcommand*{\currentname}{\TR@currentTitle}
%\makeatother

\definecolor{light-light-gray}{gray}{0.85}

%\usetheme[compress]{Berlin}
\usetheme[compress]{Dresden}
%\usetheme{Hamburg}
\setbeamerfont{headline}{size=\large}
\setbeamerfont*{section in head/foot}{size=\tiny}
\setbeamercovered{transparent}
\setbeamertemplate{bibliography item}[text]
\setbeamertemplate{itemize items}[square]
\setbeamertemplate{itemize subitem}[triangle]
\setbeamertemplate{section in toc}[square]
\setbeamertemplate{subsection in toc}[square]
\setbeamertemplate{caption}[numbered]
\setbeamercolor*{title}{use=structure,fg=white,bg=structure.fg}
\setbeamertemplate{title page}[default][colsep=-2bp]
\setbeamercolor{block title}{bg=darkred2!100,fg=white}
\setbeamercolor{block body}{bg=darkred3!25,fg=black}

\definecolor{darkred}{rgb}{0,0.55,0.8}
\definecolor{darkred2}{rgb}{0,0.41,0.6}
\definecolor{darkred3}{rgb}{0.2,0.4,0.6}
\usecolortheme[named=darkred]{structure}

\title{Kuchen}
\subtitle{}
\author{Utz Pöhlmann, Mirco Frankenberg, Louis Kobras}
\date{\footnotesize\today}
%\titlegraphic{\href{http://uni-hamburg.de/}{\includegraphics[height=.15\textheight]{./images/logo}}}


\begin{document}
% Titel und Präambel
\begin{frame}
	\titlepage\thispagestyle{empty}
\end{frame}
\begin{frame}
	\centering
	\Large{Willkommen im Konditorzug}\\
	\-\\
	\large{Ihre Stewardessen:}\\
	\large{Utz Pöhlmann, Mirco Frankenberg, Louis Kobras}
\end{frame}
\begin{frame}{Der Fahrplan}
	\tableofcontents[hidesubsections]
\end{frame}
% Content
	\section{Erste Station: Kuchenau}
	\subsection*{}
		\begin{frame}
			\centering
			\Large{Was ist ein Kuchen?}
		\end{frame}
	\section{Stadtplan: Cake Districts}
	\subsection{Chrissi's Cheesecake}
		\begin{frame}
		\begin{itemize}
			\item (Mager-)Quark/Philadelphia/Frischkäse
			\item Eier
			\item Milch
			\item Zucker
			\item Zitronensaft
			\item Vanillepudding und Vanillezucker
		\end{itemize}
		\end{frame}
	\subsection{New York Fudge}
		\begin{frame}
		\begin{itemize}
			\item Walnüsse
			\item Butter
			\item Vanille
			\item Eier
			\item Mehl
		\end{itemize}
		\end{frame}
	\subsection{Bienenstich}
		\begin{frame}
		\begin{itemize}
			\item Hefeboden
			\item Mandelblättchen
			\item Honig
			\item Zucker
			\item Schlagsahne
		\end{itemize}
		\end{frame}
	\subsection{Früchtekuchen}
		\begin{frame}
		\begin{itemize}
			\item Äpfel
			\item Zitrone
			\item Vanille
			\item Mandeln
		\end{itemize}
		\end{frame}
	\subsection{Baumkuchen}
		\begin{frame}
		\begin{itemize}
			\item Mandel
			\item Rum
			\item Schokolade
		\end{itemize}
		\end{frame}
	\subsection{Kalter Hund}
		\begin{frame}
		\begin{itemize}
			\item Kokosfett
			\item Kakaopulver
			\item Rum (Schuss)
			\item Butterkekse
		\end{itemize}
		\end{frame}
	\subsection{Donau-Welle}
		\begin{frame}
		\begin{itemize}
			\item Sauerkirschen
			\item Vanille
			\item Schokolade
			\item Nutella
			\item Kuchenteig
		\end{itemize}
		\end{frame}
	\section{Dritte Station: Tortenhausen}
		\subsection{Sehenswürdigkeit: Torten}
			\begin{frame}
				\frametitle{Torten}
				\begin{itemize}
					\item Mürbeteig
					\item Sahnecreme
					\item Teig und Creme abwechselnd schichten
					\item Lagern im Kühlschrank
				\end{itemize}
			\end{frame}
		\subsection{Tourismusbezirk: Schwarzwälder Kirsch-Torte}
		\begin{frame}
			\begin{itemize}
				\item Mürbeteigboden mit Kakao
				\item Sauerkirschen
				\item Sahne
				\item Roter Tortenguss
				\item Schokoraspeln
			\end{itemize}
		\end{frame}
	\section{Endstation: Lieblingsbezirk}
	\subsection*{}
		\begin{frame}
			\centering
			\Large{Sammlung: Lieblingsbezirk}
		\end{frame}
		\begin{frame}
			\centering
			\Large{Sammlung}
		\end{frame}
		\begin{frame}
			\centering
			\Large{Einigung}
		\end{frame}
\end{document}