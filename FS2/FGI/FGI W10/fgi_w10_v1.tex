\documentclass[parskip,12pt,paper=a4,sffamily]{article}
%alternate documentclass:
%\documentclass[parskip,12pt,paper=a4,sffamily]{scrartl}
\usepackage[utf8]{inputenc}
\usepackage[ngerman]{babel}
\usepackage{lastpage}
\usepackage{color}   %May be necessary if you want to color links
\usepackage{hyperref}
% code snippets
\usepackage{listings}
% listing captions
\usepackage{caption}
% tikz being tikz
\usepackage{tikz}
\usetikzlibrary{arrows,automata}
\usepackage{pgf}
% import math packages
\usepackage{amsmath}
\usepackage{amsfonts}
\usepackage{amssymb}
\usepackage{amsthm}
% contradiction lightning
\usepackage{stmaryrd}
% alignment options
\usepackage{ragged2e}
% page margins
\usepackage[margin=2.5cm]{geometry}

\definecolor{pblue}{rgb}{0.13,0.13,1}
\definecolor{pgreen}{rgb}{0,0.5,0}


\lstset{ %
language=Java,   							% choose the language of the code
basicstyle=\small\ttfamily,  				% the size of the fonts that are used for the code
numbers=left,                   			% where to put the line-numbers
numbersep=5pt,                  			% how far the line-numbers are from the code
backgroundcolor=\color{light-light-gray},   % choose the background color. You must add
frame=lrtb,           						% adds a frame around the code
tabsize=4,          						% sets default tabsize to 2 spaces
captionpos=b,           					% sets the caption-position to bottom
breaklines=true,        					% sets automatic line breaking
xleftmargin=1.5cm,							% space from the left paper edge
commentstyle=\color{pgreen},
keywordstyle=\color{pblue},
literate=%
    {Ö}{{\"O}}1
    {Ä}{{\"A}}1
    {Ü}{{\"U}}1
    {ß}{{\ss}}1
    {ü}{{\"u}}1
    {ä}{{\"a}}1
    {ö}{{\"o}}1
    {~}{{\textasciitilde}}1
}
\renewcommand{\lstlistingname}{Code}
\captionsetup[lstlisting]{font={footnotesize},margin=1.5cm,singlelinecheck=false } % removes "Listing 1: "
\definecolor{light-light-gray}{gray}{0.95}
\let\stdsection\section
\renewcommand\section{\stdsection}

% add line break for subtitle (size: large)
\title{Formale Grundlagen der Informatik I
    \\\large{
        Abgabe der Hausaufgaben\\
        Übungsgruppe 24 am Freitag, d. \today
    }
}
\author{~\\
	\Large{Louis Kobras}\\
	\large{6658699}\\ %Matrikelnummer; wenn nicht für Uni, auskommentieren
	\large{4kobras@informatik.uni-hamburg.de}\\
	\\
	\Large{Utz Pöhlmann}\\
	\large{6663579}\\ %Matrikelnummer; wenn nicht für Uni, auskommentieren
	\large{4poehlma@informatik.uni-hamburg.de}\\
	\\
	\Large{Philipp Quach}\\
	\large{6706421}\\ %Matrikelnummer; wenn nicht für Uni, auskommentieren
	\large{4quach@informatik.uni-hamburg.de}
}

% leave empty for no date on title page
% comment for auto-generated date
\date{\today}


\begin{document}
	\maketitle\thispagestyle{empty}
	\newpage
	\pagenumbering{arabic}
	\newgeometry{top=2.5cm, bottom =2.5cm, left=2cm, right=3.5cm}
	\section*{Aufgabe 10.4}
	\label{sec:a10.4}
	\begin{flushright}
	\large{[~~~~/2]}
	\end{flushright}
	%-----AUFGABE-----%
	Beweisen Sie, dass eine Inferenzregel $R=\frac{F_1,\dots,F_n}{G}$ genau dann korrekt ist,
	wenn $\{F_1,\dots,F_n\} \vDash G$ gilt.
	(Nutzen Sie dazu die Definition der Korrektheit einer Inferenzregel auf Folie 31.)
	\-\\\\\\
	%-----RECHNUNG-----%
	Definition: Wenn $M \vdash_R H$ (durch Benutzen von \textit{R} wird aus einer Formel \textit{M} die Formel \textit{H}), dann auch $M \vDash H$ (jede Belegung, die \textit{M} wahr macht, macht auch \textit{H} wahr).\\
	Daraus folgt, dass $M \vdash_R H$ gleichbedeutend ist mit:\\
	Wir haben eine Menge aus Formeln $A_1,\dots,A_m$, eine Inferenzregel $R=\frac{B_1,\dots,B_m}{C}$ und eine Formel \textit{H}.\\
	Wir formen um:\\
	\begin{table}[h]
	\centering
	\begin{tabular}{ll}
	$R=\frac{B_1,\dots,B_m}{C}$	&	Sei $n \leq m \land A_1,\dots,A_{m-x} \equiv B_1,\dots,B_n~~|x\geq0$	\\
	$R=\frac{A_1,\dots,A_m}{C}$	&	Da $A_1,\dots,A_{m-x}$ die benutzen Formeln aus \textit{M} sind,		\\
								&	sei nun \textit{C} die geschlussfolgerte Formel (also H)				\\
	$R=\frac{A_1,\dots,A_m}{H}$	&	$A_1,\dots,A_{m-x}\in M \Rightarrow M \geq \{A_1,\dots,A_{m-x}\}$ 		\\
								&	$\Rightarrow$ Für $A_1,\dots,A_{m-x}$ kann auch M eingesetzt werden, 	\\
								&	da die zusätzlichen Formeln nicht benutzt werden müssen.				\\
	$R=\frac{M}{H}$
	\end{tabular}
	\end{table}\\
	Am Anfang wurde definiert, dass $M \vDash H$ gilt.
	Nun ist:
	\begin{equation*}
	\begin{split}
		\frac{F_1,\dots,F_n}{G}=R=\frac{M}{H}
	\end{split}
	{}~~~~~{}
	\begin{split}
		\text{s. links:}\begin{cases}
			M\hat{=}F_1,\dots,F_n\\
			H\hat{=}G
		\end{cases}
	\end{split}
	\end{equation*}
	$\Rightarrow M \vDash H \equiv \{F_1,\dots,F_n\} \vDash H $
	%++++++++++++++++++++++++++++++++++++++++++++++++++++++++++%
	\section*{Aufgabe 10.5}
	\label{sec:a10.5}
	\begin{flushright}
	\large{[~~~~/3]}
	\end{flushright}
	%-----AUFGABE-----%
	\subsection*{Aufgabe 10.5.1}
	\label{ssec:a10.5.1}
	Seien $F=((A \Leftrightarrow B ) \land B \land \lnot C)$ und $G=((B \lor \lnot C) \Leftrightarrow \lnot C ) \land \lnot C \land \lnot (B \lor \lnot C)$.
	Geben Sie eine Substitution \textit{sub} an mit $sub(F)=G$ oder begründen Sie, warum dies nicht möglich ist.
	\-\\\\\\
	%-----RECHNUNG-----%
	Da nur atomare Formeln substituiert werden können, muss der Bijunktionspfeil erhalten bleiben, da in beiden Formeln nur jeweils einer vorkommt.
	$\Rightarrow \operatorname{sup}(A)=(B \lor \lnot C)$\\
	Die Position der Formeln ergibt dann $\operatorname{dub}(B)=(\lnot C)$.
	Durch $\operatorname{sub}(C)=(B \lor \lnot C)$ wird aus \textit{F} \textit{G}.
	%-----AUFGABE-----%
	\subsection*{Aufgabe 10.5.2}
	\label{ssec:a10.5.2}
	Zeigen Sie, dass für jede Formel \textit{F} und jede Substitution \textit{sub} gilt:
	Wenn \textit{F} eine Tautologie ist, dann ist auch $sub(F)$ eine Tautologie.
	Vervollständigen Sie dazu den Beweis aus der Vorlesung.
	Führen Sie insb. die dort nicht ausgeführte strukturelle Induktion.
	\-\\\\\\
	%-----RECHNUNG-----%
	Seien $A_1,\dots,A_n$ die in \textit{F} vorkommenden Aussagensymbole und $\mathcal{A}$ eine Belegung.
	Sei $\mathcal{A}'$ eine neue Belegung mit $\mathcal{A}'(A_i):=\mathcal{A}(\operatorname{sub}(A_i))$.\\
	Dies ist möglich, da alle $A_i$ kontingent sind.\\
	Sei \textit{B} eine Behauptung: $\mathcal{A}'(F)=\mathcal{A}(\operatorname{sub}(F))$.
	\begin{enumerate}
		\item Induktionsanfang: B gilt für jede atomare Formel (gegeben durch die Definition von $\mathcal{A}'$)
		\item Induktionsannahme: $"B(C)" \land "B(D)"$ gelte für $"C" \land "D"$.
		\item Induktionsschritt: Unter Annahme von (2) gilt:
	\end{enumerate}
	$B(\lnot C) \overset{\text{laut Def. v. }\mathcal{A}}{=}\operatorname{sub}(\lnot C)\overset{\text{s. Vl 17 S. 5}}{=}\lnot \operatorname{sub}(C)\overset{l.Def.v.\mathcal{A}}{=}\lnot B(C)\text{ (gilt wegen (2))}$\\
	$B(C \circ D) \overset{l.Def.v.\mathcal{A}}{=} \operatorname{sub}((C \circ D)\overset{\text{s. Vl 17 S. 5}}{=}\operatorname{sub}(C)\circ\operatorname{sub}(D)\overset{l.Def.v.\mathcal{A}}{=}B(C)\circ B(D)\text{ (gilt wegen (2))}$\\
	$\circ \in \{\lor, \land, \Rightarrow, \Leftrightarrow\}$
	%++++++++++++++++++++++++++++++++++++++++++++++++++++++++++%
	\section*{Aufgabe 10.6}
	\label{sec:a10.6}
	\begin{flushright}
	\large{[~~~~/7]}
	\end{flushright}
	%-----AUFGABE-----%
	\subsection*{Aufgabe 10.6.1}
	\label{ssec:10.6.1}
	Zeigen oder Widerlegen Sie, dass die folgenden Inferenzregeln korrekt sind:
	\begin{equation*}
		\begin{split}
			\frac{A \Rightarrow B, B \Rightarrow A}{\lnot B \lor A}
		\end{split}
		{}~~~~~~~~~~~~~~~~{}
		\begin{split}
			\frac{(A \lor B) \Rightarrow C, \lnot C \land \lnot B}{A \lor B}
		\end{split}
	\end{equation*}
	\-\\\\\\
	%-----RECHNUNG-----%
	\begin{equation*}
		\begin{array}{cc|ccccc}
			A	&	B	&	A \Rightarrow B	&	B \Rightarrow A	&	(B \Rightarrow A) \land (A \Rightarrow B)	&	\lnot B	&	\lnot B \lor A	\\	\hline
			0	&	0	&	1	&	1	&	1	&	1	&	1	\\
			0	&	1	&	1	&	0	&	0	&	0	&	0	\\
			1	&	0	&	0	&	1	&	0	&	1	&	1	\\
			1	&	1	&	1	&	1	&	1	&	0	&	1
		\end{array}
	\end{equation*}
	$\Rightarrow$ bewiesen, da $(\lnot B \lor A)$ auch dann wahr ist, wenn $((A \Rightarrow B) \land (B \Rightarrow A))$ wahr ist.
	\begin{equation*}
		\begin{array}{ccc|ccccc}
			A	&	B	&	C	&	A \lor B	&	(A \lor B)\Rightarrow C	&	\lnot C	&	\lnot B	&	\lnot C \land \lnot B	\\	\hline
			0	&	0	&	0	&	0	&	0	&	1	&	1	&	1	\\
			0	&	0	&	1	&	0	&	1	&	0	&	1	&	0	\\
			0	&	1	&	0	&	1	&	0	&	1	&	0	&	0	\\
			0	&	1	&	1	&	1	&	1	&	0	&	0	&	0	\\
			1	&	0	&	0	&	1	&	0	&	1	&	1	&	1	\\
			1	&	0	&	1	&	1	&	1	&	0	&	1	&	0	\\
			1	&	1	&	0	&	1	&	0	&	1	&	0	&	0	\\
			1	&	1	&	1	&	1	&	1	&	0	&	0	&	0
		\end{array}
	\end{equation*}
	$\Rightarrow$ widerlegt, da $(A\lor B)$ an mindestens einer Stelle wahr ist, an der $(A \lor B)\Rightarrow C$ und $C \land \lnot B$ wahr sind.
	%-----AUFGABE-----%
	\subsection*{Aufgabe 10.6.2}
	\label{ssec:10.6.2}
	Sei $\mathcal{C}=(\mathcal{L}_{AL},Ax,\mathcal{R})$ ein Kalkül der Aussagenlogik mit $Ax=\{A\Rightarrow (B\Rightarrow A)\}$ und $R=\{\frac{\lnot G,F\Rightarrow G}{\lnot F},\frac{\lnot G, F \land G}{F}\}$.
	Sei ferner $M=\{A \lor C, \lnot (E \Rightarrow C)\}$.\\
	Zeigen Sie $M\vdash_{\mathcal{c}}A$ durch Angabe einer Ableitung.
	\-\\\\\\
	%-----RECHNUNG-----%
	$R_1= \frac{\lnot G,F\Rightarrow G}{\lnot F}\hat{=}\text{Modus Tollens (MT)} \land R_2= \frac{\lnot G, F \land G}{F}\hat{=}\text{Disjunktiver Syllogismus (DS1)} $\\
	\begin{equation*}
		\begin{array}{lll}
			M	&\vdash	\lnot (E \Rightarrow C)				& [\text{aus M}] \\
				&\vdash	C \Rightarrow (E \Rightarrow C)	& [\text{\textit{Ax} mit }\operatorname{sub}(A)=C \land \operatorname{sub}(B)=E]	\\
				&\vdash	\lnot C							& [\text{MT mit }\operatorname{sub}(G)=(E \Rightarrow C) \land \operatorname{sub}(F)=C]	\\
				&\vdash	A \lor C						& [\text{aus M}]	\\
				&\vdash	A	& [\text{DS1 mit }\operatorname{sub}(G)=\lnot C \land \operatorname{sub}(F)=A]
		\end{array}
	\end{equation*}
	%++++++++++++++++++++++++++++++++++++++++++++++++++++++++++%
\end{document}