\documentclass[parskip,12pt,paper=a4,sffamily]{scrartcl}
%alternate documentclass:
%\documentclass[parskip,12pt,paper=a4,sffamily]{article}
\usepackage[utf8]{inputenc}
\usepackage[ngerman]{babel}
\usepackage{lastpage}
\usepackage{color}   %May be necessary if you want to color links
\usepackage{hyperref}
% code snippets
\usepackage{listings}
% listing captions
\usepackage{caption}
% font: times new roman
\usepackage{times}
% tikz being tikz
\usepackage{tikz}
% import math packages
\usepackage{amsmath}
\usepackage{amsfonts}
\usepackage{amssymb}
\usepackage{amsthm}
% contradiction lightning
\usepackage{stmaryrd}
% alignment options
\usepackage{ragged2e}
% page margins
\usepackage[margin=2.5cm]{geometry}

\definecolor{pblue}{rgb}{0.13,0.13,1}
\definecolor{pgreen}{rgb}{0,0.5,0}


\lstset{ %
language=Java,   							% choose the language of the code
basicstyle=\small\ttfamily,  				% the size of the fonts that are used for the code
numbers=left,                   			% where to put the line-numbers
numbersep=5pt,                  			% how far the line-numbers are from the code
backgroundcolor=\color{light-light-gray},   % choose the background color. You must add
frame=lrtb,           						% adds a frame around the code
tabsize=4,          						% sets default tabsize to 2 spaces
captionpos=b,           					% sets the caption-position to bottom
breaklines=true,        					% sets automatic line breaking
xleftmargin=1.5cm,							% space from the left paper edge
commentstyle=\color{pgreen},
keywordstyle=\color{pblue},
literate=%
    {Ö}{{\"O}}1
    {Ä}{{\"A}}1
    {Ü}{{\"U}}1
    {ß}{{\ss}}1
    {ü}{{\"u}}1
    {ä}{{\"a}}1
    {ö}{{\"o}}1
    {~}{{\textasciitilde}}1
}
\renewcommand{\lstlistingname}{Code}
\captionsetup[lstlisting]{font={footnotesize},margin=1.5cm,singlelinecheck=false } % removes "Listing 1: "
\definecolor{light-light-gray}{gray}{0.95}
\let\stdsection\section
\renewcommand\section{\stdsection}

% add line break for subtitle (size: large)
\title{Hausaufgaben zum 21. Mai 2015
    \\\large{
        Mathematik für Studierende der Informatik II\\
        (Analysis und Lineare Algebra)
    }
}
\author{~\\
	\Large{Louis Kobras}\\
	\large{6658699}\\ %Matrikelnummer; wenn nicht für Uni, auskommentieren
	\large{4kobras@informatik.uni-hamburg.de} \\
	\\
	\Large{Utz Pöhlmann}\\
	\large{6663579}\\
	\large{4poehlma@informatik.uni-hamburg.de}\\
	\\
	\Large{Jennifer Hartmann}\\
	\large{6706472}\\
	\large{fwuy089@studium.uni-hamburg.de}
}

% leave empty for no date on title page
% comment for auto-generated date
\date{\today}


\begin{document}
	\maketitle
	\newpage
	\section*{Aufgabe 1}
	\label{sec:a1}
	%%%%---AUFGABE---%%%%
	Berechnen Sie das Inverse der Matrix
	\[
	    A=
	    \begin{pmatrix}
	        1 & 1 & -1 \\
	        0 & 1 &  1 \\
	        2 & 2 &  1
	    \end{pmatrix}.
	\]\\\\
	%%%%---RECHNUNG---%%%%
	Um eine Matrix zu invertieren, ist ebenjene Matrix $B$ gesucht, mit der die gegebene Matrix multipliziert werden muss, um die Einheitsmatrix zu erhalten.\\
	\begin{table}[h]
	    \centering
        \begin{tabular}{cc}
                 & $\begin{pmatrix}a_{11}&a_{12}&a_{13}\\a_{21}&a_{22}&a_{23}\\a_{31}&a_{32}&a_{33}\end{pmatrix}$ \\
            $\begin{pmatrix}1&1&-1\\0&1&1\\2&2&1 \end{pmatrix}$  & $\begin{pmatrix}1~~~&0~~~&0\\0~~~&1~~~&0\\0~~~&0~~~&1 \end{pmatrix}$
        \end{tabular}
    \end{table}\\
    Daraus ergeben sich folgende neun Gleichungen:\\
    \[
        \begin{array}{lll}
           % first line times undefined matrix
            I~~~~~~&1=1\cdot a_{11}+1\cdot a_{21}+(-1)\cdot a_{31}~~~&\Rightarrow a_{31}+1=a_{11}+a_{21}\\
            II~~~~~&0=1\cdot a_{12}+1\cdot a_{22}+(-1)\cdot a_{32}~~~&\Rightarrow a_{32}=a_{12}+a_{22}\\
            III~~~~&0=1\cdot a_{13}+1\cdot a_{23}+(-1)\cdot a_{33}~~~&\Rightarrow a_{33}=a_{13}+a_{23}\\
        \\ % second line times undefined matrix
            IV~~~~~&0=0\cdot a_{11}+1\cdot a_{21}+1\cdot a_{31}~~~&\Rightarrow a_{21}=-a_{31} \\
            V~~~~~~&1=0\cdot a_{12}+1\cdot a_{22}+1\cdot a_{32} \\
            VI~~~~~&0=0\cdot a_{13}+1\cdot a_{23}+1\cdot a_{33}~~~&\Rightarrow a_{23}=-a_{33} \\
        \\ % third line times undefined matrix
            VII~~~~&0=2\cdot a_{11}+2\cdot a_{21}+1\cdot a_{31} \\
            VIII~~~&0=2\cdot a_{12}+2\cdot a_{22}+1\cdot a_{32} \\
            IX~~~~~&1=2\cdot a_{13}+2\cdot a_{23}+1\cdot a_{33} \\
        \end{array}
    \]
    Durch die offensichtlichen Umformungen und das Einsetzen von Gleichungen ineinander ermitteln wir nun die Werte $(a_{ij})_{i,j\in[1,2,3]}$ als Lösungen dieses Gleichungssystems.\\
        \begin{align*}
            Wir~beginnen~mit~der~&ersten~Spalte:\\
                Einsetzen~von~&IV~in~I:\\
                    a_{31}+1&=a_{11}+(-a_{31})\\
                    \Rightarrow~~a_{11}&=a_{31}+a_{31}+1=2\cdot a_{31}+1\\
                Einsetzen~von~&a_{11}~und~a_{21}~in~VII:\\
                    0&=2\cdot (2\cdot a_{31}+1)+2\cdot(-a_{31})+1\cdot a_{31}=4\cdot a_{31}+2-2\cdot a_{31}+a_{31}\\
                    &=2+3\cdot a_{31}\\
                    \Rightarrow~~a_{31}&=-\frac{2}{3}\\
                Einsetzen~von~&a_{31}~in~a_{11}~und~a_{21}:\\
                    a_{21}&=-a_{31}=\frac{2}{3}\\
                    a_{11}&=2\cdot a_{31}+1=-\frac{1}{3}\\
            Fortfahren~mit~der~&zweiten~Spalte:\\
                Einsetzen~von~&II~in~VIII:\\
                    0&=2\cdot a_{12}+2\cdot a_{22}+1\cdot (a_{12}+a_{22})\\
                    &=3\cdot a_{12}+3\cdot a_{22}\\
                    \Rightarrow~~a_{12}&=-a_{22}\\
                Einsetzen~von~&a_{32}~in~V:\\
                    1&=1\cdot a_{22}+1\cdot(a_{22}-a_{22})\\
                    \Rightarrow~~a_{22}&=1\\
                Einsetzen~von~&a_{22}~in~a_{12}~und~a_{32}:\\
                    a_{12}&=-a_{22}=-1\\
                    a_{32}&=a_{22}-a_{22}=1-1=0\\
            Fortfahren~mit~der~&dritten~Spalten:\\
                Einsetzen~von~&VI~in~III:\\
                    0&=1\cdot a_{13}+1\cdot(-a_{33}+(-1)\cdot a_{33}\\
                    \Rightarrow~~a_{13}&=2\cdot a_{33}\\
                Einsetzen~von~&a_{13}~und~a_{23}~in~IX:\\
                    1&=2\cdot(2\cdot a_{33})+2\cdot(-a_{33})+1\cdot a_{33}=4\cdot a_{33}-2\cdot a_{33}+1\cdot a_{33}=3 \cdot a_{33}\\
                    \Rightarrow~~a_{33}&=\frac{1}{3}\\
                Einsetzen~von~&a_{33}~in~a_{13}~und~a_{23}:\\
                    a_{13}&=2\cdot a_{33}=\frac{2}{3}\\
                    a_{23}&=-a_{33}=-\frac{1}{3}
        \end{align*}
    Anhand dieser Werte für $(a_{ij})_{i,j\in[1,2,3]}$ stellen wir nun die Matrix $B$ auf:
    \[
        \begin{pmatrix}
            -\frac{1}{3} & -1 &  \frac{2}{3}\\
             \frac{2}{3} &  1 & -\frac{1}{3}\\
            -\frac{2}{3} &  0 &  \frac{1}{3}
        \end{pmatrix}.
    \]
    Bei dieser Matrix handelt es sich um das Inverse der Matrix $A$ aus der Aufgabe.\\
    \\
    \textit{Probe:}
    \begin{table}[h]
	    \centering
        \begin{tabular}{cc}
                 & $\begin{pmatrix}-\frac{1}{3} & -1 &  \frac{2}{3}\\ \frac{2}{3} &  1 & -\frac{1}{3}\\ -\frac{2}{3} &  0 &  \frac{1}{3}\end{pmatrix}$ \\
            $\begin{pmatrix}1&1&-1\\0&1&1\\2&2&1 \end{pmatrix}$  & $\begin{pmatrix}a_{11}~&a_{12}~&a_{13}\\a_{21}~&a_{22}~&a_{23}\\a_{31}~&a_{32}~&a_{33} \end{pmatrix}$
        \end{tabular}
    \end{table}\\
    \begin{table}[h]
    	\centering
    	\begin{tabular}{lllll}
        $I$    &    $a_{11}=$ & $-\frac{1}{3}\cdot 1~+~\frac{2}{3}\cdot 1~-~\frac{2}{3}\cdot 1$		&	$=~~-\frac{1}{3}~+~\frac{2}{3}~+~\frac{2}{3}$	&	$=~~~~1$\\
        $II$   &    $a_{11}=$ & $-1\cdot 1~+~1\cdot 1~+~0\cdot 1 $									&	$=~~1~-~1$										&	$=~~~~0$\\
        $III$  &    $a_{11}=$ & $\frac{2}{3}\cdot 1~-~\frac{1}{3}~+~\frac{1}{3}\cdot(-1) $			&	$=~~\frac{2}{3}~+~\frac{1}{3}~+~\frac{1}{3}$	&	$=~~~~0$\\
        \\
        $IV$   &    $a_{11}=$ & $-\frac{1}{3}\cdot 0~+~\frac{2}{3}\cdot 1~-~\frac{2}{3}\cdot 1 $	&	$=~~\frac{2}{3}~-~\frac{2}{3}$					&	$=~~~~0$\\
        $V$    &    $a_{11}=$ & $-1\cdot 0~+~1\cdot 1~+~0\cdot 1 $									&	$=~~0~+~1$										&	$=~~~~1$\\
        $VI$   &    $a_{11}=$ & $-\frac{2}{3}\cdot 0~-~\frac{1}{3}\cdot 1~+~\frac{1}{3}\cdot 1 $	&	$=~~-\frac{1}{3}~+~\frac{1}{3}$					&	$=~~~~0$\\
        \\
        $VII$  &    $a_{11}=$ & $-\frac{1}{3}\cdot 2~+~\frac{2}{3}\cdot 2~-~\frac{2}{3}\cdot 1 $	&	$=~~-\frac{2}{3}~+~\frac{4}{3}~-~\frac{2}{3}$	&	$=~~~~0$\\
        $VIII$ &    $a_{11}=$ & $-1\cdot 2~+~1\cdot 2~+~1\cdot 0$									&	$=~~-2\cdot 2$									&	$=~~~~0$\\
        $IX$   &    $a_{11}=$ & $\frac{2}{3}\cdot 2~-~\frac{1}{3}\cdot 2~+~\frac{1}{3}\cdot 1 $		&	$=~~\frac{4}{3}~-~\frac{2}{3}~+~\frac{1}{3}$	&	$=~~~~1$
        \end{tabular}
    \end{table}\\
    \[\Rightarrow~~~~\begin{pmatrix}a_{11}&a_{12}&a_{13}\\a_{21}&a_{22}&a_{23}\\a_{31}&a_{32}&a_{33} \end{pmatrix}~~~~=~~~~~\begin{pmatrix} 1&0&0\\0&1&0\\0&0&1   \end{pmatrix}\]
    Damit ist die Probe gemacht.
    \newpage
	\section*{Aufgabe 2}
	\label{sec:a2}
	%%%%---AUFGABE---%%%%
	Lösen Sie das lineare Gleichungssystem
	\[
	    Ax=\begin{pmatrix}
	        1\\2\\3
	    \end{pmatrix},
	\]
	wobei $A$ die Matrix aus Aufgabe 1 ist.\\\\
	%%%%---RECHNUNG---%%%%
	\[\begin{pmatrix}1&1&-1\\0&1&1\\2&2&1 \end{pmatrix}~~\cdot \overset{\rightarrow}{x}~=~~~\begin{pmatrix} 1\\2\\3 \end{pmatrix}\]
	\begin{equation*}
	\begin{split}
		\begin{array}{c|c}
			\begin{array}{l}
				\begin{array}{lccccl}
					I~~~~&1&1&-1&1&~	\\
					II~~~&0&1& 1&2&~	\\
					III~~&2&2& 1&3&~~~~(III-2\cdot I)	
				\end{array}\\
				\\
				\begin{array}{lccccl}
					I~~~~&1&1&-1&1&~	\\
					II~~~&0&1& 1&2&~	\\
					III~~&0&0& 3&1&~~~~(:3)	
				\end{array}\\
				\\
				\begin{array}{lccccl}
					I~~~~&1&1&-1&1&~~~~(I~-~II)	\\
					II~~~&0&1& 1&2&~	\\
					III~~&0&0& 1&\frac{1}{3}	
				\end{array}
			\end{array}
			{}~~~~~&~~~~~{}
			\begin{array}{l}
				\begin{array}{lccccl}
					I~~~~&1&0&-2&-1&~~~~(I~+~2\cdot II)	\\
					II~~~&0&1& 1& 2&~~~~(II~-~III)	\\
					III~~&0&0& 1& \frac{1}{3}
				\end{array}\\
				\\
				\begin{array}{lcccc}
					I~~~~&1&0&0&-\frac{1}{3}	\\
					II~~~&0&1&0& \frac{5}{3}	\\
					III~~&0&0&1& \frac{1}{3}	
				\end{array}
			\end{array}
		\end{array}
	\end{split}
	\end{equation*}\\
	\[\overset{\rightarrow}{x}~=~\begin{pmatrix}-\frac{1}{3}\\\frac{5}{3}\\\frac{1}{3}\end{pmatrix}\]
	\newpage
	\section*{Aufgabe 3}
	\label{sec:a3}
	%%%%---AUFGABE---%%%%
	Zeigen Sie, dass für zwei ($2~\times~2$)-Matrizen $A$ und $B$ tatsächlich die Formel
	\[
	    Det(A)Det(B)=Det(AB)
	\]
	gilt.\\\\
	%%%%---RECHNUNG---%%%%
	Zunächst werden die Determinanten von $A$ und $B$ und dann deren Produkt bestimmt.
	\[
		A~=~\begin{pmatrix}
			a_{11}&a_{12}\\
			a_{21}&a_{22}
		\end{pmatrix}~~~~
		B~=~\begin{pmatrix}
			b_{11}&b_{12}\\
			b_{21}&b_{22}
		\end{pmatrix}
	\]\\
	\[Det(A)~=~a_{11}a_{22}~-~a_{12}a_{21}\]
	\[Det(B)~=~b_{11}b_{22}~-~b_{21}b_{12}\]
	\begin{align*}
		Det(AB)~&=~(a_{11}a_{22}~-~a_{12}a_{21})(b_{11}b_{22}~-~b_{21}b_{12})\\
				\\
				&=a_{11}a_{22}b_{11}b_{22}~+~a_{12}a_{21}b_{12}b_{21}\\
				&-~a_{11}a_{22}b_{12}b_{21}~-~a_{12}a_{21}b_{11}b_{22}
	\end{align*}
	Nun wird zunächst das Produkt der Matrizen $A$ und $B$ bestimmt, anschließend die Determinante des Produktes:\\
	\[
		AB~=~\begin{pmatrix}
			(a_{11}b_{11}+a_{12}b_{21})&(a_{21}b_{11}+a_{22}b_{21})\\
			(a_{11}b_{12}+a_{12}b_{22})&(a_{21}b_{12}+a_{22}b_{22})
		\end{pmatrix}
	\]
	\begin{align*}
		Det(AB)&=~(a_{11}b_{11}+a_{12}b_{21})~\cdot~(a_{21}b_{12}+a_{22}b_{22})\\
			   &-~(a_{11}b_{12}+a_{12}b_{22})~\cdot~(a_{21}b_{11}+a_{22}b_{21})\\
			   \\
			   &=~a_{11}b_{11}a_{21}b_{12}~+~a_{11}b_{11}a_{22}b_{22}\\
			   &+~a_{12}b_{21}a_{12}b_{11}~+~a_{12}b_{21}a_{22}b_{22}\\
			   &-~a_{11}b_{12}a_{12}b_{11}~-~a_{11}b_{12}a_{22}b_{21}\\
			   &-~a_{12}b_{12}a_{21}b_{11}~-~a_{21}b_{22}a_{22}b_{21}\\
			   \\
			   &=~a_{11}a_{21}b_{11}b_{22}~+~a_{11}a_{22}b_{11}b_{22}\\
			   &+~a_{12}a_{21}b_{12}b_{21}~+~a_{12}a_{22}b_{21}b_{22}\\
			   &-~a_{11}a_{21}b_{11}b_{12}~-~a_{11}a_{22}b_{12}b_{21}\\
			   &-~a_{12}a_{21}b_{11}b_{22}~-~a_{12}a_{22}b_{21}b_{22}\\
			   \\
			   &=~a_{11}a_{22}b_{11}b_{22}~+~a_{12}a_{21}b_{12}b_{21}\\
			   &-~a_{11}a_{22}b_{12}b_{21}~-~a_{12}a_{21}b_{11}b_{22}
	\end{align*}
	Damit ist gezeigt, dass $Det(A)Det(B)=Det(AB)$.
	\newpage
	\section*{Aufgabe 4}
	\label{sec:a4}
	%%%%---AUFGABE---%%%%
	Die Folge $(\frac{1}{n}+\frac{1}{n^2})_{n\in\mathbb{N}}$ konvergiert gegen 0.
	Damit existiert $n_0\in\mathbb{N}$, so dass für alle $n \ge n_0$ die Ungleichung $|\frac{1}{n}+\frac{1}{n^2}|<\frac{1}{3}$ gilt.
	Bestimmen Sie das kleinste $n_0\in\mathbb{N}$, das das leistet.\\\\
	%%%%---RECHNUNG---%%%%
	\begin{equation*}
		\begin{array}{l}
			\frac{1}{n}+\frac{1}{n^2}~=~\frac{n+1}{n^2}~=~\frac{1}{3}\\
			n^2~=~3n+3\\
			n^2-3n-\frac{3}{2}^2~=~3+\frac{9}{4}\\
			\left(n-\frac{3}{2}\right)^2~=~\frac{21}{4}\\
			n-\frac{3}{2}~=~+-\frac{\sqrt{21}}{2}\\
			n~=~\frac{3+\sqrt{21}}{2}\\
			n~\approx~3.79
		\end{array}
	\end{equation*}\\
	Da die kleinstmögliche ganze Zahl $n$ gesucht wird, und $3<3.79$ und somit herausfällt, sowie alle Zahlen kleiner $3$, ergibt sich für $n$ der Minimalwert $n_0=4$.\\
	Probe:
	\begin{align*}
		n=3:~~~&\frac{1}{3}+\frac{1}{3^2}~=~\frac{4}{9}>\frac{1}{3}\\
		n=4:~~~&\frac{1}{4}+\frac{1}{4^2}~=~\frac{5}{16}<\frac{1}{3}
	\end{align*}
	\section*{Aufgabe 5}
	\label{sec:a5}
	%%%%---AUFGABE---%%%%
	Es sei $a\in\mathbb{R}$ irgendeine Zahl und $(a_n)_{n\in\mathbb{N}}$ eine Folge reeller Zahlen mit $a_n\rightarrow0$ für $n\rightarrow\infty$.
	Man zeige nur unter Benutzung der Definition von Konvergenz, dass die Folge $(a+a_n)_{n\in\mathbb{N}7}$ gegen $a$ konvergiert.
	%%%%---RECHNUNG---%%%%
	\begin{table}[h]
    	\centering
    	\begin{tabular}{cccc}
			$a\in\mathbb{R}$ ~~~~&~~~~ $(a_n)_{n\in\mathbb{N}}$ ~~~~&~~~~ $\underset{n\rightarrow\infty}{lim}$ ~~~~&~~~~ $a_n=0^+\vee 0^-$
		\end{tabular}
	\end{table}
	$(a_n)$ konvergiert gegen $0$, wenn $n$ gegen $\infty$ geht.\\
	$\Rightarrow$ rechnen wir zu jedem Wert $a_n$ eine Zahl a hinzu, konvergiert diese neue Folge $a_n+a$ gegen $a$, da $a_n$ gegen 0 geht und $0+a=a$.
	\begin{equation*}
		\underset{n\rightarrow\infty}{lim}a=a \wedge
		\underset{n\rightarrow\infty}{lim}a_n=0	\Rightarrow
		\underset{n\rightarrow\infty}{lim}a+a_n=a
	\end{equation*}
\end{document}