\documentclass[parskip,12pt,paper=a4,sffamily]{article}
%alternate documentclass:
%\documentclass[parskip,12pt,paper=a4,sffamily]{scrartl}
\usepackage[utf8]{inputenc}
\usepackage[ngerman]{babel}
\usepackage{lastpage}
\usepackage{color}   %May be necessary if you want to color links
\usepackage{hyperref}
% code snippets
\usepackage{listings}
% listing captions
\usepackage{caption}
% tikz being tikz
\usepackage{tikz}
\usetikzlibrary{arrows,automata}
\usepackage{pgf}
% import math packages
\usepackage{amsmath}
\usepackage{amsfonts}
\usepackage{amssymb}
\usepackage{amsthm}
% polynomial division
\usepackage{polynom}
% contradiction lightning
\usepackage{stmaryrd}
% alignment options
\usepackage{ragged2e}
% page margins
\usepackage[margin=2.5cm]{geometry}

\definecolor{pblue}{rgb}{0.13,0.13,1}
\definecolor{pgreen}{rgb}{0,0.5,0}


\lstset{ %
language=Java,   							% choose the language of the code
basicstyle=\small\ttfamily,  				% the size of the fonts that are used for the code
numbers=left,                   			% where to put the line-numbers
numbersep=5pt,                  			% how far the line-numbers are from the code
backgroundcolor=\color{light-light-gray},   % choose the background color. You must add
frame=lrtb,           						% adds a frame around the code
tabsize=4,          						% sets default tabsize to 2 spaces
captionpos=b,           					% sets the caption-position to bottom
breaklines=true,        					% sets automatic line breaking
xleftmargin=1.5cm,							% space from the left paper edge
commentstyle=\color{pgreen},
keywordstyle=\color{pblue},
literate=%
    {Ö}{{\"O}}1
    {Ä}{{\"A}}1
    {Ü}{{\"U}}1
    {ß}{{\ss}}1
    {ü}{{\"u}}1
    {ä}{{\"a}}1
    {ö}{{\"o}}1
    {~}{{\textasciitilde}}1
}

\polyset{%
  style=C,
  delims={\big(}{\big)},
  div=:
}

\renewcommand{\lstlistingname}{Code}
\captionsetup[lstlisting]{font={footnotesize},margin=1.5cm,singlelinecheck=false } % removes "Listing 1: "
\definecolor{light-light-gray}{gray}{0.95}
\let\stdsection\section
\renewcommand\section{\stdsection}

% add line break for subtitle (size: large)
\title{Mathematik für Studierende der Informatik II\\
	Analysis und Lineare Algebra\\
    \-\\\large{
    	Abgabe der Hausaufgaben zum \today
    }
}
\author{~\\
	\Large{Louis Kobras}\\
	\large{6658699}\\ %Matrikelnummer; wenn nicht für Uni, auskommentieren
	\large{4kobras@informatik.uni-hamburg.de}\\
	\\
	\Large{Utz Pöhlmann}\\
	\large{6663579}\\ %Matrikelnummer; wenn nicht für Uni, auskommentieren
	\large{4poehlma@informatik.uni-hamburg.de}
}

% leave empty for no date on title page
% comment for auto-generated date
\date{\today}


\begin{document}
	\maketitle\thispagestyle{empty}
	\newpage
	\pagenumbering{arabic}
	\newgeometry{top=2.5cm, bottom =2.5cm, left=2cm, right=3.5cm}
	\section*{Aufgabe 1}
	\label{sec:a1}
	\begin{flushright}
	\large{[~~~~/4]}
	\end{flushright}
	%-----AUFGABE-----%
	Berechnen Sie die bestimmten Integrale
	\[
	(a)~~~~\int_{-1}^1(x^2-x+2)dx~~~~~~~~~~~~(b)~~~~\int_0^{2\pi}(x+\operatorname{cos}x)dx
	\]
	\-\\\\\\
	%-----RECHNUNG-----%
	(a)
	\begin{align*}
		 &\int_{-1}^1(x^2-x+2)dx\\
		=&\left[-\frac{1}{6}x^3~-~\frac{1}{2}x^2~+~\frac{3}{2}x\right]_{-1}^{1}\\
		=&\left(-\frac{1}{6} \cdot 1^3~-~\frac{1}{2} \cdot 1^2~+~\frac{3}{2} \cdot 1 \right)~~-~~\left( -\frac{1}{6} \cdot (-1)^3~-~\frac{1}{2} \cdot (-1)^2~+~\frac{3}{2} \cdot (-1) \right)\\
		=&\left( -\frac{1}{6}-\frac{1}{2}+\frac{3}{2} \right) ~-~ \left( \frac{1}{6}-\frac{1}{2}-\frac{3}{2} \right)\\
		=&\left(-\frac{1}{6}+1\right)~-~\left(\frac{1}{6}-2\right)\\
		=&\frac{5}{6}~-~\left(-\frac{7}{6}\right)\\
		=&\frac{5}{6}~+~\frac{7}{6}\\
		=&\frac{12}{6}\\
		=&2
	\end{align*}
	(b)
	\begin{align*}
		&\int_{-1}^{2}(x^3+x^2)dx\\
		=&\left[\frac{1}{4}x^4+\frac{1}{3}x^3\right]_{-1}^{2}\\
		=&\left(\frac{1}{4} \cdot (2)^4+\frac{1}{3} \cdot (2)^3\right)	~-~\left(\frac{1}{4} \cdot (-1)^4+\frac{1}{3} \cdot (-1)^3\right)\\
		=&\left(\frac{16}{4}+\frac{8}{3}\right)~-~\left(\frac{1}{4}-\frac{1}{3}\right)	\\
		=&\left(\frac{48}{12}+\frac{32}{12}\right)~-~\left(\frac{3}{12}-\frac{4}{12}\right)	\\
		=&\left(\frac{80}{12}\right)~-~\left(-\frac{1}{12}\right)	\\
		=&\frac{81}{12}	\\
		=&\frac{27}{4}
	\end{align*}
	%++++++++++++++++++++++++++++++++++++++++++++++++++++++++++%
	\section*{Aufgabe 2}
	\label{sec:a2}
	\begin{flushright}
	\large{[~~~~/4]}
	\end{flushright}
	%-----AUFGABE-----%
	Berechnen Sie die Fläche, die zwischen der $x$-Achse und dem Graphen der Funktion $-x^2+1$ eingeschlossen ist.
	\-\\\\\\
	%-----RECHNUNG-----%
	$f(x)=-x^2+1$\\
	Nullstellen:
	\[
		0=-x^2+1 \Leftrightarrow x^2= 1 \Leftrightarrow x=\pm 1
	\]
	\begin{align*}
		&\int_{-1}^{1}(-x^2+1)dx\\
	   =&\left[-\frac{1}{3}x^3+x\right]_{-1}^{1}\\
	   =&\left(-\frac{1}{3} \cdot 1^3~+~1\right)~~-~~\left(-\frac{1}{3} \cdot (-1)^3~+~(-1)\right)\\
	   =&\left(-\frac{1}{3}+\frac{3}{3}\right)~-~\left(\frac{1}{3}-\frac{3}{3}\right)\\
	   =&\frac{2}{3}-\left(-\frac{2}{3}\right)\\
	   =&\frac{4}{3}
	\end{align*}
	%++++++++++++++++++++++++++++++++++++++++++++++++++++++++++%
	\section*{Aufgabe 3}
	\label{sec:a3}
	\begin{flushright}
	\large{[~~~~/4]}
	\end{flushright}
	%-----AUFGABE-----%
	Bestimmen Sie $x$ so, dass das bestimmte Integral genau den Wert 2 hat.
	\[
		\int_1^x\left(t^2-\frac{1}{3}\right)dt
	\]
	\-\\\\\\
	%-----RECHNUNG-----%
	\begin{align*}
		&\int_1^x\left(t^2-\frac{1}{3}\right)dt~=~2\\
	   =&\left[\frac{1}{3}t^3-\frac{1}{3}t\right]_1^x\\
	   =&\left(\frac{1}{3}x^3-\frac{1}{3}x\right)~-~\left(\frac{1}{3}\cdot1^3-\frac{1}{3}\cdot 1\right)\\
	   =&\left(\frac{1}{3}x^3-\frac{1}{3}x\right)~-~\left(\frac{1}{3}-\frac{1}{3}\right)\\
	   =&\left(\frac{1}{3}x^3-\frac{1}{3}x\right)~-~0\\
	   =&\left(\frac{1}{3}x^3-\frac{1}{3}x\right)~~~~=~~2\\
	   \Leftrightarrow~~~~&1x^3-1x=6~~~~\Leftrightarrow x^3-x-6=0\\
	   \Leftrightarrow~~~~&x(x^2-1)=6\\
	   \Leftrightarrow~~~~&x_1=2
	\end{align*}
	\[
		\polylongdiv{x^3-x-6}{x-2}
	\]
	\begin{align*}
		x_{2/3}&=-\frac{2}{2}\pm \sqrt{\frac{2}{2}^2-3}\\
			&=-1\pm\sqrt{-2}\\
			&=-1\pm\sqrt{2}i\\
			i \not \in& \mathbb{R} \Rightarrow x=2\text{ ist einzige Nullstelle}
	\end{align*}
	%++++++++++++++++++++++++++++++++++++++++++++++++++++++++++%
	\section*{Aufgabe 4}
	\label{sec:a4}
	\begin{flushright}
	\large{[~~~~/4]}
	\end{flushright}
	%-----AUFGABE-----%
	Berechnen Sie die unbestimmten Integrale:
	\[
		(a)~~~~\int x^2e^xdx~~~~~~~~~~~~(b)~~~~\int x\operatorname{ln}xdx
	\]
	Hinweis: Unter Umständen muss die partielle Integration mehrmals angewendet werden.
	\-\\\\\\
	%-----RECHNUNG-----%
	(a)
	\begin{align*}
	\int(x^2e^x)dx& \begin{cases}	
						u(x)=x^2\\
						v'(x)=e^x
					\end{cases}\\
			&=x^2e^x~-~\int(2xe^x)dx\begin{cases}
										u(x)=2x\\
										v'(x)=e^x
									\end{cases}\\
			&=x^2e^x~-~(2xe^x-\int(2e^x)dx)~~~~~~~~~~~~~~\text{(Integrieren ist linear)}\\
			&=x^2e^x-(2xe^x-2\int(e^x)dx)\\
			&=x^2e^x-(2xe^x-2e^x)\\
			&=x^2e^x-2xe^x+2e^x\\
			&=e^x(x^2-2x+2)
	\end{align*}
	(b)
	\begin{align*}
	\int(x\operatorname{ln}x)dx	&	\begin{cases}
									u(x)=\operatorname{ln}x\\
									v'(x)=x \Rightarrow v(x)=\frac{1}{2}x^2
								\end{cases}\\
								&=x\operatorname{ln}x-\int\left(\frac{1}{x}\frac{1}{2}x	2\right)dx~~~~~~~~~~~~~~\text{(Differenzieren ist linear)}\\
								&=x\operatorname{ln}x-\int\left(\frac{1}{2}x\right)dx\\
								&=x\operatorname{ln}x-\frac{1}{2}\int x dx\\
								&=x\operatorname{ln}x-\frac{1}{2}\cdot \frac{1}{2}x^2\\
								&=x\left(\operatorname{ln}x~-~\frac{1}{4}\right)
	\end{align*}
	%++++++++++++++++++++++++++++++++++++++++++++++++++++++++++%
	\section*{Aufgabe 5}
	\label{sec:a5}
	\begin{flushright}
	\large{[~~~~/4]}
	\end{flushright}
	%-----AUFGABE-----%
	Berechnen Sie die unbestimmten Integrale mit der Substitutionsmethode:
	\[
		(a)~~~~\int 4xe^{x^2-4}dx~~~~~~~~~~~~(b)~~~~\int \frac{\operatorname{ln}x}{x}dx
	\]
	\-\\\\\\
	%-----RECHNUNG-----%
	(a)
	\begin{align*}
		\int(4x&e^{x^2-4})dx~~~~~~~~~~~~u=g(x):=x^2-4\\
		g'(x)=\frac{du}{dx}&\Leftrightarrow dx=\frac{du}{g'(x)}\\
		&=\int(4xe^u)\frac{du}{(x^2-4)'}\\
		&=\int(4xe^u)\frac{du}{2x}\\
		&=\int2e^u du\\
		&=2e^u+c~~~~~~~~~~~~u\text{ resubstituieren}\\
		&=2e^{x^2-4}+c
	\end{align*}
	(b)
	\begin{align*}
	\int \frac{\operatorname{ln}x}{x}&dx~~~~~~~~~~~~&u=g(x):=\operatorname{ln}x\\
	&&g'(x)=\frac{du}{dx}\Leftrightarrow dx=\frac{du}{g'(x)}\\
	&=\int \frac{u}{x}\frac{du}{(\operatorname{ln}x)'}\\
	&=\int \frac{u}{x}\frac{du}{\frac{1}{x}}\\
	&=\int \frac{u}{x}x du\\
	&=\int u du\\
	&= u+c&u\text{ resubstituieren}\\
	&=\operatorname{ln}x+c
	\end{align*}
\end{document}