\documentclass[pdf]{beamer}
\usepackage[utf8]{inputenc}
\usepackage[ngerman]{babel}
\usepackage{tikz}
\usepackage{pgf}
\usetikzlibrary{arrows,automata}
\usepackage{amsfonts}                   % AMS Math Packet (Fonts)
\usepackage{amsmath}                    % AMS Math Packet
\usepackage{amssymb}                    % Additional mathematical symbols
\usepackage{amsthm}
\usepackage{lastpage}
\usepackage{listings}
\usepackage{caption}
\usepackage{cancel}
\usepackage{color}
\usepackage{eurosym}
\usepackage{courier} \raggedright
\usepackage{nameref}
\usepackage{hyperref}
\usepackage{url}
\usepackage{tabularx}
\usepackage{titleref}
\usepackage{soul}
\usepackage{xcolor}
\usepackage{colortbl}

\usetheme[compress]{Dresden}

\setbeamerfont{headline}{size=\large}
\setbeamerfont*{section in head/foot}{size=\tiny}
\setbeamercovered{transparent}
\setbeamertemplate{bibliography item}[text]
\setbeamertemplate{itemize items}[square]
\setbeamertemplate{itemize subitem}[triangle]
\setbeamertemplate{section in toc}[square]
\setbeamertemplate{subsection in toc}[square]
\setbeamertemplate{caption}[numbered]
\setbeamercolor*{title}{use=structure,fg=white,bg=structure.fg}
\setbeamertemplate{title page}[default][colsep=-2bp]
\setbeamercolor{block title}{bg=darkred2!100,fg=white}
\setbeamercolor{block body}{bg=darkred3!25,fg=black}

\definecolor{darkred}{rgb}{0,0.55,0.8}
\definecolor{darkred2}{rgb}{0,0.41,0.6}
\definecolor{darkred3}{rgb}{0.2,0.4,0.6}
\usecolortheme[named=darkred]{structure}

\title{Generieren dynamischer Antworten durch Parsing von natürlicher Sprache}

\author{Louis Kobras}

\begin{document}

	\begin{frame}
		\titlepage
	\end{frame}
	\begin{frame}
		\tableofcontents
	\end{frame}

	\section{Natürliche Sprache}
		\subsection{}
			\begin{frame}
				\textit{''By 'natural language' we mean a language that is used for everyday communication by humans; languages such as English, Hindi, or Portuguese.
        In contrast to artificial languages such as programming languages and mathematical notations, natural languages have evolved as they pass from generation to generation.
        and are hard to pin down with explicit rules.''}
			\end{frame}
			\begin{frame}
			
			\end{frame}
	\section{Automaten zum Parsen der Morphologie}
		\subsection{Morpheme}
			\begin{frame}
			
			\end{frame}
			\begin{frame}
			
			\end{frame}
		\subsection{DFAs}
			\begin{frame}
				\begin{figure}[h]
				\label{fig:dfa}
					\begin{tikzpicture}[->,>=stealth',shorten >=1pt,auto,node distance=3.0cm,semithick]
						\node [initial, state]		(q0)					{$q_0$};
						\node [state]				(q1)	[right of=q0]	{$q_1$};
						\node [accepting, state]	(q2)	[right of=q1]	{$q_2$};
						\node [accepting, state]	(q3)	[below of=q2]	{$q_3$};
						
						\path
							(q0)	edge	[bend right=20]	node	[align=center]	{1. Buchstabe \\ groß}			(q1)
							(q1)	edge	[bend right=20]	node	[align=center]	{Stamm in \\ Nomen}				(q2)
							(q2)	edge	[bend right=20]	node	[align=center]	{endet auf \\ \textbf{n}}		(q3)
							(q3)	edge	[loop right]	node	[align=center]	{Gültiges \\ Affix}				(q3)
						;
					\end{tikzpicture}
					\caption{DFA zur Zerlegung des Wortes Nachkommastellen in seine Morpheme}
				\end{figure}
			\end{frame}
			\begin{frame}
			
			\end{frame}
		\subsection{Umsetzung der Automaten}
			\begin{frame}
				\frametitle{Einbezug der Java-Klasse String}
			\end{frame}
			\begin{frame}
				\frametitle{Einbezug der Java-Klasse String}
			\end{frame}
			\begin{frame}
				\frametitle{Verwendung von RegEx}
			\end{frame}
			\begin{frame}
				\frametitle{Verwendung von RegEx}
			\end{frame}
		\subsection{ELIZA}
			\begin{frame}
			
			\end{frame}
			\begin{frame}
			
			\end{frame}
			\begin{frame}
			
			\end{frame}
	\section{Nutzung im Hinblick auf Spieleentwicklung}
		\subsection{}
			\begin{frame}
			
			\end{frame}
			\begin{frame}
			
			\end{frame}
\end{document}