\documentclass[15pt,a4]{article}
\usepackage[utf8]{inputenc}
\usepackage[ngerman]{babel}
\usepackage[margin=2.5cm]{geometry}

\begin{document}
	\begin{LARGE}
	\begin{itemize}
		\item Natürliche Sprache
			\begin{itemize}
				\item everyday communication:
					\begin{itemize}
	       				\item tägliche Kommunikation
				        \item das hier jetzt gerade
        			\end{itemize}
				\item have evolved:
					\begin{itemize}
						\item Sprachwandel
        				\item Archaismen und Neologismen
        				\item Änderungen von Wortbedeutungen
        				\item "eindeutschen" (i.e.)			
					\end{itemize}
				\item hard to pin down with explicit rules:
					\begin{itemize}
						\item irreguläre Fälle
				        \item Dialekte
				        \item Ausnahmeregeln
					\end{itemize}
				\item Gegensatz: formale Sprachen, mathematische Notation
				\item besteht aus 7 Komponenten:
					\begin{itemize}
						\item \textbf{Morphologie:} Lehre der Gestalten und Formen eines Objektes (Flexion, Komposition)
						\item \textbf{Syntax:} Lehre vom Bau eines Satzes und dessen formaler Struktur
						\item \textbf{Semantik:} Bedeutung und Inhalt eines Wortes, Satzes oder Textes
						\item \textbf{Pragmatik:} Beziehung zwischen Sprachbausteinen und ihren Anwendungen
						\item \textbf{Diskurs:} Gesamtheit aller sprachlichen Äußerungen eines Sprechers \textit{(eine Unterhaltung)}
						\item \textbf{Mehrdeutigkeit:} verschiedene Bedeutungen abhängig vom Kontext
						\item \textbf{Phonologie:} Funktion von Lauten
					\end{itemize}
			\end{itemize}
		\item Status Quo Textadventure
		\item ELIZA (springen zu 28?)
			\begin{itemize}
				\item Hintergrund
				\item Methodik (ggf. springen zu 26)
			\end{itemize}
		\item DFA
			\begin{itemize}
				\item erklären
				\item minimales Beispiel
			\end{itemize}
		\item RegEx
			\begin{itemize}
				\item fängt mit Großbuchstabe an?
				\item fängt mit Kleinbuchstabe an?
				\item hört auf 'n' auf? (vgl. Beispiel)
				\item enthält Nummer neben Buchstaben?
				\item ist nur ein Buchstabe lang? (kürzeste deutsche Wörter 2 BS)
				\item nur Zahlen?
			\end{itemize}
	\end{itemize}
	\end{LARGE}
\end{document}