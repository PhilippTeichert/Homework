\documentclass[12pt,twoside]{article}
\usepackage[ngerman]{babel}
%\usepackage{lmodern}
\usepackage[utf8]{inputenc}
%\usepackage[T1]{fontenc}
%%%%%%%%%%%%%%%%%%%%%%%%%%%%%%%%%%%%%%%%%%%%%%%%%%%%%%%%%%%%%
% Meta informations:
\newcommand{\trauthor}{Louis Kobras}
\newcommand{\trtype}{Seminararbeit} %{Seminararbeit} %{Proseminararbeit}
\newcommand{\trcourse}{Proseminar: Künstliche Intelligenz}
\newcommand{\trtitle}{Insert witty title here}
\newcommand{\trmatrikelnummer}{6658699}
\newcommand{\tremail}{4kobras@informatik.uni-hamburg.de}
\newcommand{\trarbeitsbereich}{Knowledge Technology, WTM}
\newcommand{\trdate}{01.04.2014}

%%%%%%%%%%%%%%%%%%%%%%%%%%%%%%%%%%%%%%%%%%%%%%%%%%%%%%%%%%%%%
% Bind packages:
\usepackage{acronym}                    % Acronyms
\usepackage{algorithmic}				% Algorithms and Pseudocode
\usepackage{algorithm}					% Algorithms and Pseudocode
\usepackage{amsfonts}                   % AMS Math Packet (Fonts)
\usepackage{amsmath}                    % AMS Math Packet
\usepackage{amssymb}                    % Additional mathematical symbols
\usepackage{amsthm}
\usepackage{booktabs}                   % Nicer tables
%\usepackage[font=small,labelfont=bf]{caption} % Numbered captions for figures
\usepackage{color}                      % Enables defining of colors via \definecolor
\definecolor{uhhRed}{RGB}{254,0,0}		% Official Uni Hamburg Red
\definecolor{uhhGrey}{RGB}{122,122,120} % Official Uni Hamburg Grey
\usepackage{fancybox}                   % Gleichungen einrahmen
\usepackage{fancyhdr}					% Packet for nicer headers
%\usepackage{fancyheadings}             % Nicer numbering of headlines

%\usepackage[outer=3.35cm]{geometry} 	% Type area (size, margins...) !!!Release version
%\usepackage[outer=2.5cm]{geometry} 	% Type area (size, margins...) !!!Print version
%\usepackage{geometry} 					% Type area (size, margins...) !!!Proofread version
\usepackage[outer=3.15cm]{geometry} 	% Type area (size, margins...) !!!Draft version
\geometry{a4paper,body={5.8in,9in}}

% changing font
%\usepackage{times}

\usepackage{graphicx}                   % Inclusion of graphics
%\usepackage{latexsym}                  % Special symbols
\usepackage{longtable}					% Allow tables over several parges
\usepackage{listings}                   % Nicer source code listings
\usepackage{multicol}					% Content of a table over several columns
\usepackage{multirow}					% Content of a table over several rows
\usepackage{rotating}					% Alows to rotate text and objects
\usepackage[hang]{subfigure}            % Allows to use multiple (partial) figures in a fig
%\usepackage[font=footnotesize,labelfont=rm]{subfig}	% Pictures in a floating environment
\usepackage{tabularx}					% Tables with fixed width but variable rows
\usepackage{url,xspace,boxedminipage}   % Accurate display of URLs
\usepackage{hyperref}
% listing captions
\usepackage{caption}
% tikz being tikz
\usepackage{tikz}
%%%%%%%%%%%%%%%%%%%%%%%%%%%%%%%%%%%%%%%%%%%%%%%%%%%%%%%%%%%%%
% Configurationen:

\hyphenation{whe-ther} 					% Manually use: "\-" in a word: Staats\-ver\-trag

%\lstloadlanguages{C}                   % Set the default language for listings
\DeclareGraphicsExtensions{.pdf,.svg,.jpg,.png,.eps} % first try pdf, then eps, png and jpg
\graphicspath{{./src/}} 								% Path to a folder where all pictures are located
\pagestyle{fancy} 											% Use nicer header and footer

% Redefine the environments for floating objects:
\setcounter{topnumber}{3}
\setcounter{bottomnumber}{2}
\setcounter{totalnumber}{4}
\renewcommand{\topfraction}{0.9} 		%Standard: 0.7
\renewcommand{\bottomfraction}{0.5}		%Standard: 0.3
\renewcommand{\textfraction}{0.1}		%Standard: 0.2
\renewcommand{\floatpagefraction}{0.8} 	%Standard: 0.5

% Tables with a nicer padding:
\renewcommand{\arraystretch}{1.2}

% custom titles for TOC and References
\renewcommand{\contentsname}{whatever}
\renewcommand{\refname}{WASD-Force}
%\renewcommand{\bibname}{DSAW-Force}

%%%%%%%%%%%%%%%%%%%%%%%%%%%%
% Additional 'theorem' and 'definition' blocks:
\theoremstyle{plain}
\newtheorem{theorem}{Theorem}[section]
%\newtheorem{theorem}{Satz}[section]		% Wenn in Deutsch geschrieben wird.
%\newtheorem{axiom}{Axiom}[section] 	
%\newtheorem{axiom}{Fakt}[chapter]		% Wenn in Deutsch geschrieben wird.
%Usage:%\begin{axiom}[optional description]%Main part%\end{fakt}

\theoremstyle{definition}
\newtheorem{definition}{Definition}[section]

%Additional types of axioms:
%\newtheorem{lemma}[axiom]{Lemma}
%\newtheorem{observation}[axiom]{Observation}

%Additional types of definitions:
\theoremstyle{remark}
\newtheorem{remark}[definition]{Bemerkung} % Wenn in Deutsch geschrieben wird.
%\newtheorem{remark}[definition]{Remark} 

%%%%%%%%%%%%%%%%%%%%%%%%%%%%
% Provides TODOs within the margin:
\newcommand{\TODO}[1]{\marginpar{\emph{\small{{\bf TODO: } #1}}}}

%%%%%%%%%%%%%%%%%%%%%%%%%%%%
% Abbreviations and mathematical symbols
\newcommand{\modd}{\text{ mod }}
\newcommand{\RS}{\mathbb{R}}
\newcommand{\NS}{\mathbb{N}}
\newcommand{\ZS}{\mathbb{Z}}
\newcommand{\dnormal}{\mathit{N}}
\newcommand{\duniform}{\mathit{U}}

\newcommand{\erdos}{Erd\H{o}s}
\newcommand{\renyi}{-R\'{e}nyi}

%%%%%%%%%%%%%%%%%%%%%%%%%%%
% Source code inclusion
\definecolor{pblue}{rgb}{0.13,0.13,1}
\definecolor{pgreen}{rgb}{0,0.5,0}
\lstset{ %
language=Python,   							% choose the language of the code
basicstyle=\small\ttfamily,  				% the size of the fonts that are used for the code
numbers=left,                   			% where to put the line-numbers
numbersep=5pt,                  			% how far the line-numbers are from the code
backgroundcolor=\color{light-light-gray},   % choose the background color. You must add
frame=lrtb,           						% adds a frame around the code
tabsize=4,          						% sets default tabsize to 2 spaces
captionpos=b,           					% sets the caption-position to bottom
breaklines=true,        					% sets automatic line breaking
xleftmargin=1.5cm,							% space from the left paper edge
commentstyle=\color{pgreen},
keywordstyle=\color{pblue},
literate=%
    {Ö}{{\"O}}1
    {Ä}{{\"A}}1
    {Ü}{{\"U}}1
    {ß}{{\ss}}1
    {ü}{{\"u}}1
    {ä}{{\"a}}1
    {ö}{{\"o}}1
    {~}{{\textasciitilde}}1
}
\renewcommand{\lstlistingname}{Code}
\captionsetup[lstlisting]{font={footnotesize},margin=1.5cm,singlelinecheck=false } % removes "Listing 1: "
\definecolor{light-light-gray}{gray}{0.95}


%%%%%%%%%%%%%%%%%%%%%%%%%%%%%%%%%%%%%%%%%%%%%%%%%%%%%%%%%%%%%
% Document:
\begin{document}
\renewcommand{\headheight}{14.5pt}

\fancyhead{}
\fancyhead[LE]{ \slshape \trauthor}
\fancyhead[LO]{}
\fancyhead[RE]{}
\fancyhead[RO]{ \slshape \trtitle}

%%%%%%%%%%%%%%%%%%%%%%%%%%%%
% Cover Header:
\begin{titlepage}
	\begin{flushleft}
		Universit\"at Hamburg\\
		Department Informatik\\
		\trarbeitsbereich\\
	\end{flushleft}
	\vspace{3.5cm}
	\begin{center}
		\huge \trtitle\\
	\end{center}
	\vspace{3.5cm}
	\begin{center}
		\normalsize\trtype\\
		[0.2cm]
		\Large\trcourse\\
		[1.5cm]
		\Large \trauthor\\
		[0.2cm]
		\normalsize Matr.Nr. \trmatrikelnummer\\
		[0.2cm]
		\normalsize\tremail\\
		[1.5cm]
		\Large \trdate
	\end{center}
	\vfill
\end{titlepage}

	%backsite of cover sheet is empty!
\thispagestyle{empty}
\hspace{1cm}
\newpage

%%%%%%%%%%%%%%%%%%%%%%%%%%%%
% Abstract:

% Abstract gives a brief summary of the main points of a paper:
\section*{Abstract}
Seit Jahrzehnten schon beschäftigen sich Forscher damit, den Computer dem Menschen näher zu bringen.
Ein wichtiger Bestandteil dieses Prozesses ist dabei die Sprachverarbeitung.
Ziel dieser Arbeit ist, Licht auf den Prozess des Verständnisses eines Computers von natürlicher Sprache zu werfen.
Zu diesem Zweck wird der Begriff der natürlichen Sprache in seine Komponenten zerlegt,
die einzelnd analysiert und in dieser Form von nichtkomplexen Automaten verarbeitet werden können.\\
\textit{Einbindung der Ergebnisse.}

% Lists:
\setcounter{tocdepth}{2} 					% depth of the table of contents (for Seminars 2 is recommented)
\tableofcontents
\pagenumbering{arabic}
\clearpage

%%%%%%%%%%%%%%%%%%%%%%%%%%%%
% Content:

% the actual content, usually separated over a number of sections
% each section is assigned a label, in order to be able to put a
% crossreference to it

\section{Einführung}
\label{sec:intro}
	\textit{„Ich wünschte, du würdest das nicht tun.
	Was ist, wenn ein integrierter Schaltkreis versagt und ich nicht rechtzeitig eingreifen kann?“\\
	„Ach, Andromeda, das würde dir ja nie passieren, denn dann brauchst du ja einen neuen Captain.“}
	\begin{flushright}
		\textit{-~~Andromeda und Captain Dylan Hunt, \cite{ADA102}}
	\end{flushright}
	\begin{itemize}
	    \item Kurzer Abriss der Geschichte
	    \item Erwähnung von ELIZA
	    \item Eingehen auf Rommie (und Cmdr. Data/HAL?)
	\end{itemize}

\section{Was ist natürliche Sprache?}
\label{sec:def_lang}
\textit{„By "natural language" we mean a language that is used for everyday communication by humans; languages such as English, Hindi, or Portuguese.
        In contrast to artificial languages such as programming languages and mathematical notations, natural languages have evolved as they pass from generation to generation.
        and are hard to pin down with explicit rules.“} \cite{Bird2009}
\begin{itemize}
    \item Analyse des Zitates mit Interpretation
    \item Auflistung und minimale Erläuterung der sechs zu erarbeitenden Eigenschaften
\end{itemize}

\section{Schrittweise Verarbeitung natürlicher Sprache}
\label{sec:meth}
tiefergehende Erläuterung der Begriffe und Begründung der gewählten Reihenfolge

\subsection{Parsing der Morphologie}
\label{ssec:morph}

\subsection{Analyse der Syntax}
\label{ssec:syntax}

\subsection{Interpretation der Semantik}
\label{sec:sem}

\subsection{Erfassung der Pragmatik}
\label{ssec:prag}

\subsection{Inbetrachtziehen des Diskurs}
\label{ssec:disc}

\subsection{Auflösung von Mehrdeutigkeit}
\label{ssec:ambi}

\section{Simon, Sirius, Jasper}
\label{sec:software}

\section{Über Haushaltsroboter bis C3PO}
\label{sec:ausb}
Es wurde gezeigt, wie Sprachverarbeitung ausgesehen hat (\ref{sec:intro}) und wie das Verständnis natürlicher Sprache funktioniert (\ref{sec:meth}).
Doch was kommt nun?
Was hat man davon, dass Computer Menschen verstehen können?
An drei Beispielen aus der Fiktion werden nun mögliche Anwendungsgebiete gezeigt.
\subsection{Das intelligente Haus}
\label{ssec:sarah}
\textit{„Herzlich Willkommen.“ - „Was ist das?“ - „DAS war Sarah.“\\
        „Selbstständig arbeitendes, rundum automatisiertes Haus. Kurz: S.A.R.A.H.“}
\begin{flushright}
    \textit{-~~Sarah, Jack Carter und Douglas Fargo, \cite{EUR102}}
\end{flushright}
\textit{Sarah} ist ein Haus aus der Science Fiction-Serie \textsc{Eureka}.
Tatsächlich ist \textit{Sarah} weniger Fiction als Science.
Es, oder Sie, ist tatsächlich eine künstliche Intelligenz, die in ein Haus eingebaut wurde.
\textit{Sarah} verwaltet dabei sämtliche Funktionen des Hauses: Lüftung, Raumtemperatur, Küche, Multimedia, Türen.
Sie kommuniziert mit den Bewohnern des Hauses durch gesprochene Sprache.
So bittet der Hausherr sie zum Beispiel mit dem Aufruf \textit{„Sarah, Tür!“} darum, die Tür zu öffnen oder zu schließen.
Doch über rudimentäre Befehle hinaus können die Bewohner auch komplexe Unterhaltungen über den Beruf, die Schule oder die Gefühle führen.
Nun wird viel nötig sein, um Empathie oder Emotionen simulieren zu können.
Jedoch ist es nicht unrealistisch, davon auszugehen, dass in nicht allzu ferner Zukunft solche Gespräche zwischen Mensch und Maschine möglich sind.
\subsection{Rechercheunterstützende Computer}
\label{ssec:data}
\textit{„Computer, erbitte alle verfügbaren Informationen über die fiktive Figur 'Dixon Hill'.“\\
        „Bitte warten. Die Figur tauchte zum ersten Mal in dem Magazin 'Erstaunliche Detektivgeschichten' auf, [...]“}
\begin{flushright}
    \textit{-~~Commander Data, \cite{TNG112}}
\end{flushright}
Recht früh in der Serie \textit{Star Trek: The Next Generation} gibt es eine Folge, in der der Android Data dem Computer des Raumschiffes \textit{Enterprise} eine Rechercheanweisung gibt.

\subsection{R2D2 und C3PO}
\label{ssec:r2d2}
\textit{„Er ist ein Protokoll-Droide und soll Mom helfen.[...]“ - „Oh! Hallo! Ich bin C-3PO, Roboter-Mensch-Kontakter.“}
\begin{flushright}
    \textit{-~~Anakin Skywalker, \cite{SWEP1}}
\end{flushright}
Das \textit{Star Wars}-Universum wimmelt von Droiden.
Und unter diesen Droiden finden sich R2D2 und C3PO, wohl zwei der bekanntesten Roboter der Filmgeschichte.

\section{Schlussfolgerung}
\label{sec:concl}
\textit{evtl mit Ausblick zusammenziehen}



%%%%%%%%%%%%%%%%%%%%%%%%%%%%%%%%%%%%%%
% hier werden - zum Ende des Textes - die bibliographischen Referenzen
% eingebunden
%
% Insbesondere stehen die eigentlichen Informationen in der Datei
% ``bib.bib''
%
\newpage
\bibliographystyle{alpha}
\addcontentsline{toc}{section}{Quellenverzeichnis}% Add to the TOC
\bibliography{bib}
\nocite{*}

\end{document}